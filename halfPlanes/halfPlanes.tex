\chapter{Ordering in the Plane}\label{chapter:HalfPlanes}
In the last chapter, we considered the theory of linear-ordering based on Hilbert's three-place $\code{between}$ relation. This culminated in a tactic for solving problems relating points along a line by reducing them to a decision procedure for linear problems of natural numbers. The use of natural numbers was explicitly permitted in Veblen's ordered geometry~\cite{Veblenphd}, but, at least, in the last chapter, we showed how they can be recovered in higher-order logic from geometry without the axiom of infinity.

Our tactic reduces geometry problems to linear arithmetic, and thus allows us to reason about the relative positions of points along a line. But what if we want to reason about the relative positions of points in a \emph{plane}? For this, we can think about a line in the plane and ask which \emph{side} of the line a point in the plane must lie on. This idea of \emph{sides} is introduced next in the \emph{Grundlagen der Geometrie}, followed by analogous ideas which culminate in the definition of a \emph{ray}. The two ideas shall be the subject of this chapter. 

\section{Definitions and Formalisation}
After stating that there are an infinite number of points on a line in THEOREM~7, Hilbert gives the following theorem and definitions:

\begin{quotation}
THEOREM~8. Every line $a$ that lies in a plane $\alpha$ separates the points which are not on the plane $\alpha$ into two regions with the following property: Every point $A$ of one region determines with every point $B$ of the other region a segment $AB$ on which there lies a point of the line $a$. However any two points $A$ and $A'$ of one and the same region determine a segment $AA'$ that contains no point of $a$.

\begin{center}\includegraphics{halfPlanes/halfPlanesDef}\end{center}

DEFINITION. \emph{The points $A$,$A'$ are said to lie in the plane $\alpha$ on one and the same side of the line $a$ and the points $A$, $B$ are said to lie in the plane $\alpha$ on different sides of $a$.}

DEFINITION. Let $A$, $A'$, $O$, $B$ be four points of the line $a$ such that $O$ lies between $A$ and $B$ but not between $A$ and $A'$. The points $A$, $A'$ are then said to lie \emph{on the line $a$ on one and the same side of the point $O$ and the points $A$, $B$ are said to lie on the line on different sides of the point $O$. The totality of points of the line $A$ that lie on one and the same side of $O$ is called a \emph{ray} emanating from $O$. Thus every point of a line partitions it into two rays.}
\end{quotation}

We are unclear why these definitions are emphasised. \emph{Rays} appear to be an important notion for later definitions and axioms, but a potential and unwieldy four-place relation $\code{same\_side}$
relating the point $O$ and line $a$ with every other pair of points did not appeal to us. Besides, once we have a notion of rays, we can just say that two points are on the line $a$ on the same side of $O$ when they lie on a \emph{ray} lying on $a$ and emanating from $O$.

We decided we would only formalise the notion of \emph{ray}, and we took a similar attitude with the first definition and focused on the two-dimensional analogue of rays, namely \emph{half-planes}. With that view, a \emph{half-plane} is the totality of points on the same side of a line $a$ in the plane $\alpha$. Two points can then be said to lie on the same side of the line $a$ precisely because they lie on a single half-plane in $\alpha$ and bounded by $a$. This appears to take us away from Hilbert's prose, but in \S\ref{sec:HalfPlaneTranslations}, we explain how the prose rendering can easily be recovered from the formalisation.

\subsection{Rays}
As we shall see in Chapters~\ref{chapter:JordanVerification1} and~\ref{chapter:JordanVerification2}, we did not find the notion of a ray a particularly useful tool in formal proofs, so in this subsection, we will need to justify keeping them in our formalisation. We believe they will be crucial in Group~III, where Hilbert introduces an axiom governing congruence of angles:

\begin{quotation}
  Let $\alpha$ be a plane and $h$,$k$ any two distinct rays emanating from $O$ in $\alpha$ and lying on {\bfseries distinct lines}. The pair of rays $h$, $k$ is called an \emph{angle} and is denoted by $\angle(h,k)$ or by $\angle(k,h)$.

\ldots

III,4. \emph{Let $\angle (h,k)$ be an angle in a plane $\alpha$ and $a'$ a line in a plane $\alpha'$ and let a definite side of $a'$ in $\alpha'$ be given. Let $h'$ be a ray on the line $a'$ that emanates from the point $O'$. Then there exists in the plane $\alpha'$ one and only one ray $k'$ such that the angle $\angle (h,k)$ is congruent or equal to the angle $\angle (h',k')$ and at the same time all interior points of the angle $\angle (h',k')$ lie on the given side of $a'$.}
\end{quotation}

The thing to note here is that Hilbert's axioms have increased substantially in complexity. Here, we have an axiom which juggles eight geometric entities, six of which are \emph{derived notions}. It is easy to make a mistake in formalising this axiom if one is starting from just Hilbert's primitives. We recommend that, if this axiom is to be reliably formalised, that the notions of \emph{angle}, and thus, the dependent notion of \emph{ray} must be fully formalised and a decent theory developed before we can trust that the definitions and axiom are correct. In simple type theory, one can gain further confidence by keeping the definitions reasonably type-safe. We shall develop some of this theory in the current chapter, where we hope it can be used for further formalisation of Group~III.

\subsection{Quotienting}\label{sec:RayQuotienting}
With a slight clarification, the ``same side'' relations in Hilbert's definitions define equivalence relations, and rays and half-planes emerge as the equivalence classes. In particular, the relation ``same side of the point $O$'' quotients the set of points in space other than a point $O$ into the set of all rays emanating from $O$, or alternatively, with \emph{origin} $O$. Note that we account for all three dimensions here, and allow rays to emanate from a point in all directions. Similarly, the relation ``same side of the line $a$'' quotients the set of points in space not on the line $a$ into the set of all half-planes bounded by $a$. 

We have had to fill in an ambiguity in Hilbert's definition, and exclude from consideration the origin $O$ for rays and the boundary line $a$ for half-planes. If we include $O$ and $a$, and we allow an arbitrary point to be on the same side of $O$ or $a$, then we will have only one equivalence class: the whole of space. If we include $O$ and $a$ but declare all other points to be on a different side of $O$ and the points of $a$, then our equivalence classes tell us that the set $\{O\}$ counts as a zero-dimensional ray, while the line $a$ counts as a one-dimensional half-plane. We exclude these possibilities, and thus make all rays and half-planes, as equivalence classes, open sets: a ray does not include its origin and a half-plane does not include its boundary. As an aside, Poincar\'{e} made the same decision, remarking parenthetically in his review of Hilbert ``I add, for precision, that I consider [the origin] as not belonging to either [half-ray]''~\cite{PoincareReview}.

\subsection{Automatic Lifting}
Now HOL~Light has several powerful procedures for automatically dealing with \emph{quotienting} and producing a strong type for the quotient sets. Assuming that $R:\alpha\rightarrow \alpha \rightarrow \alpha$ is an equivalence relation on $\alpha$, there is a procedure which splits $\alpha$ into equivalence classes. A new abstract type is then introduced in the theory, isomorphic with the class of all these equivalence classes. Additional procedures then exist which allow the user to lift HOL functions which are provably well-defined for the equivalence relation to the abstract type. We wanted to use all of these facilities to introduce the new abstract type of \emph{rays} and \emph{half-planes}, and introduce our primitive relations on these abstract types by lifting well-defined relations. 

Unfortunately, we do not have this sort of equivalence relation. The ``same side'' relations we define above are only equivalence relations on families of subsets of space. Our equivalence relations for rays are indexed by a point $O$ and have as domain the set of points in space minus $O$. Our equivalence relations for half-planes are indexed by a line $a$ and have as domain the set of points in space minus $a$. Simple type theory does not allow us to consider these families at the type-level.

We were not sure how best to tackle this problem, and we have not implemented a generic solution. Instead, in this section, we review one possible strategy which takes us to our new quotiented type via an intermediate type. Here, we will only consider the strategy applied to rays. The half-planes case is exactly analogous.

\subsubsection{Intermediate Types}
For any point $O$, we must consider the set of all points $P$ in space which are not on $O$. We can do this by pushing $O$ and $P$ into a pair and abstract their distinctness into a new type \code{arrow}. This is the type of \emph{arrows} $\overrightarrow{OP}$ where $O \neq P$. The origin of an arrow is the point $O$, and it points in the direction $P$.

The relation ``same side of'' can now be reinterpreted on these arrows. Our relation will effectively ask whether two arrows have the same position and direction. With some abuse of HOL~Light notation (we pretend that we can extract the two endpoints of an arrow with a pattern match), we have:
\begin{equation}\label{eq:EquivArrowDef}
  \begin{split}
    &\code{equiv\_arrow}\ :\ \code{arrow} \rightarrow \code{arrow}\rightarrow \code{bool}\\
    &\code{equiv\_arrow}\ \overrightarrow{OP}\ \overrightarrow{OQ}\iff O = O' \wedge (P = Q \vee \between{O}{P}{Q} \vee \between{O}{Q}{P})
  \end{split}
\end{equation}

We now just verify that this relation is an equivalence relation on the type of arrows:
\begin{equation}
  \begin{split}
    &\code{equiv\_arrow}\ s\ s\\
    &\wedge (\code{equiv\_arrow}\ s\ t \iff \code{equiv\_arrow}\ t\ s)\\
    &\wedge (\code{equiv\_arrow}\ s\ t \wedge \code{equiv\_arrow}\ t\ u \implies\code{equiv\_arrow}\ s\ u)
  \end{split}
\end{equation}

\label{sec:RayTransitivity}The only potential challenge needed when proving this theorem is dealing with transitivity. In our earlier work~\cite{ScottMScThesis}, where we tried to define rays as equivalence classes without using any automatic quotienting, the verification took some hard pen-and-paper work before we could transcribe it. We were bogged down with picky variable instantiations needed to apply \ref{eq:five}, made worse by the disjunction in our definition \eqref{eq:EquivArrowDef} which throws up several case-splits. But in our HOL~Light development, we have the linear reasoning tactic from the last chapter, which makes the matter trivial. It automatically deals with the case-splits, and can solve the goal without any explicit reference to other theorems.

Next, we will discuss some of the development of our theory of rays. This differs from our earlier work in Isabelle since we are using quotienting procedures in HOL~Light. Moreover, the theory we develop here is cleaner and has much better coverage of the important ideas. Almost all of the theorems we verified were chosen from analogous theorems in the theory of half-planes which we cover in \S\ref{sec:HalfPlaneTheory}, and in whose completeness we are confident having used the theory extensively in verifying the Polygonal Jordan Curve Theorem (see Chapters~\ref{chapter:JordanVerification1} and~\ref{chapter:JordanVerification2}).

\subsubsection{Lifting to a Theory of Rays}
With the equivalence relation verified, it is a simple matter to define the quotient type of rays. With the command
\begin{displaymath}
  \code{define\_quotient\_type}\ \code{"ray"}\ (\code{"mk\_ray"},\code{"dest\_ray"})\ \code{equiv\_arrow}
\end{displaymath}
we introduce a new type \code{ray} into the theory, together with abstraction and representation functions $\code{mk\_ray}$ and $\code{dest\_ray}$ which map between equivalence classes of arrows and the rays they represent. These functions are typical in typed theorem provers when we want to carve out a type from a subset of an existing type. They are used to prove the most elementary theorems of the new type, the ones which require reasoning in terms of the representatives.

The great thing about HOL~Light's quotienting facilities is that we have no need to deal with these particular abstraction and representation functions. All formal proofs of lifted theorems will apply these functions automatically. 

However, HOL~Light will not automatically plumb theorems about the endpoints of arrows through our intermediate type and lift them to our type of rays. Consider the relation which says that a given point lies on a given ray. If rays were an equivalence class on the space of all points, this relation would be lifted directly from the partially applied equivalence relation. Here, we must instead build an arrow, using the abstraction function for arrows, namely $\code{mk\_arrow}\ :\ (\code{point},\code{point})\rightarrow\code{arrow}$.
\begin{gather}
  \code{arrow\_origin}\ \overrightarrow{OP} = O\\
  \begin{split}
    &\code{on\_ray\_of\_arrow}\ P\ \code{a}\\
    &\qquad\iff P \neq \code{arrow\_origin}\ a\\
    &\qquad\qquad\quad\wedge \code{equiv\_arrow}\ a\ (\code{mk\_arrow}\ (\code{arrow\_origin}\ a, P))
  \end{split}
\end{gather}

It is trivial to verify that this relation is well-defined, but to use it effectively in proofs relating points of an arrow to points on the ray of an arrow, we need to manually fold and unfold the definition of arrows. It is tedious enough to verify the fact that
\begin{equation}
  \code{on\_ray\_of\_arrow}\ P\ \overrightarrow{OQ} \iff P = Q\;\vee\;\between{O}{P}{Q}\;\vee\;\between{O}{Q}{P}
\end{equation}

This could potentially be fixed by extending HOL~Light's quotienting facilities to handle indexed families of equivalence relations, and we would like to propose such extensions as further work. As it stands, we believe that Hilbert's geometry offers a nice example of where such facilities would be useful.

For completeness, we briefly mention the intermediate type for half-planes and the equivalence relation we define on this intermediate type. We will mediate the notion of half-plane by a line and a point not on that line, where a ray was mediated by a point and a distinct point. The half-plane intermediary lacks the pleasing geometric interpretation of \emph{arrows}, but the basic plumbing and proofs are similar. 

Our equivalence relation is unfortunately convoluted by constraints, since the main property saying when two points are on the same side of a line is a negative one and thus quite weak.
\begin{equation}\label{eq:SameSidePlaneDef}
  \begin{split}
    & \code{equiv\_half\_plane}\ (P,a)\ (Q,b)\\
    & \qquad\iff\ a=b \\
    & \qquad\quad\qquad \wedge (\exists \alpha. \code{on\_plane}\ P\ \alpha \wedge \code{on\_plane}\ Q\ \alpha\\
    & \qquad\quad\qquad\qquad \wedge \forall P. \code{on\_line}\ P\ a \implies \code{on\_plane}\ P\ \alpha)\\
    & \qquad\quad\qquad \wedge \neg\code{on\_line}\ P\ a \wedge \neg\code{on\_line}\ Q\ a\\
    & \qquad\quad\qquad \wedge \neg(\exists R. \code{on\_line}\ R\ a \wedge \between{P}{R}{Q})
  \end{split}
\end{equation}

The correctness of this definition may not be immediately obvious. If the reader is still unsure of its correctness after careful inspection, they can perhaps be assured by the fact that we have derived many of the expected theorems about half-planes. We consider these next.

\section{Theory of Half-Planes}\label{sec:HalfPlaneTheory}
The theory of rays is largely trivial when we have our linear reasoning tactic. Everything we want to know about linear order is bound up in THEOREM~6, from which that tactic was derived. Two-dimensional order is another story. We get this impression from Hilbert himself, who justifies the definition of half-planes with a distinguished theorem (THEOREM~8), but merely \emph{assumes} his definition of rays is sound.

As with the theory of rays, our theory is based on lifting from an intermediate type. Many of the theorems are trivial, and are only provided to link the primitive types $\code{point}$, $\code{line}$ and $\code{plane}$ with our new type of $\code{half\_plane}$. We trust that the theory developed is complete, inasmuch as all theorems about half-planes can be derived as if $\code{half\_plane}$ were entirely abstract. We have some evidence for this given that we have used the theory to verify a major theorem of ordered geometry, the Polygonal Jordan Curve Theorem without ever having to unfold our $\code{half\_plane}s$~(see Chapters~\ref{chapter:JordanVerification1} and~\ref{chapter:JordanVerification2}).

The two non-trivial theorems we need to verify are, firstly, that the relation defined above is transitive, and secondly, that there are exactly two half-planes to each plane. As we shall see, this can be understood as a strengthening of Pasch's Axiom~\eqref{eq:g24}.

\subsection{Transitivity}
Consider the transitivity problem. Suppose that the points $A$ and $B$ are on the same side of the line $a$, and that $B$ and $C$ are also on the same side. We must show that $A$ and $C$ are then on the same side.

According to our definition \eqref{eq:SameSidePlaneDef}, this means we must show that if the line $a$ does not intersect between $A$ and $B$, and does not intersect between $B$ and $C$, then it cannot intersect between $A$ and $C$. Equivalently, if there is an intersection at $A$ and $C$, then there is an intersection either between $A$ and $B$ or between $B$ and $C$. This is already very close to Pasch's axiom.

Pasch's axiom \eqref{eq:g24} asserts that, given a triangle $ABC$, if a line $a$ lies in the plane of $ABC$ and crosses the side of a triangle, and does not intersect a vertex, then it must leave by one of the other two sides. We have most of these assumptions in place. We know that our points $A$, $B$ and $C$ are planar: that assumption was made part of the definition \eqref{eq:SameSidePlaneDef}. We know that the line $a$ does not meet any vertex, since this is a defining requirement of any representative of our intermediate type. The only assumption we have not met is that $A$, $B$ and $C$ form a triangle.

But actually, this assumption on Pasch's axiom is not necessary. The conclusion holds even if $A$, $B$ and $C$ lie on a line, though we have not been able to prove it up until now. The verification, which uses THEOREM~6 via our linear reasoning tactic, is given in Figure~\ref{fig:PaschColCase}. 

Now we had proven this theorem in our earlier work in Isabelle~\cite{ScottMScThesis}, but there, we derived two lemmas whose proofs are based on numerous messy case-splits. In our new formalisation, these messy details vanish. We take this as evidence, along with our short verification of transitivity for rays in \S\ref{sec:RayTransitivity}, that our linear reasoning tactic is relieving us of some burden.

\begin{boxedfigure}
% let g24_col_case = theorem
%   "!A B C P a.
%          on_line P a /\ ~on_line C a /\ between A P B
%          /\ (?a. on_line A a /\ on_line B a /\ on_line C a)
%          ==> ?Q. on_line Q a /\ (between A Q C \/ between B Q C)"
%   [fix ["A:point";"B:point";"C:point";"P:point";"a:line"]
%   ;assume "on_line P a /\ ~on_line C a /\ between A P B" at [0]
%   ;assume "?a. on_line A a /\ on_line B a /\ on_line C a" at [1]
%   ;take ["P:point"]
%   ;thus "on_line P a" from [0]
%   ;have "~(C = P)" from [0]
%   ;hence "between A P C \/ between B P C" using ORDER_TAC `{A:point,B,C,P}` from [0;1]];;
  \begin{align*}
    & \code{assume}\ \code{on\_line}\ P\ a\wedge\neg\code{on\_line}\ C\ a \wedge \between{A}{P}{B} & 0\\
    & \code{assume}\ \exists a. \code{on\_line}\ A\ a\wedge\code{on\_line}\ B\ a\wedge\code{on\_line}\ C\ a&1\\
    & \code{take}\ P\\
    & \code{thus}\ \code{on\_line}\ P\ a\ \code{from}\ 0\\
    & \code{have}\ C \neq P\ \code{from}\ \code{from}\ 0\\
    & \code{hence}\ \between{A}{P}{C} \vee \between{B}{P}{C}\ \code{using ORDER\_TAC}\ \{A,B,C,P\}\ \code{from}\ 0,1
  \end{align*}
  \caption{Pasch's Axiom when $A$, $B$ and $C$ are collinear}
  \label{fig:PaschColCase}
\end{boxedfigure}

In the proof in Figure~\ref{fig:PaschColCase}, we have pared down the assumptions significantly. Now that we assume that the three points are collinear, there is no need to mention planes. Without the planes, the only remaining assumption is the one which says that the line $a$ does not meet any vertex. In verifying the theorem, we initially thought to throw out this assumption, believing it was as unnecessary as the planar assumption, but our linear reasoning tactic thought otherwise. It promptly told us that the resulting situation entails no contradiction. It will not give us a valid model, but with a moment's thought, we realise that if $C = P$, then the strictness of the $\code{between}$ relation means that the conclusion cannot possibly hold.

To fix this, we add back the assumption that $C$ is not on the line $a$. Notice that we then have to explicitly add a step showing that $C \neq P$, since the linear reasoning tactic will not infer this automatically: it only rewrites equalities, inequalities and betweenness claims, so we must feed it the necessary facts explicitly.

This pattern of using the linear reasoning tactic with very few assumptions, and then manually adding in facts until a contradiction is found, was our typical use-case of the tactic. We benefit from the fact that the tactic is a decision procedure, and the problems we throw at it are normally sufficiently constrained that a yes/no answer is delivered promptly. As such, the tactic can be used to explore ideas as well as verify steps that are known in advance to be valid.

\subsection{Covering}
Our next theorem shows that there are at most two half-planes to each plane. This theorem is lifted from an analogous theorem on our intermediate type, but the basic details are again a strengthening of Pasch's axiom.

We need to prove that of three points $A$, $B$ and $C$ in a plane $\alpha$ containing a line $a$, it cannot be the case that $A$, $B$ and $C$ are on mutually distinct sides of $a$. In terms of our definition \eqref{eq:SameSidePlaneDef}, this amounts to showing that $a$ cannot simultaneously intersect between the pair of points $A$ and $B$, the points $A$ and $C$ and the points $B$ and $C$. 

\label{sec:PaschInclusiveOr}Thus, if $ABC$ is a triangle, we are being asked to refute the possibility that the line $a$ intersects all three sides. This fact would be immediate if the conclusion of Pasch's axiom was rendered with the exclusive-or. 

This might well have been the case in the first edition of the text. Hilbert uses an ``either...or'' for the axiom (and the analogous construction in the German edition). By the tenth edition, the ``either'' has disappeared, and now, Hilbert makes the explicit claim that the inclusive case can be refuted. In other words, it is clear that he intends the inclusive-or in the axiom, and expects the inclusive case to be proved impossible.

Bernays thought the mere claim of a proof's existence was insufficient. In Supplement~I to the text, he gives the proof in full:

\begin{quotation}\label{sec:SupplementI}
It behooves one to deduce the proof by means of THEOREM~4. It can be carried out as follows: If the line $a$ met the segments $BC$, $CA$, $AB$ at the points $D$, $E$, $F$ then these points would be distinct. By THEOREM~4 one of these points would lie between the other two.

If, say, $D$ lay between $E$ and $F$, then an application of Axiom~II,4 to the triangle $AEF$ and the line $BC$ would show that this line would have to pass through a point of the segment $AE$ or $AF$. In both cases a contradiction of Axiom~II,3 or Axiom~I,2 would result.
\end{quotation}

\begin{figure}
  \centering\includegraphics{halfPlanes/SupplementI}
  \caption{Supplement~I}
  \label{fig:SupplementI}
\end{figure}

This is an indirect proof, effectively based on an impossible diagram. The key inference is in the second paragraph; that is, if $A$, $C$ and $E$ are collinear, then we can use Pasch's Axiom to conclude that $C$ must lie between $A$ and $E$, contradicting our assumptions. The diagram shows the situation in Figure~\ref{fig:SupplementI}, and shows how this step corresponds precisely to a use of the \emph{outer Pasch} version of the axiom~\eqref{eq:OuterPasch}.

\begin{boxedfigure}
  \begin{align*}
    &\code{assume}\ \Triangle{a}{A}{B}{C} & 0\\
    &\qquad\qquad \code{on\_line}\ D\ a \wedge \code{on\_line}\ E\ a \wedge \code{on\_line}\ F\ a & 1\\
    &\code{assume}\ \between{A}{D}{B} \wedge \between{A}{E}{C} \wedge \between{B}{F}{C} & 2\\
    &\code{obviously have}\ D\neq E \wedge D\neq F \wedge E\neq F\ \code{from}\ 0,2\\
    &\code{hence}\ \between{D}{E}{F}\vee\between{D}{F}{E}\vee\between{F}{D}{E}\ \code{from}\ 1\ \code{by} \ \eqref{eq:four}&3\\
    &\code{have}\ \forall A'\;B'\;C'\;D'\;E'\;F'. \Triangle{a}{A'}{B'}{C'}\\
    &\qquad \between{A'}{F'}{B'} \wedge \between{A'}{E'}{C'} \wedge \between{B'}{D'}{C'}\\
    &\qquad \implies \neg\between{E'}{D'}{F'}\\ 
    &\code{proof}\\
    &\qquad \code{fix}\ A',B',C',D',E',F'\\
    &\qquad \code{assume}\ \Triangle{a}{A'}{B'}{C'} & 4\\
    &\qquad \code{assume}\ \between{A'}{F'}{B'} \wedge \between{A'}{E'}{C'}\\ &\qquad\qquad\wedge\between{B'}{D'}{C'}\wedge\between{E'}{D'}{F'} & 5\\
    &\qquad\code{obviously consider}\ G\ \code{such that}\between{A'}{G}{E'} \wedge \between{B'}{D'}{G}\\
    &\qquad\qquad\code{by}\ \eqref{eq:OuterPasch},\eqref{eq:g21}\ \code{from}\ 4,5\\
    &\qquad\code{obviously qed from}\ 4,5\ \code{by}\ \eqref{eq:g23}\\
    &\code{qed from}\ 0,2,3\ \code{by}\ \eqref{eq:g21}
  \end{align*}
  \caption{Proof for Supplement~I}
  \label{fig:SupplementIProof}
\end{boxedfigure}

The formal proof is shown in Figure~\ref{fig:SupplementIProof}. Our incidence discoverer from Chapter~\ref{chapter:Automation} helps keep the proof steps almost one to one to the prose. We start by concluding as Bernays does that the points $D$, $E$ and $F$ are distinct and then apply THEOREM~4 to show that one of the points lies between the other two.

Bernays next makes a without-loss-of-generality assumption. We capture this with a subproof. There is some ugly repetition here with our $\code{assume}$ steps, but after this comes the two key inferences. Note that we use the outer Pasch axiom \eqref{eq:OuterPasch}, but, unlike Bernays, we leave out any mention of \eqref{eq:g12}. If we had to be consistently fussy in citing this axiom, it would have already appeared in the first step of the proof when showing that $D$, $E$ and $F$ are distinct. We leave it to implicit automation with our $\code{obviously}$ step.

Finally, we can strengthen this supplement by removing the assumption that $ABC$ forms a triangle. When $A$, $B$ and $C$ form a triangle, we have a linear problem, and our incidence discoverer and linear reasoning tactic can do all the work in just four steps. With this case considered, we can give Bernays' supplement in a very general form:
\begin{equation}\label{eq:SupplementI}
  \begin{split}
    &\neg\code{on\_line}\ A\ a \wedge \neg\code{on\_line}\ B\ a \wedge \neg\code{on\_line}\ C\ a\\
    &\wedge \code{on\_line}\ D\ a \wedge \code{on\_line}\ E\ a \wedge \code{on\_line}\ F\ a\\
    &\implies \neg\between{A}{D}{B} \vee \neg\between{A}{E}{C} \vee \neg\between{B}{F}{C}
  \end{split}
\end{equation}

We have now strengthened Pasch's axiom \eqref{eq:g24} in two ways: we have removed the assumption that $A$, $B$ and $C$ is a triangle, and we have removed the inclusive-or from the conclusion. Respectively, these two facts tell us that Hilbert's same-side relation for half-planes is transitive, and that there are at most two half-planes on any given plane.

\section{THEOREM~8}
It is not enough to say that at most two half-planes cover a plane. We must also show that there are \emph{at least} two half-planes in each plane. To that end, suppose we have a plane $\alpha$ and a line $a$ in $\alpha$. We find a  point $B$ on $a$, and a planar point $A$ off the line $a$. Then, with Axiom~\ref{eq:g22}, we can extend the segment $AB$ through the line $a$ to find a point $C$ on the other side. These two points will lie in distinct half-planes. Note that this proof requires that the point $B$ always exists, which is a consequence of Theorem~\ref{eq:PlaneThree} from Chapter~\ref{chapter:Axiomatics}. We noted at the time that this theorem was originally an axiom, but was later factored out. Hilbert does not explicitly say how the theorem is to be recovered, and as we suggested at the time, we do not believe the matter to be completely trivial.

Our final rendition of THEOREM~8 is a theorem lifted from our intermediate type. It relies on a number of additional lifted functions, such as $\code{line\_of\_half\_plane}$ and the incidence relation $\code{on\_half\_plane}$, whose specifications are given in Appendix~\ref{app:Group2}. 
\begin{equation}\label{eq:HalfPlaneCover}
  \begin{split}
    &(\forall P. \code{on\_line}\ P\ a \implies \code{on\_plane}\ P\ \alpha)\\
    &\implies \exists hp\; hq.\; hp \neq hq\\
    &\qquad \wedge a = \code{line\_of\_half\_plane}\ hp \wedge a = \code{line\_of\_half\_plane}\ hq\\
    &\qquad \wedge (\forall P. \code{on\_plane}\ P\ \alpha\\
    &\qquad\qquad \iff \code{on\_line}\ P\ a \vee \code{on\_half\_plane}\ hp\ P \vee \code{on\_half\_plane}\ hq\ P)
  \end{split}
\end{equation}

Note that we have had to break convention with our order of arguments for the incidence relation. While relations such as $\code{on\_line}$ have type $\code{point}\rightarrow\code{line}\rightarrow\code{bool}$, our half-plane incidence relation has the flipped type $\code{half\_plane}\rightarrow\code{point}\rightarrow\code{bool}$. We can understand why we have to put the $\code{point}$ type in second position when we consider exactly what we are saying is well-defined. It is not the two-place relation $\code{on\_half\_plane}$, but the \emph{set} of points incident with a half-plane. This set is well-defined regardless of our choice of representative in the intermediate type.

Now as predicates, point-sets are functions of type $\code{point} \rightarrow \code{bool}$, and it is a function of this form which we must verify as well-defined. We obtain the function by partial application of our intermediate incidence relation:
\begin{multline}
\code{same\_half\_plane}\ x\ y\\ \implies \code{on\_half\_plane\_intermediate}\ x = \code{on\_half\_plane\_intermediate}\ y
\end{multline}

Consequently, the type $\code{point}$ must appear last in the type of our half-plane incidence relation. This will explain our potentially confusing choice of formulation in the next few chapters.

\section{Conclusion}
In this chapter, we have given an overview of some of the challenges involved in developing a theory of rays and half-planes using HOL~Light's quotienting facilities. We would like to explore in more detail how we could automate the quotienting of partial relations as we have done in this chapter, and what technical difficulties would be involved in doing this. For now, our solution has been to create an abstract intermediate type so that the equivalence relations become total.

For half-planes, we noted that the principal justification for Hilbert's THEOREM~8 can be understood purely in terms of strengthening Pasch's Axiom~\eqref{eq:g24}. We just drop one of its hypotheses and render its disjunctive conclusion as an exclusive-or. In earlier editions, this exclusive-or was originally part of the axiom, but when it was dropped, Bernays felt that a proof of how it would be recovered needed spelling out. We would argue the same for Theorem~\ref{eq:PlaneThree}. This is also needed to prove THEOREM~8, and it was originally part of the axioms in the first edition. Now that it has been factored out, it would be useful to have its proof as another supplement to the text.

Much of the issues discussed so far in Group~II go some way to reinforcing the idea that Hilbert was not particularly concerned with ordered geometry. His axioms were highly redundant, and theorems such as THEOREM~8 are left without proof even as they require a fair amount of supplementary details. 

Hilbert is far more preoccupied with developing a theory of segment and angle congruence in Group~III, which makes up the bulk of his axiomatics. To help bring the point home, consider that Hilbert is happy to spell out the entirely trivial when it comes to working with congruence. He devotes three paragraphs to explaining that segment congruence is an equivalence relation, stressing that ``it is by no means obvious that {\bfseries every segment is congruent to itself.}'' This is quite a biased presentation. Hilbert omits all the the \emph{far} trickier detail needed to prove that his ``same-side'' relation for half-planes is an equivalence relation, and that it correctly partitions the plane.

Hilbert's presentation gets substantially weaker for Hilbert's next theorem, which he gives without any proof and yet which requires the most difficult we have considered so far. The Polygonal Jordan Curve Theorem is the subject of the next four chapters.

\begin{equation}\label{eq:halfPlaneOnPlane}
  \begin{split}
    &\code{on\_half\_plane}\ hp\ P \wedge \code{on\_plane}\ P\ \alpha \\
    &\wedge (\forall R.\;\code{on\_line}\ R\ (\code{line\_of\_half\_plane}\ hp) \implies \code{on\_plane}\ R\ \alpha) \\
    &\implies \code{on\_half\_plane}\ hp\ Q \implies \code{on\_plane}\ Q\ \alpha
  \end{split}
\end{equation}

\begin{equation}\label{eq:halfPlaneNotOnLine}
  \code{on\_half\_plane}\ hp\ P \implies \neg\code{on\_line}\ P\ (\code{line\_of\_half\_plane}\ hp)
\end{equation}

\begin{equation}\label{eq:onHalfPlaneNotBet}
  \begin{split}
    &(\forall R.\; \code{on\_half\_plane}\ hp\ R \implies \code{on\_plane}\ R\ \alpha) \wedge \code{on\_half\_plane}\ hp\ P\\
    &\implies (\code{on\_half\_plane}\ hp\ Q\\
    &\qquad \iff \neg(\exists R. \code{on\_line}\ R\ (\code{line\_of\_half\_plane}\ hp) \wedge \between{P}{R}{Q})\\
    &\qquad\qquad\quad \wedge \code{on\_plane}\ Q\ \alpha \wedge \neg\code{on\_line}\ Q\ (\code{line\_of\_half\_plane hp}))
  \end{split}
\end{equation}

\begin{equation}\label{eq:betOnHalfPlane1}
%!hp P Q R. on_line P (line_of_half_plane hp)
%      /\ on_half_plane hp Q
%      /\ (between P Q R \/ between P R Q) ==> on_half_plane hp R
  \begin{split}
    &\code{on\_line}\ P\ (\code{line\_of\_half\_plane}\ hp) \wedge \code{on\_half\_plane}\ hp\ Q\\
    &\implies \between{P}{Q}{R} \vee \between{P}{R}{Q} \implies \code{on\_half\_plane}\ hp
  \end{split}
\end{equation}

\begin{equation}\label{eq:betOnHalfPlane2}
  \begin{split}
  % "!P Q R hp. on_half_plane hp P /\ on_half_plane hp R /\ between P Q R 
  %             ==> on_half_plane hp Q"
    &\code{on\_half\_plane}\ hp\ P \wedge \code{on\_half\_plane}\ hp\ R\\
    &\implies \between{P}{Q}{R} \implies \code{on\_half\_plane}\ hp\ Q
  \end{split}
\end{equation}



%%% Local Variables: 
%%% mode: latex
%%% TeX-master: "../thesis"
%%% End: 


