\chapter{Ordering in the Plane}\label{chapter:HalfPlanes}
In the last chapter, we used linear arithmetic to settle problems of points ordered along a line. But what if we want to reason about the relative positions of points in a \emph{plane}? 

For this, we note that a line partitions a plane into two sides. So we can compare the relative position of points in the plane by asking on which side they are on. These \emph{sides} are defined in the \emph{Grundlagen der Geometrie} in Hilbert's next result, THEOREM~8, which he cites as the key theorem needed for the Polygonal Jordan Curve Theorem.

\section{Definitions and Formalisation}
\begin{quotation}
THEOREM~8. Every line $a$ that lies in a plane $\alpha$ separates the points which are not on the plane $\alpha$ into two regions with the following property: Every point $A$ of one region determines with every point $B$ of the other region a segment $AB$ on which there lies a point of the line $a$. However any two points $A$ and $A'$ of one and the same region determine a segment $AA'$ that contains no point of $a$.

\begin{center}\includegraphics{halfPlanes/halfPlanesDef}\end{center}

DEFINITION. \emph{The points $A$, $A'$ are said to lie in the plane $\alpha$ on one and the same side of the line $a$ and the points $A$, $B$ are said to lie in the plane $\alpha$ on different sides of $a$.}

DEFINITION. Let $A$, $A'$, $O$, $B$ be four points of the line $a$ such that $O$ lies between $A$ and $B$ but not between $A$ and $A'$. The points $A$, $A'$ are then said to lie \emph{on the line $a$ on one and the same side of the point $O$ and the points $A$, $B$ are said to lie on the line on different sides of the point $O$. The totality of points of the line $A$ that lie on one and the same side of $O$ is called a \emph{ray} emanating from $O$. Thus every point of a line partitions it into two rays.}

\flushright{\cite[p. 8]{FoundationsOfGeometry}}
\end{quotation}

Though we shall not need the notion of rays in our verifications, we shall describe some of its formalisation. The definition is effectively the one-dimensional analogue of THEOREM~8, and we will use it to explain the general approach to verifying that theorem. We also mention that \emph{rays} make for a useful abstraction in later definitions and axioms (see \S\ref{sec:RaysDef} below).

Now for rays, a three-place relation $\code{same\_side}$ relating a point with every other pair of points on a line is redundant when we can just relate a ray with its incident points. So we drop the relation. Similarly, we drop the relation for planes and instead introduce the two-dimensional analogue of rays, namely \emph{half-planes}. A half-plane will be the totality of points on the same side of a line $a$ in the plane $\alpha$. So when we say that two points lie on the same side of the line $a$ in the plane $\alpha$, we shall mean that they lie on a single half-plane in $\alpha$ and bounded by $a$. 

\subsection{Rays}\label{sec:RaysDef}
We briefly justify keeping the definition of rays in our formalisation. Rays are useful in Group~III, where Hilbert introduces an axiom governing congruence of angles:

\begin{quotation}
  Let $\alpha$ be a plane and $h$, $k$ any two distinct rays emanating from $O$ in $\alpha$ and lying on {\bfseries distinct lines}. The pair of rays $h$, $k$ is called an \emph{angle} and is denoted by $\angle(h,k)$ or by $\angle(k,h)$.

\ldots

III,~4. \emph{Let $\angle (h,k)$ be an angle in a plane $\alpha$ and $a'$ a line in a plane $\alpha'$ and let a definite side of $a'$ in $\alpha'$ be given. Let $h'$ be a ray on the line $a'$ that emanates from the point $O'$. Then there exists in the plane $\alpha'$ one and only one ray $k'$ such that the angle $\angle (h,k)$ is congruent or equal to the angle $\angle (h',k')$ and at the same time all interior points of the angle $\angle (h',k')$ lie on the given side of $a'$.}
\flushright{\cite[p. 11]{FoundationsOfGeometry}}
\end{quotation}

Notice how much more complicated Hilbert's axioms have become by this group. Here, we have an axiom which juggles eight geometric entities, six of which are not even primitive. It is easy to make a mistake here and we recommend that, if this axiom is to be reliably formalised, that the notions of \emph{angle}, and thus, the dependent notion of \emph{ray} must be fully formalised and a decent theory developed before we can trust that the definitions and axiom are correct. In simple type theory, one can gain further confidence by introducing rays as an abstract type, which can simplify the formalisation of Axiom~III,~4 by pushing constraints into the type-checker (see earlier work~\cite{ScottMScThesis}). 

\subsection{Quotienting}\label{sec:RayQuotienting}
With a slight clarification, the ``same side'' relations in Hilbert's definitions define equivalence relations, and rays and half-planes emerge as the equivalence classes. In particular, the relation ``same side of the point $O$'' quotients the set of points in space other than a point $O$ into the set of all rays emanating from $O$, or alternatively, with \emph{origin} $O$. Similarly, the relation ``same side of the line $a$'' quotients the set of points in space not on the line $a$ into the set of all half-planes bounded by $a$. Note that we do not restrict the dimension here, and thus allow rays and half-planes to emanate in all directions in space.

We have had to fill in an ambiguity in Hilbert's definition, and exclude the closure points or boundary from both rays and half-planes. If we include the point $O$ for a ray, or the line $a$ for a half-plane, and we allow an arbitrary point to be on the same side of $O$ or $a$, then we will have only one equivalence class: the whole of space. If we include $O$ and $a$ in every ray and half-plane, but declare all other points to be on a different side of $O$ and $a$, then our equivalence classes tell us that the set $\{O\}$ counts as a zero-dimensional ray, while the line $a$ counts as a one-dimensional half-plane. We exclude these possibilities, and thus make all rays and half-planes, as equivalence classes, open sets: a ray does not include its origin and a half-plane does not include its boundary. Incidentally, Poincar\'{e} made the same decision, remarking parenthetically in his review of Hilbert ``I add, for precision, that I consider [the origin] as not belonging to either [half-ray]''~\cite[p. 11]{PoincareReview}.

\subsection{Automatic Lifting}
HOL~Light has several powerful procedures for automatically dealing with quotienting and producing a strong type for the quotient sets. Assuming that $(\equiv)\;:\;\tau\rightarrow \tau \rightarrow \tau$ is an equivalence relation on $\tau$, there is a procedure which splits $\tau$ into equivalence classes. A new abstract type is then introduced in the theory, isomorphic with the class of all these equivalence classes. Additional procedures then exist which allow the user to \emph{lift} HOL functions which are provably well-defined for the equivalence relation to the abstract type. We wanted to use these facilities to introduce the new abstract types of rays and half-planes, and so introduce our primitive relations on these abstract types by lifting well-defined relations. 

Unfortunately, we do not have types for the domains of the equivalence relations. The ``same side'' relations Hilbert defines are only equivalence relations on families of subsets of space. Our equivalence relations for rays are indexed by a point $O$ and have as domain the set of points in space minus $O$. Our equivalence relations for half-planes are indexed by a line $a$ and have as domain the set of points in space minus $a$. Simple type theory does not allow us to consider these families at the type-level.

We were not sure how best to tackle this problem, and we have not implemented a generic solution. Instead, in this section, we review one possible strategy which takes us to our new quotiented type via an intermediate type. Here, we will only consider the strategy applied to rays. The half-planes case is exactly analogous.

\subsubsection{Intermediate Types}
For any point $O$, we must consider the set of all points $P$ in space which are not on $O$. We can do this by creating an abstract type represented by pairs $(O,P)$ where both components are distinct. We call this type \code{arrow}. It is the type of directed line-segments, or \emph{arrows} $\overrightarrow{OP}$, where $O \neq P$. The origin of the arrow is the point $O$, and the arrow \emph{points} in the direction $P$. By defining this type, we obtain abstraction and representation functions:
\begin{align*}
  \code{mk\_arrow}\ &:\ (\code{point},\code{point})\rightarrow\code{arrow}.\\
  \code{dest\_arrow}\ &:\ \code{arrow}\rightarrow(\code{point},\code{point}).
  \end{align*}
These are similar to $\code{mk\_ind}$ and $\code{dest\_ind}$ from the last chapter. They map back-and-forth between pairs of points and the arrows they represent.

The relation ``same side of'' can now be reinterpreted on these arrows. Our relation will effectively ask whether two arrows have the same position and direction. With some abuse of HOL~Light notation (we pretend that we can extract the two endpoints of an arrow with a \emph{pattern match}), we have:
\begin{equation}\label{eq:EquivArrowDef}
  \begin{split}
    &\code{equiv\_arrow}\ :\ \code{arrow} \rightarrow \code{arrow}\rightarrow \code{bool}\\
    \vdash_{def}\; &\code{equiv\_arrow}\ \overrightarrow{OP}\ \overrightarrow{O'Q}\\
    &\qquad\iff O = O' \wedge (P = Q \vee \between{O}{P}{Q} \vee \between{O}{Q}{P}).
  \end{split}
\end{equation}

We now just verify that this relation is an equivalence relation on the type of arrows:
\begin{equation*}
  \begin{split}
    \vdash&\code{equiv\_arrow}\ s\ s\\
    &\wedge (\code{equiv\_arrow}\ s\ t \iff \code{equiv\_arrow}\ t\ s)\\
    &\wedge (\code{equiv\_arrow}\ s\ t \wedge \code{equiv\_arrow}\ t\ u \implies\code{equiv\_arrow}\ s\ u).
  \end{split}
\end{equation*}

\label{sec:RayTransitivity}The only challenge when verifying this theorem is dealing with transitivity. In our earlier work~\cite{ScottMScThesis}, where we tried to define rays as equivalence classes without using any automatic quotienting, the verification took some hard pen-and-paper work before we could transcribe it. We were bogged down with picky variable instantiations needed to apply \ref{eq:five}, made worse by the disjunction in our definition \eqref{eq:EquivArrowDef} which throws up several case-splits. But in our HOL~Light development, we have the linear reasoning tactic from the last chapter, which makes the matter trivial. It automatically deals with the case-splits, and can solve the goal without any explicit reference to other theorems.

%Next, we will discuss the startegy development of our theory of rays. This differs from our earlier work in Isabelle since we are using quotienting procedures in HOL~Light. Moreover, the theory is cleaner and has much better coverage of the important ideas. Almost all of the theorems we verified were chosen from analogous theorems in the theory of half-planes which we cover in \S\ref{sec:HalfPlaneTheory}, and in whose completeness we are confident having used the theory extensively in verifying the Polygonal Jordan Curve Theorem (see Chapters~\ref{chapter:JordanVerification1} and~\ref{chapter:JordanVerification2}).

\subsubsection{Lifting to a Theory of Rays}
With the equivalence relation verified, it is a simple matter to define the quotient type of rays. With the command
\begin{displaymath}
  \code{define\_quotient\_type}\ \code{"ray"}\ (\code{"mk\_ray"},\code{"dest\_ray"})\ \code{equiv\_arrow}
\end{displaymath}
we introduce a new type \code{ray} into the theory, together with their abstraction and representation functions $\code{mk\_ray}$ and $\code{dest\_ray}$. 

The great thing about HOL~Light's quotienting facilities is that we do not need to deal with these particular abstraction and representation functions directly. When we lift theorems the functions are used automatically. 

However, HOL~Light will not automatically plumb theorems about the endpoints of arrows through our intermediate type and lift them to our type of rays. Consider the relation which says that a given point lies on a given ray. If rays were an equivalence class on the space of all points, this relation would be lifted directly from the partially applied equivalence relation. Here, we must instead build an arrow, using the abstraction function for arrows, namely $\code{mk\_arrow}$.
\begin{gather*}
  \begin{split}
    &\vdash_{def}\;\code{on\_ray\_of\_arrow}\ P\ \overrightarrow{OQ}\\
    &\qquad\iff P \neq O\\
    &\qquad\qquad\quad\wedge \code{equiv\_arrow}\ \overrightarrow{OQ}\ (\code{mk\_arrow}\ (O, P)).
  \end{split}
\end{gather*}

It is trivial to verify that this relation is well-defined, but to use it effectively in proofs relating points of an arrow to points on the ray of an arrow, we need to manually fold and unfold the definition of arrows. It is tedious enough to verify that
\begin{equation*}
  \vdash\code{on\_ray\_of\_arrow}\ P\ \overrightarrow{OQ} \iff P = Q\ \vee\ \between{O}{P}{Q}\ \vee\ \between{O}{Q}{P}.
\end{equation*}

We suspect that the creation of the intermediate \code{arrow} type and the lifting of theorems through this type could be automated, greatly improving the presentation of the theory. This would amount to an extension of HOL~Light's quotienting facilities, which would allow it to handle indexed families of equivalence relations such as that of arrows lying in the same direction, indexed by a point of origin. This would make for suitable future work, and we believe that Hilbert's geometry offers a nice example of where such facilities would be useful.

\section{Theory of Half-Planes}\label{sec:HalfPlaneTheory}
The theory of rays is largely trivial when we have our linear reasoning tactic. Everything we want to know about linear order is bound up in THEOREM~6, from which that tactic was derived. Two-dimensional order is less straightforward. We get this impression from Hilbert himself, who justifies the definition of half-planes with a distinguished theorem (THEOREM~8). He gives no corresponding theorem for rays, but instead, just assumes his definition is sound.

As with the theory of rays, our theory is based on lifting from an intermediate type. We mediate the notion of half-plane by a line and a point not on that line, where a ray was mediated by a point and a distinct point. The half-plane intermediary lacks the pleasing geometric interpretation of \emph{arrows}, but the basic plumbing and proofs are similar.
\begin{equation}\label{eq:SameSidePlaneDef}
  \begin{split}
    \vdash_{def}\;& \code{equiv\_half\_plane}\ (P,a)\ (Q,b)\\
    & \qquad\iff\ a=b \\
    & \qquad\quad\qquad \wedge (\exists \alpha.\; \code{on\_plane}\ P\ \alpha \wedge \code{on\_plane}\ Q\ \alpha\\
    & \qquad\quad\qquad\qquad \wedge \forall S.\;\code{on\_line}\ S\ a \implies \code{on\_plane}\ S\ \alpha)\\
    & \qquad\quad\qquad \wedge \neg\code{on\_line}\ P\ a \wedge \neg\code{on\_line}\ Q\ a\\
    & \qquad\quad\qquad \wedge \neg(\exists R.\; \code{on\_line}\ R\ a \wedge \between{P}{R}{Q}).
  \end{split}
\end{equation}

The equivalence relation is unfortunately convoluted by constraints, since the main property saying when two points are on the same side of a line is a negative one and thus quite weak. The correctness may not be immediately obvious. If the reader needs more confidence than simple inspection provides, they can perhaps be assured by the fact that we have derived many of the expected theorems about half-planes. 

Many of the theorems are trivial, and are only provided to link the primitive types $\code{point}$, $\code{line}$ and $\code{plane}$ with our new type of $\code{half\_plane}$. The two non-trivial theorems we need to verify are, firstly, that the relation defined above is transitive, and secondly, that there are exactly two half-planes to each plane. As we shall see, both theorems together can be understood as a strengthening of Pasch's Axiom~\eqref{eq:g24}.

\subsection{Transitivity}
Consider the transitivity problem. Suppose that the points $A$ and $B$ are on the same side of the line $a$, and that $B$ and $C$ are also on the same side. We must show that $A$ and $C$ are then on the same side.

According to our definition \eqref{eq:SameSidePlaneDef}, this means we must show that if the line $a$ does not intersect between $A$ and $B$, and does not intersect between $B$ and $C$, then it cannot intersect between $A$ and $C$. Equivalently, if there is an intersection at $A$ and $C$, then there is an intersection either between $A$ and $B$ or between $B$ and $C$. This is already very close to Pasch's axiom.

Pasch's axiom \eqref{eq:g24} asserts that, given a triangle $ABC$, if a line $a$ lies in the plane of $ABC$ and crosses the side of a triangle, and does not intersect a vertex, then it must leave by one of the other two sides. 

We have most of these assumptions in place. We know that our points $A$, $B$ and $C$ are planar: that assumption was made part of the definition \eqref{eq:SameSidePlaneDef}. We know that the line $a$ does not meet any vertex, since this is a defining requirement of any representative of our intermediate type. The only assumption we have not met is that $A$, $B$ and $C$ form a triangle.

But actually, this assumption on Pasch's axiom is not necessary. The conclusion holds even if $A$, $B$ and $C$ lie on a line, though we have not been able to prove it up until now. The verification, which uses THEOREM~6 via our linear reasoning tactic, is given in Figure~\ref{fig:PaschColCase}. 

\begin{boxedfigure}
% let g24_col_case = theorem
%   "!A B C P a.
%          on_line P a /\ ~on_line C a /\ between A P B
%          /\ (?a. on_line A a /\ on_line B a /\ on_line C a)
%          ==> ?Q. on_line Q a /\ (between A Q C \/ between B Q C)"
%   [fix ["A:point";"B:point";"C:point";"P:point";"a:line"]
%   ;assume "on_line P a /\ ~on_line C a /\ between A P B" at [0]
%   ;assume "?a. on_line A a /\ on_line B a /\ on_line C a" at [1]
%   ;take ["P:point"]
%   ;thus "on_line P a" from [0]
%   ;have "~(C = P)" from [0]
%   ;hence "between A P C \/ between B P C" using ORDER_TAC `{A:point,B,C,P}` from [0;1]];;
  \begin{align*}
    & \code{assume}\ \code{on\_line}\ P\ a\wedge\neg\code{on\_line}\ C\ a \wedge \between{A}{P}{B} & 0\\
    & \code{assume}\ \exists a.\; \code{on\_line}\ A\ a\wedge\code{on\_line}\ B\ a\wedge\code{on\_line}\ C\ a&1\\
    & \code{take}\ P\\
    & \code{thus}\ \code{on\_line}\ P\ a\ \code{from}\ 0\\
    & \code{have}\ C \neq P\ \code{from}\ \code{from}\ 0\\
    & \code{hence}\ \between{A}{P}{C} \vee \between{B}{P}{C}\ \code{using ORDER\_TAC}\ \{A,B,C,P\}\ \code{from}\ 0,1
  \end{align*}
  \caption{Pasch's Axiom when $A$, $B$ and $C$ are collinear}
  \label{fig:PaschColCase}
\end{boxedfigure}

In the proof in Figure~\ref{fig:PaschColCase}, we have pared down the assumptions significantly. Now that we assume that the three points are collinear, there is no need to mention planes. Without the planes, the only remaining assumption is the one which says that the line $a$ does not meet any vertex. In verifying the theorem, we initially thought to throw out this assumption, believing it was as unnecessary as the planar assumption, but our linear reasoning tactic thought otherwise. It promptly told us that the resulting situation entails no contradiction. It will not give us a valid model, but with a moment's thought, we realise that if $C = P$, then the strictness of the $\code{between}$ relation means that the conclusion cannot possibly hold.

To fix this, we add back the assumption that $C$ is not on the line $a$. Notice that we then have to explicitly add a step showing that $C \neq P$, since the linear reasoning tactic will not infer this automatically: it only rewrites equalities, inequalities and betweenness claims, so we must feed it the necessary facts explicitly.

This pattern of using the linear reasoning tactic with very few assumptions and lemmas, and then adding more in until the goal is solved, was our typical use-case of the tactic. We benefit from the fact that the tactic is a decision procedure, and the problems we throw at it are normally sufficiently constrained that a yes/no answer is delivered promptly. As such, the tactic can be used to explore ideas as well as verify steps that are known in advance to be valid (see \S\ref{sec:RayCasting} for further discussion).

\subsection{Covering}
Our next theorem shows that there are at most two half-planes to each plane. This theorem is lifted from an analogous theorem on our intermediate type, but the basic details are again a strengthening of Pasch's axiom.

We need to prove that of three points $A$, $B$ and $C$ in a plane $\alpha$ containing a line $a$, it cannot be the case that $A$, $B$ and $C$ are on mutually distinct sides of $a$. In terms of our definition \eqref{eq:SameSidePlaneDef}, this amounts to showing that $a$ cannot simultaneously intersect between the pair of points $A$ and $B$, the points $A$ and $C$ and the points $B$ and $C$. 

\label{sec:PaschInclusiveOr}Thus, if $ABC$ is a triangle, we are being asked to refute the possibility that the line $a$ intersects all three sides. This fact would be immediate if the conclusion of Pasch's axiom was rendered with the exclusive-or. 

This might well have been the case in the first edition of the \emph{Grundlagen}. Hilbert uses an ``either...or'' for the axiom (and the analogous construction in the German edition). By the tenth edition, the ``either'' has disappeared, and now, Hilbert makes the explicit claim that the inclusive case can be refuted. It is clear, then, that he intends the weak inclusive-or in the axiom, and expects the inclusive case to be proved impossible.

Bernays thought the mere claim of a proof's existence was insufficient. In Supplement~I to the text, he gives the proof in full:

\begin{quotation}\label{sec:SupplementI}
It behooves one to deduce the proof by means of THEOREM~4. It can be carried out as follows: If the line $a$ met the segments $BC$, $CA$, $AB$ at the points $D$, $E$, $F$ then these points would be distinct. By THEOREM~4 one of these points would lie between the other two.

If, say, $D$ lay between $E$ and $F$, then an application of Axiom~II,~4 to the triangle $AEF$ and the line $BC$ would show that this line would have to pass through a point of the segment $AE$ or $AF$. In both cases a contradiction of Axiom~II,~3 or Axiom~I,~2 would result.
\flushright{\cite[p. 200]{FoundationsOfGeometry}}
\end{quotation}

This is an indirect proof, effectively based on an impossible diagram. The key inference is in the second paragraph; that is, if $A$, $C$ and $E$ are collinear, then we can use Pasch's Axiom to conclude that $C$ must lie between $A$ and $E$, contradicting our assumptions. The diagram shows the situation in Figure~\ref{fig:SupplementI}, and shows how this step corresponds precisely to a use of the outer version of the Pasch axiom~\eqref{eq:OuterPasch}.

\begin{figure}
  \centering\includegraphics{halfPlanes/SupplementI}
  \caption{Supplement~I}
  \label{fig:SupplementI}
\end{figure}

The verification is shown in Figure~\ref{fig:SupplementIProof}. Our incidence discoverer from Chapter~\ref{chapter:Automation} helps keep the proof steps almost one to one with the prose. We start by concluding as Bernays does that the points $D$, $E$ and $F$ are distinct and then apply THEOREM~4 to show that one of the points lies between the other two.

Bernays next makes a without-loss-of-generality assumption. We capture this with a subproof. There is some ugly repetition here with our $\code{assume}$ steps, but after this comes the two key inferences. Note that we use the outer form of the Pasch Axiom \eqref{eq:OuterPasch}, but, unlike Bernays, we leave out any mention of \eqref{eq:g12}. If we had to be consistently fussy in citing this axiom, it would have already appeared in the first step of the proof when showing that $D$, $E$ and $F$ are distinct. We leave it to implicit automation with our $\code{obviously}$ step.

Finally, we can strengthen this supplement by removing the assumption that $ABC$ forms a triangle. When $A$, $B$ and $C$ are collinear, we have a linear problem, and our incidence discoverer and linear reasoning tactic can do all the work in just four steps. With this case considered, we can give Bernays' supplement in a very general form:
\begin{equation}\label{eq:SupplementI}
  \begin{split}
    \vdash&\neg\code{on\_line}\ A\ a \wedge \neg\code{on\_line}\ B\ a \wedge \neg\code{on\_line}\ C\ a\\
    &\wedge \code{on\_line}\ D\ a \wedge \code{on\_line}\ E\ a \wedge \code{on\_line}\ F\ a\\
    &\implies \neg\between{A}{D}{B} \vee \neg\between{A}{E}{C} \vee \neg\between{B}{F}{C}.
  \end{split}
\end{equation}

We have now strengthened Pasch's axiom \eqref{eq:g24} in two ways: we have removed the assumption that $A$, $B$ and $C$ is a triangle, and we have removed the inclusive-or from the conclusion. Respectively, these two facts tell us that Hilbert's same-side relation for half-planes is transitive, and that there are at most two half-planes on any given plane.


\begin{boxedfigure}
  \small
  \begin{align*}
    &\code{assume}\ \Triangle{a}{A}{B}{C} & 0\\
    &\qquad\qquad \code{on\_line}\ D\ a \wedge \code{on\_line}\ E\ a \wedge \code{on\_line}\ F\ a & 1\\
    &\code{assume}\ \between{A}{D}{B} \wedge \between{A}{E}{C} \wedge \between{B}{F}{C} & 2\\
    &\code{obviously}\ \code{(by\_ncols} \circ \code{conjuncts)}\ \code{have}\ D\neq E \wedge D\neq F \wedge E\neq F\ \code{from}\ 0,2\\
    &\code{hence}\ \between{D}{E}{F}\vee\between{D}{F}{E}\vee\between{F}{D}{E}\ \code{from}\ 1\ \code{by} \ \eqref{eq:four}&3\\
    &\code{have}\ \forall A'.\;\forall B'.\;\forall C'.\;\forall D'.\;\forall E'.\;\forall F'.\; \Triangle{a}{A'}{B'}{C'}\\
    &\qquad \between{A'}{F'}{B'} \wedge \between{A'}{E'}{C'} \wedge \between{B'}{D'}{C'}\\
    &\qquad \implies \neg\between{E'}{D'}{F'}\\ 
    &\code{proof:}\\
    &\qquad \code{fix}\ A',B',C',D',E',F'\\
    &\qquad \code{assume}\ \Triangle{a}{A'}{B'}{C'} & 4\\
    &\qquad \code{assume}\ \between{A'}{F'}{B'} \wedge \between{A'}{E'}{C'}\\ &\qquad\qquad\wedge\between{B'}{D'}{C'}\wedge\between{E'}{D'}{F'} & 5\\
    &\qquad\code{obviously}\ \code{(by\_ncols} \circ \code{conjuncts)}\ \code{consider}\ G\ \\ &\qquad\qquad\code{such that}\ \between{A'}{G}{E'} \wedge \between{B'}{D'}{G}\\ 
    &\qquad\qquad\code{by}\ \eqref{eq:OuterPasch},\eqref{eq:g21}\ \code{from}\ 4,5\\
    &\qquad\code{obviously}\ \code{(by\_eqs} \circ \code{conjuncts)}\  \code{qed}\ \code{from}\ 4,5\ \code{by}\ \eqref{eq:g23}\\
    &\code{qed from}\ 0,2,3\ \code{by}\ \eqref{eq:g21}
  \end{align*}
  \caption{Proof for Supplement~I}
  \label{fig:SupplementIProof}
\end{boxedfigure}
\section{THEOREM~8}
It is not enough to say that at most two half-planes cover a plane. We must also show that there are \emph{at least} two half-planes in each plane. To that end, suppose we have a plane $\alpha$ and a line $a$ in $\alpha$. We find a  point $A$ on $a$, and a planar point $B$ off the line $a$. Then, with Axiom~\ref{eq:g22}, we can extend the segment $BA$ through the line $a$ to find a point $C$ on the other side. The points $B$ and $C$ will then lie in distinct half-planes. 

Note that this proof requires that the point $B$ always exists, which is a consequence of Theorem~\ref{eq:PlaneThree} from Chapter~\ref{chapter:Axiomatics}. We noted at the time that this theorem was originally an axiom, but was later factored out. Hilbert does not explicitly say how the theorem is to be recovered, and as we suggested at the time, we do not believe the matter to be completely trivial.

Our final rendition of THEOREM~8 is a theorem lifted from our intermediate type. It relies on two lifted functions. The function $\code{line\_of\_half\_plane}$ returns the boundary of a given half-plane, while the function $\code{half\_plane\_contains}$ is the incidence relation for half-planes. We give some additional theorems to govern these concepts in Figure~\ref{fig:HalfPlanesAdditional}, most of which are referenced in our  account of the Polygonal Jordan Curve Theorem in Chapters~\ref{chapter:JordanVerification1} and~\ref{chapter:JordanVerification2}. Meanwhile, THEOREM~8 is verified as:
\begin{equation}\label{eq:HalfPlaneCover}
  \begin{split}
    \vdash&(\forall P.\;\code{on\_line}\ P\ a \implies \code{on\_plane}\ P\ \alpha)\\
    &\implies \exists hp.\;\exists hq.\; hp \neq hq\\
    &\quad \wedge a = \code{line\_of\_half\_plane}\ hp \wedge a = \code{line\_of\_half\_plane}\ hq\\
    &\quad \wedge (\forall P.\;\code{on\_plane}\ P\ \alpha\\
    &\quad\quad \iff \code{on\_line}\ P\ a \vee \code{half\_plane\_contains}\ hp\ P \vee \code{on\_half\_plane}\ hq\ P).
  \end{split}
\end{equation}

Note that we have broken convention with the name for our incidence relation, using 
\begin{displaymath}
\code{half\_plane\_contains}\ :\ \code{half\_plane}\rightarrow\code{point}\rightarrow\code{bool}
\end{displaymath}
and not 
\begin{displaymath}
\code{on\_half\_plane}\ :\ \code{half\_plane}\rightarrow\code{point}\rightarrow\code{bool}.
\end{displaymath}
We can understand why this is necessary by considering exactly what we would have to say is well-defined to obtain this relation. It is a set of points which are incident with a given representative of half-planes from our intermediate type. Now as predicates, point-sets are functions of type $\code{point} \rightarrow \code{bool}$, and it is a function of this form which we must verify as well-defined. We obtain the predicate by partial application of an intermediate incidence relation $\code{half\_plane\_intermediate\_contains}$:
\begin{multline*}
\vdash\code{equiv\_intermediate\_half\_plane}\ x\ y \\
\implies \code{half\_plane\_intermediate\_contains}\ x\\ =\code{on\_half\_plane\_intermediate\_contains}\ y.
\end{multline*}

Consequently, the type $\code{point}$ must appear last in the type of our half-plane incidence relation.
\begin{figure}
\begin{equation*}
  \begin{split}
    \vdash&(\forall P.\;\code{on\_line}\ P\ a \implies \code{on\_plane}\ P\ \alpha)\\
    &\implies \exists hp\;\exists hq.\; hp \neq hq\\
    &\qquad \wedge a = \code{line\_of\_half\_plane}\ hp \wedge a = \code{line\_of\_half\_plane}\ hq\\
    &\qquad \wedge (\forall P.\;\code{on\_plane}\ P\ \alpha\\
    &\qquad\quad \iff \code{on\_line}\ P\ a \vee \code{half\_plane\_contains}\ hp\ P \vee \code{on\_half\_plane}\ hq\ P)
  \end{split}
\end{equation*}
  \begin{equation}\label{eq:halfPlaneOnPlane}
    \begin{split}
      \vdash&\code{half\_plane\_contains}\ hp\ P \wedge \code{on\_plane}\ P\ \alpha \\
      &\wedge (\forall R.\;\code{on\_line}\ R\ (\code{line\_of\_half\_plane}\ hp) \implies \code{on\_plane}\ R\ \alpha) \\
      &\implies \code{half\_plane\_contains}\ hp\ Q \implies \code{on\_plane}\ Q\ \alpha
    \end{split}
  \end{equation}
  
  \begin{equation}\label{eq:halfPlaneNotOnLine}
    \vdash\code{half\_plane\_contains}\ hp\ P \implies \neg\code{on\_line}\ P\ (\code{line\_of\_half\_plane}\ hp)
  \end{equation}
  
  \begin{equation}\label{eq:onHalfPlaneNotBet}
    \begin{split}
      \vdash&(\forall R.\; \code{half\_plane\_contains}\ hp\ R \implies \code{on\_plane}\ R\ \alpha) \wedge \code{on\_half\_plane}\ hp\ P\\
      &\implies (\code{half\_plane\_contains}\ hp\ Q\\
      &\qquad \iff \neg(\exists R. \code{on\_line}\ R\ (\code{line\_of\_half\_plane}\ hp) \wedge \between{P}{R}{Q})\\
      &\qquad\qquad\quad \wedge \code{on\_plane}\ Q\ \alpha \wedge \neg\code{on\_line}\ Q\ (\code{line\_of\_half\_plane hp}))
    \end{split}
  \end{equation}
  
  \begin{equation}\label{eq:betOnHalfPlane1}
    % !hp P Q R. on_line P (line_of_half_plane hp)
    % /\ on_half_plane hp Q
    % /\ (between P Q R \/ between P R Q) ==> on_half_plane hp R
    \begin{split}
      \vdash&\code{on\_line}\ P\ (\code{line\_of\_half\_plane}\ hp) \wedge \code{half\_plane\_contains}\ hp\ Q\\
      &\implies \between{P}{Q}{R} \vee \between{P}{R}{Q} \implies \code{half\_plane\_contains}\ hp
    \end{split}
  \end{equation}
  
  \begin{equation}\label{eq:betOnHalfPlane2}
    \begin{split}
      % "!P Q R hp. on_half_plane hp P /\ on_half_plane hp R /\ between P Q R 
      % ==> on_half_plane hp Q"
      \vdash&\code{half\_plane\_contains}\ hp\ P \wedge \code{on\_half\_plane}\ hp\ R\\
      &\implies \between{P}{Q}{R} \implies \code{half\_plane\_contains}\ hp\ Q
    \end{split}
  \end{equation}
\caption{Theorems for half-Planes}
\label{fig:HalfPlanesAdditional}
\end{figure}
\section{Conclusion}
All preliminaries needed for the Polygonal Jordan Curve Theorem have now been dealt with. This chapter completes the necessary theory, showing how to use the quotienting facilities of HOL~Light to formalise the all important notion of half-planes. As we have seen, the fact that these objects exist and cover their planes can be understood as a strengthening of Pasch's Axiom.

We just note that one direction in which this axiom is weakened  was singled out by Bernays as needing an explicit proof. We feel the same way about Theorem~\ref{eq:PlaneThree}, which is also needed to prove THEOREM~8. Both supplementary proofs were unnecessary in the first edition, when the results were asserted axiomatically. But now that they have been factored out, one should be careful to derive them.

This omission is barely a chink compared to the chasm which follows. Hilbert gives no indication of how to prove THEOREM~9. A discussion of this theorem and its verification take up the next four chapters.
%%% Local Variables: 
%%% mode: latex
%%% TeX-master: "../thesis"
%%% End: 


