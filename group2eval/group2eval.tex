\chapter{Elementary Consequences in Group~II}\label{chapter:Group2Eval}
We now come to verify some theorems of Group~II, the only theorems in the first two groups which have prose proofs. Each proof uses Axiom~\ref{eq:g24}, which, as we explained in Chapter~\ref{chapter:Axiomatics}, carries several incidence preconditions. Establishing these preconditions made up the bulk of the effort in our manual verifications~\cite{ScottMScThesis} and in the verifications of Meikle and Fleuriot~\cite{MeikleFleuriotFormalizingHilbert}.

In the last chapter, we described how to implement tailored automation to handle incidence reasoning. An aim of the present chapter is to see how much this automation improves over the stock automation used in our manual verifications, in terms of writing shorter verifications which are easier to develop, and to what extent the focus on automated \emph{discovery} enables us to find alternative proofs. Our ideal for the final verifications is that their steps should match those of an ideal prose proof, here modelled by Hilbert's own arguments. These steps are just those used to define and exhibit the diagram needed to establish Hilbert's result, and are rarely used for discharging tedious incidence preconditions.

When writing our verifications, we tried our best to follow the specific steps in Hilbert's prose, and in this chapter, we shall directly compare our verifications with the originals. As we shall see, this allows us to comment on the prose in great detail.

\section{THEOREM~3}\label{sec:Theorem3}
Hilbert's first result tells us that there is a point between any two others.

\begin{quotation}
THEOREM 3. For two points $A$ and $C$ there always exists at least one point $D$ on the line $AC$ that lies between $A$ and $C$.

PROOF. By Axiom~I, 3 there exists a point $E$ outside the line $AC$ and by Axiom~II, 2 there exists on $AE$ a point $F$ such that $E$ is a point of the segment $AF$. By the same axiom and by Axiom~II, 3 there exists on $FC$ a point $G$ that does not lie on the segment $FC$. By Axiom II,~4 the line $EG$ must then intersect the segment $AC$ at a point $D$.

\vspace{0.5cm}
\centering\includegraphics{group2eval/Three}
\flushright{\emph{Foundations of Geometry}~\cite{FoundationsOfGeometry} (page 6)}
\end{quotation}

The only incidence axiom Hilbert appeals to in this proof is Axiom~I, 3, and the only order axioms are Axiom~II, 2 and Axiom~II,~4. The proof depends on more axioms than this, but Hilbert's omission is consistent with a claim we made in Chapter~\ref{chapter:Axiomatics}: Hilbert generally only cites axioms which \emph{introduce} points, omitting the others because he wants to focus on the steps which build up the diagrams. Here, we have Axiom I, 3 which can be used (indirectly) to obtain points off arbitrary lines, in this case the point $E$. We have Axiom~II, 2 which is our ``line-extension axiom'', used here to first obtain the point $F$, and then to obtain the point $G$. Finally, we have Axiom~II,~4 which finds ``exit points'' of a line passing through a triangle, in this case, the point $D$.

THEOREM~3 was not proven in the first edition of the \emph{Grundlagen der Geometrie}; it was an axiom. That it turns out to be redundant might still come as a surprise, when we realise that we are obtaining such a simple linear result from a one-dimensional order axiom \eqref{eq:g22} and a two-dimensional order axiom~\eqref{eq:g24}. This is the situation in all three proofs we consider in this chapter. We will be proving one-dimensional results by obtaining two-dimensional figures and applying Pasch's axiom.

\subsection{Verification}
Our HOL~Light verification shown in Figure~\ref{fig:ThreeVerification} improves hugely on our manual verification~\cite{ScottMScThesis}, which ran to twenty-two steps. Here, we have just five steps, and are \emph{very} close to Hilbert's prose. We have just one extra step: our final \code{qed} eliminates the unwanted disjunct from Pasch's axiom.

\begin{boxedfigure}
  \begin{align*}
    &\code{assume}\ A\neq C\\
    &\code{so consider}\ E \code{ such that }\ \Triangle{a}{A}{C}{E}\\
    &\qquad\code{by}\ \eqref{eq:g12},\eqref{eq:g13b}& 0\\
    &\code{obviously}\ \code{by\_neqs}\ \code{consider}\ F\ \code{such that}\ \between{A}{E}{F}\ \code{from}\ 0\ \code{by}\ \eqref{eq:g22} & 1\\
    &\code{obviously}\ \code{by\_neqs}\ \code{so consider}\ G\ \code{such that}\ \between{F}{C}{G}\\&\qquad \code{from}\ 0\ \code{by}\ \eqref{eq:g22} & 2\\
    &\code{obviously by\_incidence so consider}\ D\ \code{such that}\\
    &\qquad(\exists a.\; \code{on\_line}\ E\ a \wedge \code{on\_line}\ G\ a \wedge \code{on\_line}\ D\ a)\\
    &\qquad \wedge (\between{A}{D}{C}\vee\between{F}{D}{C})\\
    &\qquad \code{using K (MATCH\_MP\_TAC \eqref{eq:PaschPointSetUnfold}) from}\ 0,1\\
    &\code{obviously}\ (\code{by\_eqs}\circ\code{split})\ \code{qed from}\ 0,1,2\ \code{by}\ \eqref{eq:g21},\eqref{eq:g23}
  \end{align*}
\caption{Verification of THEOREM~3}
\label{fig:ThreeVerification}
\end{boxedfigure}

Note that the verification presented here is in its final form, packaged from an interactive verification that was developed and assisted by concurrent discoverers. We will recount the general way we used the concurrent discoverers to produce these final packaged versions in \S\ref{sec:DiscoveryAtWork}, where we consider the verification of THEOREM~5.

In the verification, we cite two axioms that Hilbert did not. His Axiom~I, 3 can only be used \emph{indirectly} to find a point off an arbitrary line. Strictly speaking, one must also appeal to Axiom~\ref{eq:g12} as we have done. We have also cited Axiom~II, 1. This is only needed for the trivial matter of switching the order of the outer arguments to the \code{between} relation. 

We reference three separate discoverers in our proof: \code{by\_incidence}, \code{by\_neqs}, and \code{by\_eqs}. The first discoverer collects all five kinds of incidence sequent considered in the last chapter into a single discoverer, while the latter two just discover inequality and equality sequents respectively. The semantics of laziness makes this a genuine optimisation: if we only pull sequents from the \code{by\_neqs} discoverer, no sequents will be pulled from the \code{by\_eqs} or \code{by\_planes} discoverers, as we can see by looking again at the dependencies in our data-flow diagram (Figure~\ref{fig:DataFlow}).

Finally, we have used \code{MATCH\_MP\_TAC}\footnote{This standard HOL~Light function matches the conclusion of an implicational theorem with the goal formula, and then generates a new subgoal to prove the antecedent.} to apply our version of Pasch's Axiom~\eqref{eq:PaschPointSetUnfold} formulated entirely in terms of points (see Appendix~\ref{app:Group2}). This is slightly ugly, but it helps \code{MESON} by directing it to the antecedents of the matched theorem. It has no other side effects that carry over to the remaining steps, and its use does not break the declarative style since we state the implicational theorem that we have matched against. Even so, such matching is an irritation, since the free variables in the matched theorem must be lined up with those in the goal, and it is easy to get the order mixed up. We shall deal with this matter in \S\ref{sec:PaschDiscoverer}.

\subsection{The Outer and Inner Pasch Axioms}
THEOREM~3 was an axiom of the first edition of the \emph{Grundlagen der Geometrie}, and the full investigation of such redundant axioms can be credited to Veblen and his supervisor E.H. Moore, who investigated ordered geometry based on axioms very similar to Hilbert's. However, if we look at Veblen's system~\cite{Veblenphd}, we see he chose a different rendering of Pasch's Axiom.

\begin{quotation}
``AXIOM VIII \emph{(Triangle traversal axiom). If three distinct points $A$, $B$, $C$ do not lie on the same line, and $D$ and $E$ are two points in the order $BCD$ and $CEA$, then a point $F$ exists in the order $AFB$, and such that $D$, $E$, $F$ lie on the same line.}''
\flushright{(page 355)}
\end{quotation}

This form of the axiom, known as the \emph{Outer Pasch Axiom}, is due to Peano. It is strong enough to replace Hilbert's own version, and has advantages in terms of incidence reasoning: it carries fewer incidence preconditions at the expense of one extra order condition, and the conclusion is not disjunctive, so we are saved the effort of eliminating an offending case.

A related axiom (which turns out to be weaker \cite{PaschForms}), is also due to Peano. This is the \emph{Inner Pasch Axiom}, which exchanges the roles of $E$ and $F$ in the Outer Pasch Axiom. It can be verified as:
\begin{multline}\label{eq:InnerPaschWeak}
  \vdash\Triangle{a}{A}{B}{C} \\
\wedge \between{B}{C}{D} \wedge \between{A}{E}{B}\\ 
\implies \exists F\;\exists a.\;  \code{on\_line}\ D\ a \wedge \code{on\_line}\ E\ a \wedge \code{on\_line}\ F \wedge \between{A}{F}{C}.
\end{multline}

It can now be seen that the diagram obtained in Hilbert's proof of THEOREM~3 is just obtaining the assumptions of this Inner Pasch Axiom. So when we verify Inner Pasch and use it as an alternative to Axiom~\ref{eq:g24}, we get the verification shown in Figure~\ref{fig:Theorem3Verification}, which does not need the final step we had before. In fact, had we spotted the factoring in our manual verification, we predict that we would have only needed eight, not twenty-two steps.

\begin{boxedfigure}
\begin{align*}
  &\code{assume}\ A\neq C\\
  &\code{so consider}\ E \code{ such that }\ \Triangle{a}{A}{C}{E}\\
  &\qquad\code{by}\ \eqref{eq:g12},\eqref{eq:g13b}& 0\\
  &\code{obviously}\ \code{by\_neqs}\ \code{consider}\ F\ \code{such that}\ \between{A}{E}{F}\ \code{from}\ 0\ \code{by}\ \eqref{eq:g22} & 1\\
  &\code{obviously}\ \code{by\_neqs}\ \code{so consider}\ G\\
  &\qquad\code{such that}\ \between{F}{C}{G}\ \code{from}\ 0\ \code{by}\ \eqref{eq:g22} & 2\\
  &\code{obviously by\_ncols}\ \code{qed from}\ 0,1\ \code{by}\ \eqref{eq:g21},\eqref{eq:InnerPaschWeak}
\end{align*}
\caption{Verification of THEOREM~3 using the derived Inner Pasch Axiom}
\label{fig:Theorem3Verification}
\end{boxedfigure}

Veblen's diagram replaces the Inner Pasch Axiom with the Outer Pasch Axiom, but is otherwise very similar. If we take a relabelling $B \mapsto C$, $C \mapsto F$, $D \mapsto G$ and $F \mapsto D$, then we see that Veblen's second use of Axiom~\ref{eq:g22} (the line extension axiom) differs from Hilbert's by finding the point $G$ on the other side of the segment $CF$. Veblen is setting himself up to use the Outer Pasch Axiom while Hilbert uses the Inner Pasch Axiom.

\begin{quotation}\label{sec:VeblenThree}
[...]. Between every two distinct points there is a third point.

Proof. Let $A$ and $B$ be the given points [figure]. [...] there is a point $E$ not lying on the line $AB$. By [the line extension axiom] points $C$ and $D$ exist, satisfying the order-relations $AEC$ and $BCD$. Hence, by [the Outer Pasch Axiom], $F$ exists in the order $AFB$.
\vspace{0.5cm}

\centering\includegraphics[scale=0.7]{group2eval/OuterPasch}
\flushright{Veblen~\cite{Veblenphd}, (page 355)}
\end{quotation}
In conclusion, for those of us who judge Hilbert's argument for THEOREM~3 to be gappy because of missing incidence reasoning, we offer a clean way to bring it up to more pedantic standards. Before THEOREM~3, we suggest one first derive the Inner Pasch Axiom~\eqref{eq:InnerPaschWeak}, after which the proof follows more fluidly. This observation will be worth bearing in mind when we come to THEOREM~5 in \S\ref{sec:Theorem5}, where we shall derive stronger versions of both the Inner and Outer Pasch axioms.

\section{THEOREM~4}
In this section, we review how we used our discovery tool in an exploratory fashion, as we examine Hilbert's THEOREM~4. This result was another axiom in the first edition of the \emph{Grundlagen der Geometrie}, or more accurately, was incorporated into Axiom~\ref{eq:g23}:
\begin{quotation}
  ``II, 3. \emph{Of any three points situated on a straight line, there is always one and only one which lies between the other two.''}
\flushright{\emph{Foundations of Geometry}~\cite{FoundationsOfGeometry} (page 5)}
\end{quotation}

The ``only one'' part is all that is retained in the tenth edition. The existence part is given in a proof which Hilbert credits to Wald.
\begin{quotation}
  THEOREM~4. Of any three points $A$, $B$, $C$ on a line there always is one that lies between the other two.

  PROOF. Let $A$ not lie between $B$ and $C$ and let also $C$ not lie between $A$ and $B$. Join a point $D$ that does not lie on the line $AC$ with $B$ and choose by Axiom~II, 2 a point $G$ on the connecting line such that $D$ lies between $B$ and $G$. By an application of Axiom~II,~4 to the triangle $BCG$ and to the line $AD$ it follows that the lines $AD$ and $CG$ intersect at a point $E$ that lies between $C$ and $G$ . In the same way, it follows that the lines $CD$ and $AG$ meet at a point $F$ that lies between $A$ and $G$.

If Axiom~II,~4 is applied now to the triangle $AEG$ and to the line $CF$ it becomes evident that $D$ lies between $A$ and $E$, and by an application of the same axiom to the triangle $AEC$ and to the line $BG$ one realises that $B$ lies between $A$ and $C$.

\centering \includegraphics{group2eval/Four}
\flushright{\emph{Foundations of Geometry}~\cite{FoundationsOfGeometry} (page 7)}
\end{quotation}

\subsection{Discovering Applications of Pasch}\label{sec:PaschDiscoverer}
With Axiom~\ref{eq:g24} used a total of four times, and with the symmetry that appears in the diagram, we wanted to explore the proof of THEOREM~4 using our automated discoverers. Currently, our discoverers only tell us about the various incidence relations implied by our assumptions. This helps us discharge preconditions on Axiom~\ref{eq:g24}, but does not tell us directly which instantiations of this axiom can be applied.

We cannot define a new discoverer which finds possible applications of Axiom~\ref{eq:g24} by using our simple forward-chaining primitives $\code{chain1}$, $\code{chain2}$ and $\code{chain3}$. The problem is that our version of this axiom in point sets~\eqref{eq:PaschPointSet} has five free variables but its antecedents involving triangles and betweenness only fix three variables at a time. The remaining two variables can only be fixed by matches up to associativity and commutativity, which \code{MATCH\_MP} does not support.

Instead, we defined a new discoverer \code{by\_pasch} with a combination of ML and monad library functions. We filter for betweenness sequents from an existing discoverer and use these to discharge one of the preconditions of Axiom~\ref{eq:g24}. We then reuse the discoverer \code{by\_ncols} to discharge the triangle preconditions, manually handling the ordering of the free variables each time a precondition is eliminated. 

The \code{by\_pasch} discoverer therefore outputs sequents which are the conclusions of Axiom~\ref{eq:g24}, telling us where the axiom can be applied. At the point in Hilbert's proof where the axiom is first used, the discoverer finds only one other possibility. Hilbert uses one after the other.
\begin{gather*}
\begin{split}
&\exists F.\; (\exists a.\; \code{on\ line}\ C\ a \wedge \code{on\ line}\ D\ a \wedge \code{on\ line}\ F\ a)\\
&\qquad \wedge (\between{A}{F}{B} \vee \between{A}{F}{G}).
\end{split}\\
\begin{split}
&\exists F.\; (\exists a.\; \code{on\ line}\ A\ a \wedge \code{on\ line}\ D\ a \wedge \code{on\ line}\ F\ a)\\
&\qquad \wedge (\between{B}{F}{C} \vee \between{C}{F}{G}).
\end{split}
\end{gather*}

After Hilbert applies the first of these, another three possibilities arise, depicted in Figure~\ref{fig:FourPossibilities}.
\begin{gather*}
\begin{split}\text{(a)}\qquad
&\exists F.\; (\exists a.\; \code{on\ line}\ C\ a \wedge \code{on\ line}\ D\ a \wedge \code{on\ line}\ F\ a)\\
&\qquad \wedge (\between{B}{F}{E} \vee \between{E}{F}{G}).
\end{split}\\
\begin{split}\text{(b)}\qquad
&\exists F.\;(\exists a.\; \code{on\ line}\ B\ a \wedge \code{on\ line}\ E\ a \wedge \code{on\ line}\ F\ a)\\
&\qquad \wedge (\between{A}{F}{C} \vee \between{A}{F}{G}).
\end{split}\\
\begin{split}\text{(c)}\qquad
&\exists F.\;(\exists a.\; \code{on\ line}\ B\ a \wedge \code{on\ line}\ E\ a \wedge \code{on\ line}\ F\ a)\\
&\qquad \wedge (\between{C}{F}{D} \vee \between{D}{F}{G}).
\end{split}\\
\end{gather*}

\begin{figure}
  \includegraphics[scale=0.9]{group2eval/FourPossibilities}
  \caption{Three possible applications of Axiom~II,~4}
  \label{fig:FourPossibilities}
\end{figure}

Cases (a) and (c) of Figure~\ref{fig:FourPossibilities} obtain the exact same point $F$, but in case (a) we should conclude that $F$ is between $B$ and $E$ while in (c) we should conclude that $F$ is between $C$ and $D$. If we apply both cases of the axiom, we will know that $F$ is between $B$ and $E$ and simultaneously between $C$ and $D$. Either of these cases yields an alternative proof of the theorem which we describe in \S\ref{sec:FourAlternative}.

It would be surprising if case (b) got us \emph{anywhere}. The only valid disjunct in its conclusion should put the point $F$ between $A$ and $C$, from which we could immediately conclude that $B = F$ and thus complete the proof. But this is all too easy. The truth of the matter is that incidence reasoning alone, according to our discoverers, cannot reject the impossible disjunct in the axiom's conclusion.

\subsection{Verifying Hilbert's Proof}
When we verified THEOREM~4 manually, it ran to sixty-nine steps. In this comparatively long verification, the structure of the basic prose argument is buried by incidence arguments which do not illuminate anything. The arguments consist mostly of applications of our point set rules given in \S\ref{fig:PointSets} with manual variable instantiations. We tried to use comments to show how the prose translated into the verification steps, but in the end, any claims of a faithful verification were weak.

But with our incidence automation, we have a HOL~Light verification in just thirteen lines, each readily understandable, and matching the prose very closely. The verification is shown in Figure~\ref{fig:FourVerification}, and we are again reasonably close to Hilbert's prose. The only warts are due to the \code{by\_pasch} discoverer not handling the case-split in the conclusion of Axiom~\ref{eq:g24}. This requires the elimination of a disjunct and the identification of the obtained point with an existing point. Both tasks are now handled in a subproof using a second discoverer, \code{by\_eqs}. 

\begin{boxedfigure}
\begin{align*}
& \code{assume}\ \exists a.\; \code{on\_line}\ A\ a\wedge\code{on\_line}\ B\ a\wedge\code{on\_line}\ C\ a & 0\\
& \code{assume}\ A\neq B \wedge A\neq C \wedge B \neq C & 1,2,3\\
& \code{assume}\ \neg\between{A}{C}{B} \wedge \neg\between{B}{A}{C} & 4\\
& \code{consider}\ D\ \code{such that}\ \Triangle{a}{A}{B}{D}\\
& \qquad\code{from}\ 1\ \code{by}\ \eqref{eq:g12},\eqref{eq:g13b}& 5\\
& \code{obviously by\_neqs}\ \code{so consider}\ G\ \code{such that}\ \between{B}{D}{G}\ \code{by}\ \eqref{eq:g22} & 6\\
& \code{consider}\ E\ \code{such that}\ (\exists a.\; \code{on\_line}\ A\ a\wedge \code{on\_line}\ D\ a\wedge \code{on\_line}\ E\ a) & 7\\
& \qquad\qquad\qquad\qquad\qquad\qquad \wedge\between{C}{E}{G}\ & 8\\
&\qquad \code{proof:}\ \code{clearly by\_pasch consider}\ E\ \code{such that}\\
&\qquad\qquad (\exists a.\; \code{on\_line}\ A\ a\wedge \code{on\_line}\ D\ a\wedge \code{on\_line}\ E\ a)\\
&\qquad\qquad \wedge (\between{B}{E}{C} \vee \between{C}{E}{G})\ \code{by}\ \eqref{eq:g21}\ \code{from}\ 0,2,3,5,6\\
&\qquad\code{obviously}\ (\code{by\_eqs}\circ\code{split})\ \code{qed from}\ 0,3,4,5\ \code{by}\ \eqref{eq:g21},\eqref{eq:g23}\\
& \code{consider}\ F\ \code{such that}\ (\exists a.\; \code{on\_line}\ C\ a\wedge \code{on\_line}\ D\ a\wedge \code{on\_line}\ F\ a) & 9\\
& \qquad\qquad\qquad\qquad\qquad\qquad \wedge\between{A}{F}{G}\ & 8\\
&\qquad[\ldots]\\
&\code{have}\ \between{A}{D}{E}\\
&\qquad \code{proof:}\ \code{obviously by\_ncols so consider}\ D'\ \code{such that}\\
&\qquad\qquad\between{C}{D'}{F}\wedge\between{E}{D'}{A}\\
&\qquad\qquad\code{using K (MATCH\_MP\_TAC \eqref{eq:InnerPasch})}\ \code{from}\ 0,2,5,6,8,10\ \code{by}\ \eqref{eq:g21}\\
&\qquad\code{obviously}\ (\code{by\_eqs}\circ\code{split})\ \code{qed from}\ 0,2,5,7,9\ \code{by}\ \eqref{eq:g21},\eqref{eq:g23}\\
&\code{obviously by\_ncols so consider}\ B'\ \code{such that}\\
&\qquad\between{G}{D}{B'}\wedge\between{C}{B'}{A}\\
&\qquad\code{using K (MATCH\_MP\_TAC \eqref{eq:OuterPasch})}\ \code{from}\ 0,2,5,7,\ \code{by}\ \eqref{eq:g21}\\
&\code{obviously}\ (\code{by\_eqs}\circ\code{split})\ \code{qed from}\ 0,2,5,6\ \code{by}\ \eqref{eq:g21},\eqref{eq:g23}\\
\end{align*}
\caption{Verification of THEOREM~4}
\label{fig:FourVerification}
\end{boxedfigure}

We now review the verification. At the very start, we set our goal formula to be
\begin{multline*}
(\exists a.\; \code{on\_line}\ A\ a\wedge\code{on\_line}\ B\ a\wedge\code{on\_line}\ C\ a) \\\implies\between{A}{B}{C}\vee\between{B}{A}{C}\vee\between{A}{C}{B}.
\end{multline*}

We then get to \code{assume}, just as Hilbert does, that $C$ is neither between $A$ or $B$ nor $A$ between $B$ or $C$. This is a very natural way to express one's assumptions mathematically. It goes through because of the way Mizar~Light implements the \code{assume} primitive: it first tries to prove the goal under the negation of our assumption. If this is successful, the assumption is used to rewrite the goal before being placed into the goal hypotheses. The upshot is that our goal formula gets rewritten to
\begin{displaymath}
(\exists a.\; \code{on\_line}\ A\ a\wedge\code{on\_line}\ B\ a\wedge\code{on\_line}\ C\ a) \implies\between{A}{B}{C}.
\end{displaymath}

Next, we look at the first two applications of Axiom~II,~4. In our verification, we have not been able to apply this axiom in either the inner or outer forms, since we have only one order hypothesis. This means we are faced with having to eliminate a disjunct in the conclusion of Axiom~\ref{eq:g24}. To do this, we find point equalities via the composed discoverer $\code{by\_eqs}\circ\code{split}$, which tells us that, in the offending cases, the point $A$ lies between $B$ and $C$ or $C$ lies between $A$ and $B$. We have explicitly hypothesised against these two possibilities, and thus, the disjuncts can be eliminated.

In these two applications of Axiom~\ref{eq:g24}, we have used our \code{clearly} keyword to ask the \code{by\_pasch} discoverer to search for a specific conclusion of the axiom. Exploiting our discoverer in this way makes things more robust than using \code{MATCH\_MP\_TAC} as we did in \S\ref{sec:Theorem3}. With matching, we have to be careful that all the variables in the axiom line up with our desired conclusion. But when we use the discoverer, we know that its outputs are always lexicographically normalised, so there is only one conclusion we could possibly be after.

The sought conclusion could usually be copied and pasted from the many results obtained by the \code{by\_pasch} discoverer when it was run concurrently during the interactive development, selecting the one which we want as an intermediate result and add to the proof context to be referenced later. Moreover, by using \code{clearly} and specifying the narrow set of justifying theorems needed to derive the conclusion, it the sole target for efficient search in proof replay.

The third and fourth applications of Pasch's axiom take the inner~\eqref{eq:InnerPasch} and outer~\eqref{eq:OuterPasch} forms respectively, and so we recommend the Outer Pasch Axiom as a useful lemma at this stage of Hilbert's exposition.

\subsection{Alternative Proof}\label{sec:FourAlternative}
Our proof tool was mostly used in a supporting role, but for Theorem~4 we allowed it to tackle the problem unaided, probing into the search space by applying Axiom~II,~4 non-deterministically. In this way, we hoped it would find alternative proofs of THEOREM~4, and, ideally, find one which required less than four applications of Axiom~II,~4. This defined a search limit for the problem: once four applications were found, it stopped searching in the relevant branch. 

We found several ``alternatives'', but most of these were symmetries of the original proof. In some cases, two independent applications of Pasch's axiom were applied in reverse order. In other cases, the proof was identical to the original up to a symmetric relabelling of the points. Only one new proof was revealed up to symmetry, and it corresponds to case (a) and (c) of Figure~\ref{fig:FourPossibilities}. We give it now in a prose formulation with an accompanying diagram.

\begin{proposition}
Between points $A$ and $C$ is a third point $B$.
\end{proposition}
\begin{proof}Assume $A$, $B$ and $C$ are collinear, with $A$ not between $B$ or $C$ and $C$ not between $A$ or $B$. We find a point $D$ off the line $AC$ and extend the segment $BD$ to $G$ using Axiom~II, 2. We then use Axiom~II,~4 on the triangle $BCG$ and the line $AD$ to find the point $E$ between $C$ and $G$. We use Axiom~II,~4 on the triangle $BEG$ and the line $CD$ to find the point $F$ between $B$ and $E$. We use the axiom again on the triangle $ABE$ and the line $CF$ to show that $D$ lies between $A$ and $E$. Finally, we can use the axiom on the triangle $ACE$ and the line $BG$ to find $B$ between $A$ and $C$.
\end{proof}
\begin{center}\includegraphics{group2eval/FourAlt}
\end{center}

The opening strategy of our alternative proof is the same as the original. We first construct a point $D$ off the line $AC$ and extend the segment $BD$ to $G$. This tells us that $D$ lies between $B$ and $G$, which gives us our first opportunities to use Axiom~\ref{eq:g24}. In both cases, our goal is to use this axiom in order to place the point $D$ between $A$ and $E$, so that a final application of Pasch's axiom to the triangle $ACE$ and the line $BG$ will place $B$ between $A$ and $C$. The two proofs only differ in how they construct the point $F$, and how they use $F$ to place $D$ between $A$ and $E$.

In Hilbert's proof, $F$ is found on the edge of the outer triangle $ACG$, and is placed symmetrically with $E$. Indeed, the proof is valid even after exchanging all references of $E$ and $F$, whereas in our proof, $F$ is placed in the interior of $ACG$ while $E$ lies asymmetrically on the triangle's edge. 

So Hilbert's proof has a lot of symmetry: $E$ could be replaced with $F$; and the third application of Pasch's axiom could be made on the triangle $CFG$ and the line $AE$, instead of $AEG$ and the line $CF$. Our proof makes it clear that, while $E$ and $F$ can be constructed symmetrically and independently, only one of these points is distinguished in the final few steps. 

It is worth drawing some attention to the subtlety of the incidence reasoning here. We could have applied Axiom~II,~4 differently to find the point $F$, using the triangle $CDG$ and the line $BE$. This would tell us that $F$ lies on the line $BE$ between $C$ and $D$ (before, it told us that $F$ lies on the line $CD$ between $B$ and $E$). Now it might seem that we can use a symmetrical application of Pasch's axiom on the same line $BE$ and the triangle $ACD$, which would solve the goal putting $B$ between $A$ and $C$. But at this stage in the proof, we must consider the possibility that $BF$ exits the triangle $ACD$ between $A$ and $D$. This possibility is not yet eliminable by incidence reasoning alone. It really does appear we need at least \emph{four} applications of Axiom~II,~4 to get this theorem.

Observations such as these are not apparent in Hilbert's proof. In his eleven uses of Axiom~II,~4 across THEOREM~3, 4 and 5, Hilbert only considers the case-split implied by the axiom twice. And yet it takes up a significant amount of combinatorial reasoning about incidence. It is difficult to justify leaving this complexity implicit, when it has consequences on the shape of the proof which we find difficult to argue as obvious. 

\section{THEOREM~5}\label{sec:Theorem5}
For the final theorem of this chapter, we shall see what is gained with the inner and outer form of Pasch's axiom, and we will look under the bonnet to see what our discoverers are actually up to.

THEOREM~5 has the most complex of the three proofs, taking up almost an entire page of the English translation. Like THEOREM~3 and THEOREM~4, it was originally an axiom in the first edition of Hilbert's text. The proof here is credited to E.H. Moore who proved it for projective geometry. Effectively, the result gives a transitivity property for point ordering. The proof is divided into three parts, though as observed by Dehlinger et al~\cite{DehlingerFOG}, it makes sense to leave the third part to the generalisation of THEOREM~6, which we cover in the next chapter.

\subsection{Part 1 of THEOREM~5}
\begin{quotation}
THEOREM 5. Given any four points on a line, it is always possible to label them $A$, $B$, $C$, $D$ in such a way that the point labelled $B$ lies between $A$ and $C$ and also between $A$ and $D$, and furthermore, that the point labelled $C$ lies between $A$ and $D$ and also between $B$ and $D$.

PROOF. Let $A$, $B$, $C$, $D$ be four points on a line $g$. The following will now be shown:

1. If $B$ lies on the segment $AC$ and $C$ lies on the segment $BD$ then the points $B$ and $C$ also lie on the segment $AD$. By Axioms~I, 3 and II, 2 choose a point $E$ that does not lie on $g$, on [sic] a point $F$ such that $E$ lies between $C$ and $F$. By repeated applications of Axioms~II, 3 and II,~4 it follows that the segments $AE$ and $BF$ meet at a point $G$, and moreover, that the line $CF$ meets the segment $GD$ at a point $H$. Since $H$ thus lies on the segment $GD$ and since, however, by Axiom~II, 3, $E$ does not lie on the segment $AG$, the line $EH$ by Axiom~II,~4 meets the segment $AD$, i.e. $C$ lies on the segment $AD$. In exactly the same way one shows analogously that $B$ also lies on this segment.

\centering\includegraphics{group2eval/Five}
\flushright{\emph{Foundations of Geometry}~\cite{FoundationsOfGeometry} (page 7)}
\end{quotation}

\subsubsection{Evaluating our Manual Verification}\label{sec:FindingAEH}
Our manual verification of this proof runs to approximately 135 lines of complicated proof steps. As we should expect by now, most of these steps were used to derive the preconditions needed for Axiom~\ref{eq:g24}. These are now handled by our incidence discoverers.

In the manual verifications, the complexity of the inferences had got the better of us. We were not able to verify Hilbert's final application of Axiom~\ref{eq:g24} with the line $EH$ and the triangle $ADG$. This requires knowing that the line $EH$ does not intersect any vertex of $ADG$, which requires, in particular, knowing that $AEH$ is a triangle.

We began to speculate that this matter was unprovable. In fact, we had produced a sketch argument that the derivation of $E\neq H$ was impossible, and we hoped that our discoverers would confirm this.

\label{sec:CombinatoryError}Instead, our discoverers refuted it. By tracking the path of inferences via a writer (see \S\ref{sec:WriterMonad}), we found that $\triangle AEH$ is derived at the end of a chain of discovered triangles starting from $\triangle ABE$. The rule linking each is $\code{colncolncol}$ from \S\ref{list:Procedures}, which can be understood as substituting points of a non-collinear triple one-at-a-time, until we have rewritten the initial triangle $\triangle ABE$ to $\triangle AEH$. We briefly discuss how.

It might seem that we can just replace the point $B$ with the point $H$ to rewrite $\triangle ABE$ to $\triangle AEH$, but the triangle introduction rule requires a hypothesis about an appropriate collinear set and an appropriate point inequality. Instead, the inference we use is less direct, and is shown in Figure~\ref{fig:FindingAEH}. At first, our discoverer substitutes $G$ for $E$ using the line $AGE$, producing $\triangle ABG$ from $\triangle ABE$. It then substitutes $D$ for $B$ using the line $ABD$. This gives us a triangle $ADG$. Next, it substitutes $H$ for $D$ using the line $DGH$, giving us $\triangle AGH$. Finally, $E$ and $H$ are shown distinct on the basis of $\triangle AGH$ and the line $AGE$, after which the discoverer substitutes $H$ for $G$ using the line $AGE$, thus obtaining $\triangle AEH$.

\begin{figure}
\centering\includegraphics{group2eval/FindingAEH}
\caption{Finding $\triangle AEH$}
\label{fig:FindingAEH}
\end{figure}

\subsubsection{Strengthening Inner and Outer Pasch}\label{sec:StrengthenedPasch}
We have a verification of THEOREM~5 using Axiom~\ref{eq:g24} directly via \code{by\_pasch}. Incidence reasoning implicitly dominates this verification, but our discoverers take on the labour. We can avoid much of this implicit reasoning though, and so in another verification, we exploit the Inner and Outer Pasch axioms, whose preconditions have only one incidence assumption: namely that the triangle on which the axiom applies exists. By using this form of the axiom, a much cleaner proof can be obtained, one which needs much less automation. Furthermore, we do not need to write any subproofs to identify points in the figure, nor eliminate disjuncts.

Our versions of Inner and Outer Pasch are strengthened from their axiomatic form. Consider Veblen's Outer Pasch Axiom, which we can formalise as
\begin{multline*}
  \Triangle{a}{A}{B}{C} \\\wedge \between{B}{C}{D} \wedge \between{A}{E}{C}\\ \implies \exists F\;\exists a.\;  \code{on\_line}\ D\ a \wedge \code{on\_line}\ E\ a \wedge \code{on\_line}\ F\ a \wedge \between{A}{F}{B}.
\end{multline*}
We can say something stronger in the conclusion here. We know that $D$, $E$ and $F$ are not merely collinear. The point $E$ must lie between $D$ and $F$ (see the diagram accompanying Veblen's proof in \S\ref{sec:VeblenThree}). This is an important corollary, as evidenced by the fact that it is the very first theorem Veblen proves after stating the axiom. Thus, we have both Inner and Outer Pasch axioms as the following strengthened theorems:
\begin{equation}\label{eq:OuterPasch}
  \begin{split}
    &\vdash\Triangle{a}{A}{B}{C} \\
    &\qquad\wedge \between{B}{C}{D}\wedge \between{A}{E}{C}\\
    &\qquad\implies \exists F.\; \between{D}{E}{F} \wedge \between{A}{F}{B}.
  \end{split}
\end{equation}
\begin{equation}\label{eq:InnerPasch}
  \begin{split}
    &\vdash\Triangle{a}{A}{B}{C} \\
    &\qquad\wedge \between{B}{C}{D} \wedge \between{A}{E}{B}\\ 
    &\qquad\implies \exists F.\; \between{D}{F}{E} \wedge \between{A}{F}{C}.
  \end{split}
\end{equation}

\subsubsection{Verification}
With these theorems now derived and with some of our automation, we obtain the verification in Figure~\ref{fig:FiveVerification1}, which is very close to the prose. We have two steps which Hilbert does not make explicit. Firstly, we note that $A\neq D$, a fact which follows from our assumptions and Axiom~II, 3, and without which we would not be able to show the existence of $\triangle ADG$ for our final application of Pasch's axiom.

The other extra step is somewhat of an irritation. Instead of applying Pasch's axiom to conclude that $C$ is between $A$ and $D$, we must instead obtain a new point $C'$ and then use incidence reasoning via \code{by\_eqs} to identify it with $C$. 

\begin{boxedfigure}
\small
  \begin{align*}
    &\code{assume}\ \between{A}{B}{C} \wedge \between{B}{C}{D} & 0,1\\
    &\code{consider}\ E\ \code{such that}\ \Triangle{a}{A}{B}{E}\\
    &\qquad\code{from}\ 0\ \code{by}\ \eqref{eq:g12},\eqref{eq:g13b},\eqref{eq:g21} & 2\\
    &\code{obviously by\_neqs so consider}\ F\ \code{such that}\\
    &\qquad\between{C}{E}{F}\ \code{from}\ 0\ \code{by}\ \eqref{eq:g22} & 3\\
    &\code{obviously by\_ncols so consider}\ G\ \code{such that}\\ 
    &\qquad\between{A}{G}{E} \wedge \between{B}{G}{F}\ \code{from}\ 0,2,3\ \code{by}\
    \eqref{eq:g21},\eqref{eq:InnerPasch}& 4,5\\    
    &\code{obviously by\_ncols so consider}\ H\ \code{such that}\\ 
    &\qquad\between{C}{H}{F} \wedge \between{D}{H}{G}\ \code{from}\ 0,1,2,3,5\ \code{by}\ \eqref{eq:g21},\eqref{eq:InnerPasch}& 6\\
    &\code{have}\ A\neq D\ \code{from}\ 0,1\ \code{by}\ \eqref{eq:g21},\eqref{eq:g23}\\
    &\code{obviously by\_ncols}\ \code{so consider}\ C'\ \code{such that}\\
    &\qquad\between{E}{H}{C'}\wedge\between{A}{C'}{D}\ \code{from}\ 0,1,2,4,6,7\ \code{by}\ \eqref{eq:OuterPasch},\eqref{eq:g21}\\
    &\code{obviously}\ (\code{by\_eqs}\circ\code{split})\ \code{qed from}\ 0,1,2,3,6,7
  \end{align*}
  \caption{THEOREM~5 verification, part 1}
  \label{fig:FiveVerification1}
\end{boxedfigure}

\subsubsection{Comparison with the Prose}
The strengthened versions of the Pasch axioms mean we can be more efficient than Hilbert, who uses an unspecified number of applications of Axiom~II,~4.

\begin{quote}
``By repeated applications of Axioms II, 3 and II,~4 it follows that the segments $AE$ and $BF$ meet at a point $G$, and moreover, that the line $CF$ meets the segment $GD$ at a point $H$.''
\flushright{\emph{Foundations of Geometry}~\cite{FoundationsOfGeometry}, page 7.}
\end{quote}

We can say exactly how many applications are needed here. In one of our verifications of THEOREM~5, which avoids the Inner and Outer Pasch axioms and so follows Hilbert most closely, we apply Axiom~II,~4 via the \code{by\_pasch} discoverer. By doing so, we see there are exactly three applications implied by Hilbert's prose (we elide the subproofs used to reject offending disjuncts).

\fbox{\begin{minipage}{\linewidth}
\small
\setlength\abovedisplayskip{-0.2cm}
\begin{align*}
  &\code{consider}\ G\ \code{such that}\ (\exists a.\; \code{on\_line}\ B\ a\wedge\code{on\_line}\ F\ a\wedge\code{on\_line}\ G\ a)\\
  &\qquad\qquad\qquad\qquad\qquad\wedge \between{A}{G}{E}\ & 4,5 \\
  &\qquad\code{proof:}\ \code{clearly by\_pasch}\ ...\\
  &\code{have}\ \between{B}{G}{F}\ & 6\\
  &\qquad\code{proof:}\ \code{clearly by\_pasch}\ ...\\
  &\code{consider}\ H\ \code{such that}\ (\exists a.\; \code{on\_line}\ C\ a\wedge\code{on\_line}\ F\ a\wedge\code{on\_line}\ H\ a)\\
  &\qquad\qquad\qquad\qquad\qquad\wedge \between{D}{H}{G}\ & 7,8\\
  &\qquad\code{proof:}\ \code{clearly by\_pasch}\ ...
\end{align*}\end{minipage}}\linebreak

That three applications are necessary is implied by the careful language used in the prose: ``the segments $AE$ and $BF$ meet at a point $G$'' while ``the \emph{line} $CF$ meets the segment $GD$ at a point $H$'' (our emphasis). Now if we are to show that two \emph{segments} intersect, we must derive \emph{two} facts of betweenness, and therefore we need \emph{two} applications of Axiom~\ref{eq:g24}. But if we are to show that a \emph{line} and a segment intersect, we need only one fact of betweenness. 

In our verification using the Inner and Outer Pasch axioms (Figure~\ref{fig:FiveVerification1}), we can trim this down. The intersection of the segments $AE$ and $BF$ is covered by just one application of the strengthened version of the Inner Pasch Axiom~\eqref{eq:InnerPasch}, which we use again to intersect the segments $CF$ and $GD$. Finally, we use the Outer Pasch Axiom~\eqref{eq:OuterPasch} to find a point $C'$ between $A$ and $D$. 

For this final application of Pasch's axiom, Hilbert writes: ``...and since, however, by Axiom II, 3, $E$ does not lie on the segment $AG$,....'' We draw attention to this remark because it is not reflected in our verification. Hilbert, for the one and only time, is explicitly eliminating the disjunct in the conclusion of Axiom~\ref{eq:g24}, by appealing to the fact that $G$ lies between $A$ and $E$. We must appeal to the same fact, but in our verification, it is just the necessary precondition of the inner  Pasch axiom~\eqref{eq:InnerPasch}.

Finally, we mention Hilbert's final step ``one shows analogously that $B$ also lies on this segment.'' This does not require an analogous \emph{proof} as the word ``show'' would imply. Instead, we can capture the analogy directly by using the theorem verified in Figure~\ref{fig:FiveVerification1} as a lemma and then applying the symmetry of betweenness. In fact, \code{MESON} takes care of this automatically:

\fbox{\begin{minipage}{\linewidth}\small
\setlength\abovedisplayskip{-0.2cm}
\begin{multline*}
\code{MESON [lemma,\eqref{eq:g21}] }
\forall A\;\forall B\;\forall C\;\forall D.\; \between{A}{B}{C} \wedge \between{B}{C}{D}\\ \implies \between{A}{B}{D} \wedge \between{A}{C}{D}
\end{multline*}\end{minipage}}\linebreak

\subsection{Discovery at work}\label{sec:DiscoveryAtWork}
In this subsection, we give an idea of how our discoverers interact with HOL~Light by showing the sequents generated concurrently as we interactively develop a declarative verification of THEOREM~5. We consider two discoverers, \code{by\_incidence}, which generates the five kinds of basic incidence sequents described in the last chapter, and \code{by\_pasch}, which finds potential applications of Pasch's axiom. The \code{by\_pasch} discoverer feeds off the generations produced by \code{by\_incidence}, so that their work is not duplicated. But there is no \emph{feedback}. That is, \code{by\_incidence} does not continue searching based on the results of \code{by\_pasch}. Instead, \emph{we}, the user, shall take responsibility for when Pasch's axiom is applied. It is a point introduction axiom explicit in the prose that we want to keep it explicit in the verification.

The discovery begins once we have stated the theorem's assumptions and obtained our first non-collinear point.

\fbox{\begin{minipage}{\linewidth}\small\setlength\abovedisplayskip{-0.2cm}
\begin{align*}
&\code{assume}\ \between{A}{B}{C} \wedge \between{B}{C}{D} & 0,1\\
&\code{consider}\ E\ \code{such that}\ \Triangle{a}{A}{B}{E}\\
&\qquad\code{from}\ 0\ \code{by}\ \eqref{eq:g12},\eqref{eq:g13b},\eqref{eq:g21} & 2
\end{align*}
\end{minipage}}
\linebreak

Concurrently, sequents are pulled from the discoverer \code{by\_incidence}, which lazily forces values according to our data-flow diagram from Figure~\ref{fig:DataFlow}. This ultimately involves pulling hypothesis sequents from the discoverer \code{monitor}, whose job it is to inspect the proof context constructed from the steps of the declarative proof, as it is written, and add any new hypotheses which appear there. Here, we have three hypotheses, which are picked up and fed through our incidence discoverers to produce the seven generations of incidence sequents shown in Figure~\ref{fig:FirstGenerations}. If the stream is forced beyond this, only empty generations appear, indicating that no more inference is possible.

The seven generations of sequents are delivered within 0.31 seconds.\footnote{We have tested this on an Intel Core 2 with a 2.53GHz clock speed.}. We write the dependent hypotheses in parentheses, and omit the turnstile $(\vdash)$ and sequent context. Note that sequents can be repeated if they can be derived in more than one way. When we are not tracking dependent hypotheses, such repetitions are automatically filtered out.

As we can see, two of the seven generations are empty. This happens because of filtering: some of the inferences we use turn out to generate sequents which have already appeared, and duplicates are always removed from the discoverer. This filtering sometimes leaves generations completely empty. 

Besides the selection of triangles here, we have a sequent which says that all points in our figure lie in the same plane. We can see how this sequent has grown over the generations, with larger and larger planar sets found, via rule~$\code{planeplane}$ from \S\ref{list:Procedures}. 

\begin{figure}[H]
\doublebox{\begin{minipage}{\linewidth}\footnotesize\setlength\abovedisplayskip{0cm}
    \begin{displaymath}
  \begin{split}
    &\left\{\begin{aligned}
          &\exists a.\;\code{on\_line}\ B\ a\wedge\code{on\_line}\ C\ a\wedge\code{on\_line}\ D\ a, &(1)\\
          &\exists a.\;\code{on\_line}\ A\ a\wedge\code{on\_line}\ B\ a\wedge\code{on\_line}\ C\ a, &(0)\\
          &\Triangle{a}{A}{B}{E}, &(2)\\
          &B\neq C, B\neq D, C\neq D, & (1)\\
          &A\neq B, A\neq C, B\neq C, & (0)\\
          &A\neq B, A\neq E, B\neq E, & (2)\\
          &\exists \alpha.\;\code{on\_plane}\ B\ \alpha\wedge \code{on\_plane}\ C\ \alpha\wedge \code{on\_plane}\ D\ \alpha, &(1)\\
          &\exists \alpha.\;\code{on\_plane}\ A\ \alpha\wedge \code{on\_plane}\ B\ \alpha\wedge \code{on\_plane}\ C\ \alpha, &(0)\\
          &\exists \alpha.\;\code{on\_plane}\ A\ \alpha\wedge \code{on\_plane}\ B\ \alpha\wedge \code{on\_plane}\ E\ \alpha &(2)\end{aligned}\right\},\\
  &\{\exists \alpha.\;\code{on\_plane}\ A\ \alpha\wedge \code{on\_plane}\ B\ \alpha\wedge \code{on\_plane}\ C\ \alpha\wedge\code{on\_plane}\ D\ \alpha \quad (0,1)\},\\
  &\{\},\\
  &\left\{\begin{aligned}
          &\Triangle{a}{A}{C}{E}, & (0,2)\\
          &\Triangle{a}{B}{C}{E}, & (0,2)\\
          &C\neq E, & (0,2)\\
          &\exists \alpha.\;\code{on\_plane}\ A\ \alpha\wedge \code{on\_plane}\ B\ \alpha\wedge \code{on\_plane}\ C\ \alpha\wedge \code{on\_plane}\ E\ \alpha & (0,2)
        \end{aligned}\right\},\\
      &\{\exists a.\;\code{on\_line}\ A\ a\wedge \code{on\_line}\ B\ a\wedge \code{on\_line}\ C\ a\wedge\code{on\_line}\ D\ a \quad (0,1)\},\\
      &\{\},\\
    &\left\{\begin{aligned}
        &\Triangle{a}{B}{D}{E}, & (0,1,2)\\
        &\Triangle{a}{C}{D}{E}, & (0,1,2)\\
        &D\neq E, & (0,1,2)\\
      &\exists \alpha.\;\code{on\_plane}\ A\ \alpha\wedge \code{on\_plane}\ B\ \alpha\wedge \code{on\_plane}\ C\ \alpha\wedge\code{on\_plane}\ D\ \alpha\wedge\code{on\_plane}\ E\ \alpha & (0,1,2)
    \end{aligned}\right\}\\
\end{split}
\end{displaymath}
\end{minipage}}
\caption{First Generations of Discovered Sequents}
\label{fig:FirstGenerations}
\end{figure}

To follow Hilbert's proof at this stage, all we need to know is that $C\neq E$, a fact which is delivered in the fourth generation. We are told that its derivation depends on hypothesis $0$, namely $\between{A}{B}{C}$, and hypothesis $2$, which is 
\begin{displaymath}
\Triangle{a}{A}{B}{C}.
\end{displaymath}
Since this was the \emph{last} hypothesis to enter the proof context, we can add it as justification to the declarative proof we are building by using the Mizar~Light keyword \code{so}. Appealing to the pertinent hypotheses in this way gives the reader more information about the dependencies within the proof, and enables the discoverer to work more efficiently in replay, where it will use only those hypotheses that have been marked as justification.

\fbox{\begin{minipage}{\linewidth}\small\setlength\abovedisplayskip{-0.2cm}
\begin{align*}
\code{obviously by\_neqs so consider}\ F\ \code{such that}\ \between{C}{E}{F}\ \code{from}\ 0\ \code{by}\ \eqref{eq:g22} \quad 3
\end{align*}
\end{minipage}}
\linebreak

\begin{figure}[H]
\doublebox{\begin{minipage}{\linewidth}\footnotesize\setlength\abovedisplayskip{0cm}
\begin{displaymath}
  \begin{split}
    &\left\{\begin{aligned}
        &\exists a.\;\code{on\_line}\ C\ a\wedge\code{on\_line}\ E\ a\wedge\code{on\_line}\ F\ a, &(3)\\
        &C \neq E, C\neq F, E\neq F, &(3)\\
        &\exists \alpha.\; \code{on\_plane}\ C\ \alpha\wedge\code{on\_plane}\ E\ \alpha\wedge\code{on\_plane}\ F\ \alpha&(3)\end{aligned}\right\},\\
    &\left\{\begin{aligned}
      &\exists \alpha.\;\code{on\_plane}\ B\ \alpha\wedge \code{on\_plane}\ C\ \alpha\wedge \code{on\_plane}\ D\ \alpha\wedge\code{on\_plane}\ E\ \alpha\wedge\code{on\_plane}\ F\ \alpha, & (1,3)\\
      &\exists \alpha.\;\code{on\_plane}\ A\ \alpha\wedge \code{on\_plane}\ B\ \alpha\wedge \code{on\_plane}\ C\ \alpha\wedge\code{on\_plane}\ E\ \alpha\wedge\code{on\_plane}\ F\ \alpha & (0,3)
      \end{aligned}\right\},\\
    &\left\{\begin{aligned}
        &\exists\alpha.\; \code{on\_plane}\ B\ \alpha\wedge\code{on\_plane}\ C\ \alpha\wedge\code{on\_plane}\ D\ \alpha\wedge\code{on\_plane}\ E\ \alpha\wedge\code{on\_plane}\ F\ \alpha,&(1,3)\\
        &\exists\alpha.\; \code{on\_plane}\ A\ \alpha\wedge\code{on\_plane}\ B\ \alpha\wedge\code{on\_plane}\ C\ \alpha\wedge\code{on\_plane}\ E\ \alpha\wedge\code{on\_plane}\ F\ \alpha&(0,3)\end{aligned}\right\},\\
    &\{\},\{\},\\
    &\left\{\begin{aligned}
        &\exists \alpha.\; \code{on\_plane}\ A\ \alpha\wedge\code{on\_plane}\ B\ \alpha\wedge\code{on\_plane}\ C\ \alpha\\
        &\qquad\wedge\code{on\_plane}\ D\ \alpha\wedge\code{on\_plane}\ E\ \alpha\wedge\code{on\_plane}\ F\ \alpha&(0,1,3)\end{aligned}\right\},\\
    &\{\},\\
    &\left\{A\neq F, B\neq F \quad (0,2,3)\right\},\\
    &\{\},\{\},\\
    &\{D\neq F \quad (0,1,2,3)\},\\
    &\{\},\{\},\{\},\\
    &\left\{\begin{aligned} 
        &\Triangle{a}{A}{C}{F}, & (0,2,3)\\
        &\Triangle{a}{A}{E}{F}, & (0,2,3)\\
        &\Triangle{a}{B}{C}{F}, & (0,2,3)\\
        &\Triangle{a}{B}{E}{F} & (0,2,3)\end{aligned}\right\},\\
    &\{\},\{\},\\
    &\left\{\begin{aligned} 
        &\Triangle{a}{D}{E}{F}, & (0,1,2,3)\\
        &\Triangle{a}{B}{D}{F}, & (0,1,2,3)\\
        &\Triangle{a}{C}{D}{F}, & (0,1,2,3)\\
        &\Triangle{a}{A}{B}{F} & (0,2,3)\end{aligned}\right\}
\end{split}
\end{displaymath}
\end{minipage}}
\caption{Second generations of sequents}
\label{fig:SecondGenerations}
\end{figure}

With this next declarative proof step, we add a new sequent into the proof-context, which is picked up by the \code{monitor} discoverer, to flow into the network of incidence discoverers, creating new generations of sequents. These generations are shown in Figure~\ref{fig:SecondGenerations} and are found within 1.21 seconds. Including the sequents found earlier, we have thirteen triangles identified in total, and so we are going to be interested in which applications of Axiom~\ref{eq:g24} are permissible. To find out, we look more specifically at the results of our \code{by\_pasch} discoverer. Its first eighteen generations are empty, meaning that it requires a search-depth of eighteen before all the required preconditions of Pasch have been found. We are then told that, in this early stage of the proof, there are already six possibilities to choose from. The full set is shown in Figure~\ref{fig:PaschGenerations} and is found within 2.82 seconds.

\begin{figure}
\doublebox{\begin{minipage}{\linewidth}\footnotesize\setlength\abovedisplayskip{0cm}
    \begin{displaymath}
      \begin{split}
&\{\},\{\},\{\},\{\},\{\},\{\},\{\},\{\},\{\},\{\},\{\},\{\},\{\},\{\},\{\},\{\},\{\},\{\},\\
        &\left\{
          \begin{aligned}
            &\exists G.\; (\exists a.\;\code{on\_line}\ B\ a\wedge\code{on\_line}\ E\ a\wedge\code{on\_line}\ G\ a)\\
            &\qquad\qquad\wedge (\between{A}{G}{C} \vee \between{A}{G}{F}), & (0,2,3)\\
            &\exists G.\; (\exists a.\;\code{on\_line}\ A\ a\wedge\code{on\_line}\ E\ a\wedge\code{on\_line}\ G\ a)\\
            &\qquad\qquad\wedge (\between{B}{G}{C} \vee \between{B}{G}{F}) & (0,2,3)\\
         \end{aligned}\right\},\\
       &\{\},\{\},\{\},\{\},\{\},\{\},\{\},\{\},\\
       &\left\{
         \begin{aligned}
           &\exists G.\; (\exists a.\;\code{on\_line}\ D\ a\wedge\code{on\_line}\ E\ a\wedge\code{on\_line}\ G\ a)\\
           &\qquad\qquad\wedge (\between{B}{G}{C} \vee \between{B}{G}{F}), & (0,1,2,3)\\
           &\exists G.\; (\exists a.\;\code{on\_line}\ B\ a\wedge\code{on\_line}\ E\ a\wedge\code{on\_line}\ G\ a)\\
           &\qquad\qquad\wedge (\between{C}{G}{D} \vee \between{D}{G}{F}), & (0,1,2,3)\\
           &\exists G.\; (\exists a.\;\code{on\_line}\ B\ a\wedge\code{on\_line}\ E\ a\wedge\code{on\_line}\ G\ a)\\ 
           &\qquad\qquad\wedge (\between{A}{G}{F} \vee \between{C}{G}{F}) & (0,2,3)
         \end{aligned}\right\},\\
       &\{\},\{\},\{\},\{\},\{\},\\
     &\left\{
       \begin{aligned}
       &\exists G.\; (\exists a.\;\code{on\_line}\ B\ a\wedge\code{on\_line}\ F\ a\wedge\code{on\_line}\ G\ a)\\
       &\qquad\qquad\wedge (\between{A}{G}{E} \vee \between{C}{G}{E}) & (0,2,3)
     \end{aligned}\right\}
     \end{split}
   \end{displaymath}
\end{minipage}}
\caption{Discovered applications of Pasch's Axiom}
\label{fig:PaschGenerations}
\end{figure}

The last possibility corresponds to one of Hilbert's applications of Axiom~II,~4, which we shall therefore add as an explicit proof step in the declarative proof we are building. We do this by asserting the theorem as \code{clearly} derivable, adding the three hypotheses as explanatory justification, so that the theorem is found efficiently in replay:

\fbox{\begin{minipage}{\linewidth}\small\setlength\abovedisplayskip{-0.2cm}
\begin{align*}
&\code{clearly by\_pasch so consider}\ G\ \code{such that}\\
&(\exists a.\;\code{on\_line}\ B\ a\wedge\code{on\_line}\ F\ a\wedge\code{on\_line}\ G\ a)\wedge (\between{A}{G}{E} \vee \between{C}{G}{E})\\
&\qquad \code{from}\ 0,2
\end{align*}
\end{minipage}}\linebreak

We now have a disjunction in our hypotheses. When we run the \code{by\_incidence} discoverer with the \code{split} function, it will convert this disjunction into a tree, which will combine as described in \S\ref{sec:Trees}. Our generations become proper trees, and the search space is automatically partitioned into three. 

The first partition covers inferences which are made in the root node of our trees, where no particular disjunct is assumed. These generations are shown in Figure~\ref{fig:RootGenerations}. Here, the discoverer infers that the new point $G$ must, \emph{in any case}, be distinct from $A$, $C$, $D$ and $E$, and that all seven points must be planar. 

\begin{figure}
\doublebox{\begin{minipage}{\linewidth}\footnotesize\setlength\abovedisplayskip{0cm}
    \begin{displaymath}
      \begin{split}
        &\{\code{on\_line}\ B\ a\wedge\code{on\_line}\ F\ a\wedge\code{on\_line}\ G\ a \quad (4)\},\\
        &\{\},\{\},\{\},\{\},\{\},\{\},\\
        &\left\{\begin{aligned}
            &\exists \alpha.\;\code{on\_plane}\ A\ \alpha\wedge \code{on\_plane}\ B\ \alpha\wedge \code{on\_plane}\ C\ \alpha\wedge\code{on\_plane}\ D\ \alpha\\
            &\qquad\wedge\code{on\_plane}\ E\ \alpha\ \code{on\_plane}\ F\ \alpha\ \code{on\_plane}\ G\ \alpha & (0,1,3,4)
            \end{aligned}\right\},\\
        &\{\},\\
        &\{C \neq G, E \neq G \quad (0,2,3,4)\},\\
        &\{\},\{\},\\
        &\left\{\begin{aligned}
          &D \neq G, & (0,1,2,3,4),\\
          &A \neq G & (0,2,3,4)
        \end{aligned}\right\}
     \end{split}
   \end{displaymath}
\end{minipage}}
\caption{Generations in the root of the case-split}
\label{fig:RootGenerations}
\end{figure}

\begin{figure}[H]
\doublebox{\begin{minipage}{\linewidth}\footnotesize\setlength\abovedisplayskip{0cm}
    \begin{displaymath}
      \begin{split}
        &\left\{\begin{aligned}
            &\exists a.\;\code{on\_line}\ A\ a\wedge\code{on\_line}\ E\ a\wedge\code{on\_line}\ G\ a \quad & (4),\\
            &A\neq G, A\neq E, E\neq G & (4)
          \end{aligned}\right\},\\
        &\{\},\{\},\\
        &\{B \neq G \quad (2,4)\}\\
        &\{\},\{\},\\
        &\{C\neq G\ \quad (0,2,4)\}\\
        &\{F\neq G\ \quad (0,2,3,4)\}\\
        &\{\},\{\},\{\},\{\},\{\},\{\},\{\},\{\},\{\},\{\},\{\},\{\},\{\},\{\},\\
        &\left\{\begin{aligned}
            &\Triangle{a}{A}{B}{G} & (2,4) \\
            &\Triangle{a}{B}{E}{G} & (2,4)
          \end{aligned}\right\},\\
        &\{\},\{\},\{\}\\
        &\left\{\begin{aligned}
            &\Triangle{a}{C}{E}{G} & (0,2,4)\\
            &\Triangle{a}{A}{C}{G} & (0,2,4)\\
            &\Triangle{a}{B}{C}{G} & (0,2,4)
          \end{aligned}\right\},\\
        &\{\},\{\},\\
        &\left\{\begin{aligned}
            &\Triangle{a}{A}{F}{G} & (0,2,3,4)\\
            &\Triangle{a}{C}{F}{G} & (0,2,3,4)\\
            &\Triangle{a}{E}{F}{G} & (0,2,3,4)\\
            &\Triangle{a}{B}{D}{G} & (0,1,2,4)\\
            &\Triangle{a}{C}{D}{G} & (0,1,2,4)\\
            &D\neq G & (0,1,2,4)
          \end{aligned}\right\}.
     \end{split}
   \end{displaymath}
\end{minipage}}
\caption{Generations found on the assumption $\between{A}{G}{E}$}
\label{fig:LeftGenerations}
\end{figure}

We get more information in the next partition, which is the left-branch of the case-split, shown in Figure~\ref{fig:LeftGenerations}. When the trees are flattened, the sequents in this branch carry $\between{A}{G}{E}$ as a disjunctive hypothesis. For brevity, we omit sequents about the existence of planes.

This is the consistent case. The triangles found here will be used as further justification for applications of Axiom~\ref{eq:g24}. The inconsistent case occurs in the remaining partition, carrying the disjunctive hypothesis $\between{C}{G}{E}$. See Figure~\ref{fig:RightGenerations}.

\begin{figure}[h]
\doublebox{\begin{minipage}{\linewidth}\footnotesize\setlength\abovedisplayskip{0cm}
    \begin{displaymath}
      \begin{split}
        &\left\{\begin{aligned}
            &\exists a.\;\code{on\_line}\ C\ a\wedge\code{on\_line}\ E\ a\wedge\code{on\_line}\ G\ a \quad & (4),\\
            &C\neq G, C\neq E, E\neq G & (4)
          \end{aligned}\right\},\\
        &\{\},\{\},\{\},\\
        &\{\exists a.\;\code{on\_line}\ C\ a\wedge\code{on\_line}\ E\ a\wedge\code{on\_line}\ F\ a\wedge\code{on\_line}\ G\ a, \quad (3,4)\},\\
        &\{\}\\
        &\{A\neq G, B\neq G \quad (0,2,4)\},\{\},\{\},\{D\neq G \quad (0,1,2,4)\}\\
&\{\},\{\},\{\},\{\},\{\},\{\},\{\},\{\},\{\},\{\},\{\},\{\},\{\},\{\},\{\},\{\},\{\},\{\},\{\},\{\},\\
        &\left\{\begin{aligned}
            &\Triangle{a}{A}{C}{G} & (0,2,4) \\
            &\Triangle{a}{A}{E}{G} & (0,2,4) \\
            &\Triangle{a}{B}{C}{G} \quad& (0,2,4) \\
            &\Triangle{a}{B}{E}{G} & (0,2,4) \\
          \end{aligned}\right\},\{\},\{\},\\
        &\left\{\begin{aligned}
            &\Triangle{a}{D}{E}{G} & (0,1,2,4)\\
            &\Triangle{a}{B}{D}{G} & (0,1,2,4)\\
            &\Triangle{a}{C}{D}{G}\quad & (0,1,2,4)\\
            &\Triangle{a}{A}{B}{G} & (0,2,4) \\
            &F = G & (0,2,3,4)
      \end{aligned}\right\}
     \end{split}
   \end{displaymath}
\end{minipage}}
\caption{Generations found on the assumption $\between{C}{G}{E}$}
\label{fig:RightGenerations}
\end{figure}

The use of our tree data-structure means this last sequence is generated in parallel with the other two. It takes 9.58 seconds to generate all three sequences, though most of the theorems which appear are generated in under 1~second. The theorem required to advance the proof
\begin{displaymath}
\between{C}{G}{E} \implies F = G
\end{displaymath}
is generated in 6.19~seconds. The consequent of this implication contradicts our hypothesis that $\between{C}{E}{F}$ via $\eqref{eq:g23}$, and thus, the case given by its antecedent can be discarded.

\fbox{\begin{minipage}{\linewidth}\small
  \setlength\abovedisplayshortskip{-0.3cm}
  \begin{displaymath}
   \code{obviously}\ (\code{by\_eqs}\circ\code{split})\ \code{qed by}\ \eqref{eq:g21},\eqref{eq:g23}\ \code{from}\ 0,2,3
\end{displaymath}\end{minipage}}\linebreak

The \code{obviously} keyword tells the step to collapse the stream of trees, pushing the branch labels into the sequents as antecedents. It then uses the resulting discovered sequents to justify the step.

The rest of the proof proceeds similarly. With the help of our discoverers, we reduce the 80 or so steps used in our manual verification to a verification with just 17 steps. Generally, we found a consistent reduction of roughly 80\% in proof length across all 18 theorems from our manual verifications, with the new verifications always comparing much more favourably with the prose.

\subsection{Part~2 of THEOREM~5}
We now consider the second part of THEOREM~5, our verification of which matches Hilbert's logic very closely, even if we have reordered some of the derivations. The whole proof is indirect, which explains why there is no accompanying diagram, and it is one of the few proofs where Hilbert treats the disjunction in Axiom~II,~4 symmetrically: both alternatives entail a contradiction.

\begin{quotation}2. If $B$ lies on the segment $AC$ and $C$ lies on the segment $AD$ then $C$ also lies on the segment $BD$ and $B$ also lies on the segment $AD$. Choose one point $G$ that does not lie on $g$, and another point $F$ such that $G$ lies on the segment $BF$. By Axioms~I, 2 and II, 3 the line $CF$ meets neither the segment $AB$ nor the segment $BG$ and hence, by Axiom~II,~4 again, does not meet the segment $AG$. But since $C$ lies on the segment $AD$, the straight line $CF$ meets then the segment $GD$ at a point $H$. Now by Axiom~II,~3 and II,~4 again the line $FH$ meets the segment $BD$. Hence $C$ lies on the segment $BD$. The rest of Assertion 2 thus follows from 1.
\flushright{\emph{Foundations of Geometry}~\cite{FoundationsOfGeometry} (page 5)}
\end{quotation}

Our verification is shown in Figure~\ref{fig:FiveVerification2}. In the prose, Hilbert is applying Axiom~II,~4 in its contrapositive form, so we cannot follow him literally by using our \code{by\_pasch} discoverer. Instead, we run a \emph{reductio} argument in a subproof, where we can use Axiom~II,~4 in a forward direction.

\label{sec:g12Erratic}Notice that this is the one time that Hilbert makes an explicit reference to an incidence axiom other than I, 3: he cites Axiom~I, 2, and we do not have a clear idea why. This axiom is needed in many places in these proofs, but is nearly always implicit. Bernays cites the same axiom in a proof supplementing the text (see \S\ref{sec:SupplementI}). The only obvious commonality between the two is that the citation occurs when Axiom~\ref{eq:g24} is applied and \emph{both} disjuncts in the conclusion are eliminated. 

But Axiom~I,2 is required when eliminating even \emph{one} disjunct. So perhaps there is evidence here that neither Hilbert nor Bernays are taking the case-split in Axiom~\ref{eq:g24} sufficiently seriously. If so, then it makes sense for our verifications to eliminate the disjuncts explicitly in subproofs, and it makes sense that we treat Axiom~\ref{eq:g12} uniformly, and leave its use implicit.

\begin{boxedfigure}
\small
  \begin{align*}
    &\code{assume}\ \between{A}{B}{C} \wedge \between{A}{C}{D} & 0,1\\
    &\code{consider}\ G\ \code{such that}\\
    &\qquad\Triangle{a}{A}{B}{G}\ \code{from}\ 0\ \code{by}\ \eqref{eq:g12},\eqref{eq:g13b},\eqref{eq:g21}&2\\
    &\code{obviously by\_neqs so consider}\ F\ \code{such that}\ \between{B}{G}{F}\ \code{by}\ \eqref{eq:g22}&3\\
    &\code{have}\ \neg(\exists P.\; (\exists a.\;\code{on\_line}\ C\ a\wedge\code{on\_line}\ F\ a\wedge\code{on\_line}\ P\ a)\wedge\between{A}{P}{G} & 4\\
    &\qquad\code{proof:}\ \code{otherwise consider}\ P\ \code{such that}\\
    &\qquad\qquad \exists P.\; (\exists a.\;\code{on\_line}\ C\ a\wedge\code{on\_line}\ F\ a\wedge\code{on\_line}\ P\ a) \wedge\between{A}{P}{G}& 4,5\\
    &\qquad\code{clearly by\_pasch so consider}\ Q\ \code{such that}\\
    &\qquad\qquad(\exists a.\;\code{on\_line}\ C\ a\wedge\code{on\_line}\ P\ a\wedge\code{on\_line}\ Q\ a)\\
    &\qquad\qquad\qquad(\between{A}{Q}{B} \vee \between{B}{Q}{G})\ \code{from}\ 0,2\\
    &\qquad\code{obviously}\ (\code{by\_eqs}\circ\code{split})\ \code{qed from}\ 0,2,3,4,5\ \code{by}\ \eqref{eq:g21},\eqref{eq:g23}\\
    &\code{obviously by\_pasch so consider}\ H\ \code{such that}\\
    &\qquad(\exists a.\;\code{on\_line}\ C\ a\wedge\code{on\_line}\ F\ a\wedge\code{on\_line}\ H\ a)\\
    &\qquad\qquad\wedge\between{D}{H}{G}\ \code{from}\ 0,1,2,3\ \code{at}\ 5,6\\
    &\code{have}\ B\neq D\ \code{from}\ 0,1\ \code{by}\ \eqref{eq:g21},\eqref{eq:g23} & 7\\
    &\code{clearly by\_pasch so consider}\ C'\ \code{such that}\\
    &\qquad (\exists a.\;\code{on\_line}\ C'\ a\wedge\code{on\_line}\ F\ a\wedge\code{on\_line}\ H\ a)\\
    &\qquad\qquad(\between{B}{C'}{D} \vee \between{B}{C'}{G}\ \code{from}\ 0,1,2,3,5,6\\
    &\code{obviously}\ (\code{by\_eqs}\circ\code{split})\ \code{qed from}\ 0,1,2,3,5,6,7\ \code{by}\ \eqref{eq:g21},\eqref{eq:g23}
  \end{align*}
  \caption{THEOREM~5 verification, part 2}
  \label{fig:FiveVerification2}
\end{boxedfigure}

\section{Conclusion}
Hilbert gives only three prose proofs for his first two groups in the \emph{Grundlagen der Geometrie}. None of these proofs were present in the first edition, and the results they prove, one-dimensional ordering theorems, were originally assumed as axioms. Two of the proofs were contributed by Wald and E.H. Moore, though Veblen also deserves credit in helping to investigate how linear order theorems can be derived from  two-dimensional order axioms.

We have reviewed how the automation we discussed in the last chapter enables us to write very short proofs compared to our manual verifications, without having to break from Hilbert's basic proof strategies. In fact, we obtain proofs whose steps match Hilbert's prose steps very closely. Our DeBruijn factor is almost 1, and our incidence discoverers appear to be adequate to handle all of the incidence reasoning implicit in Hilbert's proofs.

This is not merely aesthetic. That we can write such short verifications of these relatively simple theorems without trudging through a bog of incidence arguments gave us hope that we could tackle the far more complex verification of the Polygonal Jordan Curve Theorem. Indeed, we shall find that the automation described in the last chapter is used aggressively in that verification. 

We just need some further automation for dealing with linear ordering. This is described in the next chapter, which shows how THEOREM~4 and THEOREM~5 are generalised in THEOREM~6.

%%% Local Variables: 
%%% TeX-master: "../thesis"
%%% End: 
