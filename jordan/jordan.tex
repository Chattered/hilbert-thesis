\chapter{The Jordan Curve Theorem for Polygons}\label{chapter:JordanInformal}
In this chapter, we shall discuss the next theorem we encounter in Hilbert's \emph{Foundations of Geometry}, which appears as Theorem~9 in the 10th edition. This theorem is a special case of the Jordan Curve Theorem, which applies only to simple polygons. Our discussion will draw together a number of historical threads and characters connected to the \emph{Foundations of Geometry}, and point out interesting subtleties and obstacles that come with trying to rigorously prove the theorem from Hilbert's very weak axiom system. We shall consider several informal proofs, including Veblen's 1904 proof, which we believe to contain a major fault. The verification of the theorem is left to Chapter~\ref{chapter:JordanVerification}.

In sections~\ref{sec:JordanCurveHistory} and~\ref{sec:JordanCurveGenerality}, we give a short historical introduction to the polygonal case in the context of the more general theorem. In \S\ref{sec:JordanCurveExplanation}, we give Hilbert's formulation of the theorem and, because Hilbert did not supply a proof, we provide a detailed one of our own in \S\ref{sec:JordanCurveFirstProof}. Finally, in \S\ref{sec:VeblenProof}, we present another proof given by Veblen~\cite{Veblenphd}. This latter proof is probably incorrect, but it \emph{is} based on axioms very close to Hilbert's own, and contains useful ideas. Our final verification in Chapter~\ref{chapter:JordanVerification} is based on our diagnosis of the problems with Veblen's proof.

\section{Relationship with the Full Jordan Curve Theorem}\label{sec:JordanCurveHistory}
The full Jordan Curve Theorem theorem effectively says that, when it comes to closed curves that do not self-intersect (\emph{simple closed curves}), we are justified in our use of the expressions ``inside the curve'' and ``outside the curve''. The idea that mathematicians should even bother justifying this appears relatively late in the history of mathematics. In fact, it had to wait until the 19th century and the rigorous reformulations of analysis which reduced the unclear notions governing the continuum to precisely defined formulas involving the now standard $\epsilon$ and $\delta$ inequalities. Bolzano, who is credited along with Cauchy for spotting the reformulation, provided his own rigorous definitions for closed, continuous curves and what it means for curves to enclose points, so that he was then able to recommend the following for rigorous proof~\cite{BolzanoJordan}:

\begin{quote}
If a closed line lies in a plane and if by means of a connected line one joins a point of the plane which is enclosed within the closed line with a point of the plane which is not enclosed within it, then the connected line must cut the closed line.
\end{quote}

This is a significant half of the Jordan Curve Theorem, and, reading the terms ``closed lines'' intuitively, it seems so blindingly obvious that one might think it perverse to demand a proof. However, the rigorous definitions which Bolzano had in mind to replace ``closed line'' are not so immediately intuitive, but appeal to very general and abstract topological properties. The first proof that these abstract properties preserve intuitive properties such as Bolzano's conjecture was given in Jordan's 1887 \emph{Cours d'analyse}~\cite{JordanTextBook}, and to this day, the details are regarded as quite involved. One possible reason for the complexity is that the rigorous formulations are so general that they admit weird pathologies, or as Poincar\'{e} colourfully called them, \emph{monsters}, and some of these monsters, such as closed curves enclosing finite area but having infinite length, immediately thwart a number of obvious proof strategies.

Relevant to this chapter is the case against Jordan's proof: common folklore says that it is invalid, and the first correct proof was provided by Veblen in 1905~\cite{VeblenJordan}. Both Veblen and folklore point out that Jordan had to assume the polygonal case, which should have been proven as a lemma. Hales, on the other hand, has formally verified the theorem in HOL~Light, and put together a strong defence of Jordan~\cite{HalesJordansProof}, and of a basically elegant and correct proof that has been unfairly neglected. The polygonal case is supposedly completely trivial. 

Not so, according to Feferman. In his paper concerning the aforementioned \emph{monsters}~\cite{IntuitionMonsters}, he repeats the folklore that Veblen gave the first correct proof, and claims that even the polygonal case is ``devilishly difficult to prove.''

We might suggest that this controversy begins with Hilbert's 1899 edition of \emph{Foundations of Geometry}. There, Hilbert gave a formulation of the polygonal case as Theorem~6, but, as with the five preceding theorems, he did not give a proof. Instead, he assures us that with the aid of his theorem for the existence of half-planes (Theorem~5 of that edition), one can obtain the proof ``without much difficulty.'' This certainly backs up the idea that the theorem is trivial. However, by the Ninth Edition, the clause had been deleted, with the edit noted as a ``correction.'' The theorem still appears without proof, though Bernays, in a supplement to the main text, cites a detailed proof in an article by Fiegl~\cite{FeiglJordan}. Note that he does not \emph{give} the proof, as he does in other cases, such as proving that Pasch's axiom can be rendered without the inclusive-or~(see \S\ref{sec:PaschInclusiveOr}). That would have taken more than a few supplementary remarks.

Now Veblen himself, one year before publishing his proof of the Jordan Curve Theorem, had developed an axiomatic foundation for geometry which was very close to Hilbert's own~\cite{Veblenphd}. And in this thesis, he expends a great deal more effort developing a theory of order than did Hilbert. This explains why Veblen's doctoral supervisor, E. H. Moore, was contributing proofs to later editions of Hilbert's text~(see \S\ref{sec:Theorem5}). It also explains why, in Veblen's 1905 proof of the full Jordan Curve Theorem, he thought it necessary to cite a proof from his doctoral thesis showing that the Jordan Curve Theorem holds for the even more trivial case of \emph{triangles}. It seems plausible to us, therefore, that Veblen's criticism of Jordan is explainable by his particular standards of rigour and the context of an axiomatic theory of geometry. When it came to his theory of order, his standards of rigour were even higher than Hilbert's.

Another aspect we should consider is the level of generality that Veblen was attempting. He gave a proof of the full theorem in the context of ordered geometry with the addition of one topological axiom. As such, the proof was not supposed to require any assumptions about the existence of a metric. Unfortunately, as Hales points out in his defence of Jordan~\cite{HalesJordansProof}, the generality was refuted ten years later. R. L. Moore, who had been a student of Veblen's, showed that according to Veblen's axioms, all of his planes are homeomorphic to the Euclidean plane~\cite{MooreSitus}. %do, in fact, always describe a metrizable space.

But what about the polygonal case? In his doctoral thesis, Veblen gave a detailed, standalone proof of this theorem without using the topological axiom. The theorem, then, is plausibly still very general. So one question is: given its generality, is it still \emph{trivial}? We suggest not. As we discuss in \S\ref{sec:VeblenProof}, a correct proof seems to have eluded Veblen himself.

\section{Generality of the Polygonal Case}\label{sec:JordanCurveGenerality}
We can discuss the problems raised by our level of generality by considering the two proofs of the polygonal case from the book \emph{What Is Mathematics?}~\cite{WhatIsMathematics}. Hales mentioned these to highlight the triviality of the polygonal case. The first of these proofs is the so-called ``plumb-line'' proof. We begin with a simple polygon, pick an arbitrary direction which is not parallel to any of its sides, and then for any point, we cast a ray in that specified direction. By considering the number of times the ray crosses the polygon, and how this number changes as we move around the plane, we can prove the polygonal case.

This is probably a non-starter. At Group~II, we have no theorems which say that parallel lines even \emph{exist}. Nor do we have a theorem which says we can cast rays in a given direction. The needed theorems are actually developed in Group~III and Group~IV, where angle congruence is introduced as a primitive, and where we can start manipulating \emph{directions}. We also have no formulations of what it means to ``move'' along a ray. The sort of motion being considered is presumably \emph{continuous} motion, and indeed, if we look at Tverberg's proof of the theorem~\cite{TverbergJordan}, he essentially gives the same argument in rigorous form, and appeals directly to continuity. But  continuity does not appear in Hilbert's text until Group~V. 

The second proof from \emph{What Is Mathematics?} only states an approach, and does not provide a complete proof. It effectively says we can prove the Jordan Curve Theorem for polygons by computing winding numbers for the polygon. A naive formulation of this argument in terms of angles and continuous motion will, for the same reasons as above, be well outside the scope of Hilbert's first two groups of axioms. 

To reiterate, the problem we have with traditional proofs of the Polygonal Jordan Curve Theorem is that our axioms are too weak to formulate them. Another way to put this is to say that the version of the theorem we are attempting to prove is more general than the traditional versions, and must apply to \emph{any} ordered geometry. A visual way to bring this point home is to note that by ``simple polygons'', we are actually including the boundaries of all possible \emph{mazes} which do not contain loops. But worse, we are allowing the corridors of these mazes to be infinitesimally narrow, since we do not rule out non-Archimedean geometries at this stage.

Furthermore, we are tasked with navigating these mazes without being able to measure any distances. We cannot orient ourselves, rotate or compare directions. We cannot consider continuous motion. And we know nothing about the existence of parallel lines. In other words, we are navigating without a ruler, without a compass, and without being able to run a path parallel to a wall. We suspect these constraints will eliminate most trivial proofs.

\begin{figure}
\centering
\includegraphics[scale=0.5]{jordan/maze.pdf}
\caption{A Simple Polygon}
\end{figure}

\section{Polygonal Case: Formulation}\label{sec:JordanCurveExplanation}
The full Jordan Curve theorem applies to arbitrary simple closed curves, and characterises the interior and exterior in terms of path-connectedness. The polygonal version of the Jordan Curve Theorem replaces ``simple closed curve'' with ``simple polygon'', and characterises the two regions in terms of \emph{polygonal}-path connectedness. That is, the interior and exterior of the polygon are maximal sets, all of whose points can be joined by polygonal paths. 

The three primitives at Hilbert's disposal, namely two incidence relations and a betweenness relation, are sufficient to formulate the notion of polygons, interiors and exteriors, and thus the theorem. Having already defined a segment as an unordered pair of points, Hilbert can now define a \emph{polygonal segment} as follows\footnote{Veblen calls these \emph{broken lines}.}:

\begin{quote}
DEFINITION. A set of segments $AB$, $BC$, $CD$, $\ldots$, $KL$ is called a \emph{polygonal segment} that connects the points $A$ and $L$. Such a polygonal segment will also be briefly denoted by $ABCD\ldots KL$. The points inside the segments $AB$, $BC$, $CD$, $\ldots$, $KL$ as well as the points $A$, $B$, $C$, $D$, $\ldots$, $K$, $L$ are collectively called the \emph{points of the polygonal segment}. 
\end{quote}

Polygons are then polygonal segments where the points $A$ and $L$ coincide. We then refer to each of the individual segments $AB$, $BC$, $\ldots$, $KL$ as a ``side'' of the polygon. Finally, Hilbert defines \emph{simple polygons}:

\begin{quote}
If the vertices of a polygon are all distinct, none of them falls on a side and no two of its nonadjacent sides have a point in common, the polygon is called \emph{simple}.
\end{quote}

Like the rays and half-plane theorems, the polygonal Jordan Curve Theorem describes a partitioning of a space into two ``connected'' regions, where connectedness is cashed out in terms of a suitable relation. Here, we are told that the polygon partitions all other points in the plane as follows:

\begin{quote}
  THEOREM 9. Every single polygon lying in a plane $\alpha$ separates the points of the plane $\alpha$ that are not on the polygonal segment of the polygon into two regions, the \emph{interior} and the \emph{exterior}, with the following property: If $A$ is a point of the interior {\bfseries (an interior point)} and $B$ is a point of the exterior {\bfseries (an exterior point)} then every polygonal segment that lies in $\alpha$ and joins $A$ with $B$ has at least one point in common with the polygon. On the other hand if $A$, $A'$ are two points of the interior and $B$, $B'$ are two points of the exterior then there exist polygonal segments in $\alpha$ which join $A$ with $A'$ and others which join $B$ with $B'$, none of which have any point in common with the polygon. By suitable labeling of the two regions there exist lines in $\alpha$ that always lie entirely in the exterior of the polygon. However, there are no lines that lie entirely in the interior of the polygon.

  \centering\includegraphics[scale=0.8]{jordan/jordanHilbert.pdf}
\end{quote}

In the remainder of this chapter, we shall identify a polygon with its list of vertices $P_1P_2P_3\cdots P_n$.

First, we shall briefly summarise the last few sections: as with Hilbert's \emph{Foundations of Geometry}, the formulation and proof of the Jordan Curve Theorem marks a significant development in the history of rigorous, axiomatic mathematics and geometry. Its position in Hilbert's text means that it should be possible to formulate and prove it in very basic terms and from very simple axioms, based on ideas about half-planes. It is somewhat controversial whether this is trivial, but we hope that during the next few chapters, the matter will be somewhat clarified.

\section{Point-in-Polygon Proof}\label{sec:JordanCurveFirstProof}
One feature of Veblen's proof, and the proof we verify in Chapter \ref{sec:JordanVerification}, is that the treatment of the two regions defined by a simple polygon is symmetric. This means, however, that unlike the plumb-line and winding number proofs from \emph{What is Mathematics?}, Veblen's proof does not hint at any sort of point-in-polygon \emph{test}, one that could be implemented on a machine.

Before we encountered Veblen's proof, we had developed one of our own which does allow for such a test, which reduces the problem of deciding whether a point is inside a polygon to the problem of deciding whether a point is on a given side of the polygon's diagonals. The proof is inductive over the vertices of a simple polygon, and therefore captures the idea of a polygon 

We decided to try an inductive proof on the total number of vertices of a simple polygon. To do this, it is necessary to show that every simple polygon can be broken down into smaller, simple polygons, until one reaches the smallest possible polygon (the triangle). We will assume in the following that the theorem is true for triangles, leaving the formalisation and verification of this matter to our final proof in~\S\ref{sec:MainJordanProof}.

One of the nice properties of the plumb-line and winding-number proofs is that they suggest a point-in-polygon test. Initially, 

 We decided to try an inductive proof on the total number of vertices of a simple polygon. To do this, it is necessary to show that every simple polygon can be broken down into smaller, simple polygons, until one reaches the smallest possible polygon (the triangle). We will assume in the following that the theorem is true for triangles, leaving the formalisation and verification of this matter to our final proof in~\S\ref{sec:MainJordanProof}.

This idea is not too distant from the idea of triangulating a polygon, which should give us pause, since most proofs that a polygon can be triangulated assume the Jordan Curve Theorem for polygons. We are thus in danger of creating a circular argument. We avoid this danger because firstly, at each stage, we do not require that the smaller polygons are subsets of the original (thus, we do not actually identify a triangulation). And secondly, we are always able to use the Jordan Curve Theorem as an inductive hypothesis at each stage.

To split a given simple $n$-gon where $n>3$ into two simple polygons, we just pick a line connecting two non-adjacent vertices, a \emph{diagonal}, which does not intersect the polygon. The simplest example of interest is the concave quadrilateral shown in Figure~\ref{fig:quadConcave}. This quadrilateral has exactly two diagonals. The diagonal $P_2P_4$ lies in the interior of the polygon, while the diagonal $P_1P_3$ lies in the exterior.

\begin{figure}
  \centering
  \includegraphics[scale=0.8]{jordan/quadConcave.pdf}
  \caption{Concave Quadrilateral}
  \label{fig:quadConcave}
\end{figure}

If we take the diagonal $P_2P_4$, we see that the interior of the quadrilateral consists of the union of the interiors of two triangles, namely $P_1P_2P_4$ and $P_2P_3P_4$, together with the diagonal $P_2P_4$ itself (see Figure~\ref{fig:quadUnion}). If we take the diagonal $P_2P_4$, then we see the interior of the quadrilateral consists of the interior of the triangle $P_1P_3P_4$, minus the interior of $P_1P_2P_3$ and the boundary of $P_1P_2P_3$. Thus, we can define the interior of the quadrilateral in terms of the interiors of two triangles. We shall extend this to the general case for an $n$-gon with $n>3$. First, we show how to find diagonals.

\begin{figure}
  \centering 
  \subfigure[Union of Two Triangles]{\includegraphics[scale=0.8]{jordan/quadUnion1.pdf}
    \includegraphics[scale=0.8]{jordan/quadUnion2.pdf}
    \includegraphics[scale=0.8]{jordan/quadUnion3.pdf}
    \label{fig:quadUnion}}

  \subfigure[Difference of Two Triangles]{\includegraphics[scale=0.8]{jordan/quadDiff1.pdf}
    \includegraphics[scale=0.8]{jordan/quadDiff2.pdf}
    \includegraphics[scale=0.8]{jordan/quadDiff3.pdf}
    \label{fig:quadDiff}}
  \caption{Decomposing a Concave Quadrilateral}
  \label{fig:quadDecompose}
\end{figure}

\subsection{Finding a Diagonal}\label{sec:FindingDiagonal}
A simple polygon $P_1P_2 \ldots P_n$ where $n>3$ has at least one diagonal which does not intersect the polygon. To see this, we assume without loss of generality that $P_1P_2P_3$ are non-collinear, and that $P_n$ is not inside the triangle $P_1P_2P_3$. Consider the line $P_1P_3$. If this line does not intersect the polygon, then it is a suitable diagonal. Otherwise, take the vertex $P_m$ where $3 < m < n$ such that the line $P_1P_m$ intersects $P_2P_3$ in a point $X$ and such that, for any other $P_{m'}$ where $3 < m' < n$, if $P_1P_{m'}$ intersects $P_1P_2$ at $X'$, then $X$ is between $P_1$ and $X'$. We then have that $P_1P_m$ is the required diagonal (see Figure~\ref{fig:SqueezeDemo}), which yields two smaller simple polygons, shown in green and blue.

A very slightly modified version of this argument is crucial in the proof we actually verified, and so we shall explain it in more detail in \S\ref{sec:Squeeze}.

\begin{figure}
\centering
\includegraphics{jordan/diagonal.pdf}
\caption{Finding a Diagonal}
\label{fig:SqueezeDemo}
\end{figure}

\subsection{Cases}
As noted, a diagonal $D$ of a polygon $p$ which does not intersect that polygon divides that polygon into two smaller polygons $p_1$ and $p_2$. Generalising the situation from Figure~\ref{fig:quadDecompose}, we have that if $D$ is interior to $p$, then the interior of $D$ can be defined as the union of the interiors of $p_1$ and $p_2$, together with $D$. On the other hand, if $D$ is exterior to $p$, and the interior of $p_2$ is a subset of the interior of $p_1$, then we can define the interior of $p$ as the interior of $p_1$ minus the boundary and interior of $p_2$. It therefore follows that the interior of a polygon is formed by recursively adding and subtracting out the interiors of smaller polygons. Finally, at each step, we define the exterior of the polygon as the plane minus the interior and minus the polygon's boundary.

There is obviously circularity here, since whether or not $D$ is an interior or exterior diagonal should be a \emph{corollary} of our argument. To remedy this, we shall need to characterise the two cases some other way.

In Figure~\ref{fig:DiagonalCases}, we illustrate the two cases by showing a fragment of a polygon and its diagonal. Here, we assume that the polygon $p$ is of the form $\ldots$, $P_1$, $P$, $Q_1$, $\ldots$, $Q_2$, $Q$, $P_2$, $\ldots$ and has a diagonal $PQ$ which does not intersect $p$. Our recursive step involves splitting the polygon into $p_1 = \ldots, P_1, P, Q, P_2, \ldots$ and $p_2 = \ldots, Q_1, P, Q, Q_2$. The interior of $p_1$ is shaded green. The interior of $p_2$ is shaded blue. Overlapping interiors are shaded cyan.

We now take a point $X$ on the diagonal $PQ$, and we cast rays out to the points $Y$ and $Z$ on either side of $PQ$, such that $X$ is the only point of intersection of the segment $YZ$ and the polygons $p_1$ and $p_2$ (ray-casting will be discussed in \S\ref{sec:RayCast}). 

We now make the following inductive hypotheses:

\begin{description}\label{sec:FirstProofInductiveHypotheses}
\item[IH1] Interior points $A$ and $B$ of $p_1$ can be connected by a polygonal path which does not intersect $p_1$, and similarly for $p_2$.
\item[IH2] Exterior points $A$ and $B$ of $p_1$ can be connected by a polygonal path which does not intersect $p_1$, and similarly for $p_2$.
\item[IH3] Any path connecting an interior point $A$ of $p_1$ to an exterior point $B$ of $p_1$, must intersect $p_1$, and similarly for $p_2$.
\item[IH4] If $Y'$ and $Z'$ are endpoints of a segment which crosses exactly one side of $p_1$, then they lie in different regions with respect to $p_1$ (and similarly for $p_2$).
\item[IH5] If another segment $Y'Z'$ does not intersect any side of $p_1$, nor any side of $p_2$, then $Y'$ and $Z'$ lie in the same region.
\end{description}

Assuming that interiors and exteriors are non-empty and cover the plane, the first three hypotheses are equivalent to the polygonal Jordan Curve Theorem. They give us the following two cases:

\begin{enumerate}
\item Both $Y$ and $Z$ lie in the interior of exactly one of the polygons $p_1$ and $p_2$ (see Figure~\ref{fig:UnionCase}).
\item One of $Y$ and $Z$ lies in the exterior of both $p_1$ and $p_2$ (see Figure~\ref{fig:SubCase}).
\end{enumerate}

The first case corresponds to $PQ$ being an interior diagonal. The second case corresponds to $PQ$ being an exterior diagonal.

\begin{figure}
\centering
\subfigure[Union for Interior Diagonal]{\includegraphics{jordan/unionCase.pdf}
  \label{fig:UnionCase}}
\subfigure[Subtraction for Exterior Diagonal]{\includegraphics{jordan/diffCase.pdf}
  \label{fig:SubCase}}
\caption{Diagonal Cases}
\label{fig:DiagonalCases}
\end{figure}

\begin{figure}
\centering
\includegraphics{jordan/navigation1.pdf}
\caption{Navigating around the Exterior}
\label{fig:Navigation1}
\end{figure}

\begin{figure}
\centering
\subfigure{\includegraphics{jordan/shorten1.pdf}\label{fig:Shorten1}}
\subfigure{\includegraphics{jordan/shorten2.pdf}\label{fig:Shorten2}}
\caption{Paths for IH3}
\end{figure}

\subsection{Union Case}
For the case depicted in Figure~\ref{fig:UnionCase}, we will assume, without loss of generality, that $Y$ lies in the interior of $p_1$ and $Z$ lies in the interior of $p_2$. We now define the interior of $p$ as the union of the interiors of $p_1$ and $p_2$ and the interior of the diagonal $PQ$. We define the exterior as the set of points not on the polygon and not in the interior. We just need to prove IH1--IH5 for the polygon $p$. 

\begin{description}
\item[IH1] Interior points $A$ and $B$ of $p$ can be connected by a polygonal path which does not intersect~$p$.
  \begin{proof}
    Suppose $A$ and $B$ both lie inside $p_1$, or both lie inside $p_2$. We apply IH1 to $p_1$ and $p_2$, and thereby connect $A$ and $B$ with a polygonal segment lying in the interior of $p$. On the other hand, if we suppose that $A$ is inside $p_1$ and $B$ is inside $p_2$, we apply IH1 to $p_1$ in order to connect $A$ to $Y$, and we apply IH1 to $p_2$ in order to connect $B$ to $Z$. The required path can then be completed with the segment $YZ$.
  \end{proof}
\item[IH2] Exterior points $A$ and $B$ of $p$ can be connected by a polygonal path which does not intersect $p$.
  \begin{proof}
    We must have that $A$ and $B$ lie in the exteriors of both $p_1$ and $p_2$. By IH2 applied to $p_1$, there is a path through the exterior of $p_1$ which connects $A$ and $B$. If this path lies entirely in the exterior of $p_2$ then it also lies entirely in the exterior of $p$ and we are done. 

    Otherwise, we take the first point $A_1$ and the last point $A_2$ at which the path crosses the boundary of $p_2$. We then find a new path which follows the edges of the polygon until it connects these two points. See Figure~\ref{fig:Navigation1} for an illustration and see \S\ref{sec:Jordan1NavigationDiscussion} for some discussion on this part of the proof.
  \end{proof}
\item[IH3] Any path connecting an interior point $A$ of $p$ to an exterior point $B$ of $p$, must intersect $p$.
  \begin{proof}
    Assume that $A$ lies in the interior of $p_1$ and $B$ lies in the exterior of both $p_1$ and $p_2$, and suppose there is a path connecting $A$ and $B$ which does not intersect $p$. By IH3 applied to $p_1$, this path must intersect the diagonal $PQ$. Take the last point at which it intersects this diagonal. Then there is a later point $A'$ on the path which we can connect to either $Y$ or $Z$ without crossing $PQ$. By IH5 applied to $p_1$ and $p_2$, it then follows that $A'$ is interior to either $p_1$ or $p_2$, whilst being connected to $B$. Hence, by IH3 applied to $p_1$ and $p_2$, the path must intersect $p$. See Figure~\ref{fig:Shorten1} for an illustration and see \S\ref{sec:Jordan1SameSideDiscussion} for further discussion.
  \end{proof}
\item[IH4] If $Y'$ and $Z'$ are endpoints of a segment which crosses exactly one side of $p$, then one point is interior while the other is exterior with respect to $p$.
  \begin{proof}
    Suppose $Y'Z'$ crosses a side of $p$. Then if we apply IH4 to $p_1$, we have that $Y'$ and $Z'$ lie in opposite regions with respect to $p_1$. Moreover, one of the points, say $Y'$, lies inside $p$. Aiming for a contradiction, we suppose that $Z'$ also lies inside $p$. Since it does not lie inside $p_1$, nor does it lie on $PQ$, it follows that it must lie inside $p_2$. But then if we apply IH5 to $p_2$, we find that $Y'Z'$ must cross a side of $p_2$. Hence, $Y'Z'$ crosses a side shared by both $p_1$ and $p_2$. This contradicts the fact that $p$ is simple and $PQ$ does not intersect either $p_1$ or $p_2$.
  \end{proof}
\item[IH5] If a segment $Y'Z'$ does not intersect $p$, then $Y'$ and $Z'$ lie in the same region.
  \begin{proof}
    If the segment does not intersect any side of $p$ and does not intersect any sides of $p_1$ or $p_2$, then the case reduces to IH5 applied to $p_1$ and $p_2$. If it does intersect $p_1$ or $p_2$, then the side intersected must be the diagonal $PQ$. In this case, we can find a path which does not intersect $p_1$ and which connects $Y$ to $Y'$, or a path which does not intersect $p_2$ and which connects $Y$ to $Z'$. Thus, $Y'$ is interior to one of $p_1$ and $p_2$ by IH5 and thus interior to $p$. The same argument applies to $Z'$. See \S\ref{sec:Jordan1SameSideDiscussion} for some further discussion. % Note that the above disjunction is a case-split on which side of $PQ$ the points $Y'$ and $Z'$ are on. I am confident that the step corresponds exactly to the use of same_side_wall_connected in the verified proof.
  \end{proof}
\end{description}

\subsection{Subtraction Case}
The second case differs from the first in that we lose the symmetry between $p_1$ and $p_2$. One of these polygon's interiors must be contained entirely by the other's. 

We can determine which by considering whether $P_1$ lies inside $p_2$ or whether $Q_1$ lies inside $p_1$. To see why these are alternatives, suppose that $P_1$ does not lie on $p_2$ (as depicted). Then it follows from IH5 and the fact that $p$ is simple that the boundary of $p_2 - PQ$ lies outside $p_1$. By considering a path interior to $p_2$ which connects $Z$ and $Q_1$, we can therefore conclude, again by IH5, that $Q_1$ must be interior to $p_1$. Again, see \S\ref{sec:Jordan1SameSideDiscussion} for further details.

We now assume, without loss of generality, that $Q_1$ is interior to $p_1$ as depicted. We then define the interior of $p$ to be the interior of $p_1$ minus the interior of $p_2$ and its boundary. 

Note that by IH5 and the fact that $p$ is simple, we have that all points on $p_2 - PQ$ must lie inside $p_1$, and that all points on $p_1 - PQ$ must lie outside $p_2$. We now prove IH1--IH5 for the step case.

\begin{description}
\item[IH1] Interior points $A$ and $B$ of $p$ can be connected by a polygonal path which does not intersect~$p$.
  \begin{proof}
    An interior point of $p$ is an interior point of $p_1$ which is not interior to $p_2$. If we apply IH1 to $p_1$, we can find a path between $A$ and $B$ through the interior of $p_1$. It is possible that this path intersects $p_2$, and so we must appeal to the same argument used for IH2 in the previous section and depicted in Figure~\ref{fig:Navigation1}, in order to navigate through the exterior of $p_2$.
  \end{proof}
\item[IH2] Exterior points $A$ and $B$ of $p$ can be connected by a polygonal path which does not intersect $p$.
  \begin{proof}
    This argument is similar to that given for IH1 in the previous section. If both $A$ and $B$ lie in the exterior of $p_1$, then we know from IH1 applied to $p_1$ that they can be connected by a path exterior to $p_1$. And since $p_2 - PQ$ is interior to $p_1$, we know that this path does not intersect $p$.

    Similarly, if $A$ and $B$ lie in the interior of $p_2$, we know from IH1 that they can be connected by a path interior to $p_2$. And since $p_1$ is not interior to $p_2$, we know that this path does not intersect $p$.

    Finally, if $A$ is exterior to $p_1$ and $B$ interior to $p_2$, then we can join $A$ to $Y$ by a path exterior to $p_1$ and $p_2$, and join $B$ to $Z$ by a path interior to $p_1$ and $p_2$. The segment $YZ$ then completes a path connecting $A$ to $B$ which does not intersect $p$.
    \end{proof}

\item[IH3] Any path connecting an interior point $A$ of $p$ to an exterior point $B$ of $p$, must intersect $p$.
  \begin{proof}
    The proof is similar to that for the previous section. By definition, $A$ lies inside $p_1$ and outside $p_2$. The point $B$, on the other hand, might lie inside $p_2$, outside $p_1$, or it might lie on the diagonal $PQ$. In any case, we can conclude that the path intersects the diagonal $PQ$, by applying IH3 to one of $p_1$ and $p_2$. We now take the first point at which the path intersects the diagonal. Then there will be an earlier point $A'$ on the path which must either lie outside $p_1$ or lie inside $p_2$. This point is connected to $A$ by a path which does not intersect $PQ$, and thus, by applying IH3 to one of $p_1$ and $p_2$, we can find a point where the path intersects $p$. See Figure~\ref{fig:Shorten2} for an illustration.
  \end{proof}

\item[IH4] If $Y'$ and $Z'$ are endpoints of a segment which crosses exactly one side of $p$, then they lie in different regions with respect to $p$.
  \begin{proof}
     Suppose that $Y'Z'$ intersects a side of $p_1$. If we apply IH4 to $p_1$, we find that one of these points must be interior to $p_1$ and the other exterior. Moreover, since the point of intersection of $Y'Z'$ with $p_1$ is exterior to $p_2$, we can apply IH5 to $p_2$ and conclude that $Y'$ and $Z'$ are both exterior to $p_2$. It follows by the definition of $p$ that $Y'$ and $Z'$ are then in opposite regions. A similar argument applies if $Y'Z'$ intersects a side of $p_2$.
  \end{proof}
\item[IH5] If a segment $Y'Z'$ does not intersect $p$, then $Y'$ and $Z'$ lie in the same region.
  \begin{proof}
    If $Y'Z'$ does not intersect $p_1$ or $p_2$, then this reduces to the case IH5 applied to these polygons. Otherwise, we suppose that $Y'Z'$ intersects the diagonal $PQ$. 

    If $Y'$ is inside $p$, then it lies inside $p_1$ and outside $p_2$, and thus $Z'$ must be outside $p_1$ by IH4 applied to $p_1$. But this means that both $Y'$ and $Z'$ lie outside $p_2$, which contradicts IH5 applied to $p_2$. 

    On the other hand, if $Y'$ is not inside $p$, then there are two possibilities for its location:
    \begin{description}
    \item[The point $Y'$ lies inside $p_1$ and $p_2$] In this case, we can apply IH4 to $p_2$ and conclude that $Z'$ is outside $p_2$. By definition, both points are then exterior to $p$. 
    \item[The point $Y'$ lies on $PQ$] In this case, $Z'$ must also lie on $PQ$. Again by definition, both points are then exterior to $p$.
    \item[The point $Y'$ lies outside both $p_1$ and $p_2$] Here, we apply IH4 to the point $p_2$, and thus conclude that the point $Z'$ is inside $p_2$. By definition, both points are then exterior to $p$.
    \end{description}
  \end{proof}
\end{description}

\subsection{Discussion}
There are a few definite gaps in this first proof attempt. As said, we neglected to prove the base case for triangles, and we have not shown that the recursively defined interiors and exteriors are always non-empty. Finally, the without-loss-of-generality assumption made in finding the diagonal of a polygon needs to be carefully justified (it amounts to showing that a polygon cannot have a spiral shape).
 We consider these to be fairly straightforward matters, however. 

This is not the proof we chose to verify, though the one we \emph{have} verified shares many of its inferences. When we come to verify those inferences, we shall see that many details above have been glossed over. We hope, then, to convince the reader to be quite wary of the above proof as it stands. The details that we have glossed over will be put on a more solid footing when we discuss our formal verification.

\label{sec:Jordan1NavigationDiscussion}For instance, in Figure~\ref{fig:Navigation1}, we claim to be able to navigate around the exterior of a simple polygon. This intuitive idea will be needed again in our verified proof, where it needs to be carefully formulated, and the formulation needs to be carefully tied back to the premises and the desired conclusion. The details are given in \S\ref{sec:NavigationProof}.

\label{sec:Jordan1SameSideDiscussion}At several places in the above proof, we also assumed we could always find a path connecting the points $Y$ and $Z$ in Figures~\ref{fig:UnionCase} and~\ref{fig:SubCase} to various other points. A careful formulation of what characterises such points, together with a proof that the required path can be exhibited, is given in \S\ref{sec:JordanSameSideProof}.

We have retained the above proof because of its computational properties. It recursively \emph{builds} the interior of a polygon from the interiors of triangles, and thus describes the algorithm for a point-in-polygon test by reducing the problem to point-in-triangle tests, which, in turn, reduce to point in half-plane tests. Our \emph{verified} proof does not have this feature. It was inspired by Veblen's proof, and its basic structure and focus are very different. Nevertheless, it is interesting that a number of the key steps overlap, and we shall refer back to this proof when we discuss the verification of the polygonal Jordan Curve Theorem in Chapter~\ref{chapter:JordanVerification}.

\section{Veblen's Proof}\label{sec:VeblenProof}
In his 1904 doctoral thesis~\cite{Veblenphd}, Veblen set out a basic set of axioms for Euclidean Geometry. His incidence and order axioms are very similar to Hilbert's own, and it is not difficult to prove their equivalence. Unlike Hilbert, he provided a complete proof for the polygonal Jordan Curve Theorem, and we shall discuss that proof in this section, comparing it to the one we sketched in \S\ref{sec:JordanCurveFirstProof}.

Veblen does not explicitly define the interior or exterior of a polygon. Like most other proofs, he tries to show that the two regions exist implicitly by dividing the theorem into two claims: the first states that a simple polygon divides its plane into at least two regions; the second states that the polygon divides its plane into at most two regions. Both assertions can be formalised directly in terms of points and 

\begin{enumerate}
\item there are at least two points in the plane not on $p$ which cannot be connected by a polygonal segment without crossing $p$\label{list:VeblenLemma1};
\item of any three points in the plane not on $p$, at least two of them can be connected by a polygonal segment without crossing $p$.
\end{enumerate}

We will point out in advance that Veblen's proof is a two page proof and one of the most detailed in his thesis. This again brings into question Hilbert's claim that the theorem can be easily derived. We might also consider the fact that Hilbert was willing to give explicit proofs of much simpler theorems, such as his Theorem~3 (see \S\ref{sec:Theorem3}). 

Furthermore, it is questionable whether Veblen's proof is even correct. According to Reeken and Kanovei~\cite{HahnInconclusiveIndirect}, the proof was deemed ``inconclusive'' by Hahn, while Guggenheimer claims, citing Lennes and Hahn, that Veblen's proof assumes that a polygon can be triangulated and is otherwise only valid for convex polygons~\cite{GuggenheimerJordanCurve}. We could not find this criticism in Lennes~\cite{LennesPolygon}, and were initially satisfied by Veblen's proof, and so we attempted a verification of assertion (\ref{list:VeblenLemma1}) above. Eventually, as one might expect, we hit an obstacle. In the next few sections, we will try to explain the difficulty.

\subsection{Veblen's Lemma 1}\label{sec:VeblenLemma1}
The first half of Veblen's proof is given as the corollary to a very general theorem about polygons, expressed in terms of ``multiple points'', which are those points, should they exist, where a polygonal self-intersects:
\begin{quote}
Lemma 1. \emph{If a side of a polygon $q$ intersects a side of a polygon $p_n$ in a single point $O$ not a multiple point of $p_n$ or $q$, then $p_n$ and $q$, whether simple or not, have at least one other point in common.} 
\end{quote}

We can highlight the generality of this theorem by considering a particularly degenerate version of polygons. Consider the example in Figure~\ref{fig:jordanDegenerate1}. Here, we have two polygons $P_1P_2P_3$ and $Q_1Q_2Q_3$. The points of these polygons are obviously collinear and so cannot divide the plane into multiple regions. But neither are they a counterexample to Veblen's lemma, since their point of intersection $X$ is a multiple point of both, lying simultaneously on the segments $P_1P_2$ and $P_2P_3$, and also lying on the segments $Q_1Q_2$ and $Q_2Q_3$.

\begin{figure}
\centering
\includegraphics{jordan/jordanDegenerate1}
\caption{Degenerate polygons intersecting at a multiple point}
\label{fig:jordanDegenerate1}
\end{figure}

We have formally verified that this particular lemma follows from Hilbert's axioms (see \S\ref{sec:Lemma1Verification}), but we gave up trying to reproduce Veblen's argument, and believe it to be invalid. We do not have a counterexample, as such, since we do not think Veblen's proof is sufficiently detailed to say exactly where it fails. Instead, we have tried to illustrate the difficulties we faced with the example in Figure~\ref{fig:VeblenCounter1}. This example shows a simple maze polygon $p_n$ being intersected at a point $O$ by another polygon $q$ shown in red. The goal is to identify one of the other seven points of intersection. Our labelling in the diagram is consistent with the set-up to Veblen's proof:

\begin{quote}
If $n=3$ ($q$ having any number of sides, $m$) the theorem reduces to [the case for triangles]. We assume without loss of generality that no three vertices $P_{i-1}$, $P_i$, $P_{i+1}$ are collinear and prove the lemma for every $n$ by reducing to the case $n=3$. Let $p_n$ have $n$ vertices with the notation such that the side $P_1P_2$ meets $q$ in the side $Q_1'Q_2'$ where the segment $Q_2'O$ contains no interior point of the triangle $P_1P_2P_3$.\end{quote}

\begin{figure}
\centering
\includegraphics[scale=0.6]{jordan/veblenCounter1.pdf}
\caption{Intersections on a simple maze}
\label{fig:VeblenCounter1}
\end{figure}

Veblen's basic strategy is to consider each of the triangles $P_1P_2P_3$, $P_1P_3P_4$, $P_1P_4P_5$, $P_1P_5P_6$, $\ldots$. Of these triangles, all but the first and last share exactly one side with the polygon $p_n$, with the other sides being diagonals of the polygon. Veblen tries to show that if $q$ intersects one of these triangles in one diagonal, then it intersects the next triangle. In this way, intersections can be found, one after the other, down the list of triangles, until we eventually find a second point of intersection with the polygon $p_n$.

\subsection{Finding a subset of $q$}\label{sec:SubsetOfQ}
In Figure~\ref{fig:VeblenCounter2}, we show a polygonal segment  (or ``broken line'') which is a subset of the polygon $q$, with vertices $O_kQ_2'Q_3'Q_4'Q_5'Q_6'O_j$. This segment makes contact with the diagonal $P_1P_3$ exactly twice, once from the outside of the triangle $P_1P_2P_3$, and once from the inside. The polygonal segment always exist, and can be proven to intersect $P_1P_3$ in just this way. To find it, we start from the segment $Q_2'Q_3'$ and progressively add neighbouring segments until we eventually reach a segment which intersects the line $P_1P_3$. As Veblen puts it:

\begin{figure}
\centering
\includegraphics[scale=0.6]{jordan/veblenCounter2.pdf}
\caption{A chosen subset of $q$: $k=2$ and $j=6$}
\label{fig:VeblenCounter2}
\end{figure}

\begin{quote}By the case $n=3$, $q$ meets the boundary of the triangle $P_1P_2P_3$ in at least one point other than $O$. If this point is on the broken line $P_1P_2P_3$ the lemma is verified. If not, $q$ has at least one point on $P_1P_3$, and at least one of the segments $Q_1'Q_2'$, $Q_2'Q_3'$ has no point or end-point on $P_1P_3$. Let this segment be one segment of a broken line $Q_kQ_{k+1}\cdots Q_{j-1}Q_j$ of segments of $q$ not meeting $P_1P_3$ but such that $Q_{k-1}Q_k$ and $Q_jQ_{j+1}$ do each have a point or endpoint in common with $P_1P_3$ ($1 \leq k < j \leq m$; if $k = 1$, $Q_{k-1} = Q_m$; if $j = m$, $Q_{j+1} = Q_1$). If $O_j$ is the point common to $P_1P_3$ and $Q_jQ_{j+1}$ or $Q_{j+1}$, and $O_k$ is the point common to $P_1P_3$ and $Q_{k-1}Q_k$ or $Q_{k-1}$, the broken line $O_kQ_kQ_{k+1}\cdots Q_{j-1}Q_jO_j$, has a point inside and also a point outside the triangle $P_1P_2P_3$ and cuts the broken line $P_1P_2P_3$ only once. \end{quote}

Now there is nothing particularly informative about Veblen's last remark. Notice that the segment $Q_1'Q_2'$ \emph{also} has a point inside and a point outside the triangle $P_1P_2P_3$. It also cuts the polygonal segment $P_1P_2P_3$ exactly once. Something is missing here. There must be some other property had by the polygonal segment $O_kQ_kQ_{k+1}\cdots Q_{j-1}Q_jO_j$, in order for Veblen to get to his very next claim, that ``it has a point inside and a point outside any triangle of which $P_1P_3$ is a side''. The missing property is an important detail, because as we mentioned, part of the argument is supposed to be repeated down the list of triangle $\triangle P_1P_2P_3$, $\triangle P_1P_3P_4$, $\triangle P_1P_4P_5$, $\triangle P_1P_5P_6$, $\ldots$. If we are going to repeat this part of the argument, we need to know that any missing properties will be valid in the repeated case.

For now, we just note that Veblen's claim certainly follows. The crucial point is that, as we remarked earlier, the polygonal segment $O_kQ_kQ_{k+1}\cdots Q_{j-1}Q_jO_j$ touches $P_1P_3$ exactly twice, once from the outside and once from the inside. More precisely, we just note that the interior of $Q_kO_k$ is outside the triangle, while the interior of $Q_jO_j$ is inside. To establish this, we note that the points inside the segment $O_kO$ are all inside the triangle $P_1P_2P_3$~\footnote{For Veblen, this is a matter of definition. See ~\ref{sec:triangleDef}}. Thus, by the Jordan Curve Theorem applied to triangles, all points of the polygonal segment $O\cdots O_{j-1}O_j$ must be outside the triangle $P_1P_2P_3$. It is then possible to show that the points inside the segment $O_kO$ are on the opposite side of the line $P_1P_3$ as the points inside the segment $Q_{j-1}O_j$, and so must be in different regions of any triangle of which $P_1P_3$ is a side. Veblen's claim then follows: the polygonal segment $O_kQ_kQ_{k+1}\cdots Q_{j-1}Q_jO_j$ has a point inside and a point outside any triangle of which $P_1P_3$ is a side. We just needed a bit of extra work to get there.

\subsection{Veblen's Conclusion}
\begin{quote}On this account if $P_1P_3P_4$ are not collinear, and obviously, if $P_1P_3P_4$ are collinear, $q$ must meet either $P_3P_4$ or $P_4$ or $P_4P_1$. If $q$ does not meet $P_3P_4$ or $P_4$, we proceed with $P_1P_4P_5$ as we did with $P_1P_3P_4$. Continuing this process, we either verify the lemma or come by $n-2$ steps to the triangle $P_1P_{n-1}P_n$ and find that $q$ must intersect the broken line $P_{n-1}P_nP_1$, which also verifies the lemma.\end{quote}

This is Veblen's conclusion, and the situation described is illustrated in Figure~\ref{fig:VeblenCounter3}. The first step follows by the Jordan Curve Theorem applied to triangles: we know that $O_kQ_kQ_{k+1}\cdots Q_{j-1}Q_jO_j$ does not strictly\footnote{The use of phrases such as ``strictly cut'', among many other details, will be clarified in our verification. See Chapter~\ref{chapter:JordanVerification}.} \emph{cut} the line $P_1P_3$, so it must instead cut either $P_1P_4$ or $P_3P_4$. We can assume that $q$ does not meet $P_3P_4$, and so we proceed with $P_1P_4P_5$. But Veblen is not clear on exactly \emph{how} the argument repeats. The reason the first step follows in the above conclusion is because $O_kQ_kQ_{k+1}\cdots Q_{j-1}Q_jO_j$ does not cut $P_1P_3$, but we have now assumed that it \emph{does} cut $P_1P_4$, so we are not simply repeating the conclusion. 

\begin{figure}
\centering
\includegraphics[scale=0.6]{jordan/veblenCounter3.pdf}
\caption{Intersections with $P_1P_3P_4$}
\label{fig:VeblenCounter3}
\end{figure}

Notice that in this conclusion, Veblen has slipped to talking about the original polygon $q$ rather than the polygonal segment $O_kQ_kQ_{k+1}\cdots Q_{j-1}Q_jO_j$ which is a subset of $q$. Presumably then, we need to find another subset of $q$ which has the same properties as the original. And here we encounter the problem of detail mentioned at the end of \S\ref{sec:SubsetOfQ}: we were not sure to begin with what the crucial properties of the subset \emph{are}. Nevertheless, we tried earlier to fill in those details, and based on our attempt, we might assume that the desired polygonal segment for our example is the one shown in Figure~\ref{fig:VeblenCounter4}, namely $O_k'O_kQ_2'Q_3'Q_4'O_j'$. This polygonal segment can be identified by starting at the point $O_k$ and following Veblen's procedure as we did with the point $O$. 

\begin{figure}
\centering
\includegraphics[scale=0.6]{jordan/veblenCounter4.pdf}
\caption{Desired subset of $q$?}
\label{fig:VeblenCounter4}
\end{figure}

Remember that Veblen's next claim is that the polygonal segment $O_k'O_kQ_2'Q_3'Q_4'O_j'$ has a point inside and outside any triangle of which $P_1P_3$ is a side. Again, this is certainly true. But earlier, we justified it by noting that the polygonal segment touches $P_1P_4$ exactly twice, once from inside the triangle and once from outside. More formally, we can notice that the points inside the segment $O_k'Q$ lie inside the triangle $P_1P_3P_4$, while the points inside the segment $Q_4'O_j'$ lie outside the triangle. We could establish this earlier by relying on Veblen's assertion that the polygonal segment $O_k'O_kQ_2'Q_3'Q_4'O_j'$ crosses $P_1P_3$ \emph{exactly once}, but this is no longer true. Instead, the required observation is that it crosses $P_1P_3$ an \emph{odd} number of times, and so it has a non-empty suffix which lies entirely outside the polygon. 

We therefore need a proof that the number of crossings must always be odd, but we cannot see anything in Veblen's argument that would require this. We presumably need additional lemmas about the parity of crossings on a triangle's side. But as soon as we start to do \emph{that}, we can immediately identify an argument which is conceptually much more straightforward. We give this argument in full in \S\ref{sec:ParityProofInformal}.

\section{Conclusion}
In this chapter, we have looked at several proofs of the Jordan Curve Theorem for simple polygons which might be applicable to Hilbert's two groups of axioms. Our first proof started with a recursive definition of an arbitrary polygon's interior and exterior. It explained how to add and subtract smaller polygons at each recursive step until we arrive at the base-case in triangles. The rest of the proof then shows that the recursive definition always yields two distinct regions which cover the plane, and such that any two points in the regions can be joined by a polygonal segment.

We then reviewed Veblen's 1904 proof of the theorem, from axioms equivalent to Hilbert's, and which, on the face of it, is far simpler to our first proof. However, it has been claimed elsewhere~\cite{GuggenheimerJordanCurve,HahnInconclusiveIndirect} to be invalid. Based on a verification attempt, we believe we have identified the crucial problem, and have suggested that its fix leads immediately to a conceptually simple parity proof. In the next chapter, where we shall review our verification of the parity proof, we shall be able to constrast its conceptual simplicity with the complexity of its details. 

With the verification, we have now unquestionably shown that despite the seeming theoretical poverty of Hilbert's first two groups, we have sufficient primitives and axioms to prove the Jordan Curve Theorem for polygons. We also hope that our analysis of the proof shows that Hilbert was mistaken to think the theorem could be obtained ``without serious difficulty.''

%%% Local Variables: 
%%% TeX-master: "../thesis"
%%% End: 
