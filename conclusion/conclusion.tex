\chapter{Conclusion and Future Work}\label{chapter:Conclusion}
In this thesis, we tackled the first two groups of axioms from Hilbert's \emph{Grundlagen der Geometrie}, formalising the definitions and theorems and verifying the proofs in the HOL~Light Theorem Prover. We used a declarative style language, trying to bring the verifications as close to the synthetic geometry style of written mathematics, and in many cases, we achieved an almost one-to-one correspondence with informal proofs.

What we have learned, as discussed in Chapter~\ref{chapter:Axiomatics}, was that the verification must be able to cope with complex and combinatorial incidence arguments, which arise from Hilbert's first group of axioms. This should come as no surprise when we reflect on the fact that the group introduces two primitive notions governed by more axioms than are in any of the other groups. 

In Chapter~\ref{chapter:Automation}, we provided our main implementation for the project. Our aim was to follow Hilbert, who left the use of incidence axioms largely implicit. To match this, we left the verification of incidence arguments to automation. In doing so, we showed how to adapt a combinator language for unbounded search to the domain of theorem proving, and showed how to enrich the basic data-structures as tagged proof trees which could automatically handle case-splits. We argued that this use of search fits naturally with the idea of declarative proof, and allows us to use the automation not merely as a verification tool, but as a \emph{collaborative discovery} tool to explore proofs.

The main weakness with our approach to search is that, as of writing, it does not handle subsumption or normalisation in an intelligent way. Finding a way to incorporate these features into our algebra would not only make it more useful to theorem proving, but would help specifically with our incidence discovery, where theorems of collinearity and planarity are derived which subsume earlier theorems, and where multiple points are identified and must be normalised across all discovered theorems. One possible direction for future research is to find an elegant way to implement this. This done, we would then like to further explore the theoretical underpinnings of the algebra.

In Chapter~\ref{chapter:Group2Eval}, we tackled Hilbert's three prose proofs in Group~II. These three proofs are relatively late additions to the \emph{Grundlagen der Geometrie} , and were imported from proofs by Wald, Moore and Veblen whose own axiomatics was subtlely different to Hilbert's. They therefore made for interesting case-studies. We demonstrated ways to streamline the proofs by using the Inner and Outer forms of Axiom~\ref{eq:g24}, and in each proof, we were able to show how our automation brings the verifications almost one-to-one with the prose. The systematic automation further suggested that the only remaining gaps arise because Hilbert is not clear on when to cite Axiom~\ref{eq:g12} in eliminating disjunctions from Axiom~\ref{eq:g24}. 

We showed in the same chapter that our automated tool can be used to explore the proof space around these three theorems. We were able to find an alternative proof to Hilbert's Theorem~4, and to trace the complexity of incidence reasoning using our set-theoretic rules for collinearity and planarity.

In the next chapter, we dealt with Theorem~6, and considered the problem of how to implement and then verify what are essentially metatheoretical statements. In one approach, we showed how to understand Theorem~6 as a method to generate terms about ordered points and then verify them on demand. We implemented this as an ML procedure based on an efficient representation of orderings. In another approach, we showed how to understand Theorem~6 as a theorem about order preserving maps from subsets of the natural numbers to points on a line. Our formalisation immediately gave us a means to reduce linear ordering problems to HOL~Light's decision procedures for linear arithmetic.

On the way to proving Theorem~6, we eliminated the redundancy in assuming the axiom of infinity in our core logic, showing that while the existence of infinite sets cannot be established from just the first group of axioms, it is guaranteed once we have the second group. 

In Chapter~\ref{chapter:HalfPlanes}, we laid out the theories of rays and half-planes and elaborated on the details necessary to do this. Both concepts arise by abstraction from suitable equivalence relations, and we tried to exploit the quotienting facilities of HOL~Light to automate this. This we found problematic because the equivalence relations carry an extra dependency on a suitable point and line. To deal with the problem, we folded the dependency into an intermediate type which could then be quotiented. A possible avenue of future research is to enrich the quotienting facilities of HOL~Light in such a way that these intermediate types are either unnecessary or can be generated automatically.

In the remaining chapters, we introduced what we consider to be the main contribution of this thesis. This was a relatively large and complex synthetic proof of the Polygonal Jordan Curve Theorem from relatively weak axioms. The verification mostly follows Veblen's prose proof of the theorem, but with crucial corrections and elaborations. With the verification now machine checked, we have definitively demonstrated that the theorem is derivable from just the first two groups of Hilbert's axioms via a fairly simple, and hopefully very clear, informal proof.

The verification itself ties together all the work from the previous chapters. The incidence automation and linear ordering tactic are used extensively to justify the declarative proof steps, while many of the non-automated steps are justified by reasoning with the half-planes from Chapter~\ref{chapter:HalfPlanes}. This last point confirms both Hilbert and Veblen's claim that the proof of the Polygonal Jordan Curve Theorem is based largely on reasoning with these half-planes.

Finally, the verification is a stress test of the Mizar~Light combinators. It serves as an example of how HOL~Light is quite capable of formalising declarative proofs, and that Wiedijk's simple combinator language can easily scale to handle large verifications.

We end by conjecturing that the greatest challenges faced in verifying Hilbert's axiomatics lie in dealing with incidence, the largest axiom group in the whole chapter, and order, from which one proves the Polygonal Jordan Curve Theorem. While there are many more pages dedicated to the remaining three groups of Hilbert's axiomatics, we suspect this is because of Hilbert's particular interests in the elementary consequences of metrical notions of Group~III and his disinterest in ordering, parallels and proofs based on continuity. In our earlier verifications in Isabelle, we have verified some of Hilbert's early prose proofs in Group~III, and can confirm that most of the difficulty in the verification is still in dealing with incidence reasoning.

We therefore hope that any future work verifying the remaining three groups can go through with relative ease, now that we are armed with automation for incidence reasoning, linear ordering, and have demonstrated the effectiveness of this automation in verifying what is almost certainly the most complex proof of Hilbert's first chapter.
%%% Local Variables: 
%%% mode: latex
%%% TeX-master: "../thesis"
%%% End: 
