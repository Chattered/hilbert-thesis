\documentclass[phd,cisa]{infthesis}

\usepackage{proof}
\usepackage{amssymb}
\usepackage{amsmath}
\usepackage{graphicx}
\usepackage{framed}
\usepackage{subfigure}
\usepackage{float}
\usepackage{url}
\usepackage{multirow}
\usepackage{amsthm}
\usepackage{graphics}
\usepackage{color}
\usepackage{fancybox}

\definecolor{gray}{rgb}{0.9,0.9,0.9}

\floatstyle{boxed}
\newfloat{boxedfigure}{p}{lobf}
\floatname{boxedfigure}{Figure}
\restylefloat{plain}
\makeatletter
\let\c@boxedfigure\c@figure
\makeatother 
\renewcommand{\theboxedfigure}{\arabic{chapter}.\arabic{figure}}

\newtheorem*{theorem}{Theorem}
\newtheorem*{lemma}{Lemma}

\newcommand{\code}{\texttt}
\newcommand{\adjoin}[2]{\{#1\}\,\cup\,#2}
\newcommand{\cons}[2]{#1\,:\,#2}
\newcommand{\append}[2]{#1 ++ #2}
\newcommand{\el}[2]{\code{el}\ #1\ #2}
\renewcommand{\between}[3]{\code{between}\ #1\ #2\ #3}
\newcommand{\online}[2]{\code{on\_line}\ #1\ #2}
\newcommand{\onplane}[2]{\code{on\_plane}\ #1\ #2}
\newcommand{\onpolyseg}[2]{\code{on\_polyseg}\ #1\ #2}
\newcommand{\collinearthree}[3]{($\exists$a. #1 on a $\wedge$ #2 on a $\wedge$ #3 on a)}
\newcommand{\collinearfour}[4]{($\exists$a. #1 on a $\wedge$ #2 on a $\wedge$ #3 on a $\wedge$ #4 on a)}
\newcommand{\collinearfive}[5]{($\exists$a. #1 on a $\wedge$ #2 on a $\wedge$ #3 on a $\wedge$ #4 on a $\wedge$ #5 on a)}
\newcommand{\planarthree}[3]{($\exists\alpha$. #1 on $\alpha$ $\wedge$ #2 on $\alpha$ $\wedge$ #3 on a)}
\newcommand{\planarfour}[4]{($\exists\alpha$. #1 on $\alpha$ $\wedge$ #2 on $\alpha$ $\wedge$ #3 on $\alpha$ $\wedge$ #4 on a)}
\newcommand{\planarfive}[5]{($\exists\alpha$. #1 on $\alpha$ $\wedge$ #2 on $\alpha$ $\wedge$ #3 on $\alpha$ $\wedge$ #4 on $\alpha$ $\wedge$ #5 on a)}
\newcommand{\planarsix}[6]{($\exists\alpha$. #1 on $\alpha$ $\wedge$ #2 on $\alpha$ $\wedge$ #3 on $\alpha$ $\wedge$ #4 on $\alpha$ $\wedge$ #5 on a $\wedge$ #6 on a)}
\newcommand{\planarseven}[7]{($\exists\alpha$. #1 on $\alpha$ $\wedge$ #2 on $\alpha$ $\wedge$ #3 on $\alpha$ $\wedge$ #4 on $\alpha$ $\wedge$ #5 on a $\wedge$ #6 on a $\wedge$ #7 on a)}
\newcommand{\ttriangle}[3]{$\neg$($\exists$a. #1 on a $\wedge$ #2 on a $\wedge$ #3 on a)}

\newcommand{\Triangle}[4]{\neg(\exists #1.\online{#2}{a}\wedge \online{#3}{a}\wedge \online{#4}{a})}
\newcommand{\fourbet}[4]{\between{#1}{#2}{#3}\wedge\between{#1}{#3}{#4}}
\newcommand{\bounds}[3]{\text{bounds}\,#1\,#2\,#3}

\title{Ordered Geometry in Hilbert's \emph{Foundations}}
\author{Phil Scott}

%\abstract{The \emph{Foundations of Geometry} was the first systematic attempt to fill the logical gaps of Euclid's \emph{Elements}. Now with the aid of an interactive theorem prover, we can definitively conclude that story by mechanising Hilbert's axiomatics and their elementary consequences. Keeping as faithful to Hilbert's prose as possible, we use a readable mechanisation language that allows us to go beneath the proofs and reveal the deeper logical structure behind them. Where this structure is not illuminating, we develop new automated tools, tailored to Hilbert's axiomatics, so that the structure can be integrated \emph{implicitly} into a readable proof script. In achieving this, we shall describe a new language for proof automation.}

%\abstract{The \emph{Foundations of Geometry} was the first systematic attempt to fill the logical gaps of Euclid's \emph{Elements}. With the aid of a theorem prover, we take several significant steps towards concluding this story by formalising the first two groups of Hilbert's axiomatics and their elementary consequences. We endeavour to stay faithful to Hilbert's prose and use a readable declarative language in the aims of producing a critical appraisal of Hilbert's original text. We develop automation in terms of general-purpose search combinators which we then tailor to Hilbert's axiomatics. These allow us to recover implicit details that are not illuminating and to explore the proof space independently of the user. Our thesis ends by describing a verified proof of the Jordan Curve Theorem for Polygons from very weak axioms.}

\abstract{In this thesis, we initiate a formal analysis of David Hilbert's axiomatisation of ordered geometry as it appears in the tenth and current edition of his celebrated \emph{Foundations of Geometry}. Using Hilbert as a case-study, we argue that with the help of an interactive proof assistant, we are able to critically analyse the structure of mathematical arguments with an unprecedented level of confidence, provided we stay faithful to the text itself. We use a readable declarative language to support this, verify proofs synthetically as much as possible, and embed automation to recover the computational detail that a mathematician would find tedious. The thesis ends in a large case study, verifying a proof of the Polygonal Jordan Curve Theorem in the very general setting of \emph{ordered geometry}.}

\begin{document}

%% First, the preliminary pages
\begin{preliminary}

%% This creates the title page
\maketitle

%% Acknowledgements
%\begin{acknowledgements}
%Formal verification can be isolating when we intentionally make our computers the only relevant audience. Thank you to everyone in Edinburgh Informatics for balancing this!
%\end{acknowledgements}

%% Next we need to have the declaration.
\standarddeclaration

%% Create the table of contents
\tableofcontents

\end{preliminary}

\parindent 0pt\parskip 0.5ex

%\chapter{Introduction}\label{chapter:Introduction}
In this thesis, we recount our formalisation and mechanical verification of a focused subset of \emph{synthetic geometry}. This style of geometry, and indeed, this style of pure mathematics, goes back to the earliest records of the subject as we would recognise it today. The style emphasises the deduction of geometrical theorems from very simple axioms governing entities such as points and lines which otherwise have little to define them. Its canonical reference is unquestionably Euclid's \emph{Elements}~\cite{HeathElements}, possibly the most influential mathematical text ever written~\cite{BoyerEuclidInfluence}, and still exemplary of the way pure mathematics is done today, proceeding as it does from definitions to complex hierarchies of theorems.

Synthetic geometry is contrasted with \emph{analytic geometry} or coordinate geometry. In analytic geometry, we solve geometric problems by translating them into systems of algebraic equations and then solving for unknowns. The use of algebra can be highly effective, but, compared to synthetic geometry, it is much less clear how to give a geometric interpretation of the proof steps.

A synthetic proof proceeds by introducing geometric entities which can be visualised as a diagram and reasoned about directly using simple principles, meaning that proof steps have a pleasing geometric interpretation. As we present our own proofs, we urge the reader to follow along with the help of pencil and straight-edge. The subset of synthetic geometry we consider here means that no compasses are necessary!

\section{The \emph{Grundlagen der Geometrie}}
The axioms which will form the basis for our geometry are taken directly from David Hilbert's \emph{Grundlagen der Geometrie}. The text was chosen because in both modern mathematics and in the formal verification community, it has an impressive reputation. By the middle of the 20th century, it had been hailed as the most influential book in a hundred years~\cite{BirkhoffHilbertInfluence}, and by 1971 it had ten published editions. We use the second edition of the English translation, \emph{Foundations of Geometry}~\cite{FoundationsOfGeometry}, translated from the tenth edition of the \emph{Grundlagen der Geometrie}. We follow the presentation to the letter. 

The text can be seen as the spiritual successor to the axiomatics of Euclid's own \emph{Elements}. Euclid's text remains remarkable in what it accomplishes by reducing a wealth of geometric results to a handful of simple axioms and casting basic number theory in geometric terms, but Hilbert massively improves on the rigour.

Hilbert does away with Euclid's confusing list of pseudo-definitions, in which we are told, for example, that a point is ``that which has no part''. Instead, he lets the axioms exhaustively define everything we can know of points, thereby inviting us to leave our intuitions and presumptions at the door. In a famous remark, Hilbert went so far as to demand that all references to points, lines and planes in his text should be replacable with ``mug'', ``table'' and ``chair'' without affecting the logic of the arguments~\cite{TableChairMug}, thereby enforcing a principle from Pasch that all deductions must proceed without reference to the intuitive meaning of the terms involved~\cite{PaschToPeano}. Another way to put this is to say that Hilbert presents his axioms and their consequences without interpretation. If he is successful, all consequences should follow by the \emph{form} of the axioms and not by their content, something we can test by seeing whether the axioms and proofs can be unambigously translated into formal logic.

\section{Ordered Geometry}
Hilbert has five groups of axioms to describe Euclidean geometry, but in this thesis, we are interested in a much more general \emph{ordered geometry}~\cite{AxiomaticsOrderedGeometry}. The scope is defined by the first two of the five axiom groups, providing a more restrictive setting for doing proofs, where we lack a metric to talk about the distances between points, and we lack notions of angle or stipulations about parallel lines with which to discuss direction. We are without a sense of scale or orientation, but as Hilbert shows, we can still make useful definitions.

And as we shall show, we can still demonstrate important results. The main one and, we daresay, the \emph{fundamental result} of ordered geometry, is the Polygonal Jordan Curve Theorem. This theorem requires that any polygon divides the plane into exactly two connected regions. It appears as THEOREM~9 in the 10th edition of the \emph{Grundlagen}, and all the previous theorems can be seen as setting down the preliminaries required to prove it. It will be the focal point of this thesis.

While the Polygonal Jordan Curve Theorem is relatively easy to prove when we have the full resources of topology and Euclidean geometry to hand, in the very general setting of ordered geometry, the proof is quite involved. In fact, it is reasonably certain that its first published proof by Veblen is invalid, and we shall argue our case for this in Chapter~\ref{chapter:JordanInformal}. Our proof, on the other hand, is on \emph{far} firmer footing, for a major contribution of this thesis is its formal verification.

\section{Verification}
A formal verification consists in translating theorems and proofs to formal logic and then showing that all deductions are valid according to blind symbolic inference rules. The rules are so simple and few that it is easy to guarantee their validity, and thus we can guarantee the validity of any argument expressed in those rules by mechanically checking each step.

Partial verifications of Hilbert's axiomatics have been investigated by Dehlinger et al~\cite{DehlingerFOG} and Meikle and~Fleuriot~\cite{MeikleFleuriotFormalizingHilbert}. What is particularly enticing about the work of Meikle and Fleuriot is their suggestion that there are logical gaps and unstated assumptions in some of Hilbert's prose proofs. We will explain those gaps in Chapters~\ref{chapter:Axiomatics} and~\ref{chapter:Group2Eval}, where we shall try our best to justify them and vindicate Hilbert.

That said, we do not start from the assumption that Hilbert was infallible. Experience tells us that gaps are left open and logical errors easily made when doing synthetic geometry. The axioms place a severe handicap on the mathematician, preventing us initially from using geometric constructions and making observations that are so elementary that it is tempting to assume them implicitly and erroneously. With purely ordered geometry, we have an even more stringent handicap, and so we must be even more careful when trying to prove results. A diagram used to explore a proof can easily mislead by implying constraints that are not formally demonstrable, and so great care must be taken to ensure proofs are valid. We might never be fully confident without a formal verification.

Now the formalisation, if not the verification, of the axiomatics and elementary consequences of Hilbert's \emph{Grundlagen der Geometrie} was anticipated almost immediately. In his review of the text, Veblen~\cite{VeblenHilbertReview} cites Peano, who had already translated Pasch's axiomatisation of projective geometry into a symbolic form. Peano's notation survives to this day and his ideas would inspire Russell~\cite{PrinciplesOfMathematics} to produce the first major formal verification of elementary mathematics in \emph{Principia Mathematica}~\cite{Principia}.

But as Russell found out, verification can be very labour intensive, and when Poincar\'{e} saw the lengths Russell had to go just to verify that 1 is a number, he saw only ``shackles''~\cite{PoincareShackles}, and went so far as to call Peano's aims of verification ``puerile''~\cite{PoincareReview}. The criticisms were thankfully short-sighted. Verification is forcing itself on a reluctant world now that proofs have become so long and convoluted that they cannot always be verified by individual human readers~\cite{WhitherMathematics}, while the shackling pedantry required of Russell and Whitehead's research programme can be greatly alleviated with the help of machines.

\subsection{Computer Assistance}
Surprisingly, even the machine-assisted mechanical verification of Hilbert's text had been anticipated in reviews. Both Veblen and Poincar\'{e} mention mechanical logic machines that had been developed in the 19th century~\cite{LogicMachines} with which it was hoped proofs could be automatically generated. It was early days, and they both overestimated the power of such early machines --- one was limited to syllogisms --- but by the mid-1950s, Herbert Simon had a logic machine which could automatically prove all the theorems in Russell's \emph{Principia}, thereby paving the way for computer assisted verification which could relieve the poor human of the Herculean task of manually deriving the theorems. 

The success of Simon's logic machine had Russell reflect on his manual verification as ``wasted'' effort~\cite{SimonObituary}, but humans were not to be made redundant. Instead, computer assistance empowers them to tackle more complex theories, such as Hilbert's.

A decade after Simon's logic machine came DeBruijn's AUTOMATH project and the first computer \emph{assistant} for formally verifying some real mathematics. It was successfully used to verify Landau's classic text on real analysis~\cite{LandauGrundlagen,LandauAUTOMATH}, and since then, computer assisted verification has had some astounding successes. Take the Four Colour Theorem. This is a century old outstanding problem. Its first 1976 proof was assisted by inscrutable algorithms and was rightly viewed with suspicion, but the whole theorem has now been meticulously verified by Gonthier in an extremely robust verifier~\cite{GonthierFCT}. According to Hales, the verification makes the theorem one of the most well-established results in all of mathematics~\cite{HalesFormalProof}, and Gonthier went on to lead the project verifying the Feit-Thompson Theorem~\cite{FeitThompson}, a milestone in a potential verification of the classification of all finite simple groups. Finally, the verification of the outstanding four century old Kepler Conjecture has only recently been accomplished~\cite{flyspeck}. 

\subsection{Readable Verifications}
A modern verification consists of code needed to drive and guide a computerised proof assistant, which can mechanically check the validity of inferences used in a symbolic proof. We want this code to be a possible substitute for synthetic prose proofs, and retain the same visual appeal. We want readers to be able to follow the steps of the verification, perhaps drawing diagrams, and see results emerge that match our geometric intuition.

We have therefore adopted the \emph{declarative style} of verification which aims to be close to the ``mathematical vernacular''~\cite{MizarMathematicalVernacular}, and which respects the logical progressions typical of synthetic geometry. Just as in synthetic geometry, we reason to our theorems by introducing geometric entities, obtaining configurations of points, lines and planes. We then use our axioms to derive interesting properties of the configurations.

This proves challenging, because of an observation made in the AUTOMATH project which has largely stood up: there is a wide gulf between mathematical proofs as they appear in the literature and verification code. The inferential leaps that a human mathematician makes when writing a prose proof are multiplied to many formal steps in the verification, and the term ``DeBruijn factor'' was coined for the multiplier. 

The blow up can be seen in Meikle and Fleuriot's work~\cite{MeikleFleuriotFormalizingHilbert}, and our own earlier work~\cite{ScottMScThesis}, where many verification steps are needed for a single prose step, a fact which often obscures the intuition behind the proof. We did not find this acceptable. We were not prepared to sacrifice intuition on the altar of verification.

\section{Contributions and Organisation}
In the next chapter, we give an overview of the computerised proof assistant and the logic upon which we implemented all the ideas for this thesis. In Chapter~\ref{chapter:Axiomatics}, we present our formalisation of Hilbert's axioms, and try to explain why the axioms make verifications so much more long-winded than their prose counterparts. We also state and verify a theorem~(Proposition~\ref{eq:PlaneThree}) that Hilbert may have neglected and which, to our knowledge, has not been previously verified from these axioms.

In Chapter~\ref{chapter:Automation}, we present new algorithms based on streams of proof trees, which define a general-purpose algebra for partitioning and searching domains of interest. We show how to tailor this algebra to theorem-proving, and then apply it specifically to Hilbert's axioms. Then, in Chapter~\ref{chapter:Group2Eval}, we show how the automation provided by our search algebra can greatly reduce the amount of code needed to verify theorems. In fact, we show that our verifications become almost structurally identical to Hilbert's prose, with each verification step formalising an inference in the prose. These theorems have been verified elsewhere, but we provide the first verifications which match the natural language proofs so closely.

In Chapters~\ref{chapter:LinearOrder} and~\ref{chapter:HalfPlanes}, we verify Hilbert's theorems leadings up to his statement of the Polygonal Jordan Curve Theorem. In the first of these chapters, we show how to carve out the natural numbers geometrically without needing the axiom of infinity, and we then show how our verification of Hilbert's theorems allows us to reduce problems of ordering on the line to inequalities of natural numbers.

Our remaining chapters cover the verification of the Polygonal Jordan Curve Theorem from the very weak axioms of ordered geometry. In Chapter~\ref{chapter:JordanInformal}, we give a mostly informal discussion of the theorem, before discussing problems with Veblen's first proof. In Chapter~\ref{chapter:JordanFormalisation}, we present our formalisation of the theorem, leaving the details of the verification to Chapters~\ref{chapter:JordanVerification1} and~\ref{chapter:JordanVerification2}.

The proof and verification are both original, and because of the automation we provide in Chapters~\ref{chapter:Automation} and~\ref{chapter:LinearOrder} and based on the evidence from Chapter~\ref{chapter:Group2Eval}, we hope that much of our verification code is structurally similar to what we would expect of a fully elaborated prose proof.

There is plenty of commentary about Hilbert's axiomatics throughout this thesis. This is necessary, because formalising and verifying proofs forces us to make decisions with a pedantry that even mathematicians might consider debilitating. The process of teasing out logical distinctions and subtle paths of inference, and tracing dependencies between results, puts us in a strong position from which to commentate. So we include many observations about Hilbert's approach in stating theorems and proofs, some neutral, some critical, and we make these remarks with great precision and confidence: the application of formal verification and theorem proving in studying the logic of practised mathematics can be likened to the use of the microscope in studying biology.

%%% Local Variables:
%%% mode: latex
%%% TeX-master: "../thesis"
%%% End:


\chapter{System Description}
\section{Proof Assistants}\label{ProofAssistants}
The proof assistants we can use vary in a number of ways, including their support for declarative and procedural style proofs, their underlying logic, their level of automation and their support for user extensions. We give an overview of the relevant systems:

\subsection{Mizar}\label{Mizar}
The Mizar system's library is the largest body of mechanized mathematics anywhere, with articles published in a dedicated \emph{Journal of Formalized Mathematics} \cite{MizarSoftTypes}\footnote{The journal can be found at \url{http://www.cs.ualberta.ca/~piotr/Mizar/mirror/http/JFM/}}. While the logic of the system is a formal untyped first-order set theory similar to ZFC, Mizar articles are instead expressed in the form of a \emph{Mathematical Vernacular} \cite{MizarMathematicalVernacular}. The goal of this vernacular is to imitate the language and reasoning of real mathematics while still being completely formal. As such, the language is purely declarative. 

From the perspective of other theorem provers such as Isabelle, Mizar's language is interesting in its notion of adjectives. These can be understood as a powerful type system supporting subtyping and dependent typing. However, they are not part of the underlying semantics, which is still first-order set theory. Unlike most other systems, Mizar does not support user extensions and the source code is available only to members of the \emph{Association of Mizar Users}.

\subsection{PVS}
PVS is a system designed to integrate a set of powerful decision procedures, such as procedures for type-checking, reasoning about arrays and linear arithmetic, as well as user defined procedures implemented as \emph{strategies}, programmed using the full power of the host language (Common Lisp). The user of the system is able to guide the use of the procedures interactively, but automation is seen as the driving force of the proof rather than cleverly crafted proof scripts. The core system is the same higher-order logic of Isabelle's HOL but with built-in support for recursive and record types. 

Similar to Mizar, PVS supports dependent types and subtypes defined in terms of the base logic using \emph{predicate sets}. PVS also supports a module system in terms of \emph{parametric theories}, where an abstract axiomatic theory can be developed and then later interpreted by instantiating the type, function and predicate parameters. Together, these often allow for more natural mechanisations of mathematics. Indeed, real mathematics often features abstract axiomatic theories such as those of abstract algebra, while subtyping allows for the definition of partial functions common in mathematics, and allows variables to be restricted to range over a variety of sets arranged in a type hierarchy. 

While PVS theories are mostly focused on formal verifications of software and hardware, there have been significant projects mechanising mathematics \cite{IntegralCalculusPVS}.

\subsection{HOL-based Assistants}
Higher-order logic (hereafter just \emph{HOL}) is a version of Church's simple theory of types with support for ML-style polymorphism \cite{ChurchTheoryOfTypes}. While the underlying type-system of HOL is the weakest of those considered here, it has the advantage of supporting decidable type-inference, so when we formalise in HOL we do not need to annotate any terms with their types. Moreover, it is possible to simulate the more powerful type systems in HOL using predicate sets. For instance, Kamm\"{u}ller has shown how to simulate dependent typing in HOL by implementing the types directly in terms of an underlying set-theoretic semantics \cite{KammullerDependent}, and Hurd has used a similar approach to achieve a limited form of subtyping \cite{HurdSubtyping}\label{predicatesubtyping}. It is an approach which Wiedijk has also used to analyse the dependent types and subtypes of Mizar \cite{MizarSoftTypes}.

We discuss three systems based on HOL: the HOL System, Isabelle/HOL and HOL Light.

\subsubsection{HOL System}
Mike Gordon's HOL System was developed along the lines of Milner's Edinburgh LCF \cite{LCF, LCFtoHOL}. In an LCF prover, a statically typed functional language is used to define an abstract datatype of theorems whose signature defines a trusted kernel of primitive inference rules for some logic (in this case, HOL). The objects which inhabit this type are then the theorems of the logic, so proof verification reduces to type-checking. Moreover, with the full power of the underlying programming language, the user can easily define functions to create new \emph{derived} rules and proof procedures. 

While this use of functions as inference rules suggests a traditional forward style deduction system, LCF systems tend to favour a procedural style by defining goal datatypes and tactic functions. Powerful and extensible \emph{tactic languages} are written on top of this as a collection of higher-order functions, such as functions which map theorems to tactics and functions, called \emph{tacticals}, which map tactics to tactics. 

While the HOL system is largely procedural, there have been attempts to integrate declarative languages too, such as the work of Zammit \cite{ZammitDeclarative}. His language is customisable: the parsing functions and automated proof tools can be replaced as the user develops the theory. The system is distinguished by supporting a database of facts which can be extended during a proof, and which the automated tools are designed to query to discharge trivial inferences. Note, however, that this sort of style, where some of the facts used in an inference step are elided, is sometimes felt to go against the declarative style of proof\label{ZammitNotDeclarative}.

\subsubsection{Isabelle/HOL}
Isabelle is an LCF style prover for a weak logic based on polymorphic type-theory with higher-order unification \cite{Isabelle, IsabelleTypeClasses}. This is used as a formal metalevel in which object logics such as HOL and ZF are defined. One of the advantages of using a metalevel is that it is possible to distinguish schematic variables from object-level variables, so a user can easily apply proven theorems as new inference rules. 

Like PVS, Isabelle supports modular proof development with \emph{Locales}, which allow users to prove theorems relative to an abstract context of axioms. These can then be instantiated in a concrete theory \cite{IsabelleLocales}. Unlike PVS, locales merely provide a convenient syntax to write abstract theories: all theorems that result from a theory interpretation are still typed as theorems of the kernel's object logic. 

Isabelle/HOL has been used extensively for computer verification and formalised mathematics, with theory files for abstract algebra, topology, number theory and analysis\footnote{See \url{http://afp.sourceforge.net/}}.

\subsubsection{HOL Light}
Harrison reimplemented the LCF style HOL system in Ocaml \cite{HOLLight}. This time, the theorem data type corresponds to an extremely elegant foundation of HOL. Nearly all of the existing proofs in this system are written at the Ocaml level, where there is a powerful tactic language allowing for a very dense but succinct proof style. 

This proof style can be difficult to read, since the combinators are not generally applied in the linear fashion of Isabelle. However, Harrison designed a Mizar inspired declarative language for HOL Light \cite{MizarHOL} which was later adapted by Wiedijk \cite{MizarLight}. 

Wiedijk attempted to analyse the difference between procedural tactic based languages and the declarative languages. Following some of Harrison's observations, he realised that a declarative language could be implemented in just 41 lines of ML as a set of ordinary procedural tactics. By marrying procedural tactics with declarative proof, he created a system which seamlessly integrates the two approaches. 

There is already a great deal of formalised mathematics in HOL Light, including a proof of the Jordan Curve Theorem, and as of writing, it is the preferred proof assistant for the largest mechanisation effort so far, the Flyspeck Project \cite{flyspeck}\footnote{See \url{http://code.google.com/p/flyspeck/wiki/FormalText}}. 

\subsection{Coq}
Coq is based around a powerful dependent type theory supporting type operators and polymorphism \cite{Coq}. This yields an intuitionistic higher-order logic according to the Howard-Curry Isomorphism which identifies propositions with types inhabited by their proofs \cite{CalculusOfConstructions}. The system supports user definable tactics, and a module system for writing abstract theories and theory interpretation. It also supports a new extensible declarative language interface (\emph{C-zar}) as part of the official distribution.

The idea of terms as proofs leads to programs as proofs, making Coq ideal for program \emph{extraction}. For instance, given the proof that Gr\"{o}ber bases exist, Buchberger's algorithm can be automatically extracted \cite{CoqGrobner}. Coq can therefore be seen as an environment for writing computer programs that have been verified against a specification given as a type. 

But Coq is also a vehicle for mechanised mathematics, and the infamous Four Colour Theorem recently saw its first formal verification in this system \cite{GonthierFCT}. In fact, the first two groups of Hilbert's axioms from \emph{Foundations of Geometry} have been mechanised in this system by Dehlinger et al, but they focused on an intuitionistic interpretation of the text \cite{DehlingerFOG}. 

However, the naturalness of Coq for formalised mathematics is hindered by the fact that Coq's equality is \emph{intensional}: the claim that two terms are equal is a proof that they rewrite to the same normal form. This is not the equality semantics used in classical mathematics, which is what we are concerned with here.
NOT CORRECT: You can add extensionality to Coq just as you can add excluded middle.

\subsection{Logical Framework}
One of the first proof systems, De Bruijn's AUTOMATH, was based on the principle that a formal language for mathematics should abstract away from a particular choice of logical foundation \cite{AUTOMATHPTS}. In turn, this led to the principle of proofs as terms in a dependent type theory, a technique which was taken up with the Edinburgh Logical Framework and incorporated into the Twelf system \cite{Twelf}. These systems are largely concerned with software verification and programming language semantics, but the original AUTOMATH was one of the earliest systems to realise a project of mechanised mathematics, including a translation of Landau's \emph{Foundations of Analysis} \cite{LandauGrundlagen, LandauAUTOMATH}. 

\section{Proof System}
We have settled on the HOL~Light~\cite{HOLLight} theorem prover and its declarative proof language for our formalisation. In this section, we motivate and describe various extensions we have made to the basic proof language and to the automation mechanisms available in HOL~Light. 

\subsection{Declarative Proof}
Proof assistants accept formal texts in broadly two types: declarative and procedural. The difference is analogous to that between declarative and procedural programming languages. In the procedural approach, the user uses a \emph{tactic} language to compose automated tools in order to verify formalised theorems. While different tactic languages have their own styles and idioms, they tend to support both \emph{forward} reasoning from the premises of a theorem to its conclusion, and \emph{backward} reasoning, breaking down the goal conclusion into simpler subgoals. Always the focus is on procedural \emph{transformations} rather than logical formulas, which can be entirely absent from procedural proof scripts.

Declarative proof assistants on the other hand, beginning with \emph{Mizar}~\cite{MizarMathematicalVernacular}, have attempted to imitate the style of ordinary mathematics. The proof scripts express inferential relations between intermediate results that connect the premises of a theorem to its conclusion. More precisely, a declarative proof script defines a directed graph whose vertices are formulas, sources are premises, and sinks are conclusions, such that any formula is a logical consequence of its predecessors. The flow of the script always moves \emph{forward} from assumptions to goal, with the focus on \emph{what} the logical relations between formulas are, rather than \emph{how} the proof state is transformed to represent such relations. This latter detail is left to the internal operation of the proof assistant, which tries to use automated proof tools to bridge the inferential gaps.

As we explained in our proposal~\cite{ScottPhdProposal}, we wish to analyse Hilbert's text and its deep structure as formalised in higher-order logic, comparing the prose and its formalisation, looking for logical gaps, redundancies, and hopefully enriching our understanding of the informal prose. This, we have concluded, means formalising the proofs in a structure preserving way, leaving the details of inference paths intact. This strategy is most emphasised and idiomatic in the declarative style, and so we have opted to use a proof assistant which supports that style.

\subsection{HOL Light}
The proof assistant \emph{HOL~Light}~\cite{HOLLight} belongs to the LCF tradition~\cite{LCFtoHOL}. Following that tradition, the assistant encodes the syntax and inference rules of a logic as abstract algebraic data types in a functional language. In particular, the simply-typed lambda calculus~\cite{ChurchTheoryOfTypes} is encoded in Ocaml~\cite{Ocaml}. Each primitive inference rule becomes a function over the type of theorems, which are encapsulated so that simple type checking guarantees that each sequent obtained has been derived from the primitive inference rules.

With this approach, derived rules and embedded proof languages can be implemented as ordinary Ocaml functions. The user is encouraged to work at the Ocaml toplevel, exploiting the open design of the theorem prover to add new derived rules and tactics, or to make more ambitious modifications with entirely new proof tools and embedded languages.

The powerful tactic system, for instance, is implemented as a library of ML combinators which transform a data-structure representing the \emph{goal stack}. Each goal in the stack consists in a goal-formula together with associated hypotheses, which once solved will be turned into a justification function. A theorem is proven by solving all goals in the stack, and assembling the justifications into a final forward proof of the theorem.

We should mention that HOL~Light has particularly simple foundations. Its primitive syntax supports just one function, equality, as well as lambda-abstraction and application. These are governed by just ten simple inference rules and two axioms.

These meet the needs of our proposal. We seek to modify an existing declarative proof language and heavily augment its automated tools beyond the abilities of the tactic system alone. Furthermore, we suggested ways to embed a soft type system for modular theories directly in higher-order logic, disguised under a conservative implementation of the basic inference rules. The flexibility of HOL~Light and its foundational simplicity made it an excellent choice.

\subsection{Mizar~Light}\label{sec:MizarLight}
Mizar~Light~\cite{MizarLight}, developed by Wiedijk, is a declarative style proof language embedded in HOL~Light as a set of ML combinators, inspired by the primitives of the declarative proof assistant, Mizar~\cite{MizarMathematicalVernacular}. We give an overview of its primitives in Figure \ref{fig:MizarLight}. With the exception of \texttt{\bfseries using}, we have emphasised a declarative semantics: rather than describing \emph{how} each primitive affects the state of the prover, we describe \emph{what} each primitive asserts at a given point in a script.

\begin{figure}
  \begin{tabular}{|l|l|}
    \hline
    Primitive & Meaning \\
    \hline\hline
    \texttt{\bfseries theorem} $term$ & Begins a proof of $term$. \\
    \hline
    \multirow{2}{*}\texttt{\bfseries proof} $proof$ & Asserts $proof$ as a justification\\&for the current step. \\
    \hline
    \multirow{2}{*}\texttt{\bfseries assume} $term$ & Asserts $term$ as a justified \\&assumption at this point. \\
    \hline
    \multirow{2}{*}\texttt{\bfseries so} & Refers to the previous step as\\& justifying the current step.\\
    \hline
    \texttt{\bfseries have} $term$ & Asserts $term$ as derivable at this point. \\
    \hline
    \multirow{2}{*}\texttt{\bfseries thus} $term$ & Asserts $term$ as derivable at which\\&point the (sub)theorem is justified. \\
    \hline
    \texttt{\bfseries hence} $term$ & As \texttt{{\bfseries so thus} $term$} \\
    \hline
    \multirow{2}{*}\texttt{\bfseries take} $var$ & Identifies $var$ as the witness for the \\&(sub)theorem. \\
    \hline
    \multirow{2}{*}\texttt{\bfseries fix} $vars$ & Establishes $vars$ as fixed but \\&arbitrary variables.\\
    \hline
    \multirow{2}{*}\texttt{\bfseries consider} $vars$ \texttt{\bfseries st} $term$ & Introduces $vars$ witnessing \\& $term$. \\
    \hline
    \multirow{2}{*}\texttt{\bfseries from} $steps$ & Refers to proof steps $steps$ as \\&justifications for the current step.\\
    \hline
    \multirow{3}{*}\texttt{\bfseries by} $thms$ & Refers to previously established theorems \\&$thms$ as justifications for the current\\&step. \\
    \hline
    \multirow{2}{*}\texttt{\bfseries using} $tactics$ & Augments the justification of this step\\&with $tactics$.\\
    \hline
    \multirow{2}{*}\texttt{\bfseries per cases} $cases$ & Begins a case-split into $cases$ with their\\&proofs.\\
    \hline
    \multirow{2}{*}\texttt{\bfseries suppose} $term$ & Syntactic sugar to identify the\\&supposition of each $case$. \\
    \hline
    \multirow{3}{*}\texttt{\bfseries otherwise} $proof$ & Indicates that the (sub)theorem $thm$ can\\&be established by $proof$, which derives\\&a contradiction from $\neg thm$. \\
    \hline
    \texttt{\bfseries set} $bindings$ & Introduces local variable bindings.\\
    \hline
    \multirow{2}{*}\texttt{\bfseries qed} & Asserts that the (sub)theorem is justified\\&at this point.\\
    \hline
  \end{tabular}\\
  \caption{An overview of Mizar Light}
  \label{fig:MizarLight}
\end{figure}

As noticed by Harrison~\cite{MizarHOL}, the operational semantics of these proof steps can be given in terms of the existing tactic language. The goal stack becomes the state of the proof system. The goal formula becomes the (sub)theorem we wish to prove and the hypotheses become the intermediate facts derived at each step in the proof. Each declarative proof step is translated into a tactic which is applied to the goal to drive the proof \emph{forward}. For instance, 

\begin{center}
\texttt{{\bfseries consider}} $P$ \texttt{\bfseries st} $\neg$on\_line P a  \texttt{\bfseries by} construct\_triangle
\end{center}

\noindent becomes a tactic which produces the new subgoal \texttt{$\exists$P. $\neg$on\_line P a}. This goal is immediately solved using the step's justification. By default, steps are justified by HOL~Light's generic MESON tactic \cite{HarrisonMESON}, but some of the other step combinators augment the justification. Here, the \texttt{by} step is used to pass \texttt{construct\_triangle} as an additional argument to MESON. The solved goal produces an existential theorem which is fed to an existential elimination tactic, which adds \texttt{$\neg$on\_line P a} is a hypothesis to the main goal.

Wiedijk later realised that, just as with the tactics, these primitives could be implemented as ordinary ML functions. He used this observation to create the Mizar~Light combinator language. Inspired by HOL~Light's design, he made sure the data structures used by this language are public, making the Mizar~Light system highly customisable: adding a new primitive is often as simple as defining a new function, and we discuss some of our own additions in Section \ref{sec:NewPrimitives}.

\subsection{Declarative Interactivity}
Wiedijk's basic combinators are based on the original Mizar system, a batch prover, and so his Mizar~Light proof scripts are written in their entirety and then evaluated in one. We find this undesirable, firstly, because the error reporting is not rich enough to show where errors occur in the case of a failed proof. Secondly, we have chosen to implement the algorithm described in \S\ref{sec:DiscoveryImplementation} to run concurrently with the proof system, where it works best by exploiting the user's idle time during \emph{interactive} rather than \emph{batch} proof.

To illustrate the problem, here are the original combinators at work in an extract of one of Wiedijk's example proofs (the details of which are not important):

\vspace{0.5cm}
\begin{minipage}{\linewidth}
  \footnotesize
  \texttt{...}

  \texttt{have "$\forall$p1 p2. $\exists$l. p1 ON l $\wedge$ p2 ON l" at 9}

  \texttt{proof}

  \texttt{\enspace [fix ["p1:Point"; "p2:Point"];}

  \texttt{\quad per cases}

  \texttt{\quad\enspace[[suppose "p1 = p2";}

  \texttt{\qquad\enspace qed from [0] by [LEMMA1]];}

  \texttt{\qquad [suppose "$\neg$(p1 = p2)";}

  \texttt{\qquad\enspace qed from [1]]]];}

  \texttt{...}
\end{minipage}
\vspace{0.5cm}

Notice firstly that this is a subproof within a larger proof. Notice secondly that it contains two nested subproofs, case-splitting on the propositions \texttt{p1 = p2} and \texttt{$\neg$(p1 = p2)}. Now the steps of each subproof are collected in lists, which makes for a neatly structured proof document, where the proof tree is reflected by Ocaml data-structures. However, the steps of an interactive proof are supposed to be applied \emph{linearly}, one-by-one, traversing the implicit proof tree. Here is what we prefer to write at the top-level (\texttt{>} marks the ML prompt):

\vspace{0.5cm}
\begin{minipage}{\linewidth}
  \footnotesize  
  \texttt{> have "$\forall$p1 p2. $\exists$l. p1 ON l $\wedge$ p2 ON l" at 9}

  \texttt{> proof}

  \texttt{> fix ["p1:Point"; "p2:Point"]}

  \texttt{> per cases}

  \texttt{> suppose "p1 = p2"}

  \texttt{> qed from [0] by [LEMMA1]}

  \texttt{> suppose "$\neg$(p1 = p2)"}

  \texttt{> qed from [1]}
\end{minipage}
\vspace{0.5cm}

The linearisation of this structure could be achieved by rewriting Wiedijk's proof system in a continuation-passing style. However, the development of HOL~Light has emphasised backward-compatibility, and so wherever possible, we wish to build on existing implementation.

\subsubsection{Interactive Subproofs}\label{sec:NewPrimitives}
 We needed an interactive version of the \texttt{proof} step combinator. To that end, we wrote the \texttt{lemma} function, which takes a term to prove and introduces a new subgoal. Our function is not as powerful as the \texttt{proof} \emph{combinator}, which can augment an arbitrary step by asserting a subproof as its justification. For instance, it is possible to write:

\vspace{0.5cm}
\begin{minipage}{10cm}
  \texttt{consider P st $\neg$on\_line P a}

  \texttt{proof}

  \texttt{[ otherwise have "$\forall$b P. on\_line P b"}

  \texttt{\enspace\enspace\enspace hence contradiction by two\_dimensions ]}
\end{minipage}
\vspace{0.5cm}

Here, a subproof is being used to justify the existential subgoal which is created by the \texttt{\bfseries{consider}} step. Our \texttt{lemma} function does not allow this, and becomes a standalone step in its own right. It must therefore take the lemma to be proven as argument. In this way, it replaces the combination:

\vspace{0.5cm}
\begin{minipage}{\linewidth}
  \texttt{have "$\forall$p1 p2. $\exists$l. p1 ON l $\wedge$ p2 ON l" at 9}

  \texttt{proof}

  \texttt{\enspace \ldots}
\end{minipage}
\vspace{0.5cm}

\noindent with

\vspace{0.5cm}
\begin{minipage}{\linewidth}
  \texttt{> lemma "$\forall$p1 p2. $\exists$l. p1 ON l $\wedge$ p2 ON l" at 9}

  \texttt{> \enspace \ldots}
\end{minipage}
\vspace{0.5cm}

\subsubsection{Interactive Case-splits}
Case splits are ultimately justified by proving a disjunction of all considered cases. However, in the Mizar-style of proof and as is common in ordinary mathematical proof, the particular disjunction is never stated explicitly. Consider again the extract of Mizar Light code:

\vspace{0.5cm}
\begin{minipage}{\linewidth}
  \footnotesize
  \texttt{\quad per cases}

  \texttt{\quad\enspace[[suppose "p1 = p2";}

  \texttt{\qquad\enspace qed from [0] by [LEMMA1]];}

  \texttt{\qquad [suppose "$\neg$(p1 = p2)";}

  \texttt{\qquad\enspace qed from [1]]]];}
\end{minipage}
\vspace{0.5cm}

{\samepage Here, there are two cases being considered \texttt{p1 = p2} and \texttt{$\neg$(p1 = p2)}. The disjunction which justifies them as exhaustive
\begin{center}\texttt{p1 = p2 $\vee$ $\neg$(p1 = p2)}\end{center}}

\noindent does not appear in the proof text. Instead, it is assembled by the \texttt{per} combinator from the first element of each subproof. The step then folds a case-splitting tactic over the list of cases, incorporating the tactics generated by their respective subproofs.

The important point here is that the implicit disjunction must be determined before any of the subproof tactics are applied. This is possible, because \texttt{per cases} takes the full list of cases, from which the disjunction can be assembled. But this strategy will not work if we are to linearise the subproofs and apply each step interactively, since the full disjunction will not be known until all cases are interactively solved.

Harrison's original Mizar mode for HOL Light had better support for interactive case-splitting~\cite{MizarHOL}. In his system, the disjunction is assembled at the very end of the proof, when all goals have been solved and a forward proof reconstituted from the tactic justification. The drawback is that the user is only made aware of an unsuccessful case-split at the very end of the proof.

To overcome this drawback, we have implemented two functions \texttt{case} and \texttt{end}. The \texttt{case} function is used to introduce a new case term $\phi$. It then generates two subgoals, the first with $\phi$ as hypothesis, and the second with $\neg\phi$ as hypothesis. The \texttt{case} step, therefore, has performed a case-split on $\phi\vee\neg\phi$, and then invites the user to prove the goal on the hypothesis of $\phi$. Once the goal is solved, the one remaining goal will have $\neg\phi$ as its hypothesis. 

The user now proceeds by introducing the \emph{next} case, using the \texttt{case} function again with a new term, say $\psi$. Two subgoals are again generated, one with $\psi$ and the other with $\neg\psi$ as hypothesis.

By the time the user has considered and proven all cases, the one remaining subgoal will have the negations of every considered case in its hypotheses. If the cases are exhaustive, these will entail a contradiction\footnote{We assume we are only interested in \emph{classical} proofs. Otherwise, this does not necessarily follow.}. This is where the \texttt{end} step is used. It will automatically take all the negated cases, identifying them by a case-label \texttt{C} in the goal-stack, and use them as a justification for $\bot$.

Suppose, for example, that we have a theorem $\phi \longrightarrow P \vee Q \vee R$ and suppose that each of $P$, $Q$ and $R$ can solve a goal $G$ using the implicit automation built into Mizar~Light. Then we can write the proof:

\vspace{0.5cm}
\begin{minipage}{\linewidth}
  \footnotesize
  \texttt{> theorem "$G$"}

  \texttt{> case "$P$" }

  \texttt{>\quad qed}

  \texttt{> case "$Q$"}

  \texttt{>\quad qed}

  \texttt{> case "$R$"}

  \texttt{> end by $\phi$}
\end{minipage}
\vspace{0.5cm} 

The resulting tree of goal-stacks is depicted in the following figure.

% \vspace{0.5cm}
%   \Tree{
%     & \sgoal{$G$}{}{}
%     \Bk{-0.5}{2.5}{dl}_{\texttt{case}} \Bk{-0.5}{2.5}{dr}& & &\\
%     \sgoal{$G$}{}{$P$}\Bk{-0.5}{2.5}{d} & & \sgoal{$G$}{C}{$\neg P$} 
%     \Bk{-0.5}{4.5}{dl}_{\texttt{case}} \Bk{-0.5}{4.5}{dr}& &\\
%     {\texttt{qed}}&\sgoaltwo{$G$}{C}{$\neg P$}{}{$Q$}\Bk{-2.5}{2.5}{d}& &
%     \sgoaltwo{$G$}{C}{$\neg P$}{C}{$\neg Q$}
%     \Bk{-2.5}{6.5}{dl}_{\texttt{case}} \Bk{-2.5}{6.5}{dr} &\\
%     &{\texttt{qed}}&\sgoalthree{$G$}{C}{$\neg P$}{C}{$\neg Q$}{}{$R$}\Bk{-4.5}{2.5}{d}&&
%     \sgoalthree{$G$}{C}{$\neg P$}{C}{$\neg Q$}{C}{$\neg R$}\Bk{-4.5}{2.5}{d}\\
%     &&{\texttt{qed}}&&{\texttt{end}}}
% \vspace{0.5cm}

The final \texttt{end} step is justified since 

\begin{displaymath} 
P \vee Q \vee R, \neg P, \neg Q, \neg R \vdash \bot
\end{displaymath}

This approach does not have the drawback of Harrison's solution. Whenever the case splitting is not exhaustive, the \texttt{end} step will fail at the point at which it is evaluated, rather than at the very end of the proof when the final justification is assembled. 

For the sake of completeness, we return to our example proof. Our functions \texttt{case} and \texttt{end} allow us to write

\vspace{0.5cm}
\begin{minipage}{\linewidth}
  \footnotesize
  \texttt{> lemma "$\forall$p1 p2. $\exists$l. p1 ON l $\wedge$ p2 ON l" at 9}

  \texttt{>\quad fix ["p1:Point"; "p2:Point"]}

  \texttt{>\quad case "p1 = p2" }

  \texttt{>\qquad qed from [0] by [LEMMA\_1] }

  \texttt{>\quad case "$\neg$(p1 = p2)" }

  \texttt{>\qquad qed from [1]}

  \texttt{>\quad end}
\end{minipage}
\vspace{0.5cm}

for the the proof tree:

% \vspace{0.5cm}
% \Tree{
% & \goal{$\forall$p1 p2.$\exists$l.p1 ON}{\quad$\wedge$ p2 ON l}{$\vdots$}{$\vdots$}
%   \Bk{-3.5}{5.5}{d}^{\texttt{fix}} & & \\
% & \goal{$\exists$l.p1 ON l}{$\wedge$ p2 ON l}{$\vdots$}{$\vdots$}
%   \Bk{-3.5}{7.5}{dl}_{\texttt{case}} \Bk{-3.5}{7.5}{dr} & &\\
%   \goaltwo{$\exists$l.p1 ON l}{$\wedge$ p2 ON l}{}{p1 = p2}{}{$\vdots$}
%   \Bk{-5.5}{0}{d} & &
%   \goaltwo{$\exists$l.p1 ON l}{$\wedge$ p2 ON l}{C}{$\neg$p1 = p2}{}{$\vdots$}
%   \Bk{-5.5}{9.5}{dl}_{\texttt{case}} \Bk{-5.5}{9.5}{dr}& \\
%   {\text{$\vdots$}}\Bk{-2.0}{2.0}{d} & 
%   \goalthree{$\exists$l.p1 ON l}{$\wedge$ p2 ON l}{C}{$\neg$p1 = p2}{}{$\neg$p1 = p2}{}{$\vdots$}
%   \Bk{-7.5}{0}{d}
%   & & 
%   \goalthree{$\exists$l.p1 ON l}{$\wedge$ p2 ON l}{C}{$\neg$p1 = p2}{C}{$\neg\neg$p1 = p2}{}{$\vdots$}\Bk{-7.5}{0}{d}\\
%   {\texttt{qed}} & {\text{$\vdots$}}\Bk{-2.0}{2.0}{d} & & {\texttt{end}}\\
%   & {\texttt{qed}} & &}
% \vspace{0.5cm}


\chapter{Axiomatics}\label{chapter:Axiomatics}
Hilbert has only the briefest introduction to the \emph{Grundlagen der Geometrie}, before diving in with a declaration of his primitive notions and then laying out his five groups of axioms. In this chapter, we discuss the formalisation of the first two groups, a very weak axiomatic environment under which it is nevertheless possible to verify the Polygonal Jordan Curve Theorem. We also discuss the verification of a few of the elementary theorems, and how the first group of axioms in particular feature in the rest of our verifications.

\section{Primitives}
Hilbert opens his axiomatics by declaring three sets of primitive objects: a set of objects called \emph{points}, a set of objects called \emph{lines} and a set of objects called \emph{planes}. These sets are \emph{abstract}. All we can know about their inhabitants is what is specified by Hilbert's axioms.

That we call these abstract objects \emph{points}, \emph{lines} and \emph{planes} can be thought of as mere documentation. It has no real significance to the formal theory, and this lack of significance is something that Hilbert and Pasch regarded as fundamental to rigour in geometry~\cite{TableChairMug}. As Hilbert was known to remark, it would serve just as well to call the inhabitants of the three sets ``mugs'', ``tables'' and ``chairs''~\cite{PaschToPeano}.

This is the modern axiomatic method, and it is a noble sentiment if we hold rigour in such high esteem. But it is one thing to say that it is possible to run the substitution and quite another to carry it out. We will not be evaluating the matter, but we would conjecture that it would be very difficult for a human to follow the steps of Hilbert's arguments if they were literally rendered in terms of mugs, tables and chairs.

This might explain the labour involved with verified mathematics. Our computers carry out Pasch's idea of stripping away all interpretation. They might as well be reading about mugs, tables and chairs. They see nothing but abstract symbols, and must validate the arguments without the help of intuition. If this is too much for a human, then it is quite something to expect of a machine.

Having introduced points, lines and planes, Hilbert goes on to declare that ``[t]he points are also called the \emph{elements of line geometry}; the points and the lines are called the \emph{elements of plane geometry}; and the points, lines and planes are called the \emph{elements of space geometry} or the \emph{elements of space}.'' These comments do not appear to have any importance to our verification, and we are happy to ignore them, and all others like them, without worrying about jeopardising our aims of following Hilbert to the letter. Even when they take the form of definitions, we treat them as mere \emph{documentation}, useful signposts for readers, but no more.

In general, we only introduce formal definitions when they identify abstractions we intend to use in verifications.

\section{Group~I}
\subsection{Incidence Relations}
Following the abstract sets of points, lines and planes, Hilbert introduces a primitive  relation \emph{lie}, whose axioms are intended to characterise it as an incidence relation. With it, we can say that a point \emph{lies} on a line, or that a point \emph{lies} on a plane.

In a foreword to the text, Professor Goheen says that there must, in fact, be \emph{two} relations. It is not clear to us why. Perhaps Goheen is taking his perspective from first-order logic. In this case, the sets would most naturally be represented by distinct and disjoint \emph{sorts}, and thus, we would need two \emph{lie} relations, one for each sort.

There is often an implicit assumption on sorts, namely that they are inhabited. This assumption is removed in \emph{free-logics}, usually for philosophical concerns, and often at the expense of breaking standard inference rules (see Mendelson's classic text~\cite{Mendelson}).

Whether or not Hilbert was making the assumption that his sets were inhabited is unclear. Fortunately, we do not need to be too concerned, since the assumption is not needed. To formally settle this, we consider a formalisation that Goheen perhaps neglected: we will represent each of Hilbert's three sets by a predicate, and consider relativising Hilbert's axioms to these predicates. This is, in effect, the embedding of a free-logic in classical logic. It has the additional benefit of allowing us to consider just one primitive incidence relation on one primitive sort.

Before we give our formalisation, we mention that we shall be adopting Hilbert's convention throughout this work, that points are denoted by uppercase Roman, $A$, $B$, $C$, $P$, $Q$, $R$, $X$, $Y$, $Z$, and so on. Lines are denoted by lowercase Roman $a$, $b$, $c$. And planes are denoted by Greek $\alpha$, $\beta$, $\gamma$.

The formalisation of this single-sorted geometry is given in Figure~\ref{fig:InhabitedTypes}. We have formalised four of Hilbert's incidence axioms (\ref{eq:g11}, \ref{eq:g13b}, \ref{eq:g14a} and \ref{eq:g18}) as conditions on the predicate sets \code{point}, \code{line} and \code{plane} and the single relation \code{lie}. These predicates are polymorphic of a single type variable $\tau$, which formalises the single sort for all our geometric entities. We have verified that any four objects satisfying these conditions are such that the three predicate sets are inhabited.

\begin{figure}
  \begin{align*}
    &\vdash\code{Group1}\ (\code{point}\;:\;\tau\rightarrow\code{bool}, \code{line}\;:\;\tau\rightarrow\code{bool},\code{plane}\;:\;\tau\rightarrow\code{bool},\\
    &\qquad\qquad\code{lie}\;:\;\tau\rightarrow\tau\rightarrow\code{bool})\\
    &\iff (\forall A.\;\forall B.\; \code{point}\ A \wedge \code{point}\ B \wedge A \neq B \implies \exists a.\; \code{line}\ a \wedge \code{lie}\ A\ a \wedge \code{lie}\ B\ a)\\
    &\qquad\wedge (\exists A.\;\exists B.\;\exists\ C.\; \code{point}\ A \wedge \code{point}\ B \wedge \code{point}\ C\\
    &\qquad\qquad\wedge \forall a.\; \code{line}\ a \implies \neg(\code{lie}\ A\ a \wedge \code{lie}\ B\ a \wedge \code{lie}\ C\ a))\\
    &\qquad\wedge(\forall A.\;\forall B.\;\forall C.\; \code{point}\ A \wedge \code{point}\ B \wedge \code{point}\ C\\
    &\qquad\qquad\wedge (\forall a.\; \code{line}\ a \implies \neg(\code{lie}\ A\ a \wedge \code{lie}\ B\ a \wedge \code{lie}\ C\ a))\\
    &\qquad\qquad\implies\exists\alpha.\; \code{plane}\ \alpha \wedge \code{lie}\ A\ \alpha \wedge \code{lie}\ B\ \alpha \wedge \code{lie}\ C\ \alpha)\\
    &\qquad (\exists A.\;\exists B.\;\exists C.\;\exists D.\; \code{point}\ A \wedge \code{point}\ B \wedge \code{point}\ C \wedge \code{point}\ D\\
    &\qquad\qquad \wedge \forall \alpha.\; \code{plane}\ \alpha \implies \neg(\code{lie}\ A\ \alpha \wedge \code{lie}\ B\ \alpha \wedge \code{lie}\ C\ \alpha \wedge \code{lie}\ D\ \alpha)\\
  \end{align*}
  \begin{displaymath}
    \vdash \code{Group1}\ \code{point}\ \code{line}\ \code{plane}\ \code{lie} \implies (\exists A.\;\exists a.\;\exists \alpha.\; \code{point}\ A \wedge \code{line}\ a \wedge \code{plane}\ \alpha)
  \end{displaymath}
\caption{Points, lines and planes exist}
\label{fig:InhabitedTypes}
\end{figure}

\subsection{Axioms and Formalisation}
With the verification showing that Hilbert's primitive sets must be inhabited, we are free to use what we regard as a more natural formalisation of Hilbert's axioms, the one provided independently by Meikle and Fleuriot's~\cite{MeikleFleuriotFormalizingHilbert} and Dehlinger et al~\cite{DehlingerFOG}. We declare three primitive types for points, lines and planes. The implicit constraint that these types are inhabited is permissible based on the verification of the previous subsection, and we can therefore take it as faithful to Hilbert's intended interpretation.

We then consider two incidence relations: one tells us whether points lie on a line and the other whether points lie on a plane. This gives a more readable formalisation than that of Figure~\ref{fig:InhabitedTypes}, since we can drop the relativising predicates. It also improves type-safety: HOL~Light can reject axioms which do not use the primitive relations in sensible ways, and it removes the possibility of nonsense expressions such as ``a plane lies on a point.''

We now give Hilbert's incidence axioms as they appear in the second edition of the \emph{Foundations of Geometry}~\cite{FoundationsOfGeometry}, translated from the tenth edition of the \emph{Grundlagen der Geometrie}.
\begin{quotation}
\mbox{}\vspace{-\baselineskip}\begin{enumerate}
\item[I, 1] \emph{For every two points $A$, $B$ there exists a line $a$ that contains each of the points $A$, $B$.}
\item[I, 2] \emph{For every two points $A$, $B$ there exits [sic] no more than one line that contains each of the points $A$, $B$.}
\item[I, 3] \emph{There exist at least two points on a line. There exist at least three points that do not lie on a line.}
\item[I, 4] \emph{For any three points $A$, $B$, $C$ that do not lie on the same line there exits [sic] a plane $\alpha$ that contains each of the points $A$, $B$, $C$. For every plane there exists a point which it contains.}
\item[I, 5] \emph{For any three points $A$, $B$, $C$ that do not lie on one and the same line there exists no more than one plane that contains each of the three points $A$, $B$, $C$.}
\item[I, 6] \emph{If two points $A$, $B$ of a line $a$ lie in a plane $\alpha$ then every point of $a$ lies in the plane $\alpha$.}
\item[I, 7] \emph{If two planes $\alpha$, $\beta$ have a point $A$ in common then they have at least one more point $B$ in common.}
\item[I, 8] \emph{There exist at least four points which do not lie in a plane.}
\end{enumerate}
\flushright(page 4)
\end{quotation}

These axioms have undergone substantial revision since the first edition, being reordered, with some combined, some split and redundancies deleted. The fact that there are redundancies in the first place goes to show that Hilbert's later claim that his axioms are independent was never \emph{fully} investigated in the \emph{Grundlagen der Geometrie}. In fact, only a few interdependencies were ever considered.

The axioms are formalised in Figure~\ref{fig:Group1Axioms} and are asserted in HOL~Light. We just give some supplementary discussion.

\begin{figure}
\begin{equation}\label{eq:g11}
  \tag{I,~1}
    \vdash A \neq B \implies \exists a.\; \code{on\_line}\ A\ a \wedge \code{on\_line}\ B\ a
\end{equation}
\begin{equation}\label{eq:g12}
  \tag{I,~2}
  \begin{split}
    \vdash A \neq B &\wedge \code{on\_line}\ A\ a \wedge \code{on\_line}\ B\ a\\
    &\wedge \code{on\_line}\ A\ b \wedge \code{on\_line}\ B\ b\\
    &\implies a = b
  \end{split}
\end{equation}
\begin{equation}\label{eq:g13a}
  \tag{I,~3.1}
  \vdash \exists A.\;\exists B.\; A \neq B \wedge \code{on\_line}\ A\ a \wedge \code{on\_line}\ B\ a
\end{equation}
\begin{equation}\label{eq:g13b}  \tag{I,~3.2}
  \vdash\exists A.\;\exists B.\;\exists C.\; \Triangle{a}{A}{B}{C}
\end{equation}
\begin{multline}\label{eq:g14a}
  \tag{I,~4.1}
  \vdash\Triangle{a}{A}{B}{C}\\\quad \implies \exists \alpha.\; \code{on\_plane}\ A\ \alpha \wedge \code{on\_plane}\ B\ \alpha \wedge \code{on\_plane}\ C\ \alpha
\end{multline}
\begin{equation}\label{eq:g14b}
  \tag{I,~4.2}
  \vdash\exists A.\; \code{on\_plane}\ A\ \alpha
\end{equation}
\begin{equation}\label{eq:g15}
  \tag{I,~5}
  \begin{split}
    &\vdash\Triangle{a}{A}{B}{C}\\
    &\wedge \code{on\_plane}\ A\ \alpha \wedge \code{on\_plane}\ B\ \alpha \wedge \code{on\_plane}\ C\ \alpha\\
    &\wedge \code{on\_plane}\ A\ \beta \wedge \code{on\_plane}\ B\ \beta \wedge \code{on\_plane}\ C\ \beta\\
    &\implies \alpha = \beta
  \end{split}
\end{equation}
\begin{equation}\label{eq:g16}
  \tag{I,~6}
  \begin{split}
    \vdash A \neq B &\wedge \code{on\_plane}\ A\ \alpha \wedge \code{on\_plane}\ B\ \alpha\\
    &\wedge \code{on\_line}\ A\ a \wedge \code{on\_line}\ B\ a\\
    &\implies \code{on\_line}\ P\ a \implies \code{on\_plane}\ P\ \alpha
  \end{split}
\end{equation}
\begin{multline}
\label{eq:g17}
  \tag{I,~7}
   \vdash\alpha \neq \beta \wedge \code{on\_plane}\ A\ \alpha \wedge \code{on\_plane}\ A\ \beta\\
   \implies \exists B.\; A \neq B \wedge \code{on\_plane}\ B\ \alpha \wedge \code{on\_plane}\ B\ \beta
\end{multline}
\begin{multline}  
\label{eq:g18}
  \tag{I,~8}
  \vdash\exists A.\;\exists B.\;\exists C.\;\exists D.\;\\
  \neg \exists \alpha.\; (\code{on\_plane}\ A\ \alpha \wedge \code{on\_plane}\ B\ \alpha \wedge \code{on\_plane}\ C\ \alpha \wedge \code{on\_plane}\ D\ \alpha)
\end{multline}
\caption{Group~I axioms}
\label{fig:Group1Axioms}
\end{figure}

Hilbert's Axioms I, 3 and I, 4 each contain two distinct claims. We have not identified any interesting logical connection between these, and so we have split them in our formalisation into Axiom~\ref{eq:g13a}, \ref{eq:g13b}, \ref{eq:g14a} and \ref{eq:g14b}, giving a total of 10 axioms.

Axioms~\ref{eq:g11} and~\ref{eq:g12}, which were a single axiom in the first edition of the \emph{Grundlagen der Geometrie}, require that two points uniquely determine a line. Analogous axioms for planes are given by~\ref{eq:g14a} and \ref{eq:g15}. The converse, namely that a line is determined by two points, appears with the addition of Axiom~\ref{eq:g13a}. The converse for planes, that a plane is determined by three non-collinear points, was an axiom of the first-edition, but was later weakened to assert only that a plane contains at least one point in Axiom~\ref{eq:g13b}. Hilbert must have been aware that the former could be derived from the latter when he removed the redundancy, but a statement of this fact and the proof are both absent from the text. We present our own proof and verification in \S\ref{sec:PlaneThree}.

\label{sec:DanglingPoints}
We should note that there is some ambiguity in Axiom~\ref{eq:g13a}. Is Hilbert saying that there are two points and some line such that the points lie on that line, or is he saying more generally that \emph{every} line contains two points? We took the latter view, since on the weaker interpretation, we could formalise a model of the first group of axioms following the technique in \S\ref{sec:FiniteModel}, and show in this model that there is a line with \emph{no} incident points. We assume Hilbert did not intend this.

Next, we have Axiom~\ref{eq:g13b}. This is a \emph{dimension} axiom, requiring that the geometry has at least dimension 2. An analogous axiom is the last axiom \eqref{eq:g18}, which requires that the geometry has at least dimension three. We will have very little need for this last axiom, since almost all of Hilbert's proofs are basically planar.

Finally, we have the Axioms~\ref{eq:g16} and~\ref{eq:g17}. Together, these require that intersecting planes meet in a line, and thus, they restrict the dimension of the geometry to~3.

It is important to note that Hilbert adopts the uncommon convention that when he writes expressions such as ``two points'', ``three points'', or ``two lines'', ``three lines'', he is assuming that the points and lines in question are distinct\label{sec:DistinctVars}. For this reason, a number of explicit distinctness assumptions appear in our formalisation which are only implicit in the prose. Some of these distinctness assumptions can actually be dropped, such as in Axiom~I; there is a line through the points $A$ and $B$ whether or not $A$ and $B$ are distinct. We keep the weaker form as the axiom, and verify the stronger version.
\begin{displaymath}
  \vdash \exists a.\; \code{on\_line}\ A\ a \wedge \code{on\_line}\ B\ a.
\end{displaymath}

\subsection{Related Axiomatisations}
Hilbert's axiomatisation appears within a culture of related attempts to rigorise geometry. Oswald Veblen's doctoral work \cite{Veblenphd} is perhaps the most closely related, and it is clear that ideas developed by Veblen and his supervisor E. H. Moore filtered into later editions of Hilbert's text.

Veblen differs significantly from Hilbert by following a trend he identifies with Pasch and Peano. Here, the fundamental primitives of geometry are just points and the relation of \emph{betweenness}. Lines, planes and incidence are no longer primitive, but are instead derived concepts.

A similar approach was taken up by Tarksi, who developed the first formal system for elementary geometry with the benefit of modern formal logic \cite{TarskiGeometrySystem}. Like Veblen, Tarski used only one primitive sort for points, but unlike Veblen, he admitted a congruence relation on pairs of points. Nevertheless, Tarski's axioms are particularly elegant. They do not appeal to complex derived notions as Hilbert's later axioms do, and his dimension axiom has the pleasing property that it can be mechanically modified to axiomatise an arbitrary dimension.

There are mechanisations of Tarski's geometry in the Otter automated prover~\cite{QuaifeTarski} and Coq proof assistant~\cite{NarbouxTarski}. Tarski's axioms are known to be far more primitive than Hilbert's, and thus it takes much more work to carry out even simple verifications in this system. In fact, it takes significant effort just to recover Hilbert's axioms from Tarski's~\cite{NarbouxTarskiHilbert}. That said, Tarski's theory embeds in the theory of real-closed fields where it admits quantifier elimination, and thus the theory is \emph{decidable}. The axioms are still very \emph{hard} to work with. In fact, the decision procedure is doubly-exponential!~\cite{QuantifierEliminationComplexity}

\subsection{Elementary Consequences}
Hilbert highlights two results for his first group.

\begin{quotation}
  THEOREM~1. Two lines in a plane either have one point in common or none at all. Two planes have no point in common, or have one line and otherwise no other point in common. A plane and a line that does not lie in it either have one point in common or none at all.~\footnote{We need classical logic already here. The very first clause of THEOREM~1 assumes that point equality is decidable. See Dehlinger et al~\cite{DehlingerFOG}.}

  THEOREM 2. Through a line and a point that does not lie on it, as well as through two distinct lines with one point in common, there always exists one and only one plane.
\flushright{\emph{Foundations of Geometry}~\cite{FoundationsOfGeometry} (page 4)}
\end{quotation}

The proofs are straightforward. In fact, the first and last clauses in THEOREM~1 are barely rewordings of Axiom~\ref{eq:g12} and Axiom~\ref{eq:g16}, and so we did not bother to formalise them.

The middle clause, that two planes have either no point in common or otherwise one line and no other point in common is a little more involved. We start by giving it a tidier rephrasing which makes it easier to formalise: two planes with a point in common intersect exactly in some line.
\begin{multline*}
  \vdash \alpha \neq \beta \wedge \code{on\_plane}\ P\ \alpha \wedge \code{on\_plane}\ P\ \beta\\
  \implies (\exists a.\; \forall Q.\; \code{on\_plane}\ Q\ \alpha \wedge \code{on\_plane}\ Q\ \beta \iff \code{on\_line}\ Q\ a).
\end{multline*}

The verification of this theorem, and the first piece of Mizar~Light that we present, is given in Figure~\ref{fig:Theorem1}. We encourage the reader to inspect these verifications, and we hope they agree with us that they are pleasantly readable and easy to follow. 

THEOREM~2, split into two separate propositions, is easily verified. The verifications depend on Axioms~\ref{eq:g12},~\ref{eq:g13a},~\ref{eq:g14a},~\ref{eq:g15} and~\ref{eq:g16}.
\begin{displaymath}
  \vdash\neg\code{on\_line}\ P\ a \implies \exists!\alpha.\; \code{on\_plane}\ P\ \alpha \wedge \forall Q.\; \code{on\_line}\ Q\ a \implies \code{on\_plane}\ Q\ \alpha.
\end{displaymath}
\begin{multline*}
  \vdash a\neq b\wedge\code{on\_line}\ P\ a\wedge\code{on\_line}\ P\ b\\
  \implies\exists!\alpha.\; \forall P.\; \code{on\_line}\ P\ a \vee \code{on\_line}\ P\ b \implies\code{on\_plane}\ Q\ \alpha.
\end{multline*}

Note that the symbol $\exists!$ is another defined quantifier in HOL~Light. It asserts of its argument --- a predicate --- that it is satisfied \emph{uniquely}, and it can be defined thus:

\begin{displaymath}
  \vdash_{def}(\exists! P) = (\exists P) \wedge (\forall x.\;\forall y.\; P\ x \wedge P\ y \implies x = y)
\end{displaymath}

\begin{boxedfigure}
\small
  \begin{align*}
    &\code{assume}\ \alpha\neq\beta \wedge \code{on\_plane}\ P\ \alpha\wedge\code{on\_plane}\ P\ \beta & 1\\
    &\code{so consider}\ Q\ \code{such that}\ P\neq Q \wedge\code{on\_plane}\ Q\ \alpha \wedge\code{on\_plane}\ Q\ \beta\ \code{by}\ \eqref{eq:g17} & 2\\
    &\code{so consider}\ a\ \code{such that}\ \code{on\_line}\ P\ a\wedge\code{on\_line}\ Q\ a\ \code{by}\ \eqref{eq:g11} & 3\\
    &\code{take}\ a\\
    &\code{fix}\ R\\
    &\code{have}\ \code{on\_plane}\ R\ \alpha\wedge\code{on\_plane}\ R\ \beta \implies \code{on\_line}\ R\ a\\
    &\code{proof}\\
    &\qquad\code{assume}\ \code{on\_plane}\ R\ \alpha\wedge\code{on\_plane}\ R\ \beta & 4\\
    &\qquad\code{otherwise assume}\ \neg\code{on\_line}\ R\ a & 5\\
    &\qquad\code{hence}\ \Triangle{a}{P}{Q}{R}\ \code{from}\ 2,3\ \code{by}\ \eqref{eq:g12}\\
    &\qquad\code{qed from}\ 1,2,4\ \code{by}\ \eqref{eq:g15}\\
    &\code{qed from}\ 1,2,3\ \code{by}\ \eqref{eq:g16}
  \end{align*}
  \caption{Intersecting planes intersect in a line}
  \label{fig:Theorem1}
\end{boxedfigure}

We will not reproduce the verifications here, since our later verifications have no need of these theorems, and nor does Hilbert refer to THEOREM~2 again. Instead, all our verifications which appeal to incidence will be based on an alternative formulation and some formal theory which we explain in \S\ref{sec:PointSets}.

The only other theorem we need is one we use when developing our formal theory of half-planes in Chapter~\ref{chapter:HalfPlanes}. It verifies that every plane contains a non-collinear triple, an axiom of the first edition but a result that now needs to be proven. We regard it as an oversight of Hilbert's that he neglected to mention the result, let alone give its proof. It is not entirely trivial, having us consider a configuration of six points and three planes. We give a prose proof now.

\label{sec:PlaneThree}
\begin{proposition}\label{eq:PlaneThree}
Every plane $\alpha$ contains at least three non-collinear points.
\end{proposition}
\begin{proof}
By Axiom~\ref{eq:g14b} and Axiom~\ref{eq:g18}, we can take a point $A$ on $\alpha$ and a point $B$ not on $\alpha$. We connect these two points by a line $a$. By Axiom~\ref{eq:g13b}, we can take a third point $C$ off the line $a$. The three points $A$, $B$ and $C$ must determine a plane $\beta$ by Axiom~\ref{eq:g14a}.

Since the planes $\alpha$ and $\beta$ intersect at the point $A$, we can choose another intersection point $D$ by Axiom~\ref{eq:g17}, and by Axiom~\ref{eq:g18}, we can find a fifth point $E$ off the plane $\beta$. Now $A$, $B$ and $E$ must be non-collinear, and so they determine a plane $\gamma$. If this plane is $\alpha$, then $A$, $B$ and $E$ are our three points and we are done. Otherwise, we have two distinct planes intersecting at $A$ and so we can take another intersection point $F$ by Axiom~\ref{eq:g17}. This gives us three non-collinear points in $\alpha$, namely $A$, $D$ and $F$. See Figure~\ref{fig:PlaneThree}.
\begin{multline*}
\vdash\exists A.\;\exists B.\;\exists C.\; \code{on\_plane}\ A\ \alpha \wedge \code{on\_plane}\ B\ \alpha \wedge \code{on\_plane}\ C\ \alpha\\
\wedge\Triangle{a}{A}{B}{C}.
\end{multline*}
\end{proof}
\begin{figure}
\centering\includegraphics[scale=0.8]{axioms/PlaneThree}
\caption{Planes contain non-collinear points}
\label{fig:PlaneThree}
\end{figure}

This is the only three-dimensional theorem we will consider in the present work, and as such, it is the only theorem which depends on Axiom~\ref{eq:g18}.

\subsection{Absent Arguments}
After splitting conjunctions, we found that there were ten axioms in Hilbert's first group. This is twice as many axioms as the next largest group, Group~III. One would expect, then, that these axioms would feature the most in proofs. This is indeed the case with formal verifications, but in Hilbert's prose, the axioms are almost never cited.

This appears to directly contradict Weyl, who claimed that the deductions in Hilbert's geometry contain no gaps~\cite{TableChairMug}. Indeed, taking his claim at face-value, we would have to conclude with Meikle and Fleuriot~\cite{MeikleFleuriotFormalizingHilbert} that Hilbert's arguments are full of missing assumptions and lemmas. 

We do not believe Hilbert ever made this claim, given that he is happy to elide whole proofs. The only standard we hold Hilbert to is that his deductions are logical consequences of previous ones, and that he has made a reasonable balance in his presentation, carefully elucidating only the particularly tricky proofs. So far, we have confirmed that all of his deductions are indeed valid. It is less clear whether his presentation is balanced, as we shall discuss over the remaining chapters. 

What is clear, though, is that there are many weaknesses in Hilbert's presentation when viewed as a guide for mechanical verification. Hilbert's decisions to cite axioms are sometimes erratic (see \S\ref{sec:g12Erratic}), and he includes proofs which are easily verified while omitting others he claims to be easily provable when they are a challenge to verify.

There is one axiom which Hilbert \emph{is} careful to cite. This is Axiom~I, 3 (or more precisely in our formalisation, Axiom~\ref{eq:g13b}). We might suppose Hilbert cites this axiom because it is usually used to introduce points. In an informal geometry proof, where the goal is to obtain a geometric figure, one would not want to leave this sort of introduction step implicit.

\label{sec:PlanarProofs}For many of the other axioms whose citations are missing, we might still excuse Hilbert by noting again that almost all of his proofs are \emph{planar}. In effect, Hilbert makes a pervasive ``without-loss-of-generality'' assumption, which is justified because the axioms he invokes will always force the objects considered to lie in the same plane. This makes his life substantially easier, since it means he is effectively working with just Axioms~\ref{eq:g11}, \ref{eq:g12}, \ref{eq:g13a} and \ref{eq:g13b}.

It would have made our verification effort somewhat easier had we been able to make this same without-loss-of-generality assumption. A simple idea would have been to develop a purely planar theory of geometry which could then be embedded in the individual planes of a space geometry. Unfortunately, we knew of no way to do this using HOL~Light's simple type theory. Theory embeddings which depend on particular planes would suggest we need at least a dependently typed logic such as Coq's~\cite{Coq}.

%One last reason we might let Hilbert off the hook is by simply claiming that his missing lemmas and missing citations are just \emph{trivial}. If the missing detail happens to end up dominating a formal verification of Hilbert's geometry, then so much for formal verification. It just confirms Poincar\'{e}'s frank dismissal of the relevance of formal verification to mathematicians that appear in both his review of the \emph{Grundlagen der Geometrie} and his review of Russell and Whitehead's \emph{Principia Mathematicia}~\cite{PoincareReview,PoincareShackles}.

\subsection{Point sets}
In our first attempts to verify Hilbert's first few groups, using just the stock automation available to our theorem prover, we found that many steps were needed that did not appear explicitly in the prose. These steps almost always concerned incidence, and were rarely enlightening. The fully explicated verifications were difficult to read, and could not be easily used to comment on Hilbert's presentation, to identify dependencies, redundancies, circularities, missing details, or alternative proof strategies.

Unsatisfied by this, we tried to improve the situation by reformulating statements of incidence in a way which was more expressive than the basic primitives.\footnote{Note that we do not change any axioms. We simply define our new formulation in terms of the old.} Realising that the domain of incidence reasoning is inherently combinatorial, we opted to formulate incidence claims in terms of point sets. One advantage of point sets is that they at least have the possibility of being composed via basic operations on sets.

Thus, we defined two predicates:
\begin{align*}
&\code{collinear}\;:\;(\code{point}\rightarrow\code{bool})\rightarrow\code{bool}
\\\vdash_{def}\;&\code{collinear\ Ps} \iff \exists a.\; \forall P.\; P \in Ps \implies \code{on\_line}\ P\ a.
\end{align*}
\begin{align*}
&\code{planar}\;:\;(\code{point}\rightarrow\code{bool})\rightarrow\code{bool}
\\\vdash_{def}\;&\code{planar\ Ps} \iff \exists \alpha.\; \forall P.\; P \in Ps \implies \code{on\_plane}\ P\ \alpha.
\end{align*}

%Our use of sets here does not depend on the axiom of infinity, since it might still be that only finite sets exists. The use of sets is really only giving us a way, at the object level, to collapse several formulas concerning the incidence of individual points to one formula concerning a collinear or planar point set, while abstracting away the particular lines and planes with which the points are incident. %If there are still foundational concerns here, we can reassure the reader that, in the next chapter, we will be move this use of set-theory to the meta-level.

We now refer to lines and planes by using the points which uniquely identify them. So if two distinct points $A$ and $B$ lie on a line $a$, we are able to formalise a statement that two other points $C$ and $D$ also lie on $a$ by writing
\begin{displaymath}
\code{collinear}\ \{A, B, C, D\}.
\end{displaymath}

Furthermore, we can formalise a claim that another point $E$ does \emph{not} lie on the line $a$ with $\neg \code{collinear}\ \{A, B, E\}$. Notice that we only need to use three points in this formula, and that adding more points only weakens it (every non-collinear set has a three-point non-collinear subset). This means that formulas asserting non-incidence now assert the existence of \emph{triangles}.

The real advantage to be gained by using collinear and planar sets is that we were able to capture the logic of incidence reasoning in terms of individual composition theorems for sets. Our early investigations into Hilbert's verification~\cite{ScottMScThesis} indicated that these theorems could take on the bulk of the incidence reasoning needed to verify Hilbert's proofs.

\subsubsection{Incidence Reasoning with Point Sets}\label{sec:PointSets}
In Figure~\ref{fig:PointSets}, we give a set of verified theorems for reasoning with collinear and planar sets. Note that we have assumed that singleton sets are both collinear and planar, and that the empty set is assumed to be the smallest collinear and planar set.

\begin{figure}
\begin{align}
&\vdash\code{collinear}\ \, \{A,B\} \label{rule:colltwo}\\
&\vdash S \subseteq T \wedge \code{collinear}\ \, T \implies \code{collinear}\ \, S
  \label{rule:collsubset}\\
&\vdash A \neq B \wedge A,B \in S,T \implies \code{collinear}\ \, S \wedge \code{collinear}\ \, T \implies \code{collinear}\ \, (S \cup T) \label{rule:collunion}\\
&\vdash\code{planar}\ \, \{A,B,C\} \label{rule:planethree}\\
&\vdash S \subseteq T \wedge \code{planar}\ \, T \implies \code{planar}\ \, S
  \label{rule:planesubset}\\
&\vdash\neg\code{collinear}\ \, (S \cap T) \wedge \code{planar}\ \, S \wedge \code{planar}\ \, T \implies \code{planar}\ \, (S \cup T) \label{rule:planeunion}\\
&\vdash\code{collinear}\  S \implies \code{planar}\  S \label{rule:collplane}\\
&\vdash A \neq B \wedge A,B \in S,T \wedge \code{collinear}\ \, S \wedge \code{planar}\ \, T \implies \code{planar}\ \, (S \cup T) \label{rule:collplaneplane}\\
&\vdash P \in S \wedge P \in T \wedge \code{collinear}\ \, S \wedge \code{collinear}\ \, T \implies \code{planar}\ \, (S \cup T) \label{rule:collcollplane}
\end{align}
\caption{Derived incidence theorems in point sets}
\label{fig:PointSets}
\end{figure}

So long as we are restricting our attention to a finite number of points, which is the typical context for applying these theorems, we can see how they reflect those incidence axioms which do not introduce points, and thus justify them as an alternative way to understand the logic of incidence. 

We want this important detail, since it is not our intention to introduce a bunch of \emph{ad hoc} theorems which happen to work most of the time. We want a genuinely alternative way to think about incidence reasoning, one which has nice computational properties.

We explain the content of our theorems thusly: given a finite set of points $S$, let us think of a line containing the points of $S$, should it exist, as a maximal collinear superset of $S$. Similarly, let us think of a plane containing the points of $S$, should it exist, as a maximal planar superset of $S$. In this way, we can define lines and planes entirely in terms of point-sets known to be collinear and planar.

In this sense, Theorem~\ref{rule:colltwo} must assert that the points $A$ and $B$ are incident with a line $AB$ and corresponds to Axiom~\ref{eq:g11}.  Theorem~\ref{rule:collunion} effectively asserts that the expression ``the line $AB$'' is well-defined, and thus corresponds to Axiom~\ref{eq:g12}. To see this, recall that the line $AB$ is the unique maximal superset of all points containing $A$ and $B$. Now the largest possible set containing $A$ and $B$ is just the finite union of all sets containing $A$ and $B$. What Theorem~\ref{rule:collunion} is then telling us is that this set is collinear. In other words, the unique largest set containing $A$ and $B$ is the line of $AB$.

Similarly, Theorem~\ref{rule:planethree} tells us that $A$, $B$ and $C$ are incident with the plane $ABC$ while Theorem~\ref{rule:planeunion} asserts that the expression ``the plane $ABC$'' is well-defined, or more generally, that a plane is uniquely determined by any of its non-collinear subsets.

Theorem~\ref{rule:collplaneplane} is a stronger version of Axiom~\ref{eq:g16}. To see this, we again think of our lines and planes as maximal collinear and planar sets. Axiom~I, 6 says ``[i]f two points $A$, $B$ of a line $a$ lie in a plane $\alpha$ then every point of $a$ lies in the plane $\alpha$.'' Here, we take the line $a$ to be a maximal collinear set $S$ and $\alpha$ to be a maximal planar set $T$. According to Theorem~\ref{rule:collplaneplane}, $S \cup T$ is also planar. But $T$ is maximal, so we must have $S \cup T = T$ and thus $S \subseteq T$. In other words, all points of the line $S$ lie in the plane $T$.

Finally, Theorem~\ref{rule:collcollplane} tells us that distinct intersecting lines lie in a unique plane, the claim made in Hilbert's THEOREM~2. This is because their union must be non-collinear, since otherwise they would not be maximal sets. Thus, their union determines a plane. 

The other theorem, THEOREM~1, notes firstly that distinct lines have either no points in common or just one point in common. This follows directly from Theorem~\ref{rule:collunion}. Indeed, distinct lines as maximal collinear sets must have a non-collinear union, so they cannot have more than one point in common.

THEOREM~1 further notes that distinct planes have either no points in common or one line in common. We know that distinct planes as maximal planar sets must have non-planar unions and so must have collinear intersections by Theorem~\ref{rule:planeunion}. Moreover, we know that if the intersection contains two points, then, according to Theorems~\ref{rule:collunion} and~\ref{rule:collplaneplane}, the two points yield a maximal collinear set contained in the maximal planar set. We will not say anything about the case of a one-point intersection, since this requires the existential Axiom~\ref{eq:g17}: if two planes have one point in common, they have at least one other point in common. As mentioned above, we do not consider such point introduction axioms here.

In conclusion, we have shown how all incidence axioms which do not introduce points can be expressed and strengthened as composition theorems on collinear and planar sets. Primitive lines and planes are no longer used directly, since all the axioms governing them can be subsumed by these composition theorems.

\subsubsection{Evaluation}
Meikle's verifications of Hilbert's geometry involved a lot of tedious but necessary reasoning about incidence relations, and the derivation of many additional lemmas to support the main verifications. With Theorems~\ref{rule:colltwo}--\ref{rule:collcollplane}, the verifications are a good deal less complex. For instance, the verification of THEOREM~3, which relies heavily on incidence reasoning, needed twenty-seven special case lemmas and forty steps in Meikle's verification, while our own verification using the above theorems had twenty-two steps and no additional lemmas. 

This was an improvement, but it still left our verifications bogged down in trivial combinatorial details that made them difficult to compare to the prose. Besides, the verifications were still very difficult to obtain, since we almost always needed to figure out the specific point sets with which to manually specialise the quantifiers in Theorems~\ref{rule:colltwo}--\ref{rule:collcollplane}: the proof assistant could not figure these out for itself with generic automation. Finding these sets is not just tedious, but error-prone. When our verifications were correct, they were often suboptimal. And in one case, our difficulty in proving certain properties of a geometric configuration led us to believe, mistakenly, that Hilbert had made an error in one of his arguments (see \S\ref{sec:CombinatoryError}).

Luckily, when we reflected on our manual verifications, we realised that Theorems~\ref{rule:colltwo}--\ref{rule:collcollplane} were always applied systematically. In the next chapter, we shall describe how to make these theorems the basis for an automated tool which can completely hide the messy incidence reasoning. We can then justify Hilbert's omission of the incidence arguments by claiming that they consist merely of exhaustive combinatorial reasoning which requires no geometric insight. In a geometric proof, the important steps are those which introduce points and thereby build up the geometric configuration. Hilbert's prose proofs, and our own verifications backed up by our incidence automation, leave just those steps explicit.

We can then continue to view Hilbert's proofs as \emph{model proofs} and argue that, with our automation handling the tedium of combining point sets, our verifications will be good substitutes for their missing prose counterparts. They will be more trustworthy whilst being close to what a working mathematician would produce.

We finish by remarking that Theorems~\ref{rule:colltwo}--\ref{rule:collcollplane} were verified in a procedural rather than declarative style in HOL~Light. These theorems will become the implementation details of an incidence reasoning algorithm, rather than theorems of general geometric interest. It therefore made sense to take full advantage of HOL~Light's tactics and its simplifier, which are particularly effective when working with finite sets.

\section{Group~II}
With Hilbert's second group of axioms, we have our ordered geometry, which delimits the scope for the present work. We only have a few geometrical notions to hand, and things might seem quite restrictive, but we have enough to verify our main result: the Polygonal Jordan Curve Theorem.

Order axioms were missing in the ancient axiomatisations of geometry such as Euclid's, and their introduction marks an important milestone in the modern rigorisation of geometry. The first investigation of order axioms is credited to Pasch~\cite{AxiomaticsOrderedGeometry}, and to this day, Hilbert's one planar axiom in this group and its variants elsewhere are still referred to as \emph{Pasch's Axioms.}

Like the first group, Hilbert's second group of axioms went through substantial revision between editions. There was initially a great deal of redundancy, the investigation of which was made by other contributors. Huntingdon and Kline gave a thorough analysis of axioms for ordering along a line~\cite{AnalysisBetweenness}, while E.H. Moore and his student Veblen showed how, via Pasch's Axioms, Hilbert's main linear axiom was derivable. Veblen showed great interest in a bare ordered geometry, proving forty results compared to Hilbert's ten. He gave an early proof attempt of the Polygonal Jordan Curve Theorem (see Chapter~\ref{chapter:JordanInformal}), and later set out to recover the full metrical Euclidean geometry using only order axioms, the parallel axiom and a continuity axiom.

\subsection{Axioms and Primitive Notions}
Hilbert's second group supplies a single new primitive, namely \emph{betweenness}, with which one can form expressions such as ``the point $B$ lies \emph{between} $A$ and $C$''. We are supposed to interpret these expressions strictly, as required by Hilbert's first axiom in the group
\begin{quotation}
\mbox{}\vspace{-\baselineskip}
\begin{enumerate}
\item[II, 1] \emph{If a point $B$ lies between a point $A$ and a point $C$ then the points $A$, $B$, $C$ are three distinct points of a line, and $B$ then also lies between $C$ and $A$.}
\end{enumerate}
\flushright{\emph{Foundations of Geometry}~\cite{FoundationsOfGeometry} (page 5)}
\end{quotation}
The fact that Hilbert needs to axiomatically assert the irrelevance of the order of $A$ and $C$ tells us that we are formally working with a three place relation:
\begin{displaymath}
\code{between}\;:\;\code{point}\rightarrow\code{point}\rightarrow\code{point}\rightarrow\code{bool}.
\end{displaymath}
This first axiom is not particularly informative. It really just gives some useful conditions on the betweenness relation. The symmetry requirement could have been dropped had we instead used a predicate of type $\code{point}\rightarrow \code{pair point} \rightarrow \code{bool}$ where $\code{pair}$ is the type constructor for unordered pairs. We could have dropped the strictness requirement that all points are distinct as Tarski does in his axiomatisation, and we could have allowed degenerate betweenness assertions when the three points are non-collinear. The only consequence would be that other axioms and theorems would sometimes have to make additional non-degeneracy assumptions.

We have formalised a weaker version of this axiom than the one given by Hilbert. With the other order axioms, the stronger version is verifiable. The weakening arises because we do not conclude that all points are distinct, only that $A$ is distinct from $C$.
\begin{equation}\label{eq:g21}
 \tag{II,~1}
  \begin{split}
    \vdash\between{A}{B}{C} \implies & A \neq C\\
                               & \wedge (\exists a.\; \code{on\_line}\ A\ a \wedge \code{on\_line}\ B\ a \wedge \code{on\_line}\ C\ a)\\
                               & \wedge \between{C}{B}{A}.
  \end{split}
\end{equation}

We now give the remaining axioms.
\begin{quotation}
\mbox{}\vspace{-\baselineskip}
\begin{enumerate}
\item[II, 2] \emph{For two points $A$ and $C$, there always exists at least one point $B$ on the line $AC$ such that $C$ lies between $A$ and $B$.}
  \item[II, 3] \emph{Of any three points on a line there exists no more than one that lies between the other two.}
  \item[II, 4] \emph{Let $A$, $B$, $C$ be three points that do not lie on a line and let $a$ be a line in the plane $ABC$ which does not meet any of the points $A$, $B$, $C$. If the line $a$ passes through a point of the segment $AB$, it also passes through a point of the segment $AC$, or through a point of the segment $BC$.}
\end{enumerate}
\flushright{\emph{Foundations of Geometry}~\cite{FoundationsOfGeometry} (page 5)}
\end{quotation}

Axiom~II, 2 can be compared to Euclid's second postulate: ``To produce a finite straight line continuously in a straight line.'' Here, Hilbert is telling us that we can extend the segment $AC$ to a point $B$ in the direction $\overrightarrow{AC}$. This axiom is absolutely key in building up geometrical figures.

Axiom~II, 3 is akin to an anti-symmetry property for linear ordering. With Axiom~\ref{eq:g21}, it also shows that from $\between{A}{B}{C}$ we can infer that $A$, $B$ and $C$ are mutually distinct.

Axiom~II, 4 is Pasch's axiom. So far, it is the most complex axiom we have to apply, being a planar axiom with numerous preconditions and a disjunctive conclusion wrapped in an existential. The complexity can be measured by comparing the size of the formalisation with that for the other axioms. See Figure~\ref{fig:Group2Axioms}.

\begin{figure}
\begin{equation}
 \tag{\ref{eq:g21}}
  \begin{split}
    \vdash\between{A}{B}{C} \implies & A \neq C\\
                               & \wedge (\exists a.\; \code{on\_line}\ A\ a \wedge \code{on\_line}\ B\ a \wedge \code{on\_line}\ C\ a)\\
                               & \wedge \between{C}{B}{A}
  \end{split}
\end{equation}
\begin{equation}\label{eq:g22}
  \tag{II,~2}
  \vdash A \neq B \implies \exists C.\; \between{A}{B}{C}
\end{equation}
\begin{equation}\label{eq:g23}
  \tag{II,~3}
  \vdash\between{A}{B}{C} \implies \neg\between{A}{C}{B}
\end{equation}
\begin{equation}\label{eq:g24}
% (!A B C P a 'a.
%            ~(?a. on_line A a /\ on_line B a /\ on_line C a)
% 	   ==> on_plane A 'a /\ on_plane B 'a /\ on_plane C 'a
% 	   ==> (!P. on_line P a ==> on_plane P 'a)
% 	   ==> ~on_line A a /\ ~on_line B a /\ ~on_line C a
% 	   ==> on_line P a /\ between A P B
% 	   ==> (?Q. on_line Q a /\ (between A Q C \/ between B Q C)))`;;
  \tag{II,~4}
  \begin{split}
    \text{Pasch's Axiom}\qquad & \vdash\Triangle{b}{A}{B}{C}\\
    & \wedge \code{on\_plane}\ A\ \alpha \wedge \code{on\_plane}\ B\ \alpha \wedge \code{on\_plane}\ C\ \alpha \\
    & \wedge (\forall P.\; \code{on\_line}\ P\ a \implies \code{on\_plane}\ P\ \alpha)\\
    & \wedge \neg\code{on\_line}\ A\ a \wedge \neg\code{on\_line}\ B\ a \wedge \neg\code{on\_line}\ C\ a\\
    & \wedge \code{on\_line}\ P\ a \wedge \between{A}{P}{B}\\
    & \implies \exists Q.\; \code{on\_line}\ Q\ a \wedge (\between{A}{Q}{C} \vee \between{B}{Q}{C})
  \end{split}
\end{equation}
\caption{Group~II axioms}
\label{fig:Group2Axioms}
\end{figure}

\subsection{Pasch and Incidence Reasoning}
Our early experiences verifying Hilbert's early theorems showed that most of the effort is expended trying to verify the preconditions of Pasch's Axiom~\eqref{eq:g24}. In order to leverage our representation in point sets, we decided to derive another formalisation of the axiom in terms of collinearity and planarity. To do this, we remove all mention of the line $a$ from the axiom, and replace it by two defining points $D$ and $E$. The point $D$ will be assumed to be the point of intersection between the line $a$ and the segment $AB$, and the point $E$ will be any other point on the line $a$. See Figure~\ref{fig:PaschDiagram}.

\begin{figure}
\centering\includegraphics{axioms/Pasch}
\caption{Axiom II, 4}
\label{fig:PaschDiagram}
\end{figure}

Our preconditions can now be expressed in terms of non-collinear and planar sets, as seen in the following verified formulation of Axiom~II, 4:%\footnote{Thanks to Laura Meikle for spotting that we only need to assume three triangles here.}
\begin{equation}\label{eq:PaschPointSet}
\begin{aligned}
  \vdash&\neg\code{collinear}\ \{A,B,C\}\wedge\neg\code{collinear}\ \{A,D,E\}\wedge\neg\code{collinear}\ \{C,D,E\}\\
  &\wedge\code{planar}\ \{A,B,C,D,E\}\wedge\between{A}{D}{B}\\
  &\implies\exists F.\; \code{collinear}\ \{D,E,F\} \wedge (\between{A}{F}{C}\vee\between{B}{F}{C}).
\end{aligned}
\end{equation}

Looking at this theorem, it is hopefully clearer how the theorems from \S\ref{sec:PointSets} are needed when reasoning about incidence in Hilbert's proofs. Most of his results in Group~II require reasoning about order in the plane by applying Pasch's Axiom. Each time the axiom is applied, we must verify the preconditions of the axiom, which according to our formalised Theorem~\ref{eq:PaschPointSet}, means we must find three triangles and a planar set.

This is not all. Typically, we must also eliminate one of the disjuncts in the conclusion of the axiom, and this requires further incidence reasoning. Usually, we show how, in one branch, all points considered end up collapsing to just a single line. This will contradict our assumptions that we have at least one triangle.

Our verifications from our earlier work were laden down with steps to find triangles by repeatedly applying the theorems from \S\ref{sec:PointSets}. We will show some of the complexity in Chapter~\ref{chapter:Group2Eval}, but will see how, luckily, it can be fully automated.

\section{Conclusion}
The formalisation of Hilbert's first group of axioms and the verification of his first two theorems is straightforward, up to a few minor technical points about the choice of representation and whether we implicitly assume that the primitive sorts or types are inhabited. Otherwise, the axioms of Group~I are conspicuous only for their absence from Hilbert's proofs, especially in Group~II where incidence reasoning is needed heavily in order to apply Pasch's axiom. However we are to justify the absence, all details must be restored in our verifications. When we do so, we find our verifications are washed out with fussy incidence arguments.

By reformulating in terms of point sets, we can alleviate this somewhat, but it only takes us so far. If we are to keep our proofs as clean as Hilbert's, and have a decent chance of verifying more complex theorems, we will want to make almost all the incidence reasoning implicit. This we leave to the next chapter, where we consider automation.

%%% Local Variables:
%%% TeX-master: "../thesis"
%%% End:


\chapter{Automation}\label{chapter:Automation}
In the last chapter, we considered a representation for incidence claims involving lines and planes which helped us write shorter verifications. The use of these rules also appeared ripe for automation. 

The HOL~Light theorem prover expects its users to work ``close to the metal'', often writing ML directly. Writing new tools can be comfortable in this environment, and it is easy to prototype and experimentally integrate new tools into the rest of the system and existing proof languages. In this chapter, we shall see how this integration can take the form of a concurrent and collaborative discovery tool for incidence reasoning, which can be made available as a tactic and in Mizar~light proof steps.

Much of this chapter and the subsequent chapter is an expansion and improvement on earlier work~\cite{ScottExploring,ScottComposable}.

\section{Related Work}
As far as fully automated theorem proving goes, the oldest successes are probably in geometry. The signed-area method \cite{MachineProofsInGeometry} of Chou et al, proved to be particularly capable at finding large numbers of non-trivial geometric theorems. In this section, we restrict our attention to automation that can apply to the very basic incidence theory we consider here. The signed-area method, which typically assumes an ordered field, is already outside of our scope.

\subsection{Wu's Method}
When considering general automation in Hilbert's \emph{Foundations of Geometry}, there is probably no work more relevant than Wu's method~\cite{WuMechanicalTheoremProving}. Wu, perhaps giving too much credit to Hilbert, claimed that the \emph{Foundations of Geometry} contains metatheoretical insights towards a mechanisation procedure for the whole of geometry. 

One of Wu's own insights was that the method of proof in a synthetic geometry system such as Hilbert's often falls short of absolute rigour because degenerate cases are routinely missed. Typically, when we state axioms and theorems for some geometric figure, we have in mind a particular ``genericity'' of that figure which is hard to capture formally. Moreover, the axioms and theorems may admit some generalisation to what we would regard as ``degenerate'' cases, even if this is not immediately clear at the time. We have already seen in the last group that some of Hilbert's axioms, such as Axiom~\ref{eq:g11} hold in degenerate cases (though this does not imply that the axiom should be made \emph{stronger}). We shall show in Chapter~\ref{chapter:HalfPlanes} how the particularly troublesome Axiom~\ref{eq:g24} (at least as far as incidence reasoning is concerned) can be strengthened by allowing degenerate cases.

But ``degenerate'' is not well-defined, and it is not always clear how the conditions of a theorem can be relaxed. On the other hand, when trying to rule out certain degenerate cases, there are plenty that are easily missed. When we formalise, we cannot simply neglect these degenerate cases unless we have proof tools that can do it for us. Filling in all the gaps requires enormous effort and complication of the proof, so it would seem that Wu is correct: we cannot be truly rigorous unless we can systematically deal with degenerate conditions. This provides an alternative way to diagnose the gaps in Hilbert's proofs spotted by Meikle and Fleuriot~~\cite{MeikleFleuriotFormalizingHilbert}: they were gaps concerning degeneracy conditions about point incidence.

Wu's highly celebrated method automatically inserts non-degeneracy conditions to the theorems it proves, and if possible, it automatically deletes redundant conditions to make a specific theorem more generic. The method has been used to automatically prove an enormous number of non-trivial theorems in unordered geometry \cite{MechanicalGeometryTheoremProving}. To cover ordered geometry, Wu appealed to the embedding of Euclidean geometry in real-closed fields, for which Tarksi has a well-known decision procedure. The method has much less success here, owing to the gross intractability of the decision procedure~\cite{TarksiMcNaugtonReview}. 

Unfortunately, Wu's automation is not particularly appropriate for our own work. If we were to implement Wu's method, for instance, we would need to show how each of his mechanical steps can be reduced to the axioms of Hilbert's system. But this reduction will presuppose some of the very elementary theorems which we are trying to mechanise. Indeed, Wu's method is based at least on results which rest on Desargues' theorem --- Theorem~53 in the \emph{Foundations of Geometry}. Furthermore, our focus in the present work is \emph{ordered geometry}, a domain in which Wu's method is ill-suited. 

\subsection{Ranks}
For the specific problem of incidence reasoning, Magaud et al's work~\cite{RankDesargues} is closest in spirit to our own. It too, is based on point-sets, but the authors have beautifully abstracted the core idea. 

Several of our rules for lines and planes in Figure~\ref{fig:PointSets} in the last chapter appear analogous. Magaud et al have captured the analogy formally using \emph{ranks}. The rough idea is that an $n$-dimensional set is assigned a rank of $n+1$. There are then key rules which assert that, given point sets $X$ and $Y$, the sum of the ranks of $X \cup Y$ and $X \cap Y$ is no greater than the sum of the ranks of $X$ and $Y$. This abstractly characterises our rule for taking the union of collinear and planar sets. 

Magaud's approach allows the author's to generalise to arbitrary dimension, and the elegance of the theory helps us see our own approach as less ad-hoc than it might otherwise. For the rest of this chapter, however, we shall focus on our original, more concrete representation.

\section{Inference Rules}
The rules from Figure~\ref{fig:PointSets} tell us how to reason with finite collinear and planar sets, and so can form the basis of a combinatorial algorithm. Our rules trade in four kinds of theorem: theorems about which points are distinct, which points are collinear, which triples are non-collinear, and which points are planar. These theorems will be the domain of our algorithm, whose chief procedures will be based on rules to interderive them.

Our rules already show how to introduce collinear and planar sets. Additionally, we need ways to derive the inequalities and triangles on which the rules depend. Firstly, we note that any triangle or non-collinear triple implies the mutual distinctness of its three points, giving us one way to introduce point inequalities. Another way to derive inequalities is through simple congruence reasoning: suppose we have a collinear set $S$, and a non-collinear triple sharing two points with $S$. Then the third point of the triple must be distinct from all points in $S$. 

Next, we need to introduce non-collinear triples. We based our method here on patterns of reasoning that showed up in our Isabelle formalisation. Suppose we have a collinear set $S$ and a non-collinear triple sharing two points with $S$. Then the third point forms a non-collinear triple with all pairs of points in $S$ known to be distinct.

Finally, we consider how we might infer when two points are \emph{equal} using our rules. Again, we followed a pattern of argument that showed up in the Isabelle formalisation. Given two collinear sets $S$ and $T$, which have a non-collinear union, we can infer that there intersection must be empty or a singleton. So if the intersection is a set of points $\{P_1,P_2,\ldots,P_n\}$, then we know immediately that these points are identical. This, we noted in the last chapter, is just telling us that distinct lines intersect in at most one point, as per Hilbert's Theorem~1.

We now summarise the rules and methods for introducing new theorems. 
\begin{itemize}\label{list:Procedures}
\item[$\code{ncolneq}$] Infer inequalities from non-collinear triples:
\begin{displaymath}
\neg\code{collinear}\ \{A,B,C\} \implies A \neq B \wedge A \neq C \wedge B \neq C.
\end{displaymath} (by rule \ref{rule:colltwo});
\item[$\code{colncolneq}$] Infer inequalities from a collinear set containing two points of a non-collinear triple. 
\begin{align*}
  &\code{collinear}\ S \wedge \neg\code{collinear}\ \{A,B,C\}\\
  &\qquad\wedge A,B\in S \implies \forall S. S \in As \implies C \neq S
\end{align*}(by rule~\ref{rule:collsubset}). For example, 
\begin{align*}
\code{collinear} \{A, B, C, D, E\} \wedge & \neg\code{collinear} \{A, B, P\}\\
&\implies C \neq P \wedge D \neq P \wedge E \neq P.
\end{align*}
\item[$\code{coleq}$]
Equate points in the intersection of two collinear sets which are jointly non-collinear.
\begin{align*}
&\code{collinear}\ S \wedge \code{collinear}\ T \wedge \neg\code{collinear}\ U \wedge U \subseteq S \cup T \\
&\qquad\qquad\wedge A,B \in S,T \implies A = B
\end{align*} (by rules~\ref{rule:collsubset} and \ref{rule:collunion}).
For example,
\begin{align*}\
&\code{collinear} \{A, B, C, D, E\} \,\wedge\,\code{collinear} \{A, C, E, X, Y\}\\ 
&\qquad\, \wedge\,\neg\code{collinear} \{A, B, Y\}\implies A = C \wedge A = E \wedge C = E.
\end{align*}
\item[$\code{colncolncol}$] Infer new non-collinear triples from a collinear set and another non-collinear triple.
\begin{align*}
  &\code{collinear}\ S \wedge \neg\code{collinear}\ \{A,B,C\}\\
  &\qquad\wedge X,Y,A,B \in As \wedge A \neq B\implies \neg\code{collinear}\ \{C,X,Y\}.
\end{align*} (by rules~\ref{rule:collsubset} and~\ref{rule:collunion}). For example,
\begin{align*}
A \neq C \wedge D \neq E &\wedge \code{collinear} \{A, B, C, D, E\} \wedge \neg\code{collinear} \{A, B, P\}\\
&\implies \neg\code{collinear} \{A, C, P\} \wedge \neg\code{collinear} \{D, E, P\}.
\end{align*}
\item[$\code{colcol}$] Use rule \ref{rule:collunion} to show that the union of collinear sets which intersect at more than one point is collinear.
\item[$\code{planeplane}$] Use rule \ref{rule:planeunion} to show that the union of planar sets intersecting at a non-collinear triple is planar.
\item[$\code{colplane}$] Use rule \ref{rule:collplane} to show that a collinear set is planar.
\item[$\code{colplaneplane}$] Use rule \ref{rule:collplaneplane} to show that the union of a collinear and planar set intersecting in at least two points is planar.
\item[$\code{colcolplane}$] Use rule \ref{rule:collcollplane} to show that the union of intersecting collinear sets is planar.
\end{itemize}

We have stated these rules at the object level, and they can be applied exhaustively by just using matching and term-rewriting. However, we have also implemented them directly as ML inference rules. This is preferred for efficiency, since we can be smarter about when a match should fail, and we can alleviate the burden on the rewriter.

With hard-coded ML inference rules, we can dispense with sets and represent our theorems as ordinary first-order terms. So $\code{collinear}\ \{A,B,C,D,E\}$ gets expressed directly as the existential produced by unfolding the definition of $\code{collinear}$ and the finite set constructors:
\begin{displaymath}
  \exists a. \code{on\_line}\ A\ a\wedge\code{on\_line}\ B\ a\wedge\code{on\_line}\ C\ a\wedge\code{on\_line}\ D\ a\wedge\code{on\_line}\ E\ a.
\end{displaymath}
The collinear union rule is now ML code which strips the existential and conjuncts from these formulas, and then directly applies existential rules and Axiom~\ref{eq:g12} to produce a new claim of collinearity of the same form. Our object level rules can serve as a weak sort of verification: after unfolding the predicates $\code{collinear}$ and $\code{planar}$ and unfolding the finite sets, direct matching should behave identically to our hard-coded ML.

Besides efficiency, the fact that we stick to first-order means that our declarative proofs are not loaded down having to unfold statements involving $\code{collinear}$ and $\code{planar}$ predicates, nor do they have to perform any set-theoretic reasoning.

\section{N\"{a}ive Implementation}\label{sec:NaiveImplementation}
Our first prototype was a basic forward-chaining algorithm. Forward-chaining algorithms for automatically producing readable proofs in Hilbert's geometry have recently been investigated by Stojanovic et al~\cite{ForwardChainHilbert}, and they seemed particularly suitable for our use case. We have already noted that some of our proofs were suboptimal, that we believed Hilbert might have made errors in assuming certain incidence relations, and so we wanted a tool which could \emph{explore} the proof space of incidence reasoning surrounding each of Hilbert's proofs. The idea of using forward-chaining in this sort of exploratory way also opened up the possibility of designing an automated tool which could \emph{collaborate} with the user as they develop a proof. Forward-chaining seemed quite apt, since its focus on continually growing a base of facts corresponds well with how a declarative proof is developed by continually growing a proof context. 

Our first algorithm worked on \emph{generations} made up of five lists, one for each of the different kinds of fact: point equalities, point inequalities, collinear sets, non-collinear sets, and finally planar sets. The procedures of \S\ref{list:Procedures} were then applied across all relevant combinations of theorems from the five lists, to produce a new generation of lists. In this way, the algorithm produces a sequence of generations, each a superset of its predecessor.

\subsection{Concurrency}\label{sec:NaiveConcurrency}
Typically, the automation available in interactive theorem provers is invoked on demand by the user when they evaluate individual proof steps. But when the user writes the formal proof for the first time, or comes to edit it later, they will spend most of their time \emph{thinking}, consulting texts, backtracking and typing in individual proof commands. The CPU is mostly idling during this process, and we can exploit this idle time to run automated tools concurrently.

The Isabelle Theorem prover has capitalised on this with Sledgehammer~\cite{IsabelleSledgehammer}. By invoking this command, the user can continue to work on a proof, while generic first-order tools are fired off as separate background processes, attempting to solve or refute the user's goals independently. If the tools reach a conclusion, the generated proof certificates can be automatically and seamlessly integrated into the user's proof-script. 

We argue that we can do one better, since it is almost trivial to turn our algorithm into something of a ``collaborative'' architecture. We are focusing on forward-chaining and declarative proof, and as we remarked, the way that a user builds a declarative proof is by growing a proof context towards a goal, while a forward-chaining algorithm interatively grows a set of facts. These processes are analogous, so why not splice the two? A user's proof script is then a user guided, manually crafted search through the proof space, but now one which has access to the derivations of a forward chaining algorithm. At the same time, the forward chaining algorithm works independently, but can freely incorporate the user's manually obtained deductions into its own base of facts. The two systems, the automation and the human user, can thus be seen as collaborating, sharing data as they both strive towards the goal. 

We started with a very simple implementation of this setup. We spawn a thread that is initialised with an empty generation of facts. New generations are produced iteratively through forward-chaining. At the same time, the thread  monitors the proof context, so that each time the user adds a hypothesis by performing a declarative proof step, the hypothesis can be automatically inserted into the current generation and used for further chaining.

The thread then echoes each generation of facts concurrently to a separate terminal, while the user works on the proof script. We then introduce a variable $\code{the\_facts}$ which is updated so that it always dereferences to the current generation. This variable can be used in a proof script to justify the current step with the automatically inferred facts. 

%When a fixpoint is detected, the user is notified, so that they can decide whether perhaps an assumption is missing, or whether a more sophisticated subproof is needed. In the meantime, the forward-chaining thread will sleep.

Since we are using an LCF style prover~\cite{LCF}, we always ensure that our forward-chaining derivations are \emph{fully-expansive}~\cite{FullyExpansive}. This means that to carry out its derivations, the tool directly applies inference rules to generate fully machine verified lemmas. These lemmas can then be seamlessly integrated into the user's proof script to produce a fully machine-checked proof.

\section{An Implementation in Combinators}\label{sec:DiscoveryAlgebra}
\subsection{Data-Flow}
Our prototype implementation was an adequate proof of concept, and enabled us to write some extremely short and clear proofs. We have some examples of these in the next chapter. However, our prototype currently has no theoretical underpinning. The finer details of the implementation are complex, ad-hoc, unlikely to scale, and cannot be easily modified to work on new problems.

Much of the complexity of the ML algorithm comes from the fact that each generation of theorems is partitioned into five types: point equalities, point inequalities, collinear sets, non-collinear sets and planar sets. By partitioning in this way, we greatly reduce the number of combinations of theorems we need to try with each inference rule, but there is still much wasted effort. Every time a new generation is generated, all relevant combinations are applied, \emph{including} the combinations which were applied for the \emph{last} generation. Our simple lists do not help us filter out the repetitious work.

Rather than seeing the problem as simply one of repeating inferences across a succession of generations, we need a way to model the data-flow directly. The flow of data for our problem, according to the rules of \S\ref{list:Procedures}, is given in Figure~\ref{fig:DataFlow}.

\begin{figure}
\centering\includegraphics[scale=0.5]{automation/DataFlow}
\caption{Data-flow for incidence reasoning: boxes represent our collections of various kinds of theorem, while triangles represent inference rules}
\label{fig:DataFlow}
\end{figure}

Our aim in this section is to find a way to directly express this data-flow in a suitable language. For this, we will need a distinguished set of primitives which can be combined by transformations. Following the examples of tactics and conversions~\cite{Tactics}, the hallmark of the LCF approach to theorem proving, and following a strong staple of strongly typed functional programming~\cite{CombinatorLanguages}, we shall implement our language in \emph{combinators}. The advantage of choosing a combinator language is that it fully integrates with the host programming language, and is therefore easy to integrate with the tactic combinator language and the Mizar~Light combinator language. The user is also free to extend the language and can easily inject computations into it using lambda abstraction.

\subsection{Related Work}
The idea of an algebraic data-flow language was considered early on by Chen~\cite{ChenForwardChaining}, who gave a specification for a variety of primitives very similar to our own, though without an implementation. Since then, algebras for logic programming, handling unbounded and fair search have been developed~\cite{BacktrackingMonad}. The algebra we shall consider here is related and was originally conceived by Spivey~\cite{SearchAlgebras}. It has more rigorously developed underpinnings, and compared to other search algebras, it places stronger constraints on the order in which the values are generated. We shall be generalising Spivey's ideas somewhat, and also hope to give more a more ``operational'' motivation for the definitions.

\subsection{Streams}\label{sec:Streams}
Our overarching purpose is to output a stream of theorems, perhaps to a terminal, to a database of facts to be used during a proof, or perhaps to another consumer for further processing. If we think of this output \emph{as} the implementation, then we are dealing with procedures which lazily generate successive elements of a list. These lazy lists, or streams, shall be the primitives of our data-flow algebra. 

For now, we leave unspecified what computations are used to generate the primitive streams. It might be that a stream simply echos a list of precomputed theorems; it might generate theorems using input data; it might generate them from some other automated tool. We shall focus instead on transformations for streams, and in how we might lift typical symbolic manipulation used in theorem proving to the level of streams.

One reason why lists and streams are a good choice is that they are \emph{the} ubiquitous data-structure of ML and its dialects, and have rich interfaces to manipulate them. A second reason why lists are an obvious choice is that they have long been known to satisfy a simple set of algebraic identities and thus to constitute a monad~\cite{MonadWadler}. We can interpret this monad as decorating computations with non-deterministic choice and backtracking search. 

Monads themselves have become a popular and well-understood abstraction in functional programming. Formally, a monad is a type-constructor $M$ together with three operations 
\begin{align*}
&\code{return} : \alpha \rightarrow M\;\alpha\\
&\code{fmap} : (\alpha \rightarrow \beta) \rightarrow M\;\alpha \rightarrow M\;\beta\\
&\code{join} : M\;(M\;\alpha) \rightarrow M\;\alpha
\end{align*}
satisfying the algebraic laws given in Figure~\ref{fig:MonadLaws}.

\begin{figure}
\begin{align}
&\code{fmap}\;(\lambda x.\;x)\;m = m\notag\\
&\code{fmap}\;f \circ \code{fmap}\;g = \code{fmap}\;(f \circ g)\notag\\
&\code{fmap}\;f \circ \code{return} = \code{return}\circ f\notag\\
&\code{fmap}\;f \circ \code{join} = \code{join} \circ \code{fmap}\;(\code{fmap}\;f)\notag\\
&(\code{join} \circ \code{return})\;m = m\notag\\
&(\code{join} \circ \code{fmap}\;\code{return})\;m = m\notag\\
&\code{join} \circ \code{join} = \code{join} \circ \code{fmap}\;\code{join} \label{eq:JoinAssoc}
\end{align}
\caption{The Monad Laws}
\label{fig:MonadLaws}
\end{figure}

The list monad can be used for search, but takes list concatenation $\code{concat} : [[\alpha]] \rightarrow [\alpha]$ for its join operation. This makes it unsuitable for \emph{unbounded} discovery. If the list $xs$ represents one unbounded search, then we have $xs + ys = xs$\footnote{Here, $+$ is just list append.} for any $ys$, and thus, all items found by $ys$ are lost. This is an undesirable property. We want to be able to combine unbounded searches over infinite domains without losing any data.

\subsection{The Stream Monad}\label{sec:StreamMonad}
There is an alternative definition of the monad for streams (given in Spivey~\cite{SearchAlgebras}) which handles unbounded search. Here, the join function takes a possibly infinite stream of possibly infinite streams, and produces an exhaustive enumeration of \emph{all} elements. We show how to achieve this in Figure~\ref{fig:ShiftGen} using a function $\code{shift}$, which moves each stream one to the ``right'' of its predecessor. We can then exhaustively enumerate every element, by enumerating each column, one-by-one, from left-to-right. 

\begin{figure}
  \small
  \centering
  \begin{tabular}{rcl}
    shift & &
    \begin{tabular}{ccccccc}
      $[[D_{0,0},$ & $D_{0,1},$ & $D_{0,2},$ & $\ldots,$ & $D_{0,n},$ & $\ldots],$ \\
      $[[D_{1,0},$ & $D_{1,1},$ & $D_{1,2},$ & $\ldots,$ & $D_{1,n},$ & $\ldots],$ \\
      $[[D_{2,0},$ & $D_{2,1},$ & $D_{2,2},$ & $\ldots,$ & $D_{2,n},$ & $\ldots],$ \\
      $[[D_{3,0},$ & $D_{3,1},$ & $D_{3,2},$ & $\ldots,$ & $D_{3,n},$ & $\ldots],$ \\
      $[[D_{4,0},$ & $D_{4,1},$ & $D_{4,2},$ & $\ldots,$ & $D_{4,n},$ & $\ldots],$ \\
      $[[D_{5,0},$ & $D_{5,1},$ & $D_{5,2},$ & $\ldots,$ & $D_{5,n},$ & $\ldots],$ \\
    \end{tabular}\\\\
    = & &
    \begin{tabular}{ccccccccccccc}
      $[[D_{0,0},$ & $D_{0,1},$ & $D_{0,2},$ & $D_{0,3},$ & $D_{0,4},$ & $D_{0,5},$ & $D_{0,6},$ & \colorbox{gray}{$D_{0,7},$} & $D_{0,8},$ & $\ldots],$\\
      & $[D_{1,0},$ & $D_{1,1},$ & $D_{1,2},$ & $D_{1,3},$ & $D_{1,4},$ & $D_{1,5},$ & \colorbox{gray}{$D_{1,6},$} & $D_{1,7},$ &  $\ldots],$\\
      && $[D_{2,0},$ & $D_{2,1},$ & $D_{2,2},$ & $D_{2,3},$ & $D_{2,4},$ & \colorbox{gray}{$D_{2,5},$} & $D_{2,6},$ & $\ldots],$\\
      &&& $[D_{3,0},$ & $D_{3,1},$ & $D_{3,2},$ & $D_{3,3},$ & \colorbox{gray}{$D_{3,4},$} & $D_{3,5},$ & $\ldots],$\\
      &&&& $[D_{4,0},$ & $D_{4,1},$ & $D_{4,2},$ & \colorbox{gray}{$D_{4,3},$} & $D_{4,4},$ & $\ldots],$\\
      &&&&& $[D_{5,0},$ & $D_{5,1},$ & \colorbox{gray}{$D_{5,2},$} & $D_{5,3},$ & $\ldots],$\\
      &&&&&& $[D_{6,0},$ & \colorbox{gray}{$D_{6,1},$} & $D_{6,2},$ & \ldots]\\
      &&&&&&&         &        & $\vdots$ &           &        &           
    \end{tabular}
  \end{tabular}
  \caption{Shifting}
  \label{fig:ShiftGen}
\end{figure}

If we understand these streams as the outputs of discoverers, then the outer stream can be understood as the output of a discoverer which \emph{discovers discoverers}. The join function can then be interpreted as \emph{forking} each discoverer at the point of its creation and combining the results into a single discoverer. The highlighted column in Figure~\ref{fig:ShiftGen} is this combined result: a set of values generated \emph{simultaneously} and thus having no specified order (this is required to satisfy Law~\ref{eq:JoinAssoc} in Figure~\ref{fig:MonadLaws}).

However, this complicates our stream type, since we now need additional inner structure to store the combined values. We will refer to each instance of this inner structure as a \emph{generation}, following our terminology from \S\ref{sec:NaiveImplementation}. Each generation here is a finite collection of simultaneous values discovered at the same level in a breadth-first search. We just need to define the join function, taking care of this additional structure.

Suppose that generations have type $G\;\alpha$ where $\alpha$ is the element type. The manner in which we will define our shift and join functions on discoverers of generations assumes certain algebraic laws on them: firstly, they must constitute a monad; secondly, they must support a sum operation \mbox{$(+):G\;\alpha\rightarrow G\;\alpha\rightarrow G\;\alpha$} with identity $0:G\;\alpha$. The join function for discoverers must then have type $[G\;[G\;\alpha]] \rightarrow [G\;\alpha]$, sending a discoverer of generations of discoverers into a discoverer of generations of their data. 

We now denote the $k^{\text{th}}$ element of the argument to \code{join} by $gs_k = \{d_{k,0}, d_{k,1}, \ldots, d_{k,n}\} : G\;[G\;\alpha] $. Each $d_{k,i}$ is, in turn, a discoverer stream $[g^k_{i,0}, g^k_{i,1}, g^k_{i,2}, \ldots] : [G\;\alpha]$. We invert the structure of $gs_k$ using $\code{transpose}  : M[\alpha]\rightarrow [M\;\alpha]$, which we can define for arbitrary monads $M$. This generality allows us to abstract away from Spivey's bags and consider more exotic inner data-structures. We choose the name ``$\code{transpose}$'', since its definition generalises matrix transposition on square arrays (type $[[\alpha]]\rightarrow[[\alpha]]$):
\begin{displaymath}
\code{transpose}\;xs = \code{fmap}\;\code{head}\;xs :: \code{transpose}\; (\code{fmap}\;\code{tail}\;xs)
\end{displaymath}

The transpose produces a stream of generations of generations (type $[G\;(G\;\alpha)]$). If we join each of the elements, we will have a stream $[D_{k,0}, D_{k,1}, D_{k,2}, \ldots] : [G\;\alpha]$ (see Figure~\ref{fig:Transpose}), and thus, the shift function of Figure~\ref{fig:ShiftGen} will make sense. Each row is shifted relative to its predecessor by prepending the 0 generation, and the columns are combined by taking their sum.

\begin{figure}
  \begin{align*}
    &\code{map}\;\code{join}\;(\code{transpose}\;\{d_{k,0}, d_{k,1}, \ldots, d_{k,n}\})\\
  = &\code{map}\;\code{join}\;\left(\code{transpose}\;\left\{\begin{matrix}
        [g^k_{0,0}, & \colorbox{gray}{$g^k_{0,1}$}, & g^k_{0,2}, &\ldots]\\
        [g^k_{1,0}, & \colorbox{gray}{$g^k_{1,1}$}, & g^k_{1,2}, &\ldots]\\
                  & \vdots\\
        [g^k_{n,0}, & \colorbox{gray}{$g^k_{n,1}$}, & g^k_{n,2}, &\ldots]
      \end{matrix}\right\}\right)\\
      = &\left[\begin{matrix}
        \code{join}&\{g^k_{0,0}, & g^k_{1,0}, &\ldots, & g^k_{n,1}\},\\
        \code{join}&\Big\{\colorbox{gray}{$g^k_{0,1}$}, & \colorbox{gray}{$g^k_{1,1}$}, &\colorbox{gray}{$\ldots$}, & \colorbox{gray}{$g^k_{n,1}$}\Big\},\\
        \code{join}&\{g^k_{0,2}, & g^k_{1,2}, &\ldots, & g^k_{n,2}\},\\
                   & \vdots
                 \end{matrix}\right]\\
      \quad= &[D_{k,0}, D_{k,1}, D_{k,2}, \ldots]
  \end{align*}
\caption{Transpose}
\label{fig:Transpose}
\end{figure}

The type of discoverers now constitutes a monad (see Spivey~\cite{SearchAlgebras} for details). The fact that we have a monad affords us a tight integration with the host language in the following sense: we can lift arbitrary functions in the host language to functions on discoverers, and combine one discoverer $d : [G\;\alpha]$ with another discoverer\linebreak $d' : \alpha \rightarrow [G\;\alpha]$ which depends, via arbitrary computations, on each individual element of $d$. 

There is further algebraic structure in the form of a monoid: streams can be summed by summing corresponding generations, an operation whose identity is the infinite stream of empty generations. 

\section{Case-analysis}
Our algebra allows us to partition our domain into discoverers according to our own insight into a problem. For incidence reasoning, we divided the domain into five kinds of incidence theorem. We can then compose our discoverers in a way that reflects the typical reasoning patterns found in the domain.

However, when it comes to theorem-proving, the sets of theorems are further partitioned by branches on disjunctive theorems. In proof-search, when we encounter a disjunction, we will want to branch the search and associate discovered theorems in each branch with its own disjunctive hypothesis. 

Ideally, we want to leave such case-splitting as a book-keeping issue in our algebra, and so integrate it into the composition algorithm. Streams must then record a context for all of their theorems, and this context must be respected as discoverers are combined. 

Luckily, there is some flexibility in Spivey's search monad. We can choose a data-structure other than bags for the generations. To solve the problem of case-analysis, we have chosen to implement the generations as \emph{trees}. 

\subsection{Trees}
Each tree represents a generation of discovered theorems and replaces each of Spivey's bags. Each node in a tree is a bag of theorems discovered in that generation under a disjunctive hypothesis. Branches correspond to case-splits, with each branch tagged for the disjunct on which case-splitting was performed. The branch labels along any root-path therefore provide a context of disjunctive hypotheses for that subtree. Our goal is to discover theorems which hold when the case-splits are eliminated.

Thus, the tree in Figure~\ref{fig:BasicTreeExample} might represent a formula made by case-analysing\linebreak \mbox{$P \vee (Q \wedge (R \vee S))$}:
\begin{align*}
\phi_1 \wedge \phi_2 \wedge \cdots \wedge \phi_n &\wedge (P \implies \psi_1 \wedge \psi_2 \wedge \cdots \wedge \psi_n)\\
&\wedge \left(\begin{aligned} Q \implies \chi_1 \wedge \chi_2 \wedge \cdots \chi_n &\wedge (R \implies \alpha_1 \wedge \alpha_2 \wedge \cdots \wedge \alpha_n)\\ &\wedge (S \implies \beta_1 \wedge \beta_2 \wedge \cdots \wedge \beta_n)\end{aligned}\right).
\end{align*}

\subsubsection{Combining Trees}\label{fig:CombiningTrees}
\begin{figure}
\centering\includegraphics[scale=0.7]{automation/BasicTreeExample}
\caption{A Simple Tree Branching on Case-splits}
\label{fig:BasicTreeExample}
\end{figure}

The principal operation on trees is a sum function which is analogous to the append function for lists, combining all values from two trees. The simplest way to combine two trees, one which yields a monad, has us nest one tree in the other. That is, we replace the leaf nodes of one tree with copies of the other tree, and then flatten. For definiteness, and to ensure associativity, we always nest the right tree in the left. See Figure~\ref{fig:TreeNesting}.

\begin{figure}
\centering\includegraphics[scale=0.7]{automation/FullSimpTree1}
\caption{Combining Trees by Nesting}
\label{fig:TreeNesting}
\end{figure}

Even if we only had a constant number of case-splits, successive combining in this manner would yield trees of arbitrarily large topology, which is clearly not desirable. As such, we need some way to simplify the resulting trees. Our hard constraint is that we must not lose any information: if a theorem is deleted from the tree, it must be because it is trivially implied by other theorems in the tree. Our weaker and more wooly constraint is that the topologies should represent a reasonably efficient partitioning of the proof-space according to the combination of case-analyses. 

Our first means of simplification is a form of weakening. If we have tree $t$ which branches on a disjunctive hypothesis and which contains a subtree $t'$ branching on that same hypothesis, then the root path is of the form $P \implies \cdots \implies P \implies \phi$. We eliminate the redundant antecedent by folding $t'$ into its immediate parent. Any siblings of $t'$ whose branch labels are not duplicated along the root path are then discarded. They will be left in the other branches of $t$.

The situation is shown in Figure~\ref{fig:TreeWeakening}, where we have coloured duplicate hypotheses along root paths. In the result, we fold the marked subtrees into their parents, and discard the siblings. Notice that, at this stage, no theorems have been lost.

\begin{figure}
\centering\includegraphics[scale=0.7]{automation/FullSimpTree2}
\caption{Weakening}
\label{fig:TreeWeakening}
\end{figure}

Our next simplification allows us to delete a subtree $t$ if its branch case is already considered by an uncle $t'$. All theorems of $t$ will appear in $t'$ where the topology will have been simplified in our weakening step. The situation is shown in Figure~\ref{fig:TreeRedundantSplits}. The $P$ branch on on the left-hand side is uncle to the $P$ branches on the right. These latter branches are therefore pruned.

\begin{figure}
\centering\includegraphics[scale=0.7]{automation/FullSimpTree3}
\caption{Removing Redundant Case-splits}
\label{fig:TreeRedundantSplits}
\end{figure}

Finally, we can promote data that appears at all branches. In Figure~\ref{fig:TreeRedundantSplits}, the $xs'$ bag of theorems appears in every node at the same level, and so can be promoted into the parent, corresponding to disjunction elimination. 

Our final tree is shown in Figure~\ref{fig:TreePromoting}. Though we describe our simplification in several steps, they can in fact be combined into a single pass of the left hand tree. There is a limitation of the algorithm, in that it cannot promote all data. If it could, the replicated $us'$ bag in the bottom right branches of the tree could be promoted into the $Q$ branch. We will not elaborate on this implementation issue.

\begin{figure}
\centering\includegraphics[scale=0.7]{automation/FullSimpTree4}
\caption{Promoting Common Data}
\label{fig:TreePromoting}
\end{figure}

\subsubsection{Joining Trees}
Our function which combines trees is a sum function, which has an identity in the empty tree containing a single empty root node. This is enough structure to define a \emph{join} function analogous to list concatenation. Suppose we are given a tree $t$ whose nodes are themselves trees (so the type is $\code{Tree}\;(\code{Tree}\; \alpha)$). Denote the inner trees by $t_0$, $t_1$, $t_2$, $\ldots$, $t_n : \code{Tree}\;\alpha$. We now replace every node of $t$ with an empty conjunction, giving a new tree $t'$ with the \emph{new} type $\code{Tree}\;\alpha$. We can now form the sum 

\begin{displaymath}
t' + t_0 + t_1 + t_2 + \cdots + t_n
\end{displaymath}

The resulting tree will then contain discovered theorems which respect disjunctive hypotheses from their place in $t$ and from their respective inner-trees.

\section{Additional Primitives and Derived Discoverers}\label{sec:Additional}
Once we replace Spivey's bags with our trees, we have a stream data-type to represent a theorem discoverer searching a space of theorems which have been partitioned according to case-splits. In fact, discoverers generalise over whatever type of values they are discovering, and so we give them the polymorphic type $\code{discoverer}\ \alpha$.  

As we described in \S\ref{sec:StreamMonad}, these discoverers form a monoid. Operationally, the sum function runs two discoverers in parallel, respecting their mutual case-splits, while our identity discoverer generates nothing but empty generations.

\subsection{Case-splitting}
Case-splits are introduced by $\code{disjuncts}$, which is a discoverer parameterised on arbitrary theorems. Here, $\code{disjuncts}\left(\vdash P_0 \vee P_1 \vee \cdots \vee P_n\right)$ outputs a single tree with $n$ branches from the root node. The $i^{th}$ branch is labelled with the term $P_i$ and contains the single theorem $P_i \vdash P_i$. This process can be undone by flattening trees using $\code{flatten}$, which discharges all tree labels and adds them as explicit antecedents on the theorems of the respective subtrees.

\subsection{Delaying}
A generated value can be ``put off'' to a later generation using the function $\code{delay}$. In terms of streams, this function has the trivial definition
\begin{displaymath}
  \code{delay}\ xs = \cons{\emptyset}{xs}.
\end{displaymath}
The use of this function is actually essential when writing mutually recursive streams. If values of type $\alpha$ are generated in a stream from values of type $\beta$ in another stream, and \emph{vice-versa}, then one of the two streams must be delayed to avoid deadlock.

\subsection{Filtering}\label{sec:Filtering}
In many cases, we will not be interested in all the outputs generated by a discoverer. Fortunately, filtering is a library function for monads with a zero element, and can be defined as:
\begin{align*}
xs >>= f &= \code{join}\;(\code{fmap}\;f\;xs) \\
\code{filter}\;p\;xs &= xs >>= (\lambda x.\;\code{if } p\;x \code{ then } \code{return}\;x \code { else } 0)
\end{align*}

More challenging is a function to perform something akin to \emph{subsumption}. The idea here is that when a theorem is discovered which ``trivially'' entails a later theorem, the stronger theorem should take the place of the weaker. This is intended only to improve the performance of the system by discarding redundant search-paths. 

We can generalise the idea to arbitrary data-types, and parameterise the filtering by any partial-ordering on the data, subject to suitable constraints. One intuitive constraint is that a stronger item of data should only replace a weaker item so long as we don't ``lose'' anything from later discovery. Formally, we require that any function $f$ used as the first argument to \code{fmap} is monotonic with respect to the partial-order. That is, if $x \leq y$ then $f\,x \leq f\,y$. 

We then implement a ``subsumption'' function in the form of the transformation \code{maxima}. This transforms a discoverer into one which does two things: firstly, it transforms every individual generation into one containing only maxima of the partial-order. Secondly, it discards data in generations that is strictly weaker than some item of data from an earlier generation. To handle case-splits, we assert that data higher up a case-splitting tree is always considered stronger than data below it (since it carries fewer case-splitting assumptions).

\subsubsection{More Sophisticated Filtering}
There are two pieces of missing functionality. Suppose we have theorems $\vdash \phi$ and $\vdash \psi$ where $\vdash \phi \leq \;\vdash \psi$ and a discoverer $xs = \left[\{\vdash \phi\}, \{\vdash \psi\}\right].$ That is, $xs$ is a discoverer which produces two generations of theorems. In the first generation is the single theorem $\vdash \phi$ and in the second is the single theorem $\vdash \psi$. Now suppose we have a monotonic function $f$ which happens to have the mappings
\begin{align*}
\vdash\phi &\mapsto \vdash \left[\{\},\{\},\{\},\{\},\{\phi^{*}\}\right]\\
\vdash\psi &\mapsto \vdash \left[\{\psi^{*}\},\{\}\right]
\end{align*}

Here, $\psi$ is sent immediately to a new value $\psi^{*}$, while the image $\phi^{*}$ of $\phi$ takes some time to generate, represented here by a succession of four empty generations. Thus,
\begin{displaymath}
xs >>= f = \left[\{\psi^{*}\}, \{\}, \{\}, \{\}, \{\phi^{*}\}\right].
\end{displaymath}

What happens when we want to take the maxima of this result? To do this, $f$ must be monotonic, and to verify this, we need some way to partially order the images of $f$, which means knowing more generally how to partially order discoverers.

In general, if two discoveres $xs$ and $ys$ are such that $xs \leq ys$, what should we say this implies about the ordering of the items discovered by each? We should be able to say at least this much: for every $x$ in the first $n$ generations of $xs$, there should exist some $y$ in the first $n$ generations of $ys$ such that $x \leq y$. Thus, $ys$ generates facts which are at least as strong as those of $xs$, and does so at least as early in terms of the number of generations.

Thus, if by monotonicity we require that $\left[\{\},\{\},\{\},\{\},\{\phi^{*}\}\right] \leq \left[\{\psi^{*}\},\{\}\right]$, then we further require that $\phi^{*} \leq \psi^{*}$. Therefore:
\begin{displaymath}
\code{maxima}\ xs >>= f = \code{maxima}\ \left(\left[\{\psi^{*}\}, \{\}, \{\}, \{\}, \{\phi^{*}\}\right]\right) = \left[\{\psi^{*}\}, \{\}, \{\}, \{\}, \{\}\right].
\end{displaymath}
And here we have a problem: the delayed and potentially slow computation of $\vdash \phi^{*}$ was wasted. Once $\vdash \psi$ was discovered, further expansion of the discoverer which depends on $\vdash \phi$ should have been halted. This does not happen in our implementation, and leads to a great deal of wasted effort. Consider the fact that every time we successfully apply rule~\ref{rule:collunion} to take the union of sets $S$ and $T$, all further discovery based on $S$ and $T$ should be abandoned in favour of discovery using $S \cup T$. A similar issue applies to the discovery of equalities: as new equalities are discovered, ongoing discovery based on any unnormalised term should be abandoned.

A second piece of missing functionality is a form of \emph{memoisation}. Suppose two generations of a discoverer are evaluated to $[\{\phi\}, \{\psi\}]$ where $\phi \leq \psi$. With $\psi$ evaluated, we would probably prefer any reevaluation of this discoverer to actually \emph{replace} $\phi$, yielding $[\{\psi\}, \{\}]$. 

This additional functionality has not yet been implemented, and might well require significantly modifying the underlying data-structures for our streams. We leave such possibilities to future work.

\subsection{Accumulating}
We supply an accumulation function which is similar to the usual \code{fold} function on lists and streams. This threads a two-argument function through the values of a stream, starting with a base-value, and folding each generation down to a single intermediate value. Thus, we have:

\begin{align*}
\code{accum}\;(+)\;0\;[\{1,2\}, \{3,4\}, \{5\}, \{6,7,8\}] 
&= [\{0 + 3\}, \{3 + 7\}, \{10 + 5\}, \{15 + 21\}]\\
&= [\{3\}, \{10\}, \{15\}, \{36\}]
\end{align*}

One useful case of this allows us to gather up all facts discovered so far in a single collection. If the collection is lists, we just use $\code{accum}\;(\lambda xs\;x.\;x :: xs)\;[\,]$.

% \subsection{Normalisation}
% For efficiency, it will be useful to normalise with respect to any discovered equalities and simple rewrite rules. 

\subsection{Deduction}
Direct deduction, or Modus Ponens, is the basis of forward-chaining and we provide two main ways to lift it into the type of discoverers. In our theorem-prover, the Modus Ponens inference rule can throw an exception, so we first redefine \code{fmap} to filter thrown exceptions out of the discovery. It is generally undesirable to let exceptions to propagate upwards, since this would lead to an entire discoverer being halted. 

With \code{fmap} redefined as \code{fmap'}, we can define functions $\code{fmap2}'$ and $\code{fmap3}'$ which lift two and three-argument functions up to the level of discoverers, also filtering for exceptions.
\begin{align*}
\code{fmap}'\;f\;xs &= xs >>= (\lambda x.\;\code{try}\; \code{return}\;(f\;x)  \code{ with \_ } \rightarrow 0)\;xs\\
\code{fmap2}'\;f\;xs\;ys &= \code{fmap}\;f\;xs >>= (\lambda f.\;\code{fmap}'\;f\;ys)\\
\code{fmap3}'\;f\;xs\;ys\;zs &= \code{fmap}\;f\;xs >>= (\lambda f.\;\code{fmap2}'\;f\;ys\;zs)
\end{align*}
With these, we can define the following forward-deduction functions:
\begin{align*}
\code{chain1}\;imp\;xs &= \code{fmap2}'\; \code{MATCH\_MP}\; (\code{return}\;imp)\;xs \\
\code{chain2}\;imp\;xs\;ys &= \code{fmap2}'\; \code{MATCH\_MP}\; (\code{chain1}\; imp\;xs)\;ys\\
\code{chain3}\;imp\;xs\;ys\;zs &= \code{fmap2}'\; \code{MATCH\_MP}\; (\code{chain2}\;imp\;xs\;ys)\;zs\\
\code{chain}\;imps\;xs &= imps >>= (\lambda imp.\;\code{if is\_imp } imp\\ &\code{then } \code{chain}\;(\code{fmap}\;(\code{MATCH\_MP}\;imp)\;thms)\;thms\\
&\code{else } \code{return}\; imp)
\end{align*}

The function \code{is\_imp} returns true if its argument is an implication, while $\code{MATCH\_MP}\;imp$ is a function which attempts to match the antecedent of $imp$ with its argument. Thus, \code{chain1} applies a rule of the form $P \implies Q$ across a discoverer of antecedents. The function \code{chain2} applies a rule of the form $P \implies Q \implies R$ across two discoverers of antecedents. The function \code{chain3} applies a rule of the form $P \implies Q \implies R \implies S$ across three discoverers of antecedents. The final, more general function, \emph{recursively} applies rules with arbitrary numbers of curried antecedents from the discoverer \code{imps} across all possible combinations of theorems from the discoverer $xs$.%, reproducing the behaviour of \sec{sec:Example}.

Note that the discoverers \code{chain1}, \code{chain2}, \code{chain3} will not necessarily try all combinations of theorems from their arguments. They fail opportunistically, attempting Modus Ponens on each theorem from the first argument, and \emph{only if it succeeds}, attempting Modus Ponens on facts from the second argument. This is a happy feature of the data-driven semantics of monads (but see \S\ref{sec:Applicative} for its drawbacks). 

It is therefore sensible to order the antecedents of the implication according to how likely they are to succeed. Antecendents which rarely succeed should appear early to guarantee a quick failure.

% CHAIN_MP, CHAIN1, CHAIN2, CHAIN3

% Not listed are functions based on Modus Ponens (\code{mp}). These functions we have called \code{chain1}, \code{chain2}, \code{chain3}, and so on, each written in one line of code in terms of its predecessor. The function \code{chain\_n} takes an implicational rule with $n$ antecedents, and yields a function which takes $n$ antecedent discoverers to produce a conclusion discoverer. 

% Our implementation allows for some interesting control of the search when discoveres are combined in this way. Antecedents in the original rule are matched one at a time, and conjunctions are split after each match. By carefully ordering and interleaving antecedents in the original rule around conjunctions, the user can control the search strategy. For example, suppose we create functions of three chains by applying 

% \noindent\code{   chain3 : thm $\rightarrow$ thm chain $\rightarrow$ thm chain $\rightarrow$ thm chain $\rightarrow$ thm chain} 

% \noindent to the equivalent rules:
% \begin{align*}
% &(P \rightarrow S \rightarrow Q \rightarrow X) \wedge (Q \rightarrow T \rightarrow P \rightarrow Y)
% \shortintertext{and}
% &P \rightarrow Q \rightarrow (S \rightarrow X \wedge T \rightarrow Y)
% \end{align*}

% The two resulting functions will differ in how they perform search on their three argument chains, in a potentially significant way. The first will simultaneously try to match $P$ and $Q$ in the first discoverer. Successful matches will respectively match $S$ and $T$, and then finally $Q$ and $P$. However, the second function will only try to match $P$. If successful, it will then match $Q$, and if that is successful, it will try to match $S$ and $T$ simultaneously. 

% The first formulation may be preferred when $P$ and $Q$ are equally likely to match and matches against $S$ and $T$ depend respectively on whether $P$ and $Q$ match. The second formulation would be preferred if $P$ is much less likely to match than $Q$ and there is no such dependence.

% This completes our overview of the discovery language. A user can now build discovery engines directly from these functions and primitives. This is very much in the spirit of HOL~Light and its LCF~\cite{LCF} foundation, where inference rules, tactic languages and rewrite engines are ordinary functions and combinator languages~\cite{Tactics}. Shallow-embedded tools can be a challenge for new users, but they offer enormous expressive power in a rich programming language, and allow different parts of the system to seamlessly integrate. 

\section{Integration}
It is straightforward to integrate our discoverers with the rest of HOL~Light's proof tools. We can, for example, lift term-rewriting to the level of discoverers simply by lifting the rewriting functions with \code{fmap} and its derivatives. 

To use our discoverers in declarative proofs, we introduce two Mizar~Light primitives. The first, \code{obviously}, is used to assist any step in a declarative proof via a discoverer.
\begin{displaymath}
  obviously\;:\;(\code{discoverer}\ \alpha\rightarrow\code{discoverer}\ \alpha) \rightarrow step \rightarrow step.
\end{displaymath}

The expression \code{obviously f} transforms a step into one which lifts the step's justifying theorems into a discoverer, applies the transformation f to this discoverer, and then retrieves all discovered theorems to a certain depth. These are then used to supplement the theorem's justification.

The use of a function $f$ of type $\code{discoverer}\ \alpha\rightarrow\code{discoverer}\ \alpha$ allows us to easily give a discovery pipeline via function composition to justify a declarative step. To give an example, later on in our formal development we will introduce an incidence discoverer $\code{by\_incidence}$. We also have a function $\code{split}$  which recursively breaks conjunctions and splits disjunctions and, finally, we have a function $\code{add\_triangles}$ which introduces non-collinearity claims from theorems about points inside and outside triangles. The three functions can be pipelined in a single step by writing
\begin{displaymath}
\code{obviously}\ (\code{by\_incidence}\circ \code{add\_triangles}\circ \code{split}) \ldots.
\end{displaymath}

Next, we introduce a keyword $\code{clearly}$. This has the same type as $\code{obviously}$, but rather than collecting all theorems generated by a discoverer, it searches specifically for the theorem that the user is trying to justify in the current step (in the case of a $\code{qed}$ step, this is just the goal theorem). The keyword will not help in proofs unless the user knows the form of theorems produced by the discoverer, but when it can be used, it leaves the basic Mizar~light justification tactic with a trivial proof.

Finally, we have introduced a theorem-tactic \code{discover\_tac} with type
\begin{displaymath}
\code{discover\_tac}\;:\;(\code{discoverer}\ \alpha\rightarrow\code{discoverer}\ \alpha) \rightarrow [\code{thm}] \rightarrow \code{tactic}.
\end{displaymath}
As with $\code{obviously}$ and $\code{clearly}$, we take a transformation of discoverers. Then when we apply $\code{discover\_tac}\ f\ tac$,  all goal hypotheses are fed to $f$. The resulting discovered facts are then supplied to the function $tac$ (often just $\code{MESON}$).

\subsection{Concurrency}
In \S\ref{sec:NaiveConcurrency}, we described how we could easily exploit concurrency with our n\"{a}ive forward-chaining algorithm. Things are even better for our discovery algebra. Since we are using lazy lists pervasively, theorems can be output to the terminal one-by-one as they are generated, without having to wait for an entire generation at once.

It is also trivial to define a new primitive discoverer $\code{monitor}$ which can monitor the proof-state. As we stated in our introduction in \S\ref{sec:Streams}, primitive discoverers can generate their outputs in whatever manner they want. Our $\code{monitor}$ just regularly examines the proof context and outputs its hypotheses. We ensure that only unique hypotheses are ever generated by using the function $\code{maxima}$.

To complete the architecture, we supply basic commands that handle the thread communication necessary to tell an existing concurrent discoverer to pause, resume, or to change to an alternative theorem discoverer.

% \begin{figure}
% {\scriptsize\input{incidence}}
% \caption{Incidence Discovery in ML}
% \label{fig:MLEngine}
% \end{figure}

\subsection{Dependency Tracking}
While writing a proof, we would normally start an incidence discoverer concurrently, which could generate theorems as we worked. The discoverer would consider \emph{all} of our goal hypotheses, and would typically find a large number of theorems. Most of these theorems are not needed in the proof. Of those which are needed, we would like to know the specific subset of hypotheses from which they are derived. This is desirable when it comes to proof-replay, when we would like the discoverer to work more efficiently with just the hypotheses it needs. But it is also desirable in terms of writing \emph{declarative} proofs: we want to write the dependent hypotheses directly into the proof script, so they are apparent to the reader. To achieve this, we need our discoverers to  \emph{track} hypotheses.

A nice feature of monads in functional programming is that, when defined in the form of a \emph{transformer}, the extra book-keeping can be added in a clean and modular way. In this case, we create a generic \emph{writer transformer}.

\emph{Writer} is a monad with extra functions for \emph{writing} values from any monoid in a computation, and for running computations to retrieve the final written values. It can be seen as a modular way to introduce things such as \emph{logging} into computations. So if we have a computation which writes the value \code{"Hello"} and returns the value 1, and another computation which writes the value \code{" world!"} and returns the value 2, then when we lift addition over the two computations we have a computation which returns the value 3 and writes \code{"Hello world!"}. The monoid here is strings with the append operation.

Writer can easily be defined as a transformer which, when applied, is made to write values \emph{inside} any other monad. We made our writer work on a monoid of hypotheses sets with the union operation. These sets contain the dependent hypotheses for every discovered theorem, and they automatically accumulate as theorems are combined. We do not need to write any special logic for this. All the details are handled generically by the writer transformer.

All we need do is extend the $\code{monitor}$ discoverer to initialise the dependent hypotheses sets. That is, every time $\code{monitor}$ pulls a hypothesis from the goal context, it must also \emph{write} the hypothesis as a dependency on further computation. The dependencies then automatically propagate. 

\section{Implementation Details}
The following section describes some technical challenges we faced. We include it for the sake of honesty, since the presentation of the ideas so far ignores all the bumps that thwart a nice clean implementation in Ocaml. We assume some knowledge of the Ocaml language here.

We have implemented a general monad library in Ocaml, providing a signature for the minimal implementation of an arbitrary monad, and a functor to turn this minimal implementation into a richer monad interface with derived functions. Monad transformers are functors between minimal implementations of monads. The stream monad itself is a transformer from an arbitrary monad of generations. If we want to use Spivey's original implementation, we can just supply a module of bags to this transformer. If we want to use our case-splitting implementation, we supply our module of trees. 

These transformations can be stacked. In fact, in our final implementation, we first apply our writer transformer to a monoid of hypotheses sets. The writer transformer is then applied to our tree monad. Finally, the tree monad is applied to our stream monad to produce a stream which discovers theorems, handles case-splits and tracks dependent hypotheses. \footnote{Our code, including our revisions of Ocaml's lazy list library can be made available. It comes with Ocamldoc and can be installed with the \code{Ocamlfind} toolchain. Please contact \url{phil.scott@ed.ac.uk}.}

\subsection{Implementation Issues}
As of writing, HOL~Light relies heavily on modifications to Ocaml's preprocessor, one of which forces Ocaml's lexer to treat any lexeme which contains more than one upper-case letter as being a ``lower-case identifier''. This plays havoc with the CamelCase naming convention now being adopted in Ocaml's standard library, since Ocaml's parser expects all module names to be uppercase. To circumvent this, we supply our own preprocessor which allows for lower-case identifiers in module names. This is only intended as a hack over a hack until a more robust solution presents itself.

We have stressed ``concurrency'' in this chapter, rather than ``parallelisation'', since Ocaml only supports the former. This means we get no speed improvements by using threads. This should not be seen as a significant drawback, since such multithreading is inherently dangerous in a language such as Ocaml which allows for uncontrolled side-effects. HOL~Light itself was not developed with multiprocessing support in mind. Its basic \code{MESON} tactic, for instance, is not thread-safe. Even without parallelisation, the tactic must be avoided when implementing discoverers.

Another issue with threading concerns UNIX signals. The standard way to interact with HOL~Light is to run semi-decidable decision procedures on problems. If these procedures fail, the user interrupts them at the Ocaml top-level by sending \code{SIGINT}. If the user is running a concurrent discoverer when they send this signal, it might interrupt the wrong thread, possibly leaving the discoverer with dangling resources. As a quick hack around this behaviour, we trap SIGINT in the discovery thread and simply discard it.

Finally, Ocaml's lazy list library has some unexpected implementation choices. For instance, the list concatenation function $concat\;:\;[[\alpha]]\rightarrow[\alpha]$ is \emph{strict} in its outer list, while the $\code{take}$ function which takes a certain sized prefix of a given lazy list will always force the values in the prefix. These behaviours are generally inappropriate for lazy data-structures, and their use often leads to infinite looping when we come to write recursive functions which generate lazy lists. We have had to rewrite many of them.

Besides these issues, the syntax for lazy lists in Ocaml is extremely cumbersome. Primitive list functions have to explicitly force lazy constructors, while recursive lazy lists must be wrapped with a \code{lazy} keyword. One benefit, however, of Ocaml's lazy list implementation is that it detects when a recursive definition requires forcing a vicious circularity. These actually arise quite easily when writing discoverers in our framework, and are fixed with choice uses of the $\code{delay}$ function.

\section{Applicative Functors}\label{sec:Applicative}
An idiom or applicative functor~\cite{Applicative} $F$ is a generalisation of the monad, which could be thought of as providing at least the ability to lift values $\alpha \rightarrow F\ \alpha$ and to lift a two argument function as we do with 
\begin{displaymath}
\code{fmap2}\;:\;(\alpha\rightarrow\beta\rightarrow\gamma)\rightarrow F\ \alpha \rightarrow F\ \beta \rightarrow F\ \gamma.
\end{displaymath}

This is actually quite limiting compared to the monad (see \cite{IdiomsArrowsMonads} for a detailed analysis). We can no longer write data-driven computations, such as our $\code{chain}$ functions which automatically fail at the first non-matching antecedent.

That said, because it does not allow data-driven computation, an implementation of the applicative functor interface can potentially be more efficient than the implementation it derives as a generalisation of a monad. This was something we spotted when fixing a performance issue with our trees.

In Figure~\ref{fig:ApplicativeSimp}, we show what happens when we lift a two argument function $f$ over two trees with the same topology. The result is correctly simplified, so that it appears that the function $f$ is evaluated exactly five times. But, bizarrely, when we came to profile our code, we found that the function $f$ was evaluated \emph{nine} times, once for each possible combination of values in the two trees. The results were discarded in simplification, but only after they were evaluated. This completely breaks the intended purpose of the trees, which is to partition the values so that the discarded computations never take place.

\begin{figure}
\centering\includegraphics[scale=0.7]{automation/BasicTreeSimp}
\caption{Applicative Simplification}
\label{fig:ApplicativeSimp}
\end{figure}

The inefficiency is actually unavoidable with the monad implementation. A computation in a monad produces structure dependent on \emph{data} within another structure. When we lift a two argument function over two structures, it happens that the structure of the final value is independent of this data, but the independence cannot be guaranteed by the type system.

This is the advantage of the applicative functor. In defining it, we are not allowed to ``peek'' at the data inside our structures, so when lifting a two argument function, the type system can guarantee that the final structure can only depend on the structures of the two inputs. 

For our trees in Figure~\ref{fig:ApplicativeSimp}, we know, without looking at the data, what the topology of the final result should be. We know, in advance, what simplifications should come in, and we therefore know, in advance, how many times the function $f$ will be applied. But this knowledge is only guaranteed for applicative functors. When we use the derived monad implementation, $f$ must be applied across all computations and only \emph{then}, when all dependencies on the data have been take into account,  can simplification take place.

A simple fix is to provide an explicit applicative implementation which will override the derived monad implementation. The resulting overrided interface is identical from the perspective of client code, but now, when the client does not require data driven computations, such as with the applicative functions $\code{fmap2}$, $\code{fmap3}$ and so on, the more efficient implementation is used.

\section{The Problem Revisited}\label{sec:Solution}
We now return to our original data-flow problem. In Figure~\ref{fig:IncidenceDiscoveryCode}, we use our stream algebra to capture the complex data-flow network from Figure~\ref{fig:DataFlow}. The five kinds of theorem now correspond to five primitive discoverers. Inference rules are applied across the theorems by lifting them into the type of discoverers, and mutual dependencies of the network are captured by mutual recursion\footnote{The inference rule \code{CONJUNCTS} sends a conjunctive theorem to list of its conjuncts.}.
\begin{figure}
\small
\begin{align*}
&\hspace{-2.55cm}\code{ssum} = \code{foldr}\ (+)\  0\circ \code{map}\ \code{return}\\\\
\code{by\_incidence}\ thms = \\
\quad \code{let rec}\ collinear &= \code{maxima}\ (\code{filter}\ \code{is\_collinear}\ thms\\
&\qquad + \code{fmap3'}\ \code{colcol}\ (\code{delay}\ collinear)\ (\code{delay}\ collinear)\ neqs)\\
\code{and}\ non\_collinear &= \code{maxima}\ (\code{filter}\ \code{is\_non\_collinear}\ thms\\
&\quad + \code{fmap3'}\ \code{colncolncol}\ collinear\ (\code{delay}\ non\_collinear)\ neqs)\\
\code{and}\ eqs &= \code{filter}\ \code{is\_eq}\ thms\\
&\quad\qquad+ \code{maxima} (\code{sum}\ (\code{fmap3'}\ \code{coleq}\\
&\quad\ collinears\ collinear\ non\_collinear))\\
\code{and}\ neqs &= \code{maxima} (\code{filter}\ \code{is\_neq}\ thms\\
&\quad + \code{sum}\ (\code{fmap2'}\ \code{colncolneq}\ collinear\ (\code{delay}\ non\_collinear))\\
&\quad + \code{sum}\ (\code{fmap'}\ \code{CONJUNCTS}\ (\code{chain1}\ \code{ncolneq}\ non\_collinear)))\\
\code{and}\ planes &= \code{maxima}\ (\code{filter}\ is\_plane\ thms\\
&\quad + \code{fmap3'}\ \code{planeplane}\ (\code{delay}\ planes)\ (\code{delay}\ planes)\\
&\qquad\qquad\qquad\ non\_collinear\\
&\quad + \code{fmap3'}\ \code{colplaneplane}\ collinear\ (\code{delay}\ planes)\ neqs\\
&\quad+ \code{fmap2'}\ \code{colcolplane}\ collinear\ collinear\\
&\quad+ \code{fmap'}\ \code{colplane}\ collinear\\
&\quad+ \code{fmap'}\ \code{ncolplane}\ non\_collinear)\\
\quad \code{in}\,collinear + &non\_collinear + eqs + neqs + planes
% &\code{collinear}\ thms = \code{fix}\ (\lambda cs.\  \code{chain3}\ \code{col\_union}\ (\code{not\_eqs}\ thms)\ cs\ cs)\ \\ &\quad(\code{filter}\ \code{is\_collinear}\ thms + (0 :: \code{rewrite}\ eqs\ (\code{collinear}\ thms)))\\
% \\
% &\code{planar}\ thms = \code{chain2}\ \code{colcol\_plane}\ (\code{collinear}\  thms)\ (\code{collinear}\ thms)\\
% &\quad + \code{fix}\;(\lambda ps.\;\code{chain3}\;\code{plane\_union}\;(\code{not\_collinear}\;thms)\;ps\;ps\\
% &\qquad + \code{chain3}\;\code{colplaneplane}\;(\code{not\_eqs}\;thms)\;(\code{collinear}\;thms)\;ps)\\
% &\quad + (\code{filter}\;\code{is\_planar}\;thms + (0 :: \code{rewrite}\;eqs\;(\code{planar}\;thms)))
\end{align*}
\caption{Incidence Discovery}
\label{fig:IncidenceDiscoveryCode}
\end{figure}

One advantage of our algebra is that it is almost trivial to refine the discovery system. For instance, we noticed that the network in Figure~\ref{fig:DataFlow} has some redundancy: point-inequalities delivered from non-collinear sets by the rule $\code{ncolneq}$ should not be used to try to infer \emph{new} non-collinear sets. We eliminated this redundancy by splitting \code{neqs} into two discoverers, \code{neqs} and \code{neqs'}. 
\begin{align*}
non\_collinear &= \code{maxima}\ (\code{filter}\ \code{is\_non\_collinear}\ thms\\
&\quad + \code{fmap3'}\ \code{colncolncol}\ collinear\ (\code{delay}\ non\_collinear)\ neqs')\\
\code{and}\ neqs &= \code{maxima}\ (\code{neqs'}\\
& + \code{fmap'}\ (\code{sum}\ (\code{CONJUNCTS}\ (\code{chain1}\ \code{ncolneq}\ (\code{delay}\ non\_collinear))))\\
\code{and}\ neqs' &= \code{maxima}\ (\code{filter}\ \code{is\_neq}\ thms\\
&\quad + \code{sum}\ (\code{fmap2'}\ \code{colncolneq}\ collinear\ (\code{delay}\ non\_collinear))\\
&\quad + \code{sum}\ (\code{fmap'}\ \code{CONJUNCTS}\ (\code{chain1}\ \code{ncolneq}\ (\code{filter}\ \code{is\_neq}\ thms))))
\end{align*}

\section{Conclusion}
In this chapter, we began by looking for a way to implement an automated incidence tool, suitable for use with declarative proof. We began with a n\"{a}ive approach, and implemented a simple forward-chaining algorithm. While very basic, the advantage of this algorithm was that it makes it almost trivial to implement a collaborative and concurrent proof system, where the user and theorem prover progress in the same direction towards the goal by sharing theorems.

To correct the various deficiencies and the ad-hoc nature of our implementation, we implemented a discovery algebra based on Spivey's stream monad. The advantages of using monads is that, firstly, they are a well-established pattern in functional programming, and secondly, they use combinators so that the defined languages integrate easily with the host language.

We showed how to integrate our algebra into the rest of HOL~Light, providing interfaces to the basic rewriter, the tactics and the Mizar~Light combinators. Finally, we showed how to express our complex data-flow for forward-chaining incidence reasoning using our combinators in what we hope is a reasonably clear manner.

We leave the detailed evaluation of our incidence reasoner to the next chapter, where we shall apply it against some proofs from Hilbert's text.

\section{Further Work}
The theoretical underpinnings of our tree data structure still need some investigation. The correctness of our simplification algorithm needs to be explored and ideally formally verified. For now, we regard this as beyond the scope of our current work. We still need to devise a way to integrate proper subsumption into our algebra as mentioned in \S\ref{sec:Filtering}. A particular challenge here is finding an effective way to integrate normalisation with respect to derived equalities. We suspect any solutions here will require modifying our basic tree data-structures, and so it is perhaps premature to start working on proofs for our algebra. The most that we can say for now is that our incidence reasoner has been extensively tested on non-trivial problems.

Our discovery language does not yet provide functions for more powerful first-order and higher-order reasoning. Our domain was a relatively simple combinatorial space of concrete incidence claims, but in the future, we would like to be able to apply the system to inductive problems, having it speculate inductive hypotheses and infer universals. Since the basic discovery data-type is polymorphic and not specific to theorem-proving, we hope that lemma speculation will just be a matter of defining appropriate search strategies. We would also like to handle existential reasoning automatically, and we are still working on a clean way to accomplish this. 


%%% Local Variables: 
%%% mode: latex
%%% TeX-master: "../thesis"
%%% End: 


\chapter{Elementary Consequences in Group~II}\label{chapter:Group2Eval}
We now come to verify some theorems of Group~II, the only theorems in the first two groups which have prose proofs in the tenth edition of the \emph{Grundlagen der Geometrie}. Each proof uses Axiom~\ref{eq:g24}, which, as we explained in Chapter~\ref{chapter:Axiomatics}, carries several incidence preconditions. Establishing these preconditions made up the bulk of the effort in our manual verifications~\cite{ScottMScThesis} and in the verifications of Meikle and Fleuriot~\cite{MeikleFleuriotFormalizingHilbert}.

In the last chapter, we described some automation to handle the incidence reasoning.
An aim of the present chapter is to see how much this automation can help us shorten
our verifications, make it easier to develop them and make it possible to find alternative proofs. Our ideal is for the verification steps to match the steps of an ideal prose proof, modelled here by Hilbert's own arguments, which focus on defining and exhibiting the diagram needed to establish a result, rather than discharging tedious incidence preconditions.

When writing our verifications, we tried our best to follow the specific steps in Hilbert's prose, and in this chapter, we shall directly compare our verifications with the originals. As we shall see, this allows us to comment on the prose in great detail.

\section{THEOREM~3}\label{sec:Theorem3}
Hilbert's first result tells us that there is a point between any two others.

\begin{quotation}
THEOREM 3. For two points $A$ and $C$ there always exists at least one point $D$ on the line $AC$ that lies between $A$ and $C$.

PROOF. By Axiom~I, 3 there exists a point $E$ outside the line $AC$ and by Axiom~II, 2 there exists on $AE$ a point $F$ such that $E$ is a point of the segment $AF$. By the same axiom and by Axiom~II, 3 there exists on $FC$ a point $G$ that does not lie on the segment $FC$. By Axiom II,~4 the line $EG$ must then intersect the segment $AC$ at a point $D$.

\vspace{0.5cm}
\centering\includegraphics{group2eval/Three}
\flushright{\cite[p. 6]{FoundationsOfGeometry}}
\end{quotation}

The only incidence axiom Hilbert appeals to in this proof is Axiom~I, 3, and the only order axioms are Axiom~II, 2 and Axiom~II,~4. The proof depends on more axioms than this, but Hilbert's omission is consistent with a claim we made in Chapter~\ref{chapter:Axiomatics}: Hilbert generally only cites axioms which \emph{introduce} points, omitting the others because he wants to focus on the steps which build up the diagrams. Here, we have Axiom I, 3 which can be used (indirectly) to obtain points off arbitrary lines, in this case the point $E$. We have Axiom~II, 2 which is our ``line-extension axiom'', used here to first obtain the point $F$, and then to obtain the point $G$. Finally, we have Axiom~II,~4 which finds ``exit points'' of a line passing through a triangle, in this case, the point $D$.

THEOREM~3 was not proven in the first edition of the \emph{Grundlagen der Geometrie}; it was an axiom. That it turns out to be redundant might still come as a surprise, when we realise that we are obtaining such a simple linear result from a one-dimensional order axiom \eqref{eq:g22} and a two-dimensional order axiom~\eqref{eq:g24}. This is the situation in all three proofs we consider in this chapter. We will be proving one-dimensional results by obtaining two-dimensional figures and applying Pasch's axiom.

\subsection{Verification}
Our HOL~Light verification shown in Figure~\ref{fig:ThreeVerification} improves hugely on our manual verification~\cite{ScottMScThesis}, which ran to twenty-two steps. Here, we have just five steps, and are \emph{very} close to Hilbert's prose. We have just one extra step: our final \code{qed} eliminates the unwanted disjunct from Pasch's axiom.

\begin{boxedfigure}
  \begin{align*}
    &\code{assume}\ A\neq C\\
    &\code{so consider}\ E \code{ such that }\ \Triangle{a}{A}{C}{E}\\
    &\qquad\code{by}\ \eqref{eq:g12},\eqref{eq:g13b}& 0\\
    &\code{obviously}\ \code{by\_neqs}\ \code{consider}\ F\ \code{such that}\ \between{A}{E}{F}\ \code{from}\ 0\ \code{by}\ \eqref{eq:g22} & 1\\
    &\code{obviously}\ \code{by\_neqs}\ \code{so consider}\ G\ \code{such that}\ \between{F}{C}{G}\\&\qquad \code{from}\ 0\ \code{by}\ \eqref{eq:g22} & 2\\
    &\code{obviously by\_incidence so consider}\ D\ \code{such that}\\
    &\qquad(\exists a.\; \code{on\_line}\ E\ a \wedge \code{on\_line}\ G\ a \wedge \code{on\_line}\ D\ a)\\
    &\qquad \wedge (\between{A}{D}{C}\vee\between{F}{D}{C})\\
    &\qquad \code{using K (MATCH\_MP\_TAC \eqref{eq:PaschPointSetUnfold}) from}\ 0,1\\
    &\code{obviously}\ (\code{by\_eqs}\circ\code{split})\ \code{qed from}\ 0,1,2\ \code{by}\ \eqref{eq:g21},\eqref{eq:g23}
  \end{align*}
\caption{Verification of THEOREM~3}
\label{fig:ThreeVerification}
\end{boxedfigure}

Note that the verification presented here is in its final form, packaged from an interactive verification that was developed and assisted by concurrent discoverers. We will recount the general way we used the concurrent discoverers to produce these final packaged versions in \S\ref{sec:DiscoveryAtWork}, where we consider the verification of THEOREM~5.

In the verification, we cite two axioms that Hilbert did not. His Axiom~I, 3 can only be used \emph{indirectly} to find a point off an arbitrary line. Strictly speaking, one must also appeal to Axiom~\ref{eq:g12} as we have done. We have also cited Axiom~II, 1. This is only needed for the trivial matter of switching the order of the outer arguments to the \code{between} relation. 

We reference three separate discoverers in our proof: \code{by\_incidence}, \code{by\_neqs}, and \code{by\_eqs}. The first discoverer collects all five kinds of incidence sequent considered in the last chapter into a single discoverer, while the latter two just discover inequality and equality sequents respectively. The semantics of laziness makes this a genuine optimisation: if we only pull sequents from the \code{by\_neqs} discoverer, no sequents will be pulled from the \code{by\_eqs} or \code{by\_planes} discoverers, as we can see by looking again at the dependencies in our data-flow diagram (Figure~\ref{fig:DataFlow}).

Finally, we have used \code{MATCH\_MP\_TAC}\footnote{This standard HOL~Light function matches the conclusion of an implicational theorem with the goal formula, and then generates a new subgoal to prove the antecedent.} to apply our version of Pasch's Axiom~\eqref{eq:PaschPointSetUnfold} formulated entirely in terms of points (see Appendix~\ref{app:Group2}). This is slightly ugly, but it helps \code{MESON} by directing it to the antecedents of the matched theorem. It has no other side effects that carry over to the remaining steps, and its use does not break the declarative style since we state the implicational theorem that we have matched against. Even so, such matching is an irritation, since the free variables in the matched theorem must be lined up with those in the goal, and it is easy to get the order mixed up. We shall deal with this matter in \S\ref{sec:PaschDiscoverer}.

\subsection{The Outer and Inner Pasch Axioms}
THEOREM~3 was an axiom of the first edition of the \emph{Grundlagen der Geometrie}, and the full investigation of such redundant axioms can be credited to Veblen and his supervisor E.H. Moore, who investigated ordered geometry based on axioms very similar to Hilbert's. However, if we look at Veblen's system, we see he chose a different rendering of Pasch's Axiom.
\begin{quote}
  ``AXIOM VIII \emph{(Triangle traversal axiom). If three distinct points $A$, $B$, $C$ do not lie on the same line, and $D$ and $E$ are two points in the order $BCD$ and $CEA$, then a point $F$ exists in the order $AFB$, and such that $D$, $E$, $F$ lie on the same line.}''
\flushright{\cite[p. 355]{Veblenphd}}
\end{quote}

This form of the axiom, known as the \emph{Outer Pasch Axiom}, is due to Peano. It is strong enough to replace Hilbert's own version, and has advantages in terms of incidence reasoning: it carries fewer incidence preconditions at the expense of one extra order condition, and the conclusion is not disjunctive, so we are saved the effort of eliminating an offending case.

A related axiom (which turns out to be weaker \cite{PaschForms}), is also due to Peano. This is the \emph{Inner Pasch Axiom}, which exchanges the roles of $E$ and $F$ in the Outer Pasch Axiom. It can be verified as:
\begin{multline}\label{eq:InnerPaschWeak}
  \vdash\Triangle{a}{A}{B}{C} \\
\wedge \between{B}{C}{D} \wedge \between{A}{E}{B}\\ 
\implies \exists F.\;\exists a.\;  \code{on\_line}\ D\ a \wedge \code{on\_line}\ E\ a \wedge \code{on\_line}\ F \wedge \between{A}{F}{C}.
\end{multline}

It can now be seen that the diagram obtained in Hilbert's proof of THEOREM~3 is just obtaining the assumptions of this Inner Pasch Axiom. So when we verify Inner Pasch and use it as an alternative to Axiom~\ref{eq:g24}, we get the verification shown in Figure~\ref{fig:Theorem3Verification}, which does not need the final step we had before. In fact, had we spotted the factoring in our manual verification, we predict that we would have only needed eight, not twenty-two steps.

\begin{boxedfigure}
\begin{align*}
  &\code{assume}\ A\neq C\\
  &\code{so consider}\ E \code{ such that }\ \Triangle{a}{A}{C}{E}\\
  &\qquad\code{by}\ \eqref{eq:g12},\eqref{eq:g13b}& 0\\
  &\code{obviously}\ \code{by\_neqs}\ \code{consider}\ F\ \code{such that}\ \between{A}{E}{F}\ \code{from}\ 0\ \code{by}\ \eqref{eq:g22} & 1\\
  &\code{obviously}\ \code{by\_neqs}\ \code{so consider}\ G\\
  &\qquad\code{such that}\ \between{F}{C}{G}\ \code{from}\ 0\ \code{by}\ \eqref{eq:g22} & 2\\
  &\code{obviously by\_ncols}\ \code{qed from}\ 0,1\ \code{by}\ \eqref{eq:g21},\eqref{eq:InnerPaschWeak}
\end{align*}
\caption{Verification of THEOREM~3 using the derived Inner Pasch Axiom}
\label{fig:Theorem3Verification}
\end{boxedfigure}

\begin{figure}
\centering\includegraphics[scale=0.7]{group2eval/OuterPasch}
\caption{Veblen's versus Hilbert's Proof}
\label{fig:VeblenHilbert}
\end{figure}

Veblen's diagram replaces the Inner Pasch Axiom with the Outer Pasch Axiom, but is otherwise very similar. If we take a relabelling $B \mapsto C$, $C \mapsto F$, $D \mapsto G$ and $F \mapsto D$, then we see that Veblen's second use of Axiom~\ref{eq:g22} (the line extension axiom) differs from Hilbert's by finding the point $G$ on the other side of the segment $CF$ (see Figure~\ref{fig:VeblenHilbert}). Where Hilbert sets himself up to use the Inner Pasch Axiom, Veblen sets himself up to use the Outer Pasch Axiom:

\begin{quotation}\label{sec:VeblenThree}
[...]. Between every two distinct points there is a third point.

Proof. Let $A$ and $B$ be the given points [figure]. [...] there is a point $E$ not lying on the line $AB$. By [the line extension axiom] points $C$ and $D$ exist, satisfying the order-relations $AEC$ and $BCD$. Hence, by [the Outer Pasch Axiom], $F$ exists in the order $AFB$.
\flushright{\cite[p. 355]{Veblenphd}}
\vspace{0.5cm}
\end{quotation}

In conclusion, for those of us who judge Hilbert's argument for THEOREM~3 to be gappy because of missing incidence reasoning, we offer a clean way to bring it up to more pedantic standards. Before THEOREM~3, we suggest one first derive the Inner Pasch Axiom~\eqref{eq:InnerPaschWeak}, after which the proof follows more fluidly. This observation will be worth bearing in mind when we come to THEOREM~5 in \S\ref{sec:Theorem5}, where we shall derive stronger versions of both the Inner and Outer Pasch axioms.

\section{THEOREM~4}
In this section, we review how we used our discovery tool in an exploratory fashion, as we examine Hilbert's THEOREM~4. This result was another axiom in the first edition of the \emph{Grundlagen der Geometrie}, or more accurately, was incorporated into Axiom~\ref{eq:g23}:
\begin{quote}
  ``II, 3. \emph{Of any three points situated on a straight line, there is always one and only one which lies between the other two.''}
\flushright{\cite[p. 4]{FoundationsOfGeometryFst}}
\end{quote}

The ``only one'' part is all that is retained in the tenth edition. The existence part is given in a proof which Hilbert credits to Wald.
\begin{quotation}
  THEOREM~4. Of any three points $A$, $B$, $C$ on a line there always is one that lies between the other two.

  PROOF. Let $A$ not lie between $B$ and $C$ and let also $C$ not lie between $A$ and $B$. Join a point $D$ that does not lie on the line $AC$ with $B$ and choose by Axiom~II, 2 a point $G$ on the connecting line such that $D$ lies between $B$ and $G$. By an application of Axiom~II,~4 to the triangle $BCG$ and to the line $AD$ it follows that the lines $AD$ and $CG$ intersect at a point $E$ that lies between $C$ and $G$ . In the same way, it follows that the lines $CD$ and $AG$ meet at a point $F$ that lies between $A$ and $G$.

If Axiom~II,~4 is applied now to the triangle $AEG$ and to the line $CF$ it becomes evident that $D$ lies between $A$ and $E$, and by an application of the same axiom to the triangle $AEC$ and to the line $BG$ one realises that $B$ lies between $A$ and $C$.

\centering \includegraphics{group2eval/Four}
\flushright{\cite[p. 7]{FoundationsOfGeometry}}
\end{quotation}

\subsection{Discovering Applications of Pasch}\label{sec:PaschDiscoverer}
With Axiom~\ref{eq:g24} used a total of four times, and with the symmetry that appears in the diagram, we wanted to explore the proof of THEOREM~4 using our automated discoverers. Currently, our discoverers only tell us about the various incidence relations implied by our assumptions. This helps us discharge preconditions on Axiom~\ref{eq:g24}, but does not tell us directly which instantiations of this axiom can be applied.

We cannot define a new discoverer which finds possible applications of Axiom~\ref{eq:g24} by using our simple forward-chaining primitives $\code{chain1}$, $\code{chain2}$ and $\code{chain3}$. The problem is that our version of this axiom in point sets~\eqref{eq:PaschPointSet} has five free variables but its antecedents involving triangles and betweenness only fix three variables at a time. The remaining two variables can only be fixed by matches up to associativity and commutativity, which \code{MATCH\_MP} does not support.

Instead, we defined a new discoverer \code{by\_pasch} with a combination of ML and monad library functions. We filter for betweenness sequents from an existing discoverer and use these to discharge one of the preconditions of Axiom~\ref{eq:g24}. We then reuse the discoverer \code{by\_ncols} to discharge the triangle preconditions, manually handling the ordering of the free variables each time a precondition is eliminated. 

The \code{by\_pasch} discoverer therefore outputs sequents which are the conclusions of Axiom~\ref{eq:g24}, telling us where the axiom can be applied. At the point in Hilbert's proof where the axiom is first used, the discoverer finds only one other possibility. Hilbert uses one after the other.
\begingroup
\allowdisplaybreaks
\begin{gather*}
\begin{split}
&\exists F.\; (\exists a.\; \code{on\ line}\ C\ a \wedge \code{on\ line}\ D\ a \wedge \code{on\ line}\ F\ a)\\
&\qquad \wedge (\between{A}{F}{B} \vee \between{A}{F}{G}).
\end{split}\\
\begin{split}
&\exists F.\; (\exists a.\; \code{on\ line}\ A\ a \wedge \code{on\ line}\ D\ a \wedge \code{on\ line}\ F\ a)\\
&\qquad \wedge (\between{B}{F}{C} \vee \between{C}{F}{G}).
\end{split}
\end{gather*}
\endgroup

After Hilbert applies the first of these, another three possibilities arise, depicted in Figure~\ref{fig:FourPossibilities}.
\begin{gather*}
\begin{split}\text{(a)}\qquad
&\exists F.\; (\exists a.\; \code{on\ line}\ C\ a \wedge \code{on\ line}\ D\ a \wedge \code{on\ line}\ F\ a)\\
&\qquad \wedge (\between{B}{F}{E} \vee \between{E}{F}{G}).
\end{split}\\
\begin{split}\text{(b)}\qquad
&\exists F.\;(\exists a.\; \code{on\ line}\ B\ a \wedge \code{on\ line}\ E\ a \wedge \code{on\ line}\ F\ a)\\
&\qquad \wedge (\between{A}{F}{C} \vee \between{A}{F}{G}).
\end{split}\\
\begin{split}\text{(c)}\qquad
&\exists F.\;(\exists a.\; \code{on\ line}\ B\ a \wedge \code{on\ line}\ E\ a \wedge \code{on\ line}\ F\ a)\\
&\qquad \wedge (\between{C}{F}{D} \vee \between{D}{F}{G}).
\end{split}
\end{gather*}

\begin{figure}
  \includegraphics[scale=0.9]{group2eval/FourPossibilities}
  \caption{Three possible applications of Axiom~II,~4}
  \label{fig:FourPossibilities}
\end{figure}

Cases (a) and (c) of Figure~\ref{fig:FourPossibilities} obtain the exact same point $F$, but in case (a) we should conclude that $F$ is between $B$ and $E$ while in (c) we should conclude that $F$ is between $C$ and $D$. If we apply both cases of the axiom, we will know that $F$ is between $B$ and $E$ and simultaneously between $C$ and $D$. Either of these cases yields an alternative proof of the theorem which we describe in \S\ref{sec:FourAlternative}.

It would be surprising if case (b) got us \emph{anywhere}. The only valid disjunct in its conclusion should put the point $F$ between $A$ and $C$, from which we could immediately conclude that $B = F$ and thus complete the proof. But this is all too easy. The truth of the matter is that incidence reasoning alone, according to our discoverers, cannot reject the impossible disjunct in the axiom's conclusion.

\subsection{Verifying Hilbert's Proof}
When we verified THEOREM~4 manually, it ran to sixty-nine steps. In this comparatively long verification, the structure of the basic prose argument is buried by incidence arguments which do not illuminate anything. The arguments consist mostly of applications of our point set rules given in \S\ref{fig:PointSets} with manual variable instantiations. We tried to use comments to show how the prose translated into the verification steps, but in the end, any claims of a faithful verification were weak.

But with our incidence automation, we have a verification in just thirteen lines, each readily understandable, and matching the prose very closely. The verification is shown in Figure~\ref{fig:FourVerification}, and we are again reasonably close to Hilbert's prose. The only warts are due to the \code{by\_pasch} discoverer not handling the case-split in the conclusion of Axiom~\ref{eq:g24}. This requires the elimination of a disjunct and the identification of the obtained point with an existing point. Both tasks are now handled in a subproof using a second discoverer, \code{by\_eqs}. 

We now review the verification. At the very start, we set our goal formula to be
\begin{multline*}
(\exists a.\; \code{on\_line}\ A\ a\wedge\code{on\_line}\ B\ a\wedge\code{on\_line}\ C\ a) \\\implies\between{A}{B}{C}\vee\between{B}{A}{C}\vee\between{A}{C}{B}.
\end{multline*}

We then get to \code{assume}, just as Hilbert does, that $C$ is neither between $A$ or $B$ nor $A$ between $B$ or $C$. This is a very natural way to express one's assumptions mathematically. It goes through because of the way Mizar~Light implements the \code{assume} primitive: it first tries to prove the goal under the negation of our assumption. If this is successful, the assumption is used to rewrite the goal before being placed into the goal hypotheses. The upshot is that our goal formula gets rewritten to
\begin{displaymath}
(\exists a.\; \code{on\_line}\ A\ a\wedge\code{on\_line}\ B\ a\wedge\code{on\_line}\ C\ a) \implies\between{A}{B}{C}.
\end{displaymath}

Next, we look at the first two applications of Axiom~II,~4. In our verification, we have not been able to apply this axiom in either the inner or outer forms, since we have only one order hypothesis. This means we are faced with having to eliminate a disjunct in the conclusion of Axiom~\ref{eq:g24}. To do this, we find point equalities via the composed discoverer $\code{by\_eqs}\circ\code{split}$, which tells us that, in the offending cases, the point $A$ lies between $B$ and $C$ or $C$ lies between $A$ and $B$. We have explicitly hypothesised against these two possibilities, and thus, the disjuncts can be eliminated.

\begin{boxedfigure}
\begin{align*}
& \code{assume}\ \exists a.\; \code{on\_line}\ A\ a\wedge\code{on\_line}\ B\ a\wedge\code{on\_line}\ C\ a & 0\\
& \code{assume}\ A\neq B \wedge A\neq C \wedge B \neq C & 1,2,3\\
& \code{assume}\ \neg\between{A}{C}{B} \wedge \neg\between{B}{A}{C} & 4\\
& \code{consider}\ D\ \code{such that}\ \Triangle{a}{A}{B}{D}\\
& \qquad\code{from}\ 1\ \code{by}\ \eqref{eq:g12},\eqref{eq:g13b}& 5\\
& \code{obviously by\_neqs}\ \code{so consider}\ G\ \code{such that}\ \between{B}{D}{G}\ \code{by}\ \eqref{eq:g22} & 6\\
& \code{consider}\ E\ \code{such that}\ (\exists a.\; \code{on\_line}\ A\ a\wedge \code{on\_line}\ D\ a\wedge \code{on\_line}\ E\ a) & 7\\
& \qquad\qquad\qquad\qquad\qquad\qquad \wedge\between{C}{E}{G}\ & 8\\
&\qquad \code{proof:}\ \code{clearly by\_pasch consider}\ E\ \code{such that}\\
&\qquad\qquad (\exists a.\; \code{on\_line}\ A\ a\wedge \code{on\_line}\ D\ a\wedge \code{on\_line}\ E\ a)\\
&\qquad\qquad \wedge (\between{B}{E}{C} \vee \between{C}{E}{G})\ \code{by}\ \eqref{eq:g21}\ \code{from}\ 0,2,3,5,6\\
&\qquad\code{obviously}\ (\code{by\_eqs}\circ\code{split})\ \code{qed from}\ 0,3,4,5\ \code{by}\ \eqref{eq:g21},\eqref{eq:g23}\\
& \code{consider}\ F\ \code{such that}\ (\exists a.\; \code{on\_line}\ C\ a\wedge \code{on\_line}\ D\ a\wedge \code{on\_line}\ F\ a) & 9\\
& \qquad\qquad\qquad\qquad\qquad\qquad \wedge\between{A}{F}{G}\ & 8\\
&\qquad[\ldots]\\
&\code{have}\ \between{A}{D}{E}\\
&\qquad \code{proof:}\ \code{obviously by\_ncols so consider}\ D'\ \code{such that}\\
&\qquad\qquad\between{C}{D'}{F}\wedge\between{E}{D'}{A}\\
&\qquad\qquad\code{using K (MATCH\_MP\_TAC \eqref{eq:InnerPasch})}\ \code{from}\ 0,2,5,6,8,10\ \code{by}\ \eqref{eq:g21}\\
&\qquad\code{obviously}\ (\code{by\_eqs}\circ\code{split})\ \code{qed from}\ 0,2,5,7,9\ \code{by}\ \eqref{eq:g21},\eqref{eq:g23}\\
&\code{obviously by\_ncols so consider}\ B'\ \code{such that}\\
&\qquad\between{G}{D}{B'}\wedge\between{C}{B'}{A}\\
&\qquad\code{using K (MATCH\_MP\_TAC \eqref{eq:OuterPasch})}\ \code{from}\ 0,2,5,7,\ \code{by}\ \eqref{eq:g21}\\
&\code{obviously}\ (\code{by\_eqs}\circ\code{split})\ \code{qed from}\ 0,2,5,6\ \code{by}\ \eqref{eq:g21},\eqref{eq:g23}\\
\end{align*}
\caption{Verification of THEOREM~4}
\label{fig:FourVerification}
\end{boxedfigure}\newpage

In these two applications of Axiom~\ref{eq:g24}, we have used our \code{clearly} keyword to ask the \code{by\_pasch} discoverer to search for a specific conclusion of the axiom. Exploiting our discoverer in this way makes things more robust than using \code{MATCH\_MP\_TAC} as we did in \S\ref{sec:Theorem3}. With matching, we have to be careful that all the variables in the axiom line up with our desired conclusion. But when we use the discoverer, we know that its outputs are always lexicographically normalised, so there is only one conclusion we could possibly be after.

The sought conclusion could usually be copied and pasted from the many results obtained by the \code{by\_pasch} discoverer when it was run concurrently during the interactive development, thereby selecting the intermediate result we want as a step in our verification and adding it to the proof context to be referenced later. Moreover, by using \code{clearly} and specifying the narrow set of justifying theorems needed to derive the conclusion, we make the pasted result the sole target for efficient search in proof replay.

The third and fourth applications of Pasch's axiom take the inner~\eqref{eq:InnerPasch} and outer~\eqref{eq:OuterPasch} forms respectively, and so we recommend the Outer Pasch Axiom as a useful lemma at this stage of Hilbert's exposition.

\subsection{Alternative Proof}\label{sec:FourAlternative}
Our proof tool was mostly used in a supporting role, but for Theorem~4 we allowed it to tackle the problem unaided, probing into the search space by applying Axiom~II,~4 non-deterministically. In this way, we hoped it would find alternative proofs of THEOREM~4, and, ideally, find one which required less than four applications of Axiom~II,~4. This defined a search limit for the problem: once four applications were found, it stopped searching in the relevant branch. 

We found several ``alternatives'', but most of these were symmetries of the original proof. In some cases, two independent applications of Pasch's axiom were applied in reverse order. In other cases, the proof was identical to the original up to a symmetric relabelling of the points. Only one new proof was revealed up to symmetry, and it corresponds to case (a) and (c) of Figure~\ref{fig:FourPossibilities}. We give it now in a prose formulation with an accompanying diagram.

\begin{proposition}
Between points $A$ and $C$ is a third point $B$.
\end{proposition}
\begin{proof}Assume $A$, $B$ and $C$ are collinear, with $A$ not between $B$ or $C$ and $C$ not between $A$ or $B$. We find a point $D$ off the line $AC$ and extend the segment $BD$ to $G$ using Axiom~II, 2. We then use Axiom~II,~4 on the triangle $BCG$ and the line $AD$ to find the point $E$ between $C$ and $G$. We use Axiom~II,~4 on the triangle $BEG$ and the line $CD$ to find the point $F$ between $B$ and $E$. We use the axiom again on the triangle $ABE$ and the line $CF$ to show that $D$ lies between $A$ and $E$. Finally, we can use the axiom on the triangle $ACE$ and the line $BG$ to find $B$ between $A$ and $C$.
\end{proof}
\begin{center}\includegraphics{group2eval/FourAlt}
\end{center}

The opening strategy of our alternative proof is the same as the original. We first construct a point $D$ off the line $AC$ and extend the segment $BD$ to $G$. This tells us that $D$ lies between $B$ and $G$, which gives us our first opportunities to use Axiom~\ref{eq:g24}. In both cases, our goal is to use this axiom in order to place the point $D$ between $A$ and $E$, so that a final application of Pasch's axiom to the triangle $ACE$ and the line $BG$ will place $B$ between $A$ and $C$. The two proofs only differ in how they construct the point $F$, and how they use $F$ to place $D$ between $A$ and $E$.

In Hilbert's proof, $F$ is found on the edge of the outer triangle $ACG$, and is placed symmetrically with $E$. Indeed, the proof is valid even after exchanging all references of $E$ and $F$, whereas in our proof, $F$ is placed in the interior of $ACG$ while $E$ lies asymmetrically on the triangle's edge. 

So Hilbert's proof has a lot of symmetry: $E$ could be replaced with $F$; and the third application of Pasch's axiom could be made on the triangle $CFG$ and the line $AE$, instead of $AEG$ and the line $CF$. Our proof makes it clear that, while $E$ and $F$ can be constructed symmetrically and independently, only one of these points is distinguished in the final few steps. 

It is worth drawing some attention to the subtlety of the incidence reasoning here. We could have applied Axiom~II,~4 differently to find the point $F$, using the triangle $CDG$ and the line $BE$. This would tell us that $F$ lies on the line $BE$ between $C$ and $D$ (before, it told us that $F$ lies on the line $CD$ between $B$ and $E$). Now it might seem that we can use a symmetrical application of Pasch's axiom on the same line $BE$ and the triangle $ACD$, which would solve the goal putting $B$ between $A$ and $C$. But at this stage in the proof, we must consider the possibility that $BF$ exits the triangle $ACD$ between $A$ and $D$. This possibility is not yet eliminable by incidence reasoning alone. It really does appear we need at least \emph{four} applications of Axiom~II,~4 to get this theorem.

Observations such as these are not apparent in Hilbert's proof. In his eleven uses of Axiom~II,~4 across THEOREM~3, 4 and 5, Hilbert only considers the case-split implied by the axiom twice. And yet it takes up a significant amount of combinatorial reasoning about incidence. It is difficult to justify leaving this complexity implicit, when it has consequences on the shape of the proof which we find difficult to argue as obvious. 

\section{THEOREM~5}\label{sec:Theorem5}
For the final theorem of this chapter, we shall see what is gained with the inner and outer form of Pasch's axiom, and we will look under the bonnet to see what our discoverers are actually up to.

THEOREM~5 has the most complex of the three proofs, taking up almost an entire page of the English translation. Like THEOREM~3 and THEOREM~4, it was originally an axiom in the first edition of Hilbert's text. The proof here is credited to E.H. Moore who proved it for projective geometry. Effectively, the result gives a transitivity property for point ordering. The proof is divided into three parts, though as observed by Dehlinger et al~\cite{DehlingerFOG}, it makes sense to leave the third part to the generalisation of THEOREM~6, which we cover in the next chapter.

\subsection{Part 1 of THEOREM~5}
\begin{quotation}
THEOREM 5. Given any four points on a line, it is always possible to label them $A$, $B$, $C$, $D$ in such a way that the point labelled $B$ lies between $A$ and $C$ and also between $A$ and $D$, and furthermore, that the point labelled $C$ lies between $A$ and $D$ and also between $B$ and $D$.

PROOF. Let $A$, $B$, $C$, $D$ be four points on a line $g$. The following will now be shown:

1. If $B$ lies on the segment $AC$ and $C$ lies on the segment $BD$ then the points $B$ and $C$ also lie on the segment $AD$. By Axioms~I, 3 and II, 2 choose a point $E$ that does not lie on $g$, on [sic] a point $F$ such that $E$ lies between $C$ and $F$. By repeated applications of Axioms~II, 3 and II,~4 it follows that the segments $AE$ and $BF$ meet at a point $G$, and moreover, that the line $CF$ meets the segment $GD$ at a point $H$. Since $H$ thus lies on the segment $GD$ and since, however, by Axiom~II, 3, $E$ does not lie on the segment $AG$, the line $EH$ by Axiom~II,~4 meets the segment $AD$, i.e. $C$ lies on the segment $AD$. In exactly the same way one shows analogously that $B$ also lies on this segment.

\centering\includegraphics{group2eval/Five}
\flushright{\cite[p. 7]{FoundationsOfGeometry}}
\end{quotation}

\subsubsection{Evaluating our Manual Verification}\label{sec:FindingAEH}
Our manual verification of this proof runs to approximately 80 lines of complicated proof steps. As we should expect by now, most of these steps were used to derive the preconditions needed for Axiom~\ref{eq:g24}. These are now handled by our incidence discoverers.

In the manual verifications, the complexity of the inferences had got the better of us. We were not able to verify Hilbert's final application of Axiom~\ref{eq:g24} with the line $EH$ and the triangle $ADG$. This requires knowing that the line $EH$ does not intersect any vertex of $ADG$, which requires, in particular, knowing that $AEH$ is a triangle.

We began to speculate that this matter was unprovable. In fact, we had produced a sketch argument that the derivation of $E\neq H$ was impossible, and we hoped that our discoverers would confirm this.

\label{sec:CombinatoryError}Instead, our discoverers refuted it. By tracking the path of inferences via a writer (see \S\ref{sec:WriterMonad}), we found that $\triangle AEH$ is derived at the end of a chain of discovered triangles starting from $\triangle ABE$. The rule linking each is $\code{colncolncol}$ from \S\ref{list:Procedures}, which can be understood as substituting points of a non-collinear triple one-at-a-time, until we have rewritten the initial triangle $\triangle ABE$ to $\triangle AEH$. We briefly discuss how.

It might seem that we can just replace the point $B$ with the point $H$ to rewrite $\triangle ABE$ to $\triangle AEH$, but the triangle introduction rule requires a hypothesis about an appropriate collinear set and an appropriate point inequality. Instead, the inference we use is less direct, and is shown in Figure~\ref{fig:FindingAEH}. At first, our discoverer substitutes $G$ for $E$ using the line $AGE$, producing $\triangle ABG$ from $\triangle ABE$. It then substitutes $D$ for $B$ using the line $ABD$. This gives us a triangle $ADG$. Next, it substitutes $H$ for $D$ using the line $DGH$, giving us $\triangle AGH$. Finally, $E$ and $H$ are shown distinct on the basis of $\triangle AGH$ and the line $AGE$, after which the discoverer substitutes $H$ for $G$ using the line $AGE$, thus obtaining $\triangle AEH$.

\begin{figure}
\centering\includegraphics{group2eval/FindingAEH}
\caption{Finding $\triangle AEH$}
\label{fig:FindingAEH}
\end{figure}

\subsubsection{Strengthening Inner and Outer Pasch}\label{sec:StrengthenedPasch}
We have a verification of THEOREM~5 using Axiom~\ref{eq:g24} directly via \code{by\_pasch}. Incidence reasoning implicitly dominates this verification, but our discoverers take on the labour. We can avoid much of this implicit reasoning though, and so in another verification, we exploit the Inner and Outer Pasch axioms, whose preconditions have only one incidence assumption: namely that the triangle on which the axiom applies exists. By using this form of the axiom, a much cleaner proof can be obtained, one which needs much less automation. Furthermore, we do not need to write any subproofs to identify points in the figure, nor eliminate disjuncts.

Our versions of Inner and Outer Pasch are strengthened from their axiomatic form. Consider Veblen's Outer Pasch Axiom, which we can formalise as
\begin{multline*}
  \Triangle{a}{A}{B}{C} \\\wedge \between{B}{C}{D} \wedge \between{A}{E}{C}\\ \implies \exists F.\;\exists a.\;  \code{on\_line}\ D\ a \wedge \code{on\_line}\ E\ a \wedge \code{on\_line}\ F\ a \wedge \between{A}{F}{B}.
\end{multline*}
We can say something stronger in the conclusion here. We know that $D$, $E$ and $F$ are not merely collinear. The point $E$ must lie between $D$ and $F$ (see the diagram accompanying Veblen's proof in \S\ref{sec:VeblenThree}). This is an important corollary, as evidenced by the fact that it is the very first theorem Veblen proves after stating the axiom. Thus, we have both Inner and Outer Pasch axioms as the following strengthened theorems:
\begin{equation}\label{eq:OuterPasch}
  \begin{split}
    &\vdash\Triangle{a}{A}{B}{C} \\
    &\qquad\wedge \between{B}{C}{D}\wedge \between{A}{E}{C}\\
    &\qquad\implies \exists F.\; \between{D}{E}{F} \wedge \between{A}{F}{B}.
  \end{split}
\end{equation}
\begin{equation}\label{eq:InnerPasch}
  \begin{split}
    &\vdash\Triangle{a}{A}{B}{C} \\
    &\qquad\wedge \between{B}{C}{D} \wedge \between{A}{E}{B}\\ 
    &\qquad\implies \exists F.\; \between{D}{F}{E} \wedge \between{A}{F}{C}.
  \end{split}
\end{equation}

\begin{boxedfigure}
\small
  \begin{align*}
    &\code{assume}\ \between{A}{B}{C} \wedge \between{B}{C}{D} & 0,1\\
    &\code{consider}\ E\ \code{such that}\ \Triangle{a}{A}{B}{E}\\
    &\qquad\code{from}\ 0\ \code{by}\ \eqref{eq:g12},\eqref{eq:g13b},\eqref{eq:g21} & 2\\
    &\code{obviously by\_neqs so consider}\ F\ \code{such that}\\
    &\qquad\between{C}{E}{F}\ \code{from}\ 0\ \code{by}\ \eqref{eq:g22} & 3\\
    &\code{obviously by\_ncols so consider}\ G\ \code{such that}\\ 
    &\qquad\between{A}{G}{E} \wedge \between{B}{G}{F}\ \code{from}\ 0,2,3\ \code{by}\
    \eqref{eq:g21},\eqref{eq:InnerPasch}& 4,5\\    
    &\code{obviously by\_ncols so consider}\ H\ \code{such that}\\ 
    &\qquad\between{C}{H}{F} \wedge \between{D}{H}{G}\ \code{from}\ 0,1,2,3,5\ \code{by}\ \eqref{eq:g21},\eqref{eq:InnerPasch}& 6\\
    &\code{have}\ A\neq D\ \code{from}\ 0,1\ \code{by}\ \eqref{eq:g21},\eqref{eq:g23}\\
    &\code{obviously by\_ncols}\ \code{so consider}\ C'\ \code{such that}\\
    &\qquad\between{E}{H}{C'}\wedge\between{A}{C'}{D}\ \code{from}\ 0,1,2,4,6,7\ \code{by}\ \eqref{eq:OuterPasch},\eqref{eq:g21}\\
    &\code{obviously}\ (\code{by\_eqs}\circ\code{split})\ \code{qed from}\ 0,1,2,3,6,7
  \end{align*}
  \caption{THEOREM~5 verification, part 1}
  \label{fig:FiveVerification1}
\end{boxedfigure}

\subsubsection{Verification}
With these theorems now derived and with some of our automation, we obtain the verification in Figure~\ref{fig:FiveVerification1}, which is very close to the prose. We have two steps which Hilbert does not make explicit. Firstly, we note that $A\neq D$, a fact which follows from our assumptions and Axiom~II, 3, and without which we would not be able to show the existence of $\triangle ADG$ for our final application of Pasch's axiom.

The other extra step is somewhat of an irritation. Instead of applying Pasch's axiom to conclude that $C$ is between $A$ and $D$, we must instead obtain a new point $C'$ and then use incidence reasoning via \code{by\_eqs} to identify it with $C$. 

\subsubsection{Comparison with the Prose}
The strengthened versions of the Pasch axioms mean we can be more efficient than Hilbert, who uses an unspecified number of applications of Axiom~II,~4.

\begin{quote}
  ``By repeated applications of Axioms II, 3 and II,~4 it follows that the segments $AE$ and $BF$ meet at a point $G$, and moreover, that the line $CF$ meets the segment $GD$ at a point $H$.''
  \flushright{\cite[p. 7]{FoundationsOfGeometry}}
\end{quote}

We can say exactly how many applications are needed here. In one of our verifications of THEOREM~5, which avoids the Inner and Outer Pasch axioms and so follows Hilbert most closely, we apply Axiom~II,~4 via the \code{by\_pasch} discoverer. By doing so, we see there are exactly three applications implied by Hilbert's prose (we elide the subproofs used to reject offending disjuncts).

\fbox{\begin{minipage}{\boxwidth}
\small
\setlength\abovedisplayskip{-0.2cm}
\begin{align*}
  &\code{consider}\ G\ \code{such that}\ (\exists a.\; \code{on\_line}\ B\ a\wedge\code{on\_line}\ F\ a\wedge\code{on\_line}\ G\ a)\\
  &\qquad\qquad\qquad\qquad\qquad\wedge \between{A}{G}{E}\ & 4,5 \\
  &\qquad\code{proof:}\ \code{clearly by\_pasch}\ ...\\
  &\code{have}\ \between{B}{G}{F}\ & 6\\
  &\qquad\code{proof:}\ \code{clearly by\_pasch}\ ...\\
  &\code{consider}\ H\ \code{such that}\ (\exists a.\; \code{on\_line}\ C\ a\wedge\code{on\_line}\ F\ a\wedge\code{on\_line}\ H\ a)\\
  &\qquad\qquad\qquad\qquad\qquad\wedge \between{D}{H}{G}\ & 7,8\\
  &\qquad\code{proof:}\ \code{clearly by\_pasch}\ ...
\end{align*}\end{minipage}}\linebreak

That three applications are necessary is implied by the careful language used in the prose: ``the segments $AE$ and $BF$ meet at a point $G$'' while ``the \emph{line} $CF$ meets the segment $GD$ at a point $H$'' (our emphasis). Now if we are to show that two \emph{segments} intersect, we must derive \emph{two} facts of betweenness, and therefore we need \emph{two} applications of Axiom~\ref{eq:g24}. But if we are to show that a \emph{line} and a segment intersect, we need only one fact of betweenness. 

In our verification using the Inner and Outer Pasch axioms (Figure~\ref{fig:FiveVerification1}), we can trim this down. The intersection of the segments $AE$ and $BF$ is covered by just one application of the strengthened version of the Inner Pasch Axiom~\eqref{eq:InnerPasch}, which we use again to intersect the segments $CF$ and $GD$. Finally, we use the Outer Pasch Axiom~\eqref{eq:OuterPasch} to find a point $C'$ between $A$ and $D$. 

For this final application of Pasch's axiom, Hilbert writes: ``...and since, however, by Axiom II, 3, $E$ does not lie on the segment $AG$,....'' We draw attention to this remark because it is not reflected in our verification. Hilbert, for the one and only time, is explicitly eliminating the disjunct in the conclusion of Axiom~\ref{eq:g24}, by appealing to the fact that $G$ lies between $A$ and $E$. We must appeal to the same fact, but in our verification, it is just the necessary precondition of the inner  Pasch axiom~\eqref{eq:InnerPasch}.

Finally, we mention Hilbert's final step ``one shows analogously that $B$ also lies on this segment.'' This does not require an analogous \emph{proof} as the word ``show'' would imply. Instead, we can capture the analogy directly by using the theorem verified in Figure~\ref{fig:FiveVerification1} as a lemma and then applying the symmetry of betweenness. In fact, \code{MESON} takes care of this automatically:

\fbox{\begin{minipage}{\boxwidth}\small
\setlength\abovedisplayskip{-0.2cm}
\begin{multline*}
\code{MESON [lemma,\eqref{eq:g21}] }
\forall A.\;\forall B.\;\forall C.\;\forall D.\; \between{A}{B}{C} \wedge \between{B}{C}{D}\\ \implies \between{A}{B}{D} \wedge \between{A}{C}{D}
\end{multline*}\end{minipage}}\linebreak

\subsection{Discovery at work}\label{sec:DiscoveryAtWork}
In this subsection, we give an idea of how our discoverers interact with HOL~Light by showing the sequents generated concurrently as we interactively develop a declarative verification of THEOREM~5. We consider two discoverers, \code{by\_incidence}, which generates the five kinds of basic incidence sequents described in the last chapter, and \code{by\_pasch}, which finds potential applications of Pasch's axiom. The \code{by\_pasch} discoverer feeds off the generations produced by \code{by\_incidence}, so that their work is not duplicated. But there is no \emph{feedback}. That is, \code{by\_incidence} does not continue searching based on the results of \code{by\_pasch}. Instead, we, the user, shall take responsibility for when Pasch's axiom is applied. It is a point introduction axiom explicit in the prose that we want to keep explicit in the verification.

The discovery begins once we have stated the theorem's assumptions and obtained our first non-collinear point.

\fbox{\begin{minipage}{\boxwidth}\small\setlength\abovedisplayskip{-0.2cm}
\begin{align*}
&\code{assume}\ \between{A}{B}{C} \wedge \between{B}{C}{D} & 0,1\\
&\code{consider}\ E\ \code{such that}\ \Triangle{a}{A}{B}{E}\\
&\qquad\code{from}\ 0\ \code{by}\ \eqref{eq:g12},\eqref{eq:g13b},\eqref{eq:g21} & 2
\end{align*}
\end{minipage}}
\linebreak

Concurrently, sequents are pulled from the discoverer \code{by\_incidence}, which lazily forces values according to our data-flow diagram from Figure~\ref{fig:DataFlow}. This ultimately involves pulling hypothesis sequents from the discoverer \code{monitor}, whose job it is to inspect the proof context constructed from the steps of the declarative proof, as it is written, and add any new hypotheses which appear there. Here, we have three hypotheses, which are picked up and fed through our incidence discoverers to produce the seven generations of incidence sequents shown in Figure~\ref{fig:FirstGenerations}. If the stream is forced beyond this, only empty generations appear, indicating that no more inference is possible.

The seven generations of sequents are delivered within 0.31 seconds.\footnote{We have tested this on an Intel Core 2 with a 2.53GHz clock speed.}. We write the dependent hypotheses in parentheses, and omit the turnstile $(\vdash)$ and sequent context. Note that sequents can be repeated if they can be derived in more than one way. When we are not tracking dependent hypotheses, such repetitions are automatically filtered out.

As we can see, two of the seven generations are empty. This happens because of filtering: some of the inferences we use turn out to generate sequents which have already appeared, and duplicates are always removed from the discoverer. This filtering sometimes leaves generations completely empty. 

Besides the selection of triangles here, we have a sequent which says that all points in our figure lie in the same plane. We can see how this sequent has grown over the generations, with larger and larger planar sets found, via rule~$\code{planeplane}$ from \S\ref{list:Procedures}. 

To follow Hilbert's proof at this stage, all we need to know is that $C\neq E$, a fact which is delivered in the fourth generation. We are told that its derivation depends on hypothesis $0$, namely $\between{A}{B}{C}$, and hypothesis $2$, which is 
\begin{displaymath}
\Triangle{a}{A}{B}{C}.
\end{displaymath}
Since this was the \emph{last} hypothesis to enter the proof context, we can add it as justification to the declarative proof we are building by using the Mizar~Light keyword \code{so}. Again, appealing to the pertinent hypotheses gives the reader more information about the dependencies within the proof, and enables the discoverer to work more efficiently in replay, where it will use only those hypotheses that have been marked as justification.

\fbox{\begin{minipage}{\boxwidth}\small\setlength\abovedisplayskip{-0.2cm}
\begin{align*}
\code{obviously by\_neqs so consider}\ F\ \code{such that}\ \between{C}{E}{F}\ \code{from}\ 0\ \code{by}\ \eqref{eq:g22} \quad 3
\end{align*}
\end{minipage}}
\linebreak

\begin{figure}[H]
\doublebox{\begin{minipage}{\boxwidth}\footnotesize\setlength\abovedisplayskip{0cm}
    \begin{displaymath}
  \begin{split}
    &\left\{\begin{aligned}
          &\exists a.\;\code{on\_line}\ B\ a\wedge\code{on\_line}\ C\ a\wedge\code{on\_line}\ D\ a, &(1)\\
          &\exists a.\;\code{on\_line}\ A\ a\wedge\code{on\_line}\ B\ a\wedge\code{on\_line}\ C\ a, &(0)\\
          &\Triangle{a}{A}{B}{E}, &(2)\\
          &B\neq C, B\neq D, C\neq D, & (1)\\
          &A\neq B, A\neq C, B\neq C, & (0)\\
          &A\neq B, A\neq E, B\neq E, & (2)\\
          &\exists \alpha.\;\code{on\_plane}\ B\ \alpha\wedge \code{on\_plane}\ C\ \alpha\wedge \code{on\_plane}\ D\ \alpha, &(1)\\
          &\exists \alpha.\;\code{on\_plane}\ A\ \alpha\wedge \code{on\_plane}\ B\ \alpha\wedge \code{on\_plane}\ C\ \alpha, &(0)\\
          &\exists \alpha.\;\code{on\_plane}\ A\ \alpha\wedge \code{on\_plane}\ B\ \alpha\wedge \code{on\_plane}\ E\ \alpha &(2)\end{aligned}\right\},\\
  &\{\exists \alpha.\;\code{on\_plane}\ A\ \alpha\wedge \code{on\_plane}\ B\ \alpha\wedge \code{on\_plane}\ C\ \alpha\wedge\code{on\_plane}\ D\ \alpha \quad (0,1)\},\\
  &\{\},\\
  &\left\{\begin{aligned}
          &\Triangle{a}{A}{C}{E}, & (0,2)\\
          &\Triangle{a}{B}{C}{E}, & (0,2)\\
          &C\neq E, & (0,2)\\
          &\exists \alpha.\;\code{on\_plane}\ A\ \alpha\wedge \code{on\_plane}\ B\ \alpha\wedge \code{on\_plane}\ C\ \alpha\wedge \code{on\_plane}\ E\ \alpha & (0,2)
        \end{aligned}\right\},\\
      &\{\exists a.\;\code{on\_line}\ A\ a\wedge \code{on\_line}\ B\ a\wedge \code{on\_line}\ C\ a\wedge\code{on\_line}\ D\ a \quad (0,1)\},\\
      &\{\},\\
    &\left\{\begin{aligned}
        &\Triangle{a}{B}{D}{E}, & (0,1,2)\\
        &\Triangle{a}{C}{D}{E}, & (0,1,2)\\
        &D\neq E, & (0,1,2)\\
      &\exists \alpha.\;\code{on\_plane}\ A\ \alpha\wedge \code{on\_plane}\ B\ \alpha\wedge \code{on\_plane}\ C\ \alpha\wedge\code{on\_plane}\ D\ \alpha\wedge\code{on\_plane}\ E\ \alpha & (0,1,2)
    \end{aligned}\right\}\\
\end{split}
\end{displaymath}
\end{minipage}}
\caption{First Generations of Discovered Sequents}
\label{fig:FirstGenerations}
\end{figure}

With this declarative proof step, we add a new sequent into the proof-context, which is picked up by the \code{monitor} discoverer, to flow into the network of incidence discoverers, creating new generations of sequents. These generations are shown in Figure~\ref{fig:SecondGenerations} and are found within 1.21 seconds. Including the sequents found earlier, we have thirteen triangles identified in total, and so we are going to be interested in which applications of Axiom~\ref{eq:g24} are permissible. To find out, we look more specifically at the results of our \code{by\_pasch} discoverer. Its first eighteen generations are empty, meaning that it requires a search-depth of eighteen before all the required preconditions of Pasch have been found. We are then told that, in this early stage of the proof, there are already six possibilities to choose from. The full set is shown in Figure~\ref{fig:PaschGenerations} and is found within 2.82 seconds.

\begin{figure}[H]
\doublebox{\begin{minipage}{\boxwidth}\footnotesize\setlength\abovedisplayskip{0cm}
\begin{displaymath}
  \begin{split}
    &\left\{\begin{aligned}
        &\exists a.\;\code{on\_line}\ C\ a\wedge\code{on\_line}\ E\ a\wedge\code{on\_line}\ F\ a, &(3)\\
        &C \neq E, C\neq F, E\neq F, &(3)\\
        &\exists \alpha.\; \code{on\_plane}\ C\ \alpha\wedge\code{on\_plane}\ E\ \alpha\wedge\code{on\_plane}\ F\ \alpha&(3)\end{aligned}\right\},\\
    &\left\{\begin{aligned}
      &\exists \alpha.\;\code{on\_plane}\ B\ \alpha\wedge \code{on\_plane}\ C\ \alpha\wedge \code{on\_plane}\ D\ \alpha\wedge\code{on\_plane}\ E\ \alpha\wedge\code{on\_plane}\ F\ \alpha, & (1,3)\\
      &\exists \alpha.\;\code{on\_plane}\ A\ \alpha\wedge \code{on\_plane}\ B\ \alpha\wedge \code{on\_plane}\ C\ \alpha\wedge\code{on\_plane}\ E\ \alpha\wedge\code{on\_plane}\ F\ \alpha & (0,3)
      \end{aligned}\right\},\\
    &\left\{\begin{aligned}
        &\exists\alpha.\; \code{on\_plane}\ B\ \alpha\wedge\code{on\_plane}\ C\ \alpha\wedge\code{on\_plane}\ D\ \alpha\wedge\code{on\_plane}\ E\ \alpha\wedge\code{on\_plane}\ F\ \alpha,&(1,3)\\
        &\exists\alpha.\; \code{on\_plane}\ A\ \alpha\wedge\code{on\_plane}\ B\ \alpha\wedge\code{on\_plane}\ C\ \alpha\wedge\code{on\_plane}\ E\ \alpha\wedge\code{on\_plane}\ F\ \alpha&(0,3)\end{aligned}\right\},\\
    &\{\},\{\},\\
    &\left\{\begin{aligned}
        &\exists \alpha.\; \code{on\_plane}\ A\ \alpha\wedge\code{on\_plane}\ B\ \alpha\wedge\code{on\_plane}\ C\ \alpha\\
        &\qquad\wedge\code{on\_plane}\ D\ \alpha\wedge\code{on\_plane}\ E\ \alpha\wedge\code{on\_plane}\ F\ \alpha&(0,1,3)\end{aligned}\right\},\\
    &\{\},\\
    &\left\{A\neq F, B\neq F \quad (0,2,3)\right\},\\
    &\{\},\{\},\\
    &\{D\neq F \quad (0,1,2,3)\},\\
    &\{\},\{\},\{\},\\
    &\left\{\begin{aligned} 
        &\Triangle{a}{A}{C}{F}, & (0,2,3)\\
        &\Triangle{a}{A}{E}{F}, & (0,2,3)\\
        &\Triangle{a}{B}{C}{F}, & (0,2,3)\\
        &\Triangle{a}{B}{E}{F} & (0,2,3)\end{aligned}\right\},\\
    &\{\},\{\},\\
    &\left\{\begin{aligned} 
        &\Triangle{a}{D}{E}{F}, & (0,1,2,3)\\
        &\Triangle{a}{B}{D}{F}, & (0,1,2,3)\\
        &\Triangle{a}{C}{D}{F}, & (0,1,2,3)\\
        &\Triangle{a}{A}{B}{F} & (0,2,3)\end{aligned}\right\}
\end{split}
\end{displaymath}
\end{minipage}}
\caption{Second generations of sequents}
\label{fig:SecondGenerations}
\end{figure}

\begin{figure}
\doublebox{\begin{minipage}{\boxwidth}\footnotesize\setlength\abovedisplayskip{0cm}
    \begin{displaymath}
      \begin{split}
&\{\},\{\},\{\},\{\},\{\},\{\},\{\},\{\},\{\},\{\},\{\},\{\},\{\},\{\},\{\},\{\},\{\},\{\},\\
        &\left\{
          \begin{aligned}
            &\exists G.\; (\exists a.\;\code{on\_line}\ B\ a\wedge\code{on\_line}\ E\ a\wedge\code{on\_line}\ G\ a)\\
            &\qquad\qquad\wedge (\between{A}{G}{C} \vee \between{A}{G}{F}), & (0,2,3)\\
            &\exists G.\; (\exists a.\;\code{on\_line}\ A\ a\wedge\code{on\_line}\ E\ a\wedge\code{on\_line}\ G\ a)\\
            &\qquad\qquad\wedge (\between{B}{G}{C} \vee \between{B}{G}{F}) & (0,2,3)\\
         \end{aligned}\right\},\\
       &\{\},\{\},\{\},\{\},\{\},\{\},\{\},\{\},\\
       &\left\{
         \begin{aligned}
           &\exists G.\; (\exists a.\;\code{on\_line}\ D\ a\wedge\code{on\_line}\ E\ a\wedge\code{on\_line}\ G\ a)\\
           &\qquad\qquad\wedge (\between{B}{G}{C} \vee \between{B}{G}{F}), & (0,1,2,3)\\
           &\exists G.\; (\exists a.\;\code{on\_line}\ B\ a\wedge\code{on\_line}\ E\ a\wedge\code{on\_line}\ G\ a)\\
           &\qquad\qquad\wedge (\between{C}{G}{D} \vee \between{D}{G}{F}), & (0,1,2,3)\\
           &\exists G.\; (\exists a.\;\code{on\_line}\ B\ a\wedge\code{on\_line}\ E\ a\wedge\code{on\_line}\ G\ a)\\ 
           &\qquad\qquad\wedge (\between{A}{G}{F} \vee \between{C}{G}{F}) & (0,2,3)
         \end{aligned}\right\},\\
       &\{\},\{\},\{\},\{\},\{\},\\
     &\left\{
       \begin{aligned}
       &\exists G.\; (\exists a.\;\code{on\_line}\ B\ a\wedge\code{on\_line}\ F\ a\wedge\code{on\_line}\ G\ a)\\
       &\qquad\qquad\wedge (\between{A}{G}{E} \vee \between{C}{G}{E}) & (0,2,3)
     \end{aligned}\right\}
     \end{split}
   \end{displaymath}
\end{minipage}}
\caption{Discovered applications of Pasch's Axiom}
\label{fig:PaschGenerations}
\end{figure}

The last possibility corresponds to one of Hilbert's applications of Axiom~II,~4, which we shall therefore add as an explicit proof step in the declarative proof we are building. We do this by asserting the theorem as \code{clearly} derivable, adding the three hypotheses as explanatory justification, so that the theorem is found efficiently in replay:

\fbox{\begin{minipage}{\boxwidth}\small\setlength\abovedisplayskip{-0.2cm}
\begin{align*}
&\code{clearly by\_pasch so consider}\ G\ \code{such that}\\
&(\exists a.\;\code{on\_line}\ B\ a\wedge\code{on\_line}\ F\ a\wedge\code{on\_line}\ G\ a)\wedge (\between{A}{G}{E} \vee \between{C}{G}{E})\\
&\qquad \code{from}\ 0,2
\end{align*}
\end{minipage}}\linebreak

We now have a disjunction in our hypotheses. When we run the \code{by\_incidence} discoverer with the \code{split} function, it will convert this disjunction into a tree, which will combine as described in \S\ref{sec:Trees}. Our generations become proper trees, and the search space is automatically partitioned into three. 

The first partition covers inferences which are made in the root node of our trees, where no particular disjunct is assumed. These generations are shown in Figure~\ref{fig:RootGenerations}. Here, the discoverer infers that the new point $G$ must, \emph{in any case}, be distinct from $A$, $C$, $D$ and $E$, and that all seven points must be planar.

\begin{figure}
\doublebox{\begin{minipage}{\boxwidth}\footnotesize\setlength\abovedisplayskip{0cm}
    \begin{displaymath}
      \begin{split}
        &\{\code{on\_line}\ B\ a\wedge\code{on\_line}\ F\ a\wedge\code{on\_line}\ G\ a \quad (4)\},\\
        &\{\},\{\},\{\},\{\},\{\},\{\},\\
        &\left\{\begin{aligned}
            &\exists \alpha.\;\code{on\_plane}\ A\ \alpha\wedge \code{on\_plane}\ B\ \alpha\wedge \code{on\_plane}\ C\ \alpha\wedge\code{on\_plane}\ D\ \alpha\\
            &\qquad\wedge\code{on\_plane}\ E\ \alpha\ \code{on\_plane}\ F\ \alpha\ \code{on\_plane}\ G\ \alpha & (0,1,3,4)
            \end{aligned}\right\},\\
        &\{\},\\
        &\{C \neq G, E \neq G \quad (0,2,3,4)\},\\
        &\{\},\{\},\\
        &\left\{\begin{aligned}
          &D \neq G, & (0,1,2,3,4),\\
          &A \neq G & (0,2,3,4)
        \end{aligned}\right\}
     \end{split}
   \end{displaymath}
\end{minipage}}
\caption{Generations in the root of the case-split}
\label{fig:RootGenerations}
\end{figure}

\begin{figure}[H]
\doublebox{\begin{minipage}{\boxwidth}\footnotesize\setlength\abovedisplayskip{0cm}
    \begin{displaymath}
      \begin{split}
        &\left\{\begin{aligned}
            &\exists a.\;\code{on\_line}\ A\ a\wedge\code{on\_line}\ E\ a\wedge\code{on\_line}\ G\ a \quad & (4),\\
            &A\neq G, A\neq E, E\neq G & (4)
          \end{aligned}\right\},\\
        &\{\},\{\},\\
        &\{B \neq G \quad (2,4)\}\\
        &\{\},\{\},\\
        &\{C\neq G\ \quad (0,2,4)\}\\
        &\{F\neq G\ \quad (0,2,3,4)\}\\
        &\{\},\{\},\{\},\{\},\{\},\{\},\{\},\{\},\{\},\{\},\{\},\{\},\{\},\{\},\\
        &\left\{\begin{aligned}
            &\Triangle{a}{A}{B}{G} & (2,4) \\
            &\Triangle{a}{B}{E}{G} & (2,4)
          \end{aligned}\right\},\\
        &\{\},\{\},\{\}\\
        &\left\{\begin{aligned}
            &\Triangle{a}{C}{E}{G} & (0,2,4)\\
            &\Triangle{a}{A}{C}{G} & (0,2,4)\\
            &\Triangle{a}{B}{C}{G} & (0,2,4)
          \end{aligned}\right\},\\
        &\{\},\{\},\\
        &\left\{\begin{aligned}
            &\Triangle{a}{A}{F}{G} & (0,2,3,4)\\
            &\Triangle{a}{C}{F}{G} & (0,2,3,4)\\
            &\Triangle{a}{E}{F}{G} & (0,2,3,4)\\
            &\Triangle{a}{B}{D}{G} & (0,1,2,4)\\
            &\Triangle{a}{C}{D}{G} & (0,1,2,4)\\
            &D\neq G & (0,1,2,4)
          \end{aligned}\right\}.
     \end{split}
   \end{displaymath}
\end{minipage}}
\caption{Generations found on the assumption $\between{A}{G}{E}$}
\label{fig:LeftGenerations}
\end{figure}

We get more information in the next partition, which is the left-branch of the case-split, shown in Figure~\ref{fig:LeftGenerations}. When the trees are flattened, the sequents in this branch carry $\between{A}{G}{E}$ as a disjunctive hypothesis. For brevity, we omit sequents about the existence of planes.

This is the consistent case. The triangles found here will be used as further justification for applications of Axiom~\ref{eq:g24}. The inconsistent case occurs in the remaining partition, carrying the disjunctive hypothesis $\between{C}{G}{E}$. See Figure~\ref{fig:RightGenerations}.

\begin{figure}[h]
\doublebox{\begin{minipage}{\boxwidth}\footnotesize\setlength\abovedisplayskip{0cm}
    \begin{displaymath}
      \begin{split}
        &\left\{\begin{aligned}
            &\exists a.\;\code{on\_line}\ C\ a\wedge\code{on\_line}\ E\ a\wedge\code{on\_line}\ G\ a \quad & (4),\\
            &C\neq G, C\neq E, E\neq G & (4)
          \end{aligned}\right\},\\
        &\{\},\{\},\{\},\\
        &\{\exists a.\;\code{on\_line}\ C\ a\wedge\code{on\_line}\ E\ a\wedge\code{on\_line}\ F\ a\wedge\code{on\_line}\ G\ a, \quad (3,4)\},\\
        &\{\}\\
        &\{A\neq G, B\neq G \quad (0,2,4)\},\{\},\{\},\{D\neq G \quad (0,1,2,4)\}\\
&\{\},\{\},\{\},\{\},\{\},\{\},\{\},\{\},\{\},\{\},\{\},\{\},\{\},\{\},\{\},\{\},\{\},\{\},\{\},\{\},\\
        &\left\{\begin{aligned}
            &\Triangle{a}{A}{C}{G} & (0,2,4) \\
            &\Triangle{a}{A}{E}{G} & (0,2,4) \\
            &\Triangle{a}{B}{C}{G} \quad& (0,2,4) \\
            &\Triangle{a}{B}{E}{G} & (0,2,4) \\
          \end{aligned}\right\},\{\},\{\},\\
        &\left\{\begin{aligned}
            &\Triangle{a}{D}{E}{G} & (0,1,2,4)\\
            &\Triangle{a}{B}{D}{G} & (0,1,2,4)\\
            &\Triangle{a}{C}{D}{G}\quad & (0,1,2,4)\\
            &\Triangle{a}{A}{B}{G} & (0,2,4) \\
            &F = G & (0,2,3,4)
      \end{aligned}\right\}
     \end{split}
   \end{displaymath}
\end{minipage}}
\caption{Generations found on the assumption $\between{C}{G}{E}$}
\label{fig:RightGenerations}
\end{figure}

The use of our tree data-structure means this last sequence is generated in parallel with the other two. It takes 9.58 seconds to generate all three sequences, though most of the theorems which appear are generated in under 1~second. The theorem required to advance the proof
\begin{displaymath}
\between{C}{G}{E} \implies F = G
\end{displaymath}
is generated in 6.19~seconds. The consequent of this implication contradicts our hypothesis that $\between{C}{E}{F}$ via $\eqref{eq:g23}$, and thus, the case given by its antecedent can be discarded.

\fbox{\begin{minipage}{\boxwidth}\small
  \setlength\abovedisplayshortskip{-0.3cm}
  \begin{displaymath}
   \code{obviously}\ (\code{by\_eqs}\circ\code{split})\ \code{qed by}\ \eqref{eq:g21},\eqref{eq:g23}\ \code{from}\ 0,2,3
\end{displaymath}\end{minipage}}\linebreak

The \code{obviously} keyword tells the step to collapse the stream of trees, pushing the branch labels into the sequents as antecedents. It then uses the resulting discovered sequents to justify the step.

The rest of the proof proceeds similarly. With the help of our discoverers, we reduce the 80 or so steps used in our manual verification to a verification with just 17 steps. Generally, we found a consistent reduction of roughly 80\% in proof length across all 18 theorems from our manual verifications, with the new verifications always comparing much more favourably with the prose.

\subsection{Part~2 of THEOREM~5}
We now consider the second part of THEOREM~5, our verification of which matches Hilbert's logic very closely, even if we have reordered some of the derivations. The whole proof is indirect, which explains why there is no accompanying diagram, and it is one of the few proofs where Hilbert treats the disjunction in Axiom~II,~4 symmetrically: both alternatives entail a contradiction.

\begin{quotation}2. If $B$ lies on the segment $AC$ and $C$ lies on the segment $AD$ then $C$ also lies on the segment $BD$ and $B$ also lies on the segment $AD$. Choose one point $G$ that does not lie on $g$, and another point $F$ such that $G$ lies on the segment $BF$. By Axioms~I, 2 and II, 3 the line $CF$ meets neither the segment $AB$ nor the segment $BG$ and hence, by Axiom~II,~4 again, does not meet the segment $AG$. But since $C$ lies on the segment $AD$, the straight line $CF$ meets then the segment $GD$ at a point $H$. Now by Axiom~II,~3 and II,~4 again the line $FH$ meets the segment $BD$. Hence $C$ lies on the segment $BD$. The rest of Assertion 2 thus follows from 1.
\flushright{\cite[p. 7]{FoundationsOfGeometry}}
\end{quotation}

Our verification is shown in Figure~\ref{fig:FiveVerification2}. In the prose, Hilbert is applying Axiom~II,~4 in its contrapositive form, so we cannot follow him literally by using our \code{by\_pasch} discoverer. Instead, we run a \emph{reductio} argument in a subproof, where we can use Axiom~II,~4 in a forward direction.

\label{sec:g12Erratic}Notice that this is the one time that Hilbert makes an explicit reference to an incidence axiom other than I, 3: he cites Axiom~I, 2, and we do not have a clear idea why. This axiom is needed in many places in these proofs, but is nearly always implicit. Bernays cites the same axiom in a proof supplementing the text (see \S\ref{sec:SupplementI}). The only obvious commonality between the two is that the citation occurs when Axiom~\ref{eq:g24} is applied and \emph{both} disjuncts in the conclusion are eliminated. 

But Axiom~I,2 is required when eliminating even \emph{one} disjunct. So perhaps there is evidence here that neither Hilbert nor Bernays are taking the case-split in Axiom~\ref{eq:g24} sufficiently seriously. If so, then it makes sense for our verifications to eliminate the disjuncts explicitly in subproofs, and it makes sense that we treat Axiom~\ref{eq:g12} uniformly, and leave its use implicit.

\begin{boxedfigure}
\small
  \begin{align*}
    &\code{assume}\ \between{A}{B}{C} \wedge \between{A}{C}{D} & 0,1\\
    &\code{consider}\ G\ \code{such that}\\
    &\qquad\Triangle{a}{A}{B}{G}\ \code{from}\ 0\ \code{by}\ \eqref{eq:g12},\eqref{eq:g13b},\eqref{eq:g21}&2\\
    &\code{obviously by\_neqs so consider}\ F\ \code{such that}\ \between{B}{G}{F}\ \code{by}\ \eqref{eq:g22}&3\\
    &\code{have}\ \neg(\exists P.\; (\exists a.\;\code{on\_line}\ C\ a\wedge\code{on\_line}\ F\ a\wedge\code{on\_line}\ P\ a)\wedge\between{A}{P}{G} & 4\\
    &\qquad\code{proof:}\ \code{otherwise consider}\ P\ \code{such that}\\
    &\qquad\qquad \exists P.\; (\exists a.\;\code{on\_line}\ C\ a\wedge\code{on\_line}\ F\ a\wedge\code{on\_line}\ P\ a) \wedge\between{A}{P}{G}& 4,5\\
    &\qquad\code{clearly by\_pasch so consider}\ Q\ \code{such that}\\
    &\qquad\qquad(\exists a.\;\code{on\_line}\ C\ a\wedge\code{on\_line}\ P\ a\wedge\code{on\_line}\ Q\ a)\\
    &\qquad\qquad\qquad(\between{A}{Q}{B} \vee \between{B}{Q}{G})\ \code{from}\ 0,2\\
    &\qquad\code{obviously}\ (\code{by\_eqs}\circ\code{split})\ \code{qed from}\ 0,2,3,4,5\ \code{by}\ \eqref{eq:g21},\eqref{eq:g23}\\
    &\code{obviously by\_pasch so consider}\ H\ \code{such that}\\
    &\qquad(\exists a.\;\code{on\_line}\ C\ a\wedge\code{on\_line}\ F\ a\wedge\code{on\_line}\ H\ a)\\
    &\qquad\qquad\wedge\between{D}{H}{G}\ \code{from}\ 0,1,2,3\ \code{at}\ 5,6\\
    &\code{have}\ B\neq D\ \code{from}\ 0,1\ \code{by}\ \eqref{eq:g21},\eqref{eq:g23} & 7\\
    &\code{clearly by\_pasch so consider}\ C'\ \code{such that}\\
    &\qquad (\exists a.\;\code{on\_line}\ C'\ a\wedge\code{on\_line}\ F\ a\wedge\code{on\_line}\ H\ a)\\
    &\qquad\qquad(\between{B}{C'}{D} \vee \between{B}{C'}{G}\ \code{from}\ 0,1,2,3,5,6\\
    &\code{obviously}\ (\code{by\_eqs}\circ\code{split})\ \code{qed from}\ 0,1,2,3,5,6,7\ \code{by}\ \eqref{eq:g21},\eqref{eq:g23}
  \end{align*}
  \caption{THEOREM~5 verification, part 2}
  \label{fig:FiveVerification2}
\end{boxedfigure}

\section{Conclusion}
Hilbert gives only three prose proofs for his first two groups in the \emph{Grundlagen der Geometrie}. None of these proofs were present in the first edition, and the results they prove, one-dimensional ordering theorems, were originally assumed as axioms. Two of the proofs were contributed by Wald and E.H. Moore, though Veblen also deserves credit in helping to investigate how linear order theorems can be derived from  two-dimensional order axioms.

We have reviewed how the automation we discussed in the last chapter enables us to write very short proofs compared to our manual verifications, without having to break from Hilbert's basic proof strategies. In fact, we obtain proofs whose steps match Hilbert's prose steps very closely. Our DeBruijn factor is almost 1, and our incidence discoverers appear to be adequate to handle all of the incidence reasoning implicit in Hilbert's proofs.

This is not merely aesthetic. That we can write such short verifications of these relatively simple theorems without trudging through a bog of incidence arguments gave us hope that we could tackle the far more complex verification of the Polygonal Jordan Curve Theorem. Indeed, we shall find that the automation described in the last chapter is used aggressively in that verification. 

We just need some further automation for dealing with linear ordering. This is described in the next chapter, which shows how THEOREM~4 and THEOREM~5 are generalised in THEOREM~6.

%%% Local Variables: 
%%% TeX-master: "../thesis"
%%% End: 


\chapter{Infinity and Linear Ordering}\label{chapter:LinearOrder}
The next theorem on the agenda is Theorem~6, and in this chapter, we will look at two different ways to deal with it formally. The theorem itself tells us that any finite set of points on a line are linearly ordered in terms of betweenness. One approach to formalising this is at the metalevel, where it can be treated as an algorithm for enumerating cases of betweenness. In another approach, we can formalise the theorem at the object level and verify it. In so doing, we shall derive the Axiom of Infinity. All the development up to now has been independent of this axiom, which we manually removed from HOL~Light as explained in \S\ref{sec:NoInfinity}.

\section{Theorem 6 at the Metalevel}\label{sec:Theorem6}
We ended the last chapter by discussing two parts to Hilbert's proof of Theorem~5. The final part of this proof can be generalised to formalise and verify Theorem~6.

\begin{quote}THEOREM 6 (generalization of Theorem~5). Given any finite number of points on a line it is always possible to label them $A$, $B$, $C$, $D, \ldots, K$ in such a way that the point labelled $B$ lies between $A$, and $C$,$D$,$E, \ldots, K$, the point labelled $C$ lies between $A$, $B$ and $D$,$E,\ldots K,$ $D$ lies between $A$,$B$,$C$ and $E, \ldots,$ etc. Besides this order of labelling there is only the reverse one that has the same property.
\end{quote}

A \emph{labelling} of a point is, of course, a syntactic device, \emph{used} to \emph{mention} points. As with many logicians of the time, Hilbert was likely away of the distinction between \emph{use} and \emph{mention}, even though he was working before it had been formalised.

Thinking in terms of formal languages, we could identify a \emph{labelling} as the assignment of an existing value to a new variable. The existence of a labelling is then the existence of these assignments. With this in mind, we could semi-formalise Theorem~6 to say that, given some points on a line $A'$, $B'$, $C'$, $D'$, $\ldots$, $K'$, there exists an appropriate assignment to $A$, $B$, $C$, $D$, $\ldots$, $K$:
\begin{equation}
  \begin{split}
    &\exists a. \code{on\_line}\ A'\ a\wedge\code{on\_line}\ B'\ a\wedge\code{on\_line}\ C'\ a\wedge\code{on\_line}\ D'\ a\\
    &\qquad\wedge\code{on\_line}\ E'\ a\wedge\cdots\wedge\code{on\_line}\ K'\ a\\
    &\implies\exists A\ B\ C\ D\ \ldots\ K. \\
    &\qquad\qquad A = A' \wedge B = B' \wedge C = C' \wedge D = D' \wedge E = E' \wedge \cdots \wedge K = K'\\
    &\qquad\quad\wedge\left(\between{A}{B}{C} \wedge \between{A}{B}{D} \wedge \between{A}{B}{E} \wedge \cdots \wedge \between{A}{B}{K}\right) \\
    &\qquad\quad\wedge
    \left(\begin{split}&\between{A}{C}{D} \wedge \between{A}{C}{E} \wedge \cdots \wedge \between{A}{C}{K}\\
        &\wedge\between{B}{C}{D} \wedge \between{B}{C}{E} \wedge \between{B}{C}{K}\\
        &\qquad\wedge \cdots \wedge \between{B}{C}{K}
      \end{split}\right)\\  
    &\qquad\quad\wedge
    \left(\begin{split}&\between{A}{D}{E} \wedge \cdots \wedge \between{A}{D}{K} \\
        &\wedge\between{B}{D}{E} \wedge \cdots \wedge \between{B}{D}{K} \\
        &\wedge\between{C}{D}{E} \wedge \cdots \wedge \between{B}{D}{K}
        \end{split}\right)\\
    &\qquad\quad\wedge\code{between}\ldots.
  \end{split}
\end{equation} 

Obviously, we cannot capture this directly in first-order logic. The fact that we quantify over an unspecified number of variables already renders that goal hopeless. Instead, the use of ellipses here tells us that this is a \emph{scheme}. As we fix our choice of variables in $A, B, C, D, E, \ldots, K$, we are expected to fill in the holes by continuing a syntactic pattern.

In short, this is a metatheorem, something perhaps surprisingly typical of mathematics. As noted by Harrison~\cite{FormalizedMathematics}, the existence of such theorems must be borne in mind when formalising, and he mentions the simple example of an object theorem about certain operations being associative and commutative (say, addition), which entails the \emph{metatheorem} that brackets (bits of syntax) can be dropped without ambiguity.

Hilbert was working before any distinction had been made between metalevel and object level, let alone the sort of hard distinction we now have between HOL and the programming metalanguage (ML) which encodes its syntax and inference rules. With such a distinction, we immediately realise that our schematic above can be formalised as an ML function which inputs a list of points, fills in the scheme to create a true HOL term, and then derives it as a theorem.

\subsection{Representation}
Consider applying Theorem~6 to the special case of Theorem~5 where there are only four points $A'$, $B'$, $C'$, $D'$ and four labels $A$, $B$, $C$ and $D$. The theorem we must generate is then:

\begin{equation}
  \begin{split}
    &\exists a. \code{on\_line}\ A\ a\wedge\code{on\_line}\ B\ a\wedge\code{on\_line}\ C\ a\wedge\code{on\_line}\ D\ a\\
    &\implies\exists A\ B\ C\ D.\ A = A' \wedge B = B' \wedge C = C' \wedge D = D'\\
    &\qquad\quad\wedge\between{A}{B}{C} \wedge \between{A}{B}{D}\\
    &\qquad\quad\wedge\between{B}{C}{D}
  \end{split}
\end{equation} 

Now we can unwind this existential into the disjunction given in Figure~\ref{fig:Theorem5CasesFormalised}, and thus this sufficies as a formalisation of Theorem~5. It is rather verbose though, and as we increase the number of points, the number of disjuncts will explode. 

Hilbert does not state Theorem~6 with any efficiency in mind, as he lists all possible betweenness relations among the points considered. We shall not be so wasteful, and will control this blow-up by finding a concise, canonical representation of the linear order of points on a line. The constraint is that we should be able to quickly derive any member of the full set of betweenness relations from the canonical representation.

\begin{figure}
  \begin{align*}
    &\fourbet{A}{B}{C}{D}\\
    &\vee\fourbet{B}{A}{C}{D}\\
    &\vee\fourbet{B}{C}{A}{D}\\
    &\vee\fourbet{A}{D}{C}{B}\\
    &\vee\fourbet{A}{C}{B}{D}\\
    &\vee\fourbet{C}{A}{B}{D}\\
    &\vee\fourbet{C}{B}{A}{D}\\
    &\vee\fourbet{A}{D}{B}{C}\\
    &\vee\fourbet{A}{B}{D}{C}\\
    &\vee\fourbet{B}{A}{D}{C}\\
    &\vee\fourbet{B}{D}{A}{C}\\
    &\vee\fourbet{B}{D}{C}{A}
  \end{align*}
\caption{Theorem~5 Case Split}
\label{fig:Theorem5CasesFormalised}
\end{figure}

Typically, a geometry will have a specified origin and a specified axis on which to define the preferred orientation. Even coordinate free methods such as the signed-area method~\cite{SignedAreaMethod} specify a preferred orientation. Not so in Hilbert's geometry. But we \emph{can} treat the first argument to the between predicate as a parameter giving the preferred origin and direction, and then regard the partially applied relation as a two place binary order. In other words, we can read $\between{A}{B}{C}$ as saying that $B<C$ from the perspective of origin $A$ and direction $\overrightarrow{AB}$.

We can then represent total orders as conjunctions ordering adjacent points. So, if $A,B,C,D,E,F,G,H$ occur along a line in that order, we just need the six conjuncts
\begin{multline}\label{theorem:OrderRepExample}
\between{A}{B}{C} \wedge \between{A}{C}{D} \wedge \between{A}{D}{E}\\
\wedge\between{A}{E}{F}\wedge\between{A}{F}{G}\wedge\between{A}{G}{H}.
\end{multline}

We want to retrieve all other between relations implied by this term quickly. Before we describe how we do this, we will interpret Theorem~5, whose verification we considered in the last chapter, as two separate rules. The first rule \eqref{eq:five2a} will ``move the origin'' in a total order. The second rule \eqref{eq:five2b} is just transitivity if we think of the between relation as a parameterised binary relation.
\begin{gather}
\label{eq:five2a}\between{A}{B}{C} \wedge \between{A}{C}{D} \implies \between{B}{C}{D}\\
\label{eq:five2b}\between{A}{B}{C} \wedge \between{A}{C}{D} \implies \between{A}{B}{D}
\end{gather}

We now explain our strategy by way of example. Suppose our goal is to derive $\between{C}{F}{H}$ from \eqref{theorem:OrderRepExample}. Our first task is to ``move the origin'' from $C$ to $A$. To do so, we match against \eqref{eq:five2a}, to yield:
\begin{displaymath}
\between{A}{C}{F} \wedge \between{A}{F}{H}.
\end{displaymath}

We now just reason transitively from the conjuncts of the ordering. We split \eqref{theorem:OrderRepExample} into the two suborderings either side of $F$
\begin{displaymath}
\between{A}{C}{D} \wedge \between{A}{D}{E} \wedge\between{A}{E}{F}
\end{displaymath}
and
\begin{displaymath}
\between{A}{F}{G}\wedge\between{A}{G}{H}.
\end{displaymath}

and then obtain $\between{A}{C}{F}$ and $\between{A}{F}{H}$ by folding Rule~\eqref{eq:five2b} across their conjuncts:
\begin{align*}
\between{A}{C}{D} &\wedge \between{A}{D}{E} \wedge\between{A}{E}{F}\\
&\longrightarrow \between{A}{C}{E} \wedge \between{A}{E}{F}\\
&\longrightarrow \between{A}{C}{F}
\end{align*}
and
\begin{align*}
\between{A}{F}{G} &\wedge \between{A}{G}{H}\\
&\longrightarrow \between{A}{F}{H}.
\end{align*}

We have implemented this procedure as a completely deterministic tactic, taking a theorem such as \eqref{theorem:OrderRepExample} and a term such as $\between{C}{F}{H}$, and then deriving the term from the order in one pass of the order conjunction.

\subsection{Enumerating Possible Orderings}
Even with the more concise representation of orderings, there are $\frac{1}{2}n!$ orderings to consider for $n$ points. In practice, when we want to apply the theorem, there are usually additional constraints on the ordering that allow us to eliminate most of the cases by appealing to Axiom~II,3. For instance, if we know that $\between{B}{A}{C}$ and $\between{B}{D}{C}$, then there are only two ways to order $A$, $B$, $C$ and $D$:
\begin{align*}
&\fourbet{B}{A}{D}{C}\\
&\vee\fourbet{B}{D}{A}{C}
\end{align*}

Factoring in these constraints not only cuts down the size of the final conclusion, but significantly speeds up the calculation of the possibilities. Thus, the procedure we have defined to enumerate all possible orderings takes both a list of the points we want it to order, and a list of betweeness constraints on those points. 

This algorithm nicely captures the purpose of a metatheorem such as Hilbert's if we see it as a computation on labellings. It also keeps us well within the scope of first-order logic. However, our ML algorithm is only a verification of the particular instances it generates. The algorithm, the metatheorem itself, is unverified. To verify it, we must bring the metalevel down to the object level.

\section{Theorem~6 at the Object Level}
Any logic which can formalise Theorem~6 will need to include a domain of labellings, which is potentially infinite. One such domain appears in Dehlinger et al's verification~\cite{DehlingerFOG} in the form of lists. Our formalisation is equivalent. We treat a labelling as an assignment from an initial prefix of natural numbers to the points being labelled. Formally:
\label{sec:OrderingDef}
\begin{equation}\label{eq:OrderingDef}
  \begin{split}
    \code{ordering}\ f\ X &\iff X = f\left(\left\{n\ \vert\ \code{finite}\ X \implies n < \left|X\right|\right\}\right)\\
    &\wedge \forall n\ n'\ n''. \code{finite}\ X \implies n < \left|X\right| \wedge n' < \left|X\right| \wedge n'' < \left|X\right|\\
    &\qquad\wedge n < n' \wedge n' < n'' \wedge \between{(f\ n)}{(f\ n')}{(f\ n'')}.
    \end{split}
\end{equation}

Our definition here allows for the possibility that $f$ is an ordering of a (countably) infinite set $X$. In this infinite case, we just drop the constraint that $n$, $n'$ and $n''$ are bounded by its finite cardinality.\footnote{In HOL~Light, the \code{finite} sets are defined to be the empty set, and all adjoins to all finite sets.}

Theorem~6 can then be formalised concisely as:
\begin{equation}
\label{eq:Infinity}
  \code{finite}\ X \wedge\code{collinear}\ X \implies \exists f. \code{ordering}\ f\ X.
\end{equation}

% In the recursive case, we must order $n+1$ points by ordering sets of $n$ points. Suppose the \mbox{$n+1$} points are $P_1$, $P_2$, $\ldots$, $P_n$, $P_{n + 1}$. We first invoke the function recursively to order $P_1$, $P_2$, $\ldots$, $P_{n}$. For each ordering $r$ returned, we invoke the function recursively again to order $P_1$, $P_3$, $\ldots$, $P_{n+1}$ (the totality of points with the exception of $P_2$) subject to the constraints implied by $r$. This gives $r$ possibilities, one for each placement of $P_{n+1}$ relative to $P_1$, $P_2$, $\ldots$, $P_n$ (see Figure \ref{fig:PointOrdering})

% \begin{figure}
%   % \begin{pspicture}(-1,0)(5,3)
%   %   \put(1.2,0.6){\parbox{5cm}{\scriptsize$P_1$}}
%   %   \put(3.8,0.6){\parbox{5cm}{\scriptsize$P_2$}}
%   %   \put(5.0,0.6){\parbox{5cm}{\scriptsize$P_3$}}
%   %   \put(7.0,0.6){\parbox{5cm}{\scriptsize$P_{n-1}$}}
%   %   \put(8.2,0.6){\parbox{5cm}{\scriptsize$P_n$}}

%   %   \put(0.5,1.5){\parbox{5cm}{\scriptsize$P_{n+1}$}}
%   %   \put(3.8,1.5){\parbox{5cm}{\scriptsize$P_{n+1}$}}
%   %   \put(6.0,1.5){\parbox{5cm}{\scriptsize$P_{n+1}$}}
%   %   \put(7.5,1.5){\parbox{5cm}{\scriptsize$P_{n+1}$}}
%   %   \put(9.0,1.5){\parbox{5cm}{\scriptsize$P_{n+1}$}}

%   %   \psdot(1.4,1,0)
%   %   \psdot(4.0,1,0)
%   %   \psdot(5.2,1,0)
%   %   \psdot(7.2,1,0)
%   %   \psdot(8.2,1,0)

%   %   \psline[linestyle=dashed](0.75,1.0)(0.75,1.4)
%   %   \psline[linestyle=dashed](3.8,1.0)(4.0,1.4)
%   %   \psline[linestyle=dashed](4.2,1.0)(4.0,1.4)
%   %   \psline[linestyle=dashed](6.25,1.0)(6.25,1.4)
%   %   \psline[linestyle=dashed](7.75,1.0)(7.75,1.4)
%   %   \psline[linestyle=dashed](9.25,1.0)(9.25,1.4)

%   %   \psline[linewidth=0.25mm](0.0,1.0)(9.5,1.0)
%   % \end{pspicture}
% \caption{Ordered points}
% \label{fig:PointOrdering}
% \end{figure}

% The final case-split is required to deal with the case:

% \begin{displaymath}
% \between{P_1}{P_2}{P_3}\,\wedge\,\between{P_1}{P_3}{P_4}\,\wedge\,\cdots\,\between{P_1}{P_{n-1}}{P_n}
% \end{displaymath}
% and
% \begin{displaymath}
% \between{P_1}{P_{n+1}}{P_3}\,\wedge\,\between{P_1}{P_3}{P_4}\,\wedge\,\cdots\,\between{P_1}{P_{n-1}}{P_n}
% \end{displaymath}

% This means we know that $P_2$ and $P_{n+1}$ both lie between $P_1$ and $P_3$, but we cannot say which comes first. We need to use Theorem~4 on $P_1$, $P_2$ and $P_{n+1}$, from which we have one of $\between{P_1}{P_2}{P_{n+1}}$, $\between{P_2}{P_1}{P_{n+1}}$ and $\between{P_1}{P_{n+1}}{P_2}$. The second of these is contradictory, leaving only the two outer cases and the two possible orderings
% \begin{displaymath}
% \between{P_1}{P_2}{P_{n+1}} \,\wedge\,\between{P_1}{P_{n+1}}{P_3}\,\wedge\,\cdots\,\between{P_1}{P_{n-1}}{P_n}
% \end{displaymath}
% and
% \begin{displaymath}
% \between{P_1}{P_{n+1}}{P_2}\,\wedge\,\between{P_1}{P_2}{P_3}\,\wedge\,\cdots\,\between{P_1}{P_{n-1}}{P_n}
% \end{displaymath}

% respectively. The result is now a disjunction of total orders of the original $n+1$ points.

% Now if we think of Theorem~6 as a metatheorem, then we can implement it in our theorem prover's metalanguage, as a procedure which takes a given list of points and then produces all possible orderings of those points. A list of possible orderings can be expressed using an existential theorem. Thus, if we take the finite set of points to be labelled as $S$, we can write
% \begin{align*}\label{theorem:Theorem6Example}
% \exists A\,B\,C&\,D\,\ldots\,I\,J\,K \in S.\tag{$\ddagger$}\\
%                 &\between{A}{B}{C} \wedge \between{A}{C}{D} \wedge \cdots \wedge \between{I}{J}{K}.                 
% \end{align*}

% But note that we are only working with finitely many points, so we can unfold this existential into a disjunction, against which we can immediately perform case-splits without having to reason about properties of finite sets. 

\section{Natural Numbers}
Our definition assumes the existence of natural numbers, but Hilbert, writing before the investigations of the logicists, was not clear whether he took these to be logically primitive. Veblen, writing some years later, is explicit: ``[The axioms] presuppose only the validity of the operations of logic and of counting (ordinal number)''. Hilbert \emph{seems} to be making the assumption implicitly when he states the Archimedean Axiom in his Group~V, though it could be argued that the following is only meant schematically:
\begin{quote}
  If $AB$ and $CD$ are any segments then there exists a \textbf{number} $n$ such that $n$ segments $CD$ constructed contiguously from $A$, along the ray from $A$ through $B$, will pass beyond the point $B$.
[emphasis added]\end{quote}

Euclid does much the same when he expresses the same property using the word ``multiplied'' (though Euclid mistakes this property for a \emph{definition}):
\begin{quote}
  Magnitudes are said to have a ratio to one another which can, when multiplied, exceed one another.\cite{Aleph0Elements}
\end{quote}

Yet elsewhere, Euclid will discuss natural numbers in geometrical terms: his books on number theory literally identify numbers with line segments. Hilbert is faithful to this idea. He shows how to recover arithmetic operations from geometrical figures by exploiting Pascal's and Desargues' Theorems. So why would they want to assume the existence of natural numbers?

The question of foundations here, whether natural numbers are a primitive logical concept required of the theory, or whether they are to be recovered geometrically, is difficult to answer. So when it comes to formalisation, we have tried to base our decisions on the broad philosophical and historical aims of the text. Pasch, Peano, Hilbert and Veblen are all supposed to be rigorising the synthetic geometry of Euclid's \emph{Elements}. There is a story that Euclid, following on from the first crisis in the foundations of mathematics and the discovery of incommensurable magnitudes, would have regarded natural numbers with suspicion, and thus kept them out of his logical foundation~\cite{EvolutionEuclideanElements}. Instead, numbers were to be grounded on secure geometrical notions.

\subsection{The Axiom of Infinity}
The theorem stating that natural numbers exist is derivable in HOL~Light assuming a domain exists which is Dedekind-infinite.
\begin{equation}\label{eq:InfinityAxiom}
  \exists f : ind\rightarrow ind. \code{one\_one}\ f \wedge \neg\code{onto}\ f.
\end{equation}

This theorem asserts that there is a one-one but not onto function $f$. From it, we can nominate an arbitrary member of the set $\{ i : ind\ \vert\ \neg\exists i'. f\ i' = i \}$ to serve as the number $0$, nominate $f$ as the successor function, and carve out the natural numbers as the smallest set containing $0$ and its own image under $f$.

The existence of the infinite domain $ind$ is not generally assumed as part of classical logic proper, and Russell, in his classic \emph{Principia Mathematica}, took it as a mere hypothesis on the theorems which required it~\cite{LogicismRevisited}. Without the axiom, one must allow for the possibility that our domains --- the sets of set theory and the types of simple type theory --- each has only finitely many inhabitants.

This should not be the case for the primitive domains of Hilbert's geometry. Hilbert insists as much for his Theorem~7, which states quite plainly that ``between any two points on a line there exists an infinite number of points.'' Thus, we should be able to derive Theorem~\ref{eq:InfinityAxiom} from Hilbert's geometric axioms. We can then obtain the natural numbers from this theorem, replacing the abstract type $ind$ with a concrete type of geometric objects. Our natural numbers will then be quite literally founded in geometry.

This can be dealt with nicely in HOL~Light. The axioms of infinity, choice and extensionality are not part of the HOL~Light kernel. Instead, they are asserted with \code{new\_axiom}, exactly as we have asserted the axioms of geometry. Thus, to replace the axiom of infinity, we just will not load its theory file. Instead of asserting the axiom, we load the theory files containing our geometric axioms, and \emph{derive it}. 

After this, we can reload the usual HOL~Light theories which depend on the axiom of infinity, thus reproducing the whole of the HOL~Light standard library from a geometric foundation.

\subsection{Models and a Finite Interpretation}
The fact that Hilbert's domains are infinite is not derivable from Group~I. We can prove this by exhibiting a finite model of the Group~I axioms. To do so formally, we define a two-place predicate \code{Group\_I} over arbitrary relations $l$ and $p$. By instantiating the polymorphic types of these relations, one supplies the domains of the interpretation. We then assert a particular model of the axioms by writing, for appropriate $l$ and $p$:
\begin{displaymath}
\code{Group\_I}\ l\ p
\end{displaymath}

Incidentally, we have used the same predicate to assert the Group~I axioms. Having declared our primitive types and our primitive relations, we write
\begin{displaymath}
  \code{new\_axiom}\ (\code{Group\_I}\ \code{on\_line}\ \code{on\_plane}).
\end{displaymath}

Besides allowing us to do this, the predicate $\code{Group\_I}$ allows us to express basic metatheoretical ideas about the Group~I axioms, and if we treat the hypothesis as a \emph{context} of axioms, we can regard it as a basic mechanism to write a module of incidence proofs. 

In general, such techniques have significant weaknesses. For one, we are limited in how much we can reason about theories of polymorphic values. We would want to be able to do this if we wanted to reason about the theory of monads in Chapter~\ref{chapter:Automation}. This would require a stronger type-theory, such as the extension provided by HOL~Omega~\cite{HOLOmega}.

For another, when we want to treat $\code{Group\_I}\ \code{on\_line}\ \code{on\_plane}$ as a module of theorems, we would still have no convenient way to make new definitions or abstract out new types. This would require a more sophisticated embedding such as the one underlying Isabelle's locales~\cite{IsabelleLocales}.

These issues are not a problem for the specific purposes of this chapter, where we only have to consider a very simple metatheoretical question, but it could cause problems reasoning in later groups such as Group~III where axioms are defined based on quite complex and derived definitions.

A finite model of Group~I, then, is realised in the four vertices, six lines, and four planes of a tetrahedron. In this finite interpretation, we can translate all our first-order axioms into propositional theorems and verify them with a tautology checker.

To capture this idea formally, we carved out two finite types. One type is inhabited by four constructors which can be used as interpretations of the four points and the four planes in our model. The other type is inhabited by six constructors, which become our six lines.
\begin{align*}
\code{ps}   &= \code{p1} \vert \code{p2} \vert\code{p3} \vert\code{p4}\\
\code{lines}&= \code{l1} \vert \code{l2} \vert\code{l3} \vert\code{l4} \vert\code{l5} \vert\code{l6}\\
\end{align*}

The type definitions are used to automatically derive an abstract type and derive two theorems, an induction theorem and a recursion theorem. For the type \code{ps}, for instance, we are given
\begin{align*}
&\forall P.\ P\ \code{p1}\ \wedge\ P\ \code{p2}\ \wedge\ P\ \code{p3}\ \wedge\ P\ \code{p4}\implies\forall p. P\ p\\
&\forall p1\ p2\ p3\ p4.\ \exists f.\ f\ \code{p1} = p1\ \wedge\ f\ \code{p2} = p2\ \wedge\ f\ \code{p3} = p3\ \wedge\ f\ \code{p4} = p4.
\end{align*}

The first (induction) theorem can be promoted to an equivalence, and then used to rewrite all universally quantified statements as finite conjunctions. Similarly, its rewrite using the infinite DeMorgan rule, $(\forall x. P\,x) \iff \neg\exists x. \neg (P\,x)$ can be used to rewrite all existentially quantified statements as finite disjunctions.

Next, by instantiating the universally quantified variables in the second (recursion) theorem with the first four natural numbers respectively, we can prove that \code{p1}, \code{p2}, \code{p3} and \code{p4} are mutually distinct. From this, every valid first-order formula over the types \code{ps} and \code{lines} can be rewritten to a mere tautology and then quickly verified.

To check our model, we first inductively defined the incidence predicates \code{on\_line} and \code{on\_plane} over the types \code{ps} and \code{lines}, as shown in Figure~\ref{fig:SmallestModel}. We then formalised the following theorem in HOL~Light 
\begin{displaymath}
\code{Group1}\ \code{online}\ \code{onplane}.
\end{displaymath}

\begin{figure}
\begin{minipage}[c]{4cm}
\begin{align*}
& \online{\code{p1}}\code{l1}\wedge \online{\code{p2}}\code{l1}\\
\wedge & \online{\code{p1}}\code{l2}\wedge \online{\code{p3}}\code{l2}\\
\wedge & \online{\code{p2}}\code{l3}\wedge \online{\code{p3}}\code{l3}\\
\wedge & \online{\code{p1}}\code{l4}\wedge \online{\code{p4}}\code{l4}\\
\wedge & \online{\code{p2}}\code{l5}\wedge \online{\code{p4}}\code{l5}\\
\wedge & \online{\code{p3}}\code{l6}\wedge \online{\code{p4}}\code{l6}\\
\\
& \onplane{\code{p1}}{\code{p1}}\wedge \onplane{\code{p2}}{\code{p1}}\wedge \onplane{\code{p3}}{\code{p1}}\\
\wedge & \onplane{\code{p1}}{\code{p2}}\wedge \onplane{\code{p2}}{\code{p2}}\wedge \onplane{\code{p4}}{\code{p2}}\\
\wedge & \onplane{\code{p1}}{\code{p3}}\wedge \onplane{\code{p3}}{\code{p3}}\wedge \onplane{\code{p4}}{\code{p3}}\\
\wedge & \onplane{\code{p2}}{\code{p4}}\wedge \onplane{\code{p3}}{\code{p4}}\wedge \onplane{\code{p4}}{\code{p1}}
\end{align*}\end{minipage}\centering\includegraphics[scale=0.8]{linearOrder/Tetra}
\caption{Smallest model}\label{fig:SmallestModel}
\end{figure}

After unfolding the definitions of $\code{on\_line}$ and $\code{on\_plane}$, it turns out that the theorem can be proven by equational reasoning alone. We replace the universals and existentials with conjunctions and disjunctions respectively. After simplifying, the remaining goals require us to show that points, lines or planes are distinct, which is dealt with by the recursion theorem.

The same method allows us to formally deal with the case of a weakened Axiom~\ref{eq:g13a} that we mentioned briefly in \S\ref{sec:DanglingPoints}. We just define a seven element finite set for the lines, and use the same inductive definitions for \code{on\_line} and \code{on\_plane}, leaving the seventh line ``dangling'' without any incident points. Again, by rewriting the axioms propositionally and simplifying, we can show that this is a (presumably inappropriate) model for the weakened axioms, and thus justify our formalisation of Axiom~\ref{eq:g13a}.

To show that the tetrahedron model is minimal, we verified that there exist four points, six lines and four planes satisfying the conditions we gave to inductively define \code{on\_line} and \code{on\_plane} in the model. We know from Theorem~\ref{eq:PlaneThree} that there are three distinct points on a plane. From axiom \code{g18}, there is a fourth point not on this plane. The remaining axioms are then sufficient to connect each pair of points by a unique line.

\section{Infinity}\label{sec:Infinity}
Our axioms now only have infinite models. Indeed, it would seem immediately that we can apply Axiom~\ref{eq:g22} indefinitely to obtain an arbitrary number of points. Starting from distinct points $A$ and $B$, we can obtain points $A$, $B$, $C$, $D$, $E$, $\ldots$, $Y$, $Z$, satisfying 

\begin{align*}
&\between{A}{B}{C}\\
&\between{A}{C}{D}\\
&\between{A}{D}{E}\\
&\ldots\\
&\between{A}{Y}{Z}
\end{align*}

With Theorem~5, we can move from theorems such as $\between{A}{B}{C}$ and $\between{A}{C}{D}$ to $\between{A}{B}{D}$, and so can prove that the points above are mutually distinct. This process therefore gives us a potentially infinite number of points.

\section{A Geometric Successor}
The function which is said to exist in the theorem of infinity is a successor function. It is possible to  witness this theorem, and, staying faithful to Hilbert's prose, we will use a witness that is effectively a function between any two points of a line segment. The witness is based on the diagram in Figure~\ref{fig:successor}.

\begin{figure}
\centering\includegraphics[scale=0.5]{linearOrder/InfinitySteps}
\caption{The successor function}
\label{fig:successor}
\end{figure}

\begin{figure}
\centering\includegraphics[scale=0.5]{linearOrder/InfinityFull}
\caption{Successors tending to $A$}
\label{fig:FullSuccessor}
\end{figure}

Formally, the diagrams are the sets of points satisfying the following constraint

\begin{align}
&\Triangle{a}{A}{B}{0} \label{theorem:IndConstraint}\\
&\qquad\wedge \between{A}{D}{B} \wedge \between{B}{0}{C}\notag\\
&\qquad\wedge (\between{A}{N}{0} \vee N = 0)\notag
\end{align}

Our natural numbers are carved out from the set of all diagrams satisying this property. We abstract this set into a type ($\code{ind}$), giving us two functions: a constructor $\code{mk\_ind}$ which promotes any diagram into the abstract type, and a destructor $\code{dest\_ind}$ which converts an inhabitant of the abstract type into its diagram representation. In order for this to be sound, we only need to prove that the type will have at least one inhabitant. This is easily settled, since the diagram for $0$ is constructed in the proof of Theorem~3 (\S\ref{sec:Theorem3}). 

Each diagram is represented by six points, five of which are fixed by the successor function, while the sixth point $N$ starts at $0$ and moves step-by-step towards $A$. The points $C_1$, $C_2$ and $C_3$, shown in Figure~\ref{fig:successor}, are not explicitly represented in our type but can be determined as the intersection of $BN$ and $CD$.

Informally, and for the purposes of explaining our formalisation, we will identify diagrams up to the equality of the five fixed points. We can then determine a diagram from the sixth point. In this way, when we talk of the object $0$, we may be referring to the \emph{diagram} 0, which is the six-tuple representing inhabitants of $\code{ind}$, or to the \emph{point} 0, which is the sixth component of this six-tuple. Similarly, we shall talk about the \emph{diagram} that is the successor of 0, as well as the \emph{point} that is the successor of 0. This is just for convenience here. The formalisation itself is always unambiguous.

Thus, the successor of $0$ is the diagram obtained by replacing $0$ with $1$, the intersection of $CD$ and $A0$. To obtain the next successor, we first find the intersection of $1D$ and $AC$, namely the point $C_1$. We then replace $1$ with the intersection of $C_1D$ and $A0$. 

In general, the successor of a diagram is obtained by finding $C'$, the intersection of $DN$ and $AC$, and then finding the intersection of $C'D$ and $A0$. Formally:

\begin{align*}
\code{ind\_suc}\, n =& \code{let } (A,B,C,D,0,N) = \code{dest\_ind}\ n \code{ in}\\
&\quad \code{mk\_ind} (A,B,C,D,0,\iota S.\\
&\quad\qquad \exists C'. (\exists a. \online{B}{a} \wedge \online{C'}{a} \wedge \online{N}{a})\\
&\qquad\qquad \wedge (\exists a. \online{C'}{a} \wedge \online{D}{a} \wedge \online{S}{a})\\
&\qquad\qquad \wedge (\exists a. \online{A}{a} \wedge \online{C}{a} \wedge \online{C'}{a})\\
&\qquad\qquad \wedge \between{A}{S}{0} \wedge \between{A}{S}{N})
\end{align*}

We have used the $\iota$ operator in this definition. This is the ``definite description'' operator, which is a weaker version of the $\epsilon$ indefinite description operator. The $\epsilon$ operator is one of the three classical axioms of higher-order logic, and equivalent to the full axiom of choice.
\begin{displaymath}
\forall P\ x. P\ x \implies P (\epsilon x. P x)
\end{displaymath}

From $\epsilon$, we can define the $\iota$ operator, which requires that the predicate $P$ is satisfied by exactly one value. 
\begin{displaymath}
  \iota x. P\ x = \epsilon x. P\ x \wedge \forall y. P\ y \implies x = y
\end{displaymath}

By using this operator and avoiding the somewhat controversial axiom of choice, we feel we are in a better position to argue that we have recovered the axiom of infinity in a logically ``secure'' way. The images of our successor function are uniquely defined from their predecessors, and the natural numbers themselves can be uniquely carved out of the type $\code{ind}$. There is exactly one object $0$ up to relabellings of the points in our figures, not an arbitrary set of possibilities in an abstract type from which we choose one example.

The price comes in the complexity of the $\code{ind}$ representatives. We cannot simply define an infinite domain of points. We need enough information in each of our figures to constrain the possible placement of successors relative to their predecessors.

\subsection{Lemmas}
Our successor function deconstructs an abstract diagram into its six points. It then chooses the unique point $S$, and finally rebuilds the diagram. 

We now need some lemmas concerning $\code{dest\_ind}$. Importantly, we need to show that the reconstructed diagram represents an abstract diagram, and thus show that the image of $\code{mk\_ind}$ in our definition is non-arbitrary. Formally:
\begin{align*}
\code{dest\_ind}\ (\code{ind\_suc}\ n) = & \code{let } (A,B,C,D,0,N) = \code{dest\_ind} n \code{ in}\\
&\quad (A,B,C,D,0,\iota S.\\
&\quad\quad \exists C'. (\exists a. \online{B}{a} \wedge \online{C'}{a} \wedge \online{N}{a})\\
&\qquad\quad \wedge (\exists a. \online{C'}{a} \wedge \online{D}{a} \wedge \online{S}{a})\\
&\qquad\quad \wedge (\exists a. \online{A}{a} \wedge \online{C'}{a} \wedge \online{D}{a})\\
&\qquad\quad \wedge \between{A}{S}{0} \wedge \between{A}{S}{N})
\end{align*} 

The key step needed to verify this theorem comes from realising that the diagrams involving the points $A,B,C,0,D$ and $A,B,C',N,D$ satisfy the same constraints. In each case, we apply Pasch's axiom \eqref{eq:g24} to $\triangle AB0$ and the line $CD$ to obtain a point $S$ between $A$ and $0$. For our initial diagram, we can apply this argument directly. For the other diagrams, we just need to find the point $C'$.

To do this, we apply Pasch's axiom \eqref{eq:g24} to $\triangle A0C$ and the line $DN$, to place the point $C'$ between $A$ and $C$. We can now use the previous argument to locate $S$ between $A$ and $N$. Finally, since $N$ is between $A$ and $0$, \ref{eq:five} shows that the point $S$ must also lie between $A$ and $0$. 

\section{Theorem of Infinity}
Finally, we must verify that our function is one-one but not onto. To verify that $\code{ind\_suc}$ is one-one, we went procedurally, but with the assistance of our discoverers via \code{discover\_tac by\_eqs}. This used incidence reasoning to automatically find equalities that arise from our diagrams. 

We verified the fact that \code{ind\_suc} is not onto declaratively. The basic verification works by noting that $0$ defines the first diagram, while all images of $\code{ind\_suc}$ use a point $S$ which is defined to be strictly between $A$ and $0$. Thus, $0$ is not in the image of the function.
\begin{equation}
\tag{\ref{eq:Infinity}}
\code{one\_one}\ \code{ind\_suc} \wedge \neg\code{onto}\ \code{ind\_suc}.
\end{equation}

This now shows that the abstract type $\code{ind}$ has an infinite domain. However, the domain is not the natural numbers. Since we are allowed to take the successor in any diagram where the point $N$ is any point lying between $A$ and $0$, or is the point $0$ itself, it is clear that most of the images of $\code{ind\_suc}$ are \emph{not} in the sequence
\begin{displaymath}
0, \code{ind\_suc } 0, \code{ind\_suc }(\code{ind\_suc } 0), \code{ind\_suc }(\code{ind\_suc }(\code{ind\_suc } 0)),  \ldots.
\end{displaymath}
For instance, no point between $0$ and $\code{ind\_suc}\ 0$ is in this sequence.

To remove the unwanted diagrams, we follow HOL~Light's construction of the natural numbers, which inductively restricts us to the smallest closure of the successor function starting from $0$. 

% \section{Infinity with Linear Reasoning}
% We now return to the more intuitive argument for an infinite set given at the beginning of \S\ref{sec:Infinity}. This argument shows that there must be an infinite number of points on a line, since we can start with any line segment and extend it indefinitely. We can prove this quite neatly. We first find a suitable type, and prove that it is inhabited.

% \begin{align*}
% \exists (A,X,B). A \neq B \wedge \forall P. P \in X \rightarrow \between{A}{P}{B}.
% \end{align*}

% Thus, we are defining an abstract type represented by triples. The first and last component of each triple are the end-points of a line segment. The second component is some subset of the points in between the end-points (our chosen example are two of the points given by Axiom~I,3 together with the empty set).

% The successor function deconstructs the abstract type, and then maps the resulting triple $(A,X,B)$ to $(A,X \cup \{B\}, C)$ where $C$ is an arbitrarily chosen point extending the segment $AB$ as per Axiom~II,2. We are only required to show that the new segment $AC$ bounds all the points in $X \cup \{B\}$, a matter which is guaranteed by Theorem~5.

% Proving that this function is not onto is again simple. The triple whose second component is the empty set cannot possibly be in the image of the successor function, since the successor function always produces a non-empty set in the second component.

% To prove that the function is one-one, we must show the following:

% \begin{displaymath}
% (A,X \cup \{B\}, \epsilon C. \between{A}{B}{C}) = (A,X' \cup \{B'\}, \epsilon C. \between{A}{B'}{C}).
% \end{displaymath}

% Well, if $B\neq B'$, it would follow that $B\in X'$ while $B'\in X$. Therefore, $\between{A}{B}{B'}$ and $\between{A}{B'}{B}$ which contradicts Axiom~II,3. Thus, \mbox{$B = B'$}, and so $X = X'$ and since the third components now choose from the same predicate, the third components are also equal.

% This gives us another proof of the theorem of infinity, using only our linear axioms and the linear Theorem~5. However, it suffers from the fact that it relies on the axiom of choice to pick out an \emph{arbitrary} extension to every line segment, while the previous proof of the theorem used only uniquely definable points.

\section{Theorem~6 Revisited}
Now that we have two ways to carve out the natural numbers from geometric objects, we can return to our verification of \S\ref{sec:Theorem6}. Recall our definition of an \emph{ordering} from \S\ref{sec:OrderingDef}:
\begin{equation}
  \tag{\ref{eq:OrderingDef}}
  \begin{split}
    \code{ordering}\ f\ X &\iff X = f\left(\left\{n\ \vert\ \code{finite}\ X \implies n < \left|X\right|\right\}\right)\\
    &\wedge \forall n\ n'\ n''. \code{finite}\ X \implies n < \left|X\right| \wedge n' < \left|X\right| \wedge n'' < \left|X\right|\\
    &\qquad\wedge n < n' \wedge n' < n'' \wedge \between{(f\ n)}{(f\ n')}{(f\ n'')}.
    \end{split}
\end{equation}

There is a lot of symmetry in this definition. Firstly, we have
\begin{displaymath}
\between{(f\ n)}{(f\ n')}{(f\ n'')} \iff \between{(f\ n'')}{(f\ n')}{(f\ n)}
\end{displaymath}
since we can always permute the first and last arguments of the \code{between} predicate. Secondly, given three distinct points $n$, $n'$ and $n''$, there are six ways to arrange them so that they are linearly ordered.

In some cases, we can take care of this latter symmetry by following a simple technique~\cite{HarrisonWLOG} in proving a suitable  without-loss-of-generality theorem. Informally, our theorem says that if a formula is true regardless of how one arranges the free-variables, then we can assume a specific ordering on those variables. Formally, for the case of three variables:
\begin{displaymath}
  \begin{split}
    &(\forall m\ n. P\ m\ m\ n)\\
    &\wedge \left(\forall m\ n\ p. P\ m\ n\ p \iff P\ n\ m\ p\right)\\
    &\wedge \left(\forall m\ n\ p. P\ m\ n\ p \iff P\ n\ p\ m\right)\\
    &\wedge (\forall m\ n\ p. m < n \wedge n < p \implies P\ m\ n\ p)\\
    &\implies \forall m\ n\ p. P\ m\ n\ p.
  \end{split}
\end{displaymath}

The second and third conditions require that the predicate $P$ holds up to any symmetry in its arguments. With these, the first condition requires that the predicate is true assuming that any two of its arguments are equal. Thus, in the fourth condition, we can assume that the the first argument is strictly less than the second which is strictly less than the third, and from this conclude that the predicate holds for any three natural numbers (which, we recall, are concretely represented by a sequence of geometric figures). 

In practice, the idea to using this theorem is to match its conclusion to the goal, discharge the first three conditions using simplification, and then attempt to prove the remaining ``without loss of generality'' condition.

%Another useful definition we shall give is for the \emph{bounds} or extremities of a set of points:

%In order to make this definition useful, we need to supply a set of lemmas and theorems that are sufficiently rich that we rarely need to unfold the definitions. Figure~\ref{fig:OrderLemmas}\footnote{The term $\text{INJ}\,f\,X$ says that $f$ is injective on the set $X$} shows the main formalisations.

% \begin{figure}
% \begin{align}
% &\forall P\,X. \bounds{P}{P}{X} \iff X = \{P\}\\
% &\forall P\,Q\,X. \bounds{P}{Q}{X} \iff \bounds{Q}{P}{X}\\
% &\forall A\,B\,P\,Q\,X. \bounds{A}{B}{X} \wedge \bounds{P}{Q}{X} \implies A = P \vee A = Q\\
% &\forall P\,Q\,X. \bounds{P}{Q}{X} \implies \forall A\,B. A \in X \wedge B \in X \implies \neg\between{A}{P}{X}\\
% &\forall P\,Q\,R\,X. \bounds{P}{Q}{X} \wedge (\forall A\,B. A \in X \wedge B \in X \wedge \neg\between{A}{R}{B})\notag\\
% &\qquad\implies R = P \vee R = Q\\
% &\forall P\,Q\,R. \code{collinear} \{P,Q,R\} \implies \exists f. ORDERING f \{P,Q,R\}\\
% &\forall f\,X. \code{ordering}{f}{X} \implies \text{INJ}\,f\,\{ n \vert FINITE X \implies n < |X| \}\\
% &\forall f\,X. \code{FINITE} X \label{eq:OrderRev}\\
% &\quad\implies (\code{ordering}{f}{X} \iff \code{ordering} (\lambda n. f\,(|X| - n - 1)) X)\\
% &\forall f\,X\,x. x \in X \wedge \code{FINITE} X \wedge \code{ordering}{f}{X}\label{eq:OrderBounds}\\
% &\quad\implies \bounds{(f 0)}{(f (CARD X - 1))}{X}\\
% &\forall f\,x\,X. \code{FINITE} X \wedge \code{ordering}{f}{X} \wedge \between{(f 0)}{(f (|X| - 1))}{x}\notag\\
% &\qquad \implies \exists g. \code{ordering}{g}{(\{x\} \cup X)}\label{eq:OrderExtend}\\
% &\forall f\,P\,Q\,X\,x. x \in X \wedge \code{FINITE} X \wedge \code{ordering}{f}{X} \wedge \bounds{P}{Q}{X}\notag\\
% &\qquad\implies\exists f'. \code{ordering}{f'}{X} \wedge f'\,0 = P \wedge f'\,(CARD X - 1) = Q\label{eq:ChooseOrder}\\
% &\forall f\,X. \code{ordering}{f}{X} \implies \code{collinear} X
% \end{align}
% \caption{Order Lemmas}
% \label{fig:OrderLemmas}
% \end{figure}

\subsection{Verification}
% \begin{displaymath}\label{eq:BoundDef}
% \bounds{P}{Q}{X} \iff P \in X \wedge Q \in X \wedge \forall R. R\in (X - \{P,Q\}) \implies \between{P}{R}{Q}.
% \end{displaymath}
In verifying \ref{eq:six}, we have tried to reinforce Hilbert's claim that it is a generalisation of Theorem~5 by showing how to generalise his \emph{proof}. Up until now, we have not even considered the special case of the proof, which makes up the third and final part of the proof of Theorem~5. This is given in Figure~\ref{fig:Theorem5Cases}

\begin{figure}
\framebox{\begin{minipage}{\linewidth}Now let any four points on a line be given. Take three of the points and label $Q$ the one which by Theorem~4 and Axiom~II,3 lies between the other two and label the other two $P$ and $R$. Finally, label $S$ the last of the four points. By Axiom~II,3 and Theorem~4 again it follows then that the following five distinct possibilities for the position of $S$ exist:
\\

\indent $R$ lies between $P$ and $S$,\\
\indent or $P$ lies between $R$ and $S$,\\
\indent or $S$ lies between $P$ and $R$ simultaneously when $Q$ lies between $P$ and $S$,\\
\indent or $S$ lies between $P$ and $Q$,\\
\indent or $P$ lies between $Q$ and $S$.
\\

The first four possibilities satisfy the hypotheses of [the second lemma] and the last one satisfies those of [the first lemma]. Theorem~5 is thus proved.\end{minipage}}
\caption{Case-Analysis for Theorem~5}
\label{fig:Theorem5Cases}
\end{figure}

For the purposes of verification, we probably want to tidy up Hilbert's case-analysis. If we take Hilbert's last three clauses to be a nested case-analysis, the last of which is contradictory and can be discarded, then we just have:
\begin{displaymath}
\begin{cases}
$R$ \text{ lies between } $P$ \text{ and } $S$,\\
$P$ \text{ lies between } $R$ \text{ and } $S$,\\
$S$ \text{ lies between } $P$ \text{ and } $R$,
\begin{cases}
  $Q$ \text{ lies between } $P$ \text{ and } $S$,\\
  $S$ \text{ lies between } $P$ \text{ and } $Q$
\end{cases}
\end{cases}
\end{displaymath}

Hilbert says the case-analysis arises from applications of \ref{eq:four}, which tells us that of three points, one lies between the other two. When we generalise from four points to $n+1$ points, we apply induction, and two of the applications of \ref{eq:four} become applications of our inductive hypothesis to $n$ points.

\begin{proof}
Now let $n+1$ points on a line be given. Take $n$ of the points and label $P$, $Q_1$, $Q_2$, $\ldots$, $Q_{n-2}$, $R$ the ones which by our inductive hypothesis are ordered along the line. Finally, label $S$ the last of the $n+1$ points. By Axiom~\ref{eq:g23} and \ref{eq:four} it follows then that the following three distinct possibilities for the position of $S$ exist:

\vspace{0.5cm}
\noident $R$ lies between $P$ and $S$,\\
\noident or $P$ lies between $R$ and $S$,\\
\noident or $S$ lies between $P$ and $R$.
\vspace{0.5cm}

In the last case, we apply our inductive hypothesis to the points $P$, $Q_1$, $Q_2$, $\ldots$, $Q_{n-2}$, $S$. In any of these cases, we can apply Theorem~\ref{eq:OrderExtend} to label all the points in order.
\begin{align}
  \label{eq:BoundsDef}
  &\code{bounds}\ P\ Q\ X \iff P,Q \in X \wedge \forall R. R \in X - \{P,Q\} \implies \between{P}{R}{Q}\\
  \label{eq:OrderExtend}
  &\code{finite}\ X \wedge \code{ordering}\ f\ X \wedge \between{(f 0)}{(f (|X| - 1))}{x}\notag\\
  &\qquad \implies \exists g. \code{ordering}\ g\ (\{x\} \cup X)\\
  \label{eq:ChooseOrder}
  &x \in X \wedge \code{finite}\ X \wedge \code{ordering}\ f\ X \wedge \code{bounds}\ P\ Q\ X\notag\\
  &\qquad\implies\exists f'. \code{ordering}\ f'\ X \wedge f'\,0 = P \wedge f'\,(|X| - 1) = Q.
\end{align}

\end{proof}
We have replaced Hilbert's reference to his earlier parts of \ref{eq:five} with a reference to its generalisation in Theorem~\ref{eq:OrderExtend}. We actually need a few other theorems. Consider that after our final application of the inductive hypothesis, we will have two orderings $f$ and $g$. The ordering $f$ applies to the points $P,$ $Q_1$, $Q_2$, $\ldots$, $Q_{n-2},$ $R$, and the ordering $g$ applies to the points $P$, $Q_1$, $Q_2$, $\ldots$, $Q_{n-2}$, $S$. But  there is some notational abuse here, as we gloss over an implicit assumption that $f(0) = g(0) = P$. In our verification, we can make this reasoning explicit with Theorem~\ref{eq:ChooseOrder}, which uses the auxiliary concept of \code{bounds}. 

Our generalised proof handles its case-analyses by one application of Theorem~4, and two applications of an induction hypothesis. This means we must apply well-founded induction rather than normal structural induction, since we must instantiate our inductive hypothesis in two different ways. 

The base case of the induction is captured in two lemmas:
\begin{displaymath}
  \begin{aligned}
    &\exists f. \code{ordering}\ f\ \emptyset\\
    &\code{collinear} \{x,y,z\} \implies \exists f. \code{ordering}\ f\ \{x,y,z\}.
  \end{aligned}
\end{displaymath}

The case for the empty set is trivial: literally any function witnesses the existential, since the conditions on the function are all vacuous. The second case actually breaks down into four separate cases, depending on whether any of the $x$, $y$ or $z$ are equal. For a single point, we can pick the constant function to that point, and for two points $x \neq y$, we pick the function which maps $0$ to $x$ and everything else to $y$.

For three points $\between{P}{Q}{R}$, we have the ordering which maps $0$ to $P$, $1$ to $Q$ and all other numbers to $R$. Since Theorem~4 requires that, of any three collinear points, one lies between the other two, it follows that there must be an ordering for any three collinear points. This takes care of the base case. And thus, we formally verify:
\begin{equation}
  \label{eq:six}
  \tag{Theorem~6}
  \begin{split}
    \code{finite}\ X \wedge \code{collinear}\ X \implies \exists f. \code{ordering}\ f\ X.
  \end{split}
\end{equation}

\subsection{Exactly Two Orderings}
Hilbert closes his statement of Theorem~6 with an obvious point which we found to have a slightly challenging formal proof and verification: ``Besides this order of labelling there is only the reverse one that has the same property.''.

To verify this, we began with a lemma: if we have two orders $f$ and $g$ where\linebreak $f(0)~=~g(0)$, then the orders are identical. 
\begin{multline}
  \code{ordering}\ f\ X \wedge \code{ordering}\ g\ X \wedge f\ 0 = g\ 0
  \wedge (\code{finite}\ x \implies n < |X|) \implies f\ n = g\ n.
\end{multline}

The verification of this theorem uses induction. Aiming for a contradiction, we assume that $f\ (n+1) \neq g\ (n+1)$ and then consider the relative positions of $f\ 0$, $f\ (n+1)$ and $g\ (n+1)$ that arise from \ref{eq:four}. For each possibility, we can apply \ref{eq:five} to show that there will end up being a point between $f\ n$ and $f\ (n+1)$, or a point between $g\ n$ and $g\ (n+1)$. But the existence of such a point contradicts the verified fact that orderings are injections:
\begin{multline}
  \code{ordering}\ f\ X \wedge (\code{finite}\ X \implies m < |X| \wedge n < |X|) \wedge f\ m = f\ n \implies m=n.
\end{multline}

Putting these facts together, we can verify Hilbert's assertion.
\begin{multline}
%`!f g X. FINITE X /\ ORDERING f X /\ ORDERING g X ==>
%               (!n. (FINITE X ==> n < CARD X) ==> f n = g n)
%               \/ !n. (FINITE X ==> n < CARD X) ==> f n = g (CARD X - n - 1)`
  \code{finite}\ X \wedge \code{ordering}\ f\ X \wedge \code{ordering}\ g\ X \wedge (\code{finite}\ X \implies n < |X|)\\
  \implies \forall n. f\ n = g\ n \vee f\ n = g (|X| - n - 1)
\end{multline}

\section{An Ordering Tactic}
\ref{eq:six} says everything one wants to know about the order of a finite number of points on a line, but it is not immediately obvious how to apply it.

One thing we can do with \ref{eq:six} is use it to convert problems involving betweenness into problems of linear arithmetic. To handle this, we will need to consider what is basically the inverse of the \code{ordering} function, given as $f$ in the following theorem:
\begin{equation}
\label{eq:OrderInverse}
\begin{split}
  &\code{finite}\ x \wedge \code{collinear}\ X\\
  &\implies \exists f. \forall A\ B\ C.\ A \in X \wedge B \in X \wedge C \in X \\
  &\qquad\qquad\implies (\between{A}{B}{C} \\
  &\qquad\qquad\qquad\qquad\iff (f\ A < f\ B \wedge f\ B < f\ C) \vee (f\ C < f\ B \wedge f\ B < f\ A))\\
  &\quad\wedge \forall A\ B.\ A \in X \wedge B \in X \implies (A = B \iff f\ A = f\ B)
\end{split}
\end{equation}

On the assumption that one has a collinear and finite set of points $X$, this theorem allows us to obtain a function $f$ with which we can take goals in terms of betweenness and equalities of points, and rewrite them into inequalities and equations of natural numbers. Once rewritten, the goal can be solved by HOL~Light's decision procedure for linear arithmetic. The procedure is not particularly efficient, but in practice, we have only considered simple betweenness problems (at most, ones involving six points).

When discharging the assumption in Theorem~\ref{eq:OrderInverse}, we expect the user to have instantiated $X$ with a concrete set enumeration (an expression of the form $\{P_1,P_2,\ldots,P_n\}$). These sets can be proven finite by simple rewriting. To prove them collinear, we reuse our incidence discoverer from Chapter~\ref{chapter:Automation}.

We package this up as the procedural tactic \code{ORDER\_TAC}, which is parameterised on a concrete set enumeration. It solves its goal by instantiating $X$ in Theorem~\ref{eq:OrderInverse}, running an incidence discoverer as a tactic via \code{discover\_tac} (see \S\ref{sec:DiscoverTac}), and finally discharging the assumption in Theorem~\ref{eq:OrderInverse} by rewriting. It then hands over to HOL~Light's decision procedure for linear arithmetic, \code{ARITH\_TAC}.

\subsection{Example: Theorem~7}
To round up this chapter, we will apply our linear reasoning tactic \code{order\_tac} to a theorem which says that there is an infinite set between any two points. 

In a sense, this has been covered indirectly by Theorem~\ref{eq:Infinity}, but our purpose with that theorem was to settle foundational questions about our logical assumptions, and provide us with enough expressive power to formally verify Theorem~6. Here, we can be more direct in our formalisation.
\begin{quotation}THEOREM~7. Between any two points on a line there exists an infinite number of points.
\end{quotation}
\begin{equation}
\label{eq:seven}
P\neq Q \implies \code{infinite}\ \{R\; \vert\; \between{P}{R}{Q}\}.
\end{equation}

We start our proof by assuming that the set of points between $P$ and $Q$ is finite. We then consider separately whether the set contains fewer than two elements, and whether it contains more than two elements. 

We describe the proof briefly. In the first case, we just use Theorem~3 twice to find two points between $P$ and $Q$, from which we can obtain a contradiction. In the second case, we get to apply \ref{eq:six}. The basic idea is as follows: we obtain an ordering $f$ of all points between $P$ and $Q$, and then take the first two elements of this ordering, namely $f\ 0$ and $f\ 1$. Via Theorem~3, we can find a point $R$ that lies strictly between them. According to our assumption, this point must be in the image of $f$. But this contradicts the definition of an ordering.

The part of our verification where we apply our linear ordering tactic might come as a surprise. It is actually used to verify a point glossed over in the informal argument, namely, that $R$ must be in the image of $f$. To show this, we must verify that $R$ lies between $P$ and $Q$.

Before we had implemented our tactic, we tried to verify this matter directly using \ref{eq:four} and \ref{eq:five}, but we gave up. The necessary case-analyses were just not intuitive to us. But \code{ORDER\_TAC} takes care of the matter elegantly.

\fbox{\begin{minipage}{\linewidth}\setlength\abovedisplayskip{0cm}
\small
\begin{align*}
  &\ldots\\
  &\code{so consider}\ R\ \code{such that}\ \between{(f\ 0)}{R}{(f 1)} & 7\\
  &\code{have}\ \between{P}{(f\ 0)}{Q} \wedge \between{P}{(f\ 1)}{Q}
  \ \code{from}\ 6\ \code{by} \ldots & 8\\
  &\code{hence}\ \between{P}{R}{Q}\ \code{from}\ 6,7\ \code{using}\ \code{ORDER\_TAC}\ \{P,Q,R,f\ 0,f\ 1\}\\
\ldots
\end{align*}\end{minipage}}\linebreak

\section{Conclusion}
In this chapter, we tackled the verification of Hilbert's Theorem~6. This is a metatheorem, so its verification requires a logic with more expressive power than is available in first-order logic. 

We aim to be conservative, so while we assume the power of higher-order logic, we do not assume the axiom of infinity from which we normally acquire the natural numbers. Instead, we have shown that Hilbert's first two groups of axioms imply infinity as a theorem. On the way, we verified that the first group alone cannot do this, since it admits a finite model.

We described two ways that we can implement Theorem~6 as a tactic. We suggest this is more faithful to Hilbert's intentions, since the theorem is metatheoretical and computational, manipulating labellings according to a scheme.

Our first tactic requires settling on a way to represent stacked betweenness claims and then writing procedures to enumerate all possible stackings for a given finite list of points. Our second strategy was to use our formally verified Theorem~6 to convert betweenness problems into arithmetical problems. That way, we can reuse decision procedures for linear arithmetic.

The second tactic has the advantage that it automatically copes with problems with equations and inequalities. In our first tactic, we assume that all points are distinct when we enumerate the possible orderings,  and we found that this assumption is too strong in practice. That said, we expect that our first tactic has potentially better performance, since it is more carefully tailored to Hilbert's geometry. We leave analysis of this matter to future work.

This concludes the description of the automation we have implemented for our verification: in summary, we use a search algebra to handle the implicit incidence reasoning from Group~I, and a tactic to handle linear reasoning from Group~II. These two automated tools were used extensively in the rest of our verification, which will be discussed in the remaining chapters. 

%The step case in our recursive metalevel procedure used two recursive calls on two different sets of points. So if we intend to do the same for the inductive hypothesis, then our induction cannot be the default one for finite sets, which has the form

% \begin{displaymath}
% \vdash \forall P. P\,\emptyset \wedge (\forall x\, X. P\,X \implies P\,(\{x\} \cup X)) \implies \forall x. P\,X.
% \end{displaymath}

%Instead, we use natural number induction on the size of the set. We assume that we can order $n$ points, and must show how to order a set $X$ containing $n+1$ points. Because of our base case, we can assume that $n \geq 3$.

%Looking at the cases for Theorem~5 from Figure~\ref{fig:Theorem5Cases}, it is clear that Hilbert is considering three case-splits based on Theorem~4. First, he considers the position of three of the points, stating ``label $Q$ the one which by Theorem~4 and Axiom II,3 lies between the other two and label the other two $P$ and $R$''. If Hilbert is talking again about ``labellings'', then we should be talking about ``orderings''. And thus, this does not generalise to an application of Theorem~4, but an application of our inductive hypothesis to $n$ of the points of $X.$ Thus, our first step is to obtain an ordering $f$ of $n$ of the points.

%From his next two cases, it is clear that Hilbert next applies Theorem~4 to the points he has labelled $P$ and $R$, and to the remaining point $S$. Let us call \emph{our} remaining point $x$. Then we recognise that $P$ and $R$ are the bounds of the ordering of $P$, $Q$ and $R$, which in the general case are the bounds $f\,0$ and $f\,(n - 1)$.

%In our generalisation we apply Theorem~4 to the points $x$, $f\,0$ and $f\,(n - 1)$. Hilbert's first two cases, where $x$ does not lie between the bounds are easily dealt with. In fact, we can assume (by normalising with Axiom~II,1 and applying \eqref{eq:OrderRev}) that there is an ordering $g$ of $X - \{x\}$ such that $\between{(g\,0)}{(g\,(n - 1)}{x}$. From this, we obtain our result directly from \eqref{eq:OrderExtend}.

%As with Hilbert's cases, when $x$ lies between the bounds, we have another case-split. Hilbert applies Theorem~4 once again, this time to the points $P$, $Q$ and $S$. In our generalisation, $P$ is $f\,0$ and $S$ is $x$. What is $Q$ however? In our generalisation, it must be \emph{all} the points between $P$ and $f(n - 1)$. So once again, this is \emph{not} an application of Theorem~4, but an application of the inductive hypothesis. That is, we are going to order the points $f\,0,f\,1,\ldots,f\,(n - 2),x$ to give us another ordering $g$.

%Using the lemmas from Figure~\ref{fig:OrderLemmas}, we can choose this ordering such that $f\,0 = g\,0$. As with Hilbert, there is another case-split to consider. Hilbert asks whether $Q$ lies between $P$ and $S$. This would make the point $S$ a bound in the ordering of $P$, $Q$ and $S$, which nicely generalises to the case $g(n - 1) = x$. We can deal with this by an immediate application of \eqref{eq:OrderExtend}.

%This brings us to Hilbert's penultimate case where $S$ lies between $P$ and $Q$. In other words, $S$ lies between the bounds of $P$ and $Q$. We generalise by noting that $f$ orders all points bar $x$. So $g(n - 1)$ must lie between $f 0$ and $f(n-1)$. Hence, a final application of \eqref{eq:OrderExtend} gives us our desired conclusion.

%Thus, we have formally generalised Hilbert's proof of Theorem~5, and seen how two of his applications of Theorem~4 are in fact uses of an induction hypothesis. Furthermore, we have shown that the final step in each case is an application of \eqref{eq:OrderExtend}.
%%% Local Variables: 
%%% TeX-master: "../thesis"
%%% End: 


\chapter{Ordering in the Plane}\label{chapter:HalfPlanes}
In the last chapter, we used linear arithmetic to settle problems of points ordered along a line. But what if we want to reason about the relative positions of points in a \emph{plane}? 

For this, we note that a line partitions a plane into two sides. So we can compare the relative position of points in the plane by asking on which side they are on. These \emph{sides} are defined in the \emph{Grundlagen der Geometrie} in Hilbert's next result, THEOREM~8, which he cites as the key theorem needed for the Polygonal Jordan Curve Theorem.

\section{Definitions and Formalisation}
\begin{quotation}
THEOREM~8. Every line $a$ that lies in a plane $\alpha$ separates the points which are not on the plane $\alpha$ into two regions with the following property: Every point $A$ of one region determines with every point $B$ of the other region a segment $AB$ on which there lies a point of the line $a$. However any two points $A$ and $A'$ of one and the same region determine a segment $AA'$ that contains no point of $a$.

\begin{center}\includegraphics{halfPlanes/halfPlanesDef}\end{center}

DEFINITION. \emph{The points $A$, $A'$ are said to lie in the plane $\alpha$ on one and the same side of the line $a$ and the points $A$, $B$ are said to lie in the plane $\alpha$ on different sides of $a$.}

DEFINITION. Let $A$, $A'$, $O$, $B$ be four points of the line $a$ such that $O$ lies between $A$ and $B$ but not between $A$ and $A'$. The points $A$, $A'$ are then said to lie \emph{on the line $a$ on one and the same side of the point $O$ and the points $A$, $B$ are said to lie on the line on different sides of the point $O$. The totality of points of the line $A$ that lie on one and the same side of $O$ is called a \emph{ray} emanating from $O$. Thus every point of a line partitions it into two rays.}

\flushright{\emph{Foundations of Geometry}~\cite{FoundationsOfGeometry} (page 8)}
\end{quotation}

Though we shall not need the notion of rays in our verifications, we shall describe some of its formalisation. The definition is effectively the one-dimensional analogue of THEOREM~8, and we will use it to explain the general approach to verifying that theorem. We also mention that \emph{rays} make for a useful abstraction in later definitions and axioms (see \S\ref{sec:RaysDef} below).

Now for rays, a three-place relation $\code{same\_side}$ relating a point with every other pair of points on a line is redundant when we can just relate a ray with its incident points. So we drop the relation. Similarly, we drop the relation for planes and instead introduce the two-dimensional analogue of rays, namely \emph{half-planes}. A half-plane will be the totality of points on the same side of a line $a$ in the plane $\alpha$. So when we say that two points lie on the same side of the line $a$ in the plane $\alpha$, we shall mean that they lie on a single half-plane in $\alpha$ and bounded by $a$. 

\subsection{Rays}\label{sec:RaysDef}
We briefly justify keeping the definition of rays in our formalisation. Rays are useful in Group~III, where Hilbert introduces an axiom governing congruence of angles:

\begin{quotation}
  Let $\alpha$ be a plane and $h$, $k$ any two distinct rays emanating from $O$ in $\alpha$ and lying on {\bfseries distinct lines}. The pair of rays $h$, $k$ is called an \emph{angle} and is denoted by $\angle(h,k)$ or by $\angle(k,h)$.

\ldots

III,~4. \emph{Let $\angle (h,k)$ be an angle in a plane $\alpha$ and $a'$ a line in a plane $\alpha'$ and let a definite side of $a'$ in $\alpha'$ be given. Let $h'$ be a ray on the line $a'$ that emanates from the point $O'$. Then there exists in the plane $\alpha'$ one and only one ray $k'$ such that the angle $\angle (h,k)$ is congruent or equal to the angle $\angle (h',k')$ and at the same time all interior points of the angle $\angle (h',k')$ lie on the given side of $a'$.}
\flushright{\emph{Foundations of Geometry}~\cite{FoundationsOfGeometry} (page 11)}
\end{quotation}

Notice how much more complicated Hilbert's axioms have become by this group. Here, we have an axiom which juggles eight geometric entities, six of which are not even primitive. It is easy to make a mistake here and we recommend that, if this axiom is to be reliably formalised, that the notions of \emph{angle}, and thus, the dependent notion of \emph{ray} must be fully formalised and a decent theory developed before we can trust that the definitions and axiom are correct. In simple type theory, one can gain further confidence by introducing rays as an abstract type, which can simplify the formalisation of Axiom~III,~4 by pushing constraints into the type-checker (see earlier work~\cite{ScottMScThesis}). 

\subsection{Quotienting}\label{sec:RayQuotienting}
With a slight clarification, the ``same side'' relations in Hilbert's definitions define equivalence relations, and rays and half-planes emerge as the equivalence classes. In particular, the relation ``same side of the point $O$'' quotients the set of points in space other than a point $O$ into the set of all rays emanating from $O$, or alternatively, with \emph{origin} $O$. Similarly, the relation ``same side of the line $a$'' quotients the set of points in space not on the line $a$ into the set of all half-planes bounded by $a$. Note that we do not restrict the dimension here, and thus allow rays and half-planes to emanate from a line in all directions in space. 

We have had to fill in an ambiguity in Hilbert's definition, and exclude the closure points or boundary from both rays and half-planes. If we include the point $O$ for a ray, or the line $a$ for a half-plane, and we allow an arbitrary point to be on the same side of $O$ or $a$, then we will have only one equivalence class: the whole of space. If we include $O$ and $a$ in every ray and half-plane, but declare all other points to be on a different side of $O$ and $a$, then our equivalence classes tell us that the set $\{O\}$ counts as a zero-dimensional ray, while the line $a$ counts as a one-dimensional half-plane. We exclude these possibilities, and thus make all rays and half-planes, as equivalence classes, open sets: a ray does not include its origin and a half-plane does not include its boundary. Incidentally, Poincar\'{e} made the same decision, remarking parenthetically in his review of Hilbert ``I add, for precision, that I consider [the origin] as not belonging to either [half-ray]''~\cite{PoincareReview}.

\subsection{Automatic Lifting}
HOL~Light has several powerful procedures for automatically dealing with quotienting and producing a strong type for the quotient sets. Assuming that $(\equiv)\;:\;\tau\rightarrow \tau \rightarrow \tau$ is an equivalence relation on $\tau$, there is a procedure which splits $\tau$ into equivalence classes. A new abstract type is then introduced in the theory, isomorphic with the class of all these equivalence classes. Additional procedures then exist which allow the user to \emph{lift} HOL functions which are provably well-defined for the equivalence relation to the abstract type. We wanted to use these facilities to introduce the new abstract types of rays and half-planes, and so introduce our primitive relations on these abstract types by lifting well-defined relations. 

Unfortunately, we do not have types for the domains of the equivalence relations. The ``same side'' relations Hilbert defines are only equivalence relations on families of subsets of space. Our equivalence relations for rays are indexed by a point $O$ and have as domain the set of points in space minus $O$. Our equivalence relations for half-planes are indexed by a line $a$ and have as domain the set of points in space minus $a$. Simple type theory does not allow us to consider these families at the type-level.

We were not sure how best to tackle this problem, and we have not implemented a generic solution. Instead, in this section, we review one possible strategy which takes us to our new quotiented type via an intermediate type. Here, we will only consider the strategy applied to rays. The half-planes case is exactly analogous.

\subsubsection{Intermediate Types}
For any point $O$, we must consider the set of all points $P$ in space which are not on $O$. We can do this by creating an abstract type of pairs $(O,P)$ where both components are distinct. We call this type \code{arrow}. It is the type of directed line-segments, or \emph{arrows} $\overrightarrow{OP}$, where $O \neq P$. The origin of the arrow is the point $O$, and the arrow \emph{points} in the direction $P$. By defining this type, we obtain abstraction and representation functions $\code{mk\_arrow}$ and $\code{dest\_arrow}$, similar to $\code{mk\_ind}$ and $\code{dest\_ind}$ from the last chapter. These map back-and-forth between pairs of points and the arrows they represent.

The relation ``same side of'' can now be reinterpreted on these arrows. Our relation will effectively ask whether two arrows have the same position and direction. With some abuse of HOL~Light notation (we pretend that we can extract the two endpoints of an arrow with a \emph{pattern match}), we have:
\begin{equation}\label{eq:EquivArrowDef}
  \begin{split}
    &\code{equiv\_arrow}\ :\ \code{arrow} \rightarrow \code{arrow}\rightarrow \code{bool}\\
    \vdash_{def}\; &\code{equiv\_arrow}\ \overrightarrow{OP}\ \overrightarrow{O'Q}\\
    &\qquad\iff O = O' \wedge (P = Q \vee \between{O}{P}{Q} \vee \between{O}{Q}{P}).
  \end{split}
\end{equation}

We now just verify that this relation is an equivalence relation on the type of arrows:
\begin{equation*}
  \begin{split}
    \vdash&\code{equiv\_arrow}\ s\ s\\
    &\wedge (\code{equiv\_arrow}\ s\ t \iff \code{equiv\_arrow}\ t\ s)\\
    &\wedge (\code{equiv\_arrow}\ s\ t \wedge \code{equiv\_arrow}\ t\ u \implies\code{equiv\_arrow}\ s\ u).
  \end{split}
\end{equation*}

\label{sec:RayTransitivity}The only challenge when verifying this theorem is dealing with transitivity. In our earlier work~\cite{ScottMScThesis}, where we tried to define rays as equivalence classes without using any automatic quotienting, the verification took some hard pen-and-paper work before we could transcribe it. We were bogged down with picky variable instantiations needed to apply \ref{eq:five}, made worse by the disjunction in our definition \eqref{eq:EquivArrowDef} which throws up several case-splits. But in our HOL~Light development, we have the linear reasoning tactic from the last chapter, which makes the matter trivial. It automatically deals with the case-splits, and can solve the goal without any explicit reference to other theorems.

%Next, we will discuss the startegy development of our theory of rays. This differs from our earlier work in Isabelle since we are using quotienting procedures in HOL~Light. Moreover, the theory is cleaner and has much better coverage of the important ideas. Almost all of the theorems we verified were chosen from analogous theorems in the theory of half-planes which we cover in \S\ref{sec:HalfPlaneTheory}, and in whose completeness we are confident having used the theory extensively in verifying the Polygonal Jordan Curve Theorem (see Chapters~\ref{chapter:JordanVerification1} and~\ref{chapter:JordanVerification2}).

\subsubsection{Lifting to a Theory of Rays}
With the equivalence relation verified, it is a simple matter to define the quotient type of rays. With the command
\begin{displaymath}
  \code{define\_quotient\_type}\ \code{"ray"}\ (\code{"mk\_ray"},\code{"dest\_ray"})\ \code{equiv\_arrow}
\end{displaymath}
we introduce a new type \code{ray} into the theory, together with their abstraction and representation functions $\code{mk\_ray}$ and $\code{dest\_ray}$. 

The great thing about HOL~Light's quotienting facilities is that we do not need to deal with these particular abstraction and representation functions directly. When we lift theorems the functions are used automatically. 

However, HOL~Light will not automatically plumb theorems about the endpoints of arrows through our intermediate type and lift them to our type of rays. Consider the relation which says that a given point lies on a given ray. If rays were an equivalence class on the space of all points, this relation would be lifted directly from the partially applied equivalence relation. Here, we must instead build an arrow, using the abstraction function for arrows, namely $\code{mk\_arrow}\ :\ (\code{point},\code{point})\rightarrow\code{arrow}$.
\begin{gather*}
  \vdash_{def}\code{arrow\_origin}\ \overrightarrow{OP} = O.\\
  \begin{split}
    &\vdash_{def}\code{on\_ray\_of\_arrow}\ P\ \code{a}\\
    &\qquad\iff P \neq \code{arrow\_origin}\ a\\
    &\qquad\qquad\quad\wedge \code{equiv\_arrow}\ a\ (\code{mk\_arrow}\ (\code{arrow\_origin}\ a, P)).
  \end{split}
\end{gather*}

It is trivial to verify that this relation is well-defined, but to use it effectively in proofs relating points of an arrow to points on the ray of an arrow, we need to manually fold and unfold the definition of arrows. It is tedious enough to verify that
\begin{equation*}
  \vdash\code{on\_ray\_of\_arrow}\ P\ \overrightarrow{OQ} \iff P = Q\ \vee\ \between{O}{P}{Q}\ \vee\ \between{O}{Q}{P}.
\end{equation*}

We suspect that the creation of the intermediate \code{arrow} type and the lifting of theorems through this type could be automated. That is, we propose that the presentation of the theory of rays and half-planes could be improved by extending HOL~Light's quotienting facilities, and in particular, extending it to handle indexed families of equivalence relations, such as the equivalence relation of arrows lying in the same direction, indexed by a point of origin. This would make for suitable future work, and we believe that Hilbert's geometry offers a nice example of where such facilities would be useful.

\section{Theory of Half-Planes}\label{sec:HalfPlaneTheory}
The theory of rays is largely trivial when we have our linear reasoning tactic. Everything we want to know about linear order is bound up in THEOREM~6, from which that tactic was derived. Two-dimensional order is less straight forward. We get this impression from Hilbert himself, who justifies the definition of half-planes with a distinguished theorem (THEOREM~8). He gives no corresponding theorem for rays, but instead, just assumes his definition is sound.

As with the theory of rays, our theory is based on lifting from an intermediate type. We mediate the notion of half-plane by a line and a point not on that line, where a ray was mediated by a point and a distinct point. The half-plane intermediary lacks the pleasing geometric interpretation of \emph{arrows}, but the basic plumbing and proofs are similar.
\begin{equation}\label{eq:SameSidePlaneDef}
  \begin{split}
    \vdash_{def}\;& \code{equiv\_half\_plane}\ (P,a)\ (Q,b)\\
    & \qquad\iff\ a=b \\
    & \qquad\quad\qquad \wedge (\exists \alpha.\; \code{on\_plane}\ P\ \alpha \wedge \code{on\_plane}\ Q\ \alpha\\
    & \qquad\quad\qquad\qquad \wedge \forall S.\;\code{on\_line}\ S\ a \implies \code{on\_plane}\ S\ \alpha)\\
    & \qquad\quad\qquad \wedge \neg\code{on\_line}\ P\ a \wedge \neg\code{on\_line}\ Q\ a\\
    & \qquad\quad\qquad \wedge \neg(\exists R.\; \code{on\_line}\ R\ a \wedge \between{P}{R}{Q}).
  \end{split}
\end{equation}

The equivalence relation is unfortunately convoluted by constraints, since the main property saying when two points are on the same side of a line is a negative one and thus quite weak. The correctness may not be immediately obvious. If the reader needs more confidence than simple inspection provides, they can perhaps be assured by the fact that we have derived many of the expected theorems about half-planes. 

Many of the theorems are trivial, and are only provided to link the primitive types $\code{point}$, $\code{line}$ and $\code{plane}$ with our new type of $\code{half\_plane}$. The two non-trivial theorems we need to verify are, firstly, that the relation defined above is transitive, and secondly, that there are exactly two half-planes to each plane. As we shall see, both theorems together can be understood as a strengthening of Pasch's Axiom~\eqref{eq:g24}.

\subsection{Transitivity}
Consider the transitivity problem. Suppose that the points $A$ and $B$ are on the same side of the line $a$, and that $B$ and $C$ are also on the same side. We must show that $A$ and $C$ are then on the same side.

According to our definition \eqref{eq:SameSidePlaneDef}, this means we must show that if the line $a$ does not intersect between $A$ and $B$, and does not intersect between $B$ and $C$, then it cannot intersect between $A$ and $C$. Equivalently, if there is an intersection at $A$ and $C$, then there is an intersection either between $A$ and $B$ or between $B$ and $C$. This is already very close to Pasch's axiom.

Pasch's axiom \eqref{eq:g24} asserts that, given a triangle $ABC$, if a line $a$ lies in the plane of $ABC$ and crosses the side of a triangle, and does not intersect a vertex, then it must leave by one of the other two sides. 

We have most of these assumptions in place. We know that our points $A$, $B$ and $C$ are planar: that assumption was made part of the definition \eqref{eq:SameSidePlaneDef}. We know that the line $a$ does not meet any vertex, since this is a defining requirement of any representative of our intermediate type. The only assumption we have not met is that $A$, $B$ and $C$ form a triangle.

But actually, this assumption on Pasch's axiom is not necessary. The conclusion holds even if $A$, $B$ and $C$ lie on a line, though we have not been able to prove it up until now. The verification, which uses THEOREM~6 via our linear reasoning tactic, is given in Figure~\ref{fig:PaschColCase}. 

Now we had proven the transitivity relation for half-planes in our earlier work in Isabelle~\cite{ScottMScThesis}, but there, we derived two lemmas whose proofs are based on numerous case-splits. In our new formalisation, these messy details are handled automatically, and we see the usefulness of our linear reasoning tactic once again.

\begin{boxedfigure}
% let g24_col_case = theorem
%   "!A B C P a.
%          on_line P a /\ ~on_line C a /\ between A P B
%          /\ (?a. on_line A a /\ on_line B a /\ on_line C a)
%          ==> ?Q. on_line Q a /\ (between A Q C \/ between B Q C)"
%   [fix ["A:point";"B:point";"C:point";"P:point";"a:line"]
%   ;assume "on_line P a /\ ~on_line C a /\ between A P B" at [0]
%   ;assume "?a. on_line A a /\ on_line B a /\ on_line C a" at [1]
%   ;take ["P:point"]
%   ;thus "on_line P a" from [0]
%   ;have "~(C = P)" from [0]
%   ;hence "between A P C \/ between B P C" using ORDER_TAC `{A:point,B,C,P}` from [0;1]];;
  \begin{align*}
    & \code{assume}\ \code{on\_line}\ P\ a\wedge\neg\code{on\_line}\ C\ a \wedge \between{A}{P}{B} & 0\\
    & \code{assume}\ \exists a.\; \code{on\_line}\ A\ a\wedge\code{on\_line}\ B\ a\wedge\code{on\_line}\ C\ a&1\\
    & \code{take}\ P\\
    & \code{thus}\ \code{on\_line}\ P\ a\ \code{from}\ 0\\
    & \code{have}\ C \neq P\ \code{from}\ \code{from}\ 0\\
    & \code{hence}\ \between{A}{P}{C} \vee \between{B}{P}{C}\ \code{using ORDER\_TAC}\ \{A,B,C,P\}\ \code{from}\ 0,1
  \end{align*}
  \caption{Pasch's Axiom when $A$, $B$ and $C$ are collinear}
  \label{fig:PaschColCase}
\end{boxedfigure}

In the proof in Figure~\ref{fig:PaschColCase}, we have pared down the assumptions significantly. Now that we assume that the three points are collinear, there is no need to mention planes. Without the planes, the only remaining assumption is the one which says that the line $a$ does not meet any vertex. In verifying the theorem, we initially thought to throw out this assumption, believing it was as unnecessary as the planar assumption, but our linear reasoning tactic thought otherwise. It promptly told us that the resulting situation entails no contradiction. It will not give us a valid model, but with a moment's thought, we realise that if $C = P$, then the strictness of the $\code{between}$ relation means that the conclusion cannot possibly hold.

To fix this, we add back the assumption that $C$ is not on the line $a$. Notice that we then have to explicitly add a step showing that $C \neq P$, since the linear reasoning tactic will not infer this automatically: it only rewrites equalities, inequalities and betweenness claims, so we must feed it the necessary facts explicitly.

This pattern of using the linear reasoning tactic with very few assumptions and lemmas, and then adding more in until the goal is solved, was our typical use-case of the tactic. We benefit from the fact that the tactic is a decision procedure, and the problems we throw at it are normally sufficiently constrained that a yes/no answer is delivered promptly. As such, the tactic can be used to explore ideas as well as verify steps that are known in advance to be valid.

\subsection{Covering}
Our next theorem shows that there are at most two half-planes to each plane. This theorem is lifted from an analogous theorem on our intermediate type, but the basic details are again a strengthening of Pasch's axiom.

We need to prove that of three points $A$, $B$ and $C$ in a plane $\alpha$ containing a line $a$, it cannot be the case that $A$, $B$ and $C$ are on mutually distinct sides of $a$. In terms of our definition \eqref{eq:SameSidePlaneDef}, this amounts to showing that $a$ cannot simultaneously intersect between the pair of points $A$ and $B$, the points $A$ and $C$ and the points $B$ and $C$. 

\label{sec:PaschInclusiveOr}Thus, if $ABC$ is a triangle, we are being asked to refute the possibility that the line $a$ intersects all three sides. This fact would be immediate if the conclusion of Pasch's axiom was rendered with the exclusive-or. 

This might well have been the case in the first edition of the \emph{Grundlagen}. Hilbert uses an ``either...or'' for the axiom (and the analogous construction in the German edition). By the tenth edition, the ``either'' has disappeared, and now, Hilbert makes the explicit claim that the inclusive case can be refuted. It is clear, then, that he intends the weak inclusive-or in the axiom, and expects the inclusive case to be proved impossible.

Bernays thought the mere claim of a proof's existence was insufficient. In Supplement~I to the text, he gives the proof in full:

\begin{quotation}\label{sec:SupplementI}
It behooves one to deduce the proof by means of THEOREM~4. It can be carried out as follows: If the line $a$ met the segments $BC$, $CA$, $AB$ at the points $D$, $E$, $F$ then these points would be distinct. By THEOREM~4 one of these points would lie between the other two.

If, say, $D$ lay between $E$ and $F$, then an application of Axiom~II,~4 to the triangle $AEF$ and the line $BC$ would show that this line would have to pass through a point of the segment $AE$ or $AF$. In both cases a contradiction of Axiom~II,~3 or Axiom~I,~2 would result.
\flushright{\emph{Foundations of Geometry}~\cite{FoundationsOfGeometry} (page 200)}
\end{quotation}

\begin{figure}
  \centering\includegraphics{halfPlanes/SupplementI}
  \caption{Supplement~I}
  \label{fig:SupplementI}
\end{figure}

This is an indirect proof, effectively based on an impossible diagram. The key inference is in the second paragraph; that is, if $A$, $C$ and $E$ are collinear, then we can use Pasch's Axiom to conclude that $C$ must lie between $A$ and $E$, contradicting our assumptions. The diagram shows the situation in Figure~\ref{fig:SupplementI}, and shows how this step corresponds precisely to a use of the outer version of the Pasch axiom~\eqref{eq:OuterPasch}.

The verification is shown in Figure~\ref{fig:SupplementIProof}. Our incidence discoverer from Chapter~\ref{chapter:Automation} helps keep the proof steps almost one to one with the prose. We start by concluding as Bernays does that the points $D$, $E$ and $F$ are distinct and then apply THEOREM~4 to show that one of the points lies between the other two.

Bernays next makes a without-loss-of-generality assumption. We capture this with a subproof. There is some ugly repetition here with our $\code{assume}$ steps, but after this comes the two key inferences. Note that we use the outer form of the Pasch Axiom \eqref{eq:OuterPasch}, but, unlike Bernays, we leave out any mention of \eqref{eq:g12}. If we had to be consistently fussy in citing this axiom, it would have already appeared in the first step of the proof when showing that $D$, $E$ and $F$ are distinct. We leave it to implicit automation with our $\code{obviously}$ step.

Finally, we can strengthen this supplement by removing the assumption that $ABC$ forms a triangle. When $A$, $B$ and $C$ are collinear, we have a linear problem, and our incidence discoverer and linear reasoning tactic can do all the work in just four steps. With this case considered, we can give Bernays' supplement in a very general form:
\begin{equation}\label{eq:SupplementI}
  \begin{split}
    \vdash&\neg\code{on\_line}\ A\ a \wedge \neg\code{on\_line}\ B\ a \wedge \neg\code{on\_line}\ C\ a\\
    &\wedge \code{on\_line}\ D\ a \wedge \code{on\_line}\ E\ a \wedge \code{on\_line}\ F\ a\\
    &\implies \neg\between{A}{D}{B} \vee \neg\between{A}{E}{C} \vee \neg\between{B}{F}{C}.
  \end{split}
\end{equation}

We have now strengthened Pasch's axiom \eqref{eq:g24} in two ways: we have removed the assumption that $A$, $B$ and $C$ is a triangle, and we have removed the inclusive-or from the conclusion. Respectively, these two facts tell us that Hilbert's same-side relation for half-planes is transitive, and that there are at most two half-planes on any given plane.


\begin{boxedfigure}
  \small
  \begin{align*}
    &\code{assume}\ \Triangle{a}{A}{B}{C} & 0\\
    &\qquad\qquad \code{on\_line}\ D\ a \wedge \code{on\_line}\ E\ a \wedge \code{on\_line}\ F\ a & 1\\
    &\code{assume}\ \between{A}{D}{B} \wedge \between{A}{E}{C} \wedge \between{B}{F}{C} & 2\\
    &\code{obviously}\ \code{(by\_ncols} \circ \code{conjuncts)}\ \code{have}\ D\neq E \wedge D\neq F \wedge E\neq F\ \code{from}\ 0,2\\
    &\code{hence}\ \between{D}{E}{F}\vee\between{D}{F}{E}\vee\between{F}{D}{E}\ \code{from}\ 1\ \code{by} \ \eqref{eq:four}&3\\
    &\code{have}\ \forall A'\;\forall B'\;\forall C'\;\forall D'\;\forall E'\;\forall F'.\; \Triangle{a}{A'}{B'}{C'}\\
    &\qquad \between{A'}{F'}{B'} \wedge \between{A'}{E'}{C'} \wedge \between{B'}{D'}{C'}\\
    &\qquad \implies \neg\between{E'}{D'}{F'}\\ 
    &\code{proof:}\\
    &\qquad \code{fix}\ A',B',C',D',E',F'\\
    &\qquad \code{assume}\ \Triangle{a}{A'}{B'}{C'} & 4\\
    &\qquad \code{assume}\ \between{A'}{F'}{B'} \wedge \between{A'}{E'}{C'}\\ &\qquad\qquad\wedge\between{B'}{D'}{C'}\wedge\between{E'}{D'}{F'} & 5\\
    &\qquad\code{obviously}\ \code{(by\_ncols} \circ \code{conjuncts)}\ \code{consider}\ G\ \\ &\qquad\qquad\code{such that}\ \between{A'}{G}{E'} \wedge \between{B'}{D'}{G}\\ 
    &\qquad\qquad\code{by}\ \eqref{eq:OuterPasch},\eqref{eq:g21}\ \code{from}\ 4,5\\
    &\qquad\code{obviously}\ \code{(by\_eqs} \circ \code{conjuncts)}\  \code{qed}\ \code{from}\ 4,5\ \code{by}\ \eqref{eq:g23}\\
    &\code{qed from}\ 0,2,3\ \code{by}\ \eqref{eq:g21}
  \end{align*}
  \caption{Proof for Supplement~I}
  \label{fig:SupplementIProof}
\end{boxedfigure}
\section{THEOREM~8}
It is not enough to say that at most two half-planes cover a plane. We must also show that there are \emph{at least} two half-planes in each plane. To that end, suppose we have a plane $\alpha$ and a line $a$ in $\alpha$. We find a  point $A$ on $a$, and a planar point $B$ off the line $a$. Then, with Axiom~\ref{eq:g22}, we can extend the segment $BA$ through the line $a$ to find a point $C$ on the other side. The points $B$ and $C$ will then lie in distinct half-planes. 

Note that this proof requires that the point $B$ always exists, which is a consequence of Theorem~\ref{eq:PlaneThree} from Chapter~\ref{chapter:Axiomatics}. We noted at the time that this theorem was originally an axiom, but was later factored out. Hilbert does not explicitly say how the theorem is to be recovered, and as we suggested at the time, we do not believe the matter to be completely trivial.

Our final rendition of THEOREM~8 is a theorem lifted from our intermediate type. It relies on two lifted functions. The function $\code{line\_of\_half\_plane}$ returns the boundary of a given half-plane, while the function $\code{half\_plane\_contains}$ is the incidence relation for half-planes. We give some additional theorems to govern these concepts in Figure~\ref{fig:HalfPlanesAdditional}, most of which are referenced in our  account of the Polygonal Jordan Curve Theorem in Chapters~\ref{chapter:JordanVerification1} and~\ref{chapter:JordanVerification2}. Meanwhile, THEOREM~8 is verified as:
\begin{equation}\label{eq:HalfPlaneCover}
  \begin{split}
    \vdash&(\forall P.\;\code{on\_line}\ P\ a \implies \code{on\_plane}\ P\ \alpha)\\
    &\implies \exists hp\;\exists hq.\; hp \neq hq\\
    &\quad \wedge a = \code{line\_of\_half\_plane}\ hp \wedge a = \code{line\_of\_half\_plane}\ hq\\
    &\quad \wedge (\forall P.\;\code{on\_plane}\ P\ \alpha\\
    &\quad\quad \iff \code{on\_line}\ P\ a \vee \code{half\_plane\_contains}\ hp\ P \vee \code{on\_half\_plane}\ hq\ P).
  \end{split}
\end{equation}

Note that we have broken convention with the name for our incidence relation, using 
\begin{displaymath}
\code{half\_plane\_contains}\ :\ \code{half\_plane}\rightarrow\code{point}\rightarrow\code{bool}
\end{displaymath}
and not 
\begin{displaymath}
\code{on\_half\_plane}\ :\ \code{half\_plane}\rightarrow\code{point}\rightarrow\code{bool}.
\end{displaymath}
We can understand why this is necessary by considering exactly what we would have to say is well-defined to obtain this relation. It is a set of points which are incident with a given representative of half-planes from our intermediate type. Now as predicates, point-sets are functions of type $\code{point} \rightarrow \code{bool}$, and it is a function of this form which we must verify as well-defined. We obtain the predicate by partial application of an intermediate incidence relation $\code{half\_plane\_intermediate\_contains}$:
\begin{multline*}
\vdash\code{equiv\_intermediate\_half\_plane}\ x\ y \\
\implies \code{half\_plane\_intermediate\_contains}\ x =\\ \code{on\_half\_plane\_intermediate\_contains}\ y.
\end{multline*}

Consequently, the type $\code{point}$ must appear last in the type of our half-plane incidence relation.
\begin{figure}
\begin{equation*}
  \begin{split}
    \vdash&(\forall P.\;\code{on\_line}\ P\ a \implies \code{on\_plane}\ P\ \alpha)\\
    &\implies \exists hp\;\exists hq.\; hp \neq hq\\
    &\qquad \wedge a = \code{line\_of\_half\_plane}\ hp \wedge a = \code{line\_of\_half\_plane}\ hq\\
    &\qquad \wedge (\forall P.\;\code{on\_plane}\ P\ \alpha\\
    &\qquad\quad \iff \code{on\_line}\ P\ a \vee \code{half\_plane\_contains}\ hp\ P \vee \code{on\_half\_plane}\ hq\ P)
  \end{split}
\end{equation*}
  \begin{equation}\label{eq:halfPlaneOnPlane}
    \begin{split}
      \vdash&\code{half\_plane\_contains}\ hp\ P \wedge \code{on\_plane}\ P\ \alpha \\
      &\wedge (\forall R.\;\code{on\_line}\ R\ (\code{line\_of\_half\_plane}\ hp) \implies \code{on\_plane}\ R\ \alpha) \\
      &\implies \code{half\_plane\_contains}\ hp\ Q \implies \code{on\_plane}\ Q\ \alpha
    \end{split}
  \end{equation}
  
  \begin{equation}\label{eq:halfPlaneNotOnLine}
    \vdash\code{half\_plane\_contains}\ hp\ P \implies \neg\code{on\_line}\ P\ (\code{line\_of\_half\_plane}\ hp)
  \end{equation}
  
  \begin{equation}\label{eq:onHalfPlaneNotBet}
    \begin{split}
      \vdash&(\forall R.\; \code{half\_plane\_contains}\ hp\ R \implies \code{on\_plane}\ R\ \alpha) \wedge \code{on\_half\_plane}\ hp\ P\\
      &\implies (\code{half\_plane\_contains}\ hp\ Q\\
      &\qquad \iff \neg(\exists R. \code{on\_line}\ R\ (\code{line\_of\_half\_plane}\ hp) \wedge \between{P}{R}{Q})\\
      &\qquad\qquad\quad \wedge \code{on\_plane}\ Q\ \alpha \wedge \neg\code{on\_line}\ Q\ (\code{line\_of\_half\_plane hp}))
    \end{split}
  \end{equation}
  
  \begin{equation}\label{eq:betOnHalfPlane1}
    % !hp P Q R. on_line P (line_of_half_plane hp)
    % /\ on_half_plane hp Q
    % /\ (between P Q R \/ between P R Q) ==> on_half_plane hp R
    \begin{split}
      \vdash&\code{on\_line}\ P\ (\code{line\_of\_half\_plane}\ hp) \wedge \code{half\_plane\_contains}\ hp\ Q\\
      &\implies \between{P}{Q}{R} \vee \between{P}{R}{Q} \implies \code{half\_plane\_contains}\ hp
    \end{split}
  \end{equation}
  
  \begin{equation}\label{eq:betOnHalfPlane2}
    \begin{split}
      % "!P Q R hp. on_half_plane hp P /\ on_half_plane hp R /\ between P Q R 
      % ==> on_half_plane hp Q"
      \vdash&\code{half\_plane\_contains}\ hp\ P \wedge \code{on\_half\_plane}\ hp\ R\\
      &\implies \between{P}{Q}{R} \implies \code{half\_plane\_contains}\ hp\ Q
    \end{split}
  \end{equation}
\caption{Theorems for half-Planes}
\label{fig:HalfPlanesAdditional}
\end{figure}
\section{Conclusion}
All preliminaries needed for the Polygonal Jordan Curve Theorem have now been dealt with. This chapter completes the necessary theory, showing how to use the quotienting facilities of HOL~Light to formalise the all important notion of half-planes. As we have seen, the fact that these objects exist and cover their planes can be understood as a strengthening of Pasch's Axiom.

We just note that one direction in which this axiom is weakened  was singled out by Bernays as needing an explicit proof. We feel the same way about Theorem~\ref{eq:PlaneThree}, which is also needed to prove THEOREM~8. Both supplementary proofs were unnecessary in the first edition, when the results were asserted axiomatically. But now that they have been factored out, one should be careful to derive them.

This omission is barely a glimmer compared to the next. Hilbert gives no indication how to prove THEOREM~9. A discussion of this theorem and its verification take up the next four chapters.
%%% Local Variables: 
%%% mode: latex
%%% TeX-master: "../thesis"
%%% End: 




CC\chapter{The Jordan Curve Theorem for Polygons}\label{chapter:JordanInformal}
In this chapter, we shall discuss the next theorem we encounter in Hilbert's \emph{Grundlagen der Geometrie}, which appears as THEOREM~9 in the 10th edition. This theorem is a special case of the Jordan Curve Theorem, which applies only to simple polygons. Our discussion will draw together a number of historical threads and characters connected to the \emph{Grundlagen der Geometrie}, and point out interesting subtleties and obstacles that come with trying to rigorously prove the theorem from Hilbert's very weak axioms. We shall consider several informal proofs, including Veblen's 1903 proof, which we believe to contain a major fault. The verification of the theorem is left to Chapters~\ref{chapter:JordanVerification1} and \ref{chapter:JordanVerification2}.

In sections~\ref{sec:JordanCurveHistory} and~\ref{sec:JordanCurveGenerality}, we give a short historical introduction to the polygonal case of Jordan Curve Thorem in the context of the more general theorem. In \S\ref{sec:JordanCurveExplanation}, we give Hilbert's formulation of the theorem and, because Hilbert did not supply a proof, we provide a detailed one of our own in \S\ref{sec:JordanCurveFirstProof}. Finally, in \S\ref{sec:VeblenProof}, we present another proof given by Veblen~\cite{Veblenphd}. This latter proof is probably incorrect, but it \emph{is} based on axioms very close to Hilbert's own, and contains useful ideas. Our final verification in Chapters~\ref{chapter:JordanVerification1} and~\ref{chapter:JordanVerification2} is based on our diagnosis of the problems with Veblen's proof.

\section{Relationship with the Full Jordan Curve Theorem}\label{sec:JordanCurveHistory}
The full Jordan Curve Theorem theorem effectively says that, when it comes to closed curves that do not self-intersect (\emph{simple closed curves}), we are justified in our use of the expressions ``inside the curve'' and ``outside the curve''. The idea that mathematicians should even bother justifying this appears relatively late in the history of mathematics. In fact, it had to wait until the 19th century and the rigorous reformulations of analysis which reduced the unclear notions governing the continuum to precisely defined formulas involving the now standard $\epsilon$ and $\delta$ inequalities. Bolzano, who is credited along with Cauchy for spotting the reformulation, provided his own rigorous definitions for closed, continuous curves and what it means for curves to enclose points, so that he was then able to recommend the following for rigorous proof~\cite{BolzanoJordan}:

\begin{quote}
If a closed line lies in a plane and if by means of a connected line one joins a point of the plane which is enclosed within the closed line with a point of the plane which is not enclosed within it, then the connected line must cut the closed line.
\end{quote}

This is a significant half of the Jordan Curve Theorem, and, reading the terms ``closed lines'' intuitively, it seems so blindingly obvious that one might think it perverse to demand a proof. However, the rigorous definitions which Bolzano had in mind to replace ``closed line'' are not so immediately intuitive, but appeal to very general and abstract topological properties. The first proof that these abstract properties preserve intuitive properties such as Bolzano's conjecture was given in Jordan's 1887 \emph{Cours d'analyse}~\cite{JordanTextBook}. To this day, the details are regarded as quite involved. One possible reason for the complexity is that the rigorous formulations are so general that they admit weird pathologies, or as Poincar\'{e} colourfully called them, \emph{monsters}, and some of these monsters, such as closed curves enclosing finite area but having infinite length, immediately thwart a number of obvious proof strategies.

Relevant to this chapter is the case against Jordan's proof: common folklore says that it is invalid, and that the first correct proof was provided by Veblen in 1905~\cite{VeblenJordan}. Both Veblen and folklore point out that Jordan had to assume the polygonal case, and argue that he should have proven this as a lemma. Hales, on the other hand, has formally verified the theorem in HOL~Light, and put together a strong defence of Jordan~\cite{HalesJordansProof}, and of a basically elegant and correct proof that has been unfairly neglected. The polygonal case is supposedly completely trivial. 

Not so, according to Feferman. In his paper concerning the aforementioned \emph{monsters}~\cite{FefermanDevilishJCT}, he repeats the folklore that Veblen gave the first correct proof, and claims that even the polygonal case is ``devilishly difficult to prove.''

We might suggest that this controversy begins with Hilbert's 1899 edition of the \emph{Grundlagen der Geometrie}. There, Hilbert gave a formulation of the polygonal case as THEOREM~6, but, as with the five preceding theorems, he did not give a proof. Instead, he assures us that with the aid of his theorem for the existence of half-planes (THEOREM~5 of that edition), one can obtain the proof ``without much difficulty.'' This certainly backs up the idea that the theorem is trivial. However, by the Ninth Edition, the clause had been deleted, with the edit noted as a ``correction.'' The theorem still appears without proof, though Bernays, in a supplement to the main text, cites a detailed proof by Fiegl~\cite{FeiglJordan}. Note that Bernays does not \emph{include} the proof, as he does in other cases, such as proving that Pasch's axiom can be rendered without the inclusive-or~(see \S\ref{sec:PaschInclusiveOr}). That would have taken more than a few supplementary remarks.

Now Veblen himself, one year before publishing his proof of the Jordan Curve Theorem, had developed an axiomatic foundation for geometry which was very close to Hilbert's own~\cite{Veblenphd}, and in this thesis, he expends a great deal more effort developing a theory of order than did Hilbert. This explains why Veblen's doctoral supervisor, E.~H.~Moore, was contributing proofs to later editions of Hilbert's text~(see \S\ref{sec:Theorem5}). It also explains why, in Veblen's 1905 proof of the full Jordan Curve Theorem, he thought it necessary to cite a proof from his doctoral thesis showing that the Jordan Curve Theorem holds for the even more trivial case of \emph{triangles}. It seems plausible to us, therefore, that Veblen's criticism of Jordan is explainable by his particular standards of rigour and the context of an axiomatic theory of geometry. When it came to his theory of order, his standards of rigour were even higher than Hilbert's.

Another aspect we should consider is the level of generality that Veblen was attempting. He gave a proof of the full theorem in the context of ordered geometry with the addition of one topological axiom. As such, the proof was not supposed to require any assumptions about the existence of a metric. Unfortunately, as Hales points out in his defence of Jordan~\cite{HalesJordansProof}, the generality was refuted ten years later. R. L. Moore\footnote{Not to be confused with Veblen's supervisor E.~H.~Moore.}, who had been a student of Veblen's. R.~L.~Moore showed that, according to Veblen's axioms, all of his planes are homeomorphic to the Euclidean plane~\cite{MooreSitus}, and thus his axioms always describe a metrisable space.

But what about the polygonal case? In his doctoral thesis, Veblen gave a detailed, standalone proof of this theorem without using the topological axiom. The theorem, then, is plausibly still very general. So one question is: given its generality, is it still \emph{trivial}? We suggest not. As we discuss in \S\ref{sec:VeblenProof}, a correct proof seems to have eluded Veblen himself.

\section{Generality of the Polygonal Case}\label{sec:JordanCurveGenerality}
We can discuss the problems raised by our level of generality by considering the two proofs of the polygonal case from the book \emph{What Is Mathematics?}~\cite{WhatIsMathematics}. Hales mentioned these to highlight the triviality of the polygonal case. The first of these proofs is the so-called ``plumb-line'' proof. We begin with a simple polygon, pick an arbitrary direction which is not parallel to any of its sides, and then for any point, we cast a ray in that specified direction. By considering the number of times the ray crosses the polygon, and how this number changes as we move around the plane, we can prove the polygonal case.

This is probably a non-starter. At Group~II, we have no theorems which say that parallel lines even \emph{exist}, nor do we have a theorem which says we can cast rays parallel to given lines. The needed theorems are actually developed in Group~III and Group~IV, where angle congruence is introduced as a primitive, and where we can start manipulating \emph{directions}. We also have no formulations of what it means to ``move'' along a ray. The sort of motion being considered is presumably \emph{continuous} motion, and indeed, if we look at Tverberg's proof of the theorem~\cite{TverbergJordan}, he essentially gives the same argument in rigorous form, and appeals directly to continuity. But  continuity does not appear in Hilbert's text until Group~V. 

The second proof from \emph{What Is Mathematics?} only states an approach, and does not provide a complete proof. It effectively says we can prove the Jordan Curve Theorem for polygons by computing winding numbers for the polygon. A naive formulation of this argument in terms of angles and continuous motion will, for the same reasons as above, be well outside the scope of Hilbert's first two groups of axioms. 

To reiterate, the problem we have with traditional proofs of the Polygonal Jordan Curve Theorem is that our axioms are too weak to formulate them. Another way to put this is to say that the version of the theorem we are attempting to prove is more general than the traditional versions, and must apply to \emph{any} ordered geometry. A visual way to bring this point home is to note that by ``simple polygons'', we are actually including the boundaries of all possible \emph{mazes} which do not contain loops. But worse, we are allowing the corridors of these mazes to be infinitesimally narrow, since we do not rule out non-Archimedean geometries at this stage.

Furthermore, we are tasked with navigating these mazes without being able to measure any distances. We cannot orient ourselves, rotate or compare directions. We cannot consider continuous motion. And we know nothing about the existence of parallel lines. In other words, we are navigating without a ruler, without a compass, and without being able to run a path parallel to a wall. We suspect these constraints will eliminate most trivial proofs.

\begin{figure}
\centering
\includegraphics[scale=0.5]{jordan/maze.pdf}
\caption{A Simple Polygon}
\end{figure}

\section{Polygonal Case: Formulation}\label{sec:JordanCurveExplanation}
The full Jordan Curve theorem applies to arbitrary simple closed curves, and characterises the interior and exterior in terms of path-connectedness. The polygonal version of the Jordan Curve Theorem replaces ``simple closed curve'' with ``simple polygon'', and characterises the two regions in terms of \emph{polygonal}-path connectedness. That is, the interior and exterior of the polygon are the largest sets all of whose points can be joined by polygonal paths. 

The three primitives at Hilbert's disposal, namely two incidence relations and a betweenness relation, are sufficient to formulate the notion of polygons, interiors and exteriors, and thus the theorem. Having already defined a segment as an unordered pair of points, Hilbert can now define a \emph{polygonal segment} as follows\footnote{Veblen calls these \emph{broken lines}.}:

\begin{quote}
DEFINITION. A set of segments $AB$, $BC$, $CD$, $\ldots$, $KL$ is called a \emph{polygonal segment} that connects the points $A$ and $L$. Such a polygonal segment will also be briefly denoted by $ABCD\ldots KL$. The points inside the segments $AB$, $BC$, $CD$, $\ldots$, $KL$ as well as the points $A$, $B$, $C$, $D$, $\ldots$, $K$, $L$ are collectively called the \emph{points of the polygonal segment}. 
\end{quote}

Polygons are then polygonal segments where the points $A$ and $L$ coincide. We then refer to each of the individual segments $AB$, $BC$, $\ldots$, $KL$ as a ``side'' of the polygon. Finally, Hilbert defines \emph{simple polygons}:

\begin{quote}
If the vertices of a polygon are all distinct, none of them falls on a side and no two of its nonadjacent sides have a point in common, the polygon is called \emph{simple}.
\end{quote}

Like the rays and half-plane theorems, the polygonal Jordan Curve Theorem describes a partitioning of a space into two ``connected'' regions, where connectedness is cashed out in terms of a suitable relation. Here, we are told that the polygon partitions all other points in the plane as follows:

\begin{quote}
  THEOREM 9. Every single polygon lying in a plane $\alpha$ separates the points of the plane $\alpha$ that are not on the polygonal segment of the polygon into two regions, the \emph{interior} and the \emph{exterior}, with the following property: If $A$ is a point of the interior {\bfseries (an interior point)} and $B$ is a point of the exterior {\bfseries (an exterior point)} then every polygonal segment that lies in $\alpha$ and joins $A$ with $B$ has at least one point in common with the polygon. On the other hand if $A$, $A'$ are two points of the interior and $B$, $B'$ are two points of the exterior then there exist polygonal segments in $\alpha$ which join $A$ with $A'$ and others which join $B$ with $B'$, none of which have any point in common with the polygon. By suitable labeling of the two regions there exist lines in $\alpha$ that always lie entirely in the exterior of the polygon. However, there are no lines that lie entirely in the interior of the polygon.

  \centering\includegraphics[scale=0.8]{jordan/jordanHilbert.pdf}
\end{quote}

In the remainder of this chapter, we shall identify a polygon with its list of vertices $P_1P_2P_3\cdots P_n$.

First, we shall briefly summarise the last few sections: as with Hilbert's \emph{Grundlagen der Geometrie}, the formulation and proof of the Jordan Curve Theorem marks a significant development in the history of rigorous, axiomatic mathematics and geometry. Its position in Hilbert's text means that it should be possible to formulate and prove it in very basic terms and from very simple axioms, based on ideas about half-planes. It is somewhat controversial whether this is trivial, but we hope that the next few chapters will go some way towards clarifying the matter.

\section{Point-in-Polygon Proof}\label{sec:JordanCurveFirstProof}
One feature of Veblen's proof, and the proof we verify in Chapter~\ref{chapter:JordanVerification1}, is that the treatment of the two regions defined by a simple polygon is symmetric. This means, however, that unlike the plumb-line and winding number proofs from \emph{What is Mathematics?}, Veblen's proof does not hint at any sort of point-in-polygon \emph{test}, one that could be implemented on a machine.

Before we encountered Veblen's proof, we had developed one of our own which does allow for such a test, which reduces the problem of deciding whether a point is inside a polygon to the problem of deciding whether a point is on a given side of each of the polygon's diagonals. The proof is inductive over the vertices of a simple polygon, and goes some way towards thinking of the Jordan Curve Theorem for polygons as a triangulation problem.

We decided to try an inductive proof on the total number of vertices of a simple polygon. To do this, it is necessary to show that every simple polygon can be broken down into smaller, simple polygons, until one reaches the smallest possible polygon (the triangle). We have verified the Polygonal Jordan Curve Theorem in this base case as part of our main verification in Chapters~\ref{chapter:JordanVerification1} and~\ref{chapter:JordanVerification2}.

The idea that we might be triangulating polygons in this proof should give the careful reader pause, since most proofs that a polygon can be triangulated assume the Jordan Curve Theorem for polygons~\cite{PolygonsHaveEars}. We are thus in danger of creating a circular argument. We avoid this danger because firstly, at each stage, we do not require that the smaller polygons are subsets of the original (thus, we do not actually identify a triangulation). And secondly, we are always able to use the Polygonal Jordan Curve Theorem as an inductive hypothesis at each stage.

To split a given simple $n$-gon where $n>3$ into two simple polygons, we just pick a line connecting two non-adjacent vertices, a \emph{diagonal}, which does not intersect the polygon. The simplest example of interest is the concave quadrilateral shown in Figure~\ref{fig:quadConcave}. This quadrilateral has exactly two diagonals. The diagonal $P_2P_4$ lies in the interior of the polygon, while the diagonal $P_1P_3$ lies in the exterior.

\begin{figure}
  \centering
  \includegraphics[scale=0.8]{jordan/quadConcave.pdf}
  \caption{Concave Quadrilateral}
  \label{fig:quadConcave}
\end{figure}

If we take the diagonal $P_2P_4$, we see that the interior of the quadrilateral consists of the union of the interiors of two triangles, namely $P_1P_2P_4$ and $P_2P_3P_4$, together with the diagonal $P_2P_4$ itself (see Figure~\ref{fig:quadUnion}). If we take the diagonal $P_2P_4$, then we see the interior of the quadrilateral consists of the interior of the triangle $P_1P_3P_4$, minus the interior of $P_1P_2P_3$ and the boundary of $P_1P_2P_3$. Thus, we can define the interior of the quadrilateral in terms of the interiors of two triangles. We shall extend this to the general case for an $n$-gon with $n>3$.

\begin{figure}
  \centering 
  \subfigure[Union of Two Triangles]{\includegraphics[scale=0.8]{jordan/quadUnion1.pdf}
    \includegraphics[scale=0.8]{jordan/quadUnion2.pdf}
    \includegraphics[scale=0.8]{jordan/quadUnion3.pdf}
    \label{fig:quadUnion}}

  \subfigure[Difference of Two Triangles]{\includegraphics[scale=0.8]{jordan/quadDiff1.pdf}
    \includegraphics[scale=0.8]{jordan/quadDiff2.pdf}
    \includegraphics[scale=0.8]{jordan/quadDiff3.pdf}
    \label{fig:quadDiff}}
  \caption{Decomposing a Concave Quadrilateral}
  \label{fig:quadDecompose}
\end{figure}

The rest of this section contains all the core ideas of the proof. The proof has not been verified, but we consider it an interesting contribution in its own right. It is distinguished from the proof we \emph{have} verified since it yields a very simple point-in-polygon test by recursively subdividing a polygon and ultimately reducing the problem to point-in-triangle tests.

The following subsections are quite dense, and details of the proof may not be convincing on a first read. We suggest, however, that with our experience verifying the Polygonal Jordan Curve Theorem and with our hindsight, we are very confident that the proof is correct and can be verified by reusing much of the verification discussed in chapters~\ref{chapter:JordanVerification1} and~\ref{chapter:JordanVerification2}. We would like to verify this proof as further work. 

\subsection{Finding a Diagonal}\label{sec:FindingDiagonal}
A simple polygon $P_1P_2 \ldots P_n$ where $n>3$ has at least one diagonal which does not intersect the polygon. To see this, we assume without loss of generality that $P_1P_2P_3$ are non-collinear, and that $P_n$ is not inside the triangle $P_1P_2P_3$. Consider the line $P_1P_3$. If this line does not intersect the polygon, then it is a suitable diagonal. Otherwise, take the vertex $P_m$ where $3 < m < n$ such that the line $P_1P_m$ intersects $P_2P_3$ in a point $X$ and such that, for any other $P_{m'}$ where $3 < m' < n$, if $P_1P_{m'}$ intersects $P_1P_2$ at $X'$, then $X$ is between $P_1$ and $X'$. We then have that $P_1P_m$ is the required diagonal (see Figure~\ref{fig:SqueezeDemo}), which yields two smaller simple polygons, shown in green and blue.

A \emph{very} slightly modified version of this argument is a core component in the proof we actually verified. We shall explain it in more detail in \S\ref{sec:Squeeze}.

\begin{figure}
\centering
\includegraphics{jordan/diagonal.pdf}
\caption{Finding a Diagonal}
\label{fig:SqueezeDemo}
\end{figure}

\subsection{Cases}
As noted, a diagonal $D$ of a polygon $p$ that does not intersect $p$ divides it into two smaller polygons $p_1$ and $p_2$. Generalising the situation from Figure~\ref{fig:quadDecompose}, we have that if $D$ is interior to $p$, then the interior of $D$ can be defined as the union of the interiors of $p_1$ and $p_2$, together with $D$. On the other hand, if $D$ is exterior to $p$, and the interior of $p_2$ is a subset of the interior of $p_1$, then we can define, without loss-of-generality, the interior of $p$ to be the interior of $p_1$ minus the boundary and interior of $p_2$. It therefore follows that the interior of a polygon is formed by recursively adding and subtracting out the interiors of smaller polygons. Finally, at each step, we define the exterior of the polygon as the plane minus the interior and minus the polygon's boundary.

There is obviously circularity here, since whether or not $D$ is an interior or exterior diagonal should be a \emph{corollary} of our argument. To remedy this, we shall need to characterise the two cases some other way.

In Figure~\ref{fig:DiagonalCases}, we illustrate the two cases by showing a fragment of a polygon and its diagonal. Here, we assume that the polygon $p$ is of the form $P$, $P_1$, $P_2$, $P_3$, $P_4$, $P_5$, $Q$, $Q_1$, $Q_2$, $Q_3$, $Q_4$, $Q_5$ and has a diagonal $PQ$ which does not intersect $p$. Our recursive step involves splitting the polygon into $p_1 = P, P_1, P_2, P_3, P_4, P_5, Q$ and $p_2 = Q, Q_1, Q_2, Q_3, Q_4, Q_5, P.$

We now take a point $X$ on the diagonal $PQ$, and we cast rays out to the points $Y$ and $Z$ on either side of $PQ$, such that $X$ is the only point of intersection of the segment $YZ$ and the polygons $p_1$ and $p_2$ (ray-casting will be discussed in \S\ref{sec:RayCasting}). 

We now make the following inductive hypotheses:
\begin{description}\label{sec:FirstProofInductiveHypotheses}
\item[IH1] Interior points $A$ and $B$ of $p_1$ can be connected by a polygonal path which does not intersect $p_1$, and similarly for $p_2$.
\item[IH2] Exterior points $A$ and $B$ of $p_1$ can be connected by a polygonal path which does not intersect $p_1$, and similarly for $p_2$.
\item[IH3] Any path connecting an interior point $A$ of $p_1$ to an exterior point $B$ of $p_1$, must intersect $p_1$, and similarly for $p_2$.
\item[IH4] If $Y'$ and $Z'$ are endpoints of a segment which crosses exactly one side of $p_1$, then they lie in different regions with respect to $p_1$ (and similarly for $p_2$).
\item[IH5] If another segment $Y'Z'$ does not intersect any side of $p_1$, nor any side of $p_2$, then $Y'$ and $Z'$ lie in the same region.
\end{description}

Assuming that interiors and exteriors are non-empty and cover the plane, the first three hypotheses are equivalent to the polygonal Jordan Curve Theorem. They give us the following two cases:

\begin{enumerate}
\item Both $Y$ and $Z$ lie in the interior of exactly one of the polygons $p_1$ and $p_2$ (see Figure~\ref{fig:UnionCase}).
\item One of $Y$ and $Z$ lies in the exterior of both $p_1$ and $p_2$ (see Figure~\ref{fig:SubCase}).
\end{enumerate}

The first case corresponds to $PQ$ being an interior diagonal. The second case corresponds to $PQ$ being an exterior diagonal.

\begin{figure}
\centering
\subfigure[Union for Interior Diagonal]{\includegraphics[scale=0.9]{jordan/unionCase.pdf}
  \label{fig:UnionCase}}
\subfigure[Subtraction for Exterior Diagonal]{\includegraphics[scale=0.9]{jordan/diffCase.pdf}
  \label{fig:SubCase}}
\caption{Diagonal Cases}
\label{fig:DiagonalCases}
\end{figure}

\begin{figure}
\centering
\includegraphics{jordan/navigation1.pdf}
\caption{Navigating around the Exterior}
\label{fig:Navigation1}
\end{figure}

\begin{figure}
\centering
\subfigure[Union Case]{\includegraphics[scale=0.9]{jordan/shorten1.pdf}\label{fig:Shorten1}}
\subfigure[Subtraction Case]{\includegraphics[scale=0.9]{jordan/shorten2.pdf}\label{fig:Shorten2}}
\caption{Paths for IH3}
\end{figure}

\subsection{Union Case}
For the case depicted in Figure~\ref{fig:UnionCase}, we will assume, without loss of generality, that $Y$ lies in the interior of $p_1$ and $Z$ lies in the interior of $p_2$. We now define the interior of $p$ as the union of the interiors of $p_1$ and $p_2$ and the interior of the diagonal $PQ$. We define the exterior as the set of points not on the polygon and not in the interior. We just need to reprove IH1--IH5 as IH$'$--IH5$'$ for the polygon $p$.

\begin{description}
\item[IH{1}$'$] Interior points $A$ and $B$ of $p$ can be connected by a polygonal path which does not intersect~$p$.
  \begin{proof}
    Suppose $A$ and $B$ both lie inside $p_1$, or both lie inside $p_2$. We apply IH1 to $p_1$ and $p_2$, and thereby connect $A$ and $B$ with a polygonal segment lying in the interior of $p$. On the other hand, if we suppose (without-loss-of-generality) that $A$ is inside $p_1$ and $B$ is inside $p_2$, then we apply IH1 to $p_1$ in order to connect $A$ to $Y$, and we apply IH1 to $p_2$ in order to connect $B$ to $Z$. The required path can then be completed with the segment $YZ$.
  \end{proof}
\item[IH{2}$'$] Exterior points $A$ and $B$ of $p$ can be connected by a polygonal path which does not intersect $p$.
  \begin{proof}
    We must have that $A$ and $B$ lie in the exteriors of both $p_1$ and $p_2$. By IH2 applied to $p_1$, there is a path through the exterior of $p_1$ which connects $A$ and $B$. If this path lies entirely in the exterior of $p_2$ then it also lies entirely in the exterior of $p$ and we are done. 

    Otherwise, we take the first point $A_1$ and the last point $A_2$ at which the path crosses the boundary of $p_2$. We then find a new path which follows the edges of the polygon until it connects these two points. See Figure~\ref{fig:Navigation1} for an illustration and see \S\ref{sec:Jordan1NavigationDiscussion} for some discussion on this part of the proof.
  \end{proof}
\item[IH{3}$'$] Any path connecting an interior point $A$ of $p$ to an exterior point $B$ of $p$, must intersect $p$.
  \begin{proof}
    Assume, without loss-of-generality, that $A$ lies in the interior of $p_1$ and $B$ lies in the exterior of both $p_1$ and $p_2$, and consider any polygonal path connecting $A$ and $B$. Since $B$ lies in the exterior of $p_1$, the path must intersect $p_1$ by IH3. If it intersects any side of the polygon $p_1$ other than $PQ$, we are done.

Otherwise, suppose the path intersects $PQ$ as in Figure~\ref{fig:Shorten1}, and take the last such point of intersection $R$ as we move towards $B$. We then consider a point $A'$ just beyond $R$ such that $A'Z$ does not intersect $p_2$ and is therefore in the interior of $p_2$ by IH5 (see \S\ref{sec:Jordan1SameSideDiscussion} for some discussion). The subpath from $A'$ to $B$ now connects a point interior to $p_2$ to a point exterior to $p_2$ which does not cross $PQ$. Hence, by IH3, the subpath must intersect $p_2 - PQ$ at a point $S$. This point $S$ must be a point on~$p$.
  \end{proof}
\item[IH{4}$'$] If $Y'$ and $Z'$ are endpoints of a segment which crosses exactly one side of $p$, then one point is interior while the other is exterior with respect to $p$.
  \begin{proof}
    Suppose $Y'Z'$ crosses a side of $p$. Without loss-of-generality, we can assume that this is a side of $p_1$, and then apply IH4. This tells us that $Y'$ and $Z'$ lie in opposite regions with respect to $p_1$. One of the points, say $Y'$ lies inside $p_1$ and thus inside $p$. Aiming for a contradiction, we suppose that $Z'$ also lies inside $p$. Since it does not lie inside $p_1$, nor does it lie on $PQ$, it follows that it must lie inside $p_2$. At the same time, $Y'$, being inside $p_1$, must lie outside $p_2$.

Now since $Y'$ and $Z'$ lie in opposite regions with respect to $p_2$, it must be the case that $Y'Z'$ intersects $p_2$. For if there is no such intersection, we would obtain a contradiction by IH5 applied to $p_2$. 

Thus, we now have \emph{two} points at which $Y'Z'$ intersect $p$, and this contradicts our hypothesis.
  \end{proof}
\item[IH{5}$'$] If a segment $Y'Z'$ does not intersect $p$, then $Y'$ and $Z'$ lie in the same region.
  \begin{proof}
    If the segment does not intersect any side of $p$ and does not intersect any sides of $p_1$ or $p_2$, then the case reduces to IH5 applied to $p_1$ and $p_2$. If it does intersect $p_1$ or $p_2$, then the side intersected must be the diagonal $PQ$. In this case, we can find a path which does not intersect $p_1$ and which connects $Y$ to $Y'$, or a path which does not intersect $p_2$ and which connects $Y$ to $Z'$. Thus, $Y'$ is interior to one of $p_1$ and $p_2$ by IH5 and thus interior to $p$. The same argument applies to $Z'$. See \S\ref{sec:Jordan1SameSideDiscussion} for some further discussion. % Note that the above disjunction is a case-split on which side of $PQ$ the points $Y'$ and $Z'$ are on. I am confident that the step corresponds exactly to the use of same_side_wall_connected in the verified proof.
  \end{proof}
\end{description}

\subsection{Subtraction Case}
The second case differs from the first in that we lose the symmetry between $p_1$ and $p_2$. One of the polygon's interior must be contained entirely by the others. 

We can determine which by considering whether $P_1$ lies inside $p_2$ or whether $Q_1$ lies inside $p_1$. To see why these are alternatives, suppose that $P_1$ does not lie inside $p_2$ (as depicted in Figure~\ref{fig:SubCase}). Then it follows from IH5 and the fact that $p$ is simple that $p_1 - PQ$ lies outside $p_2$. 

We next consider a path between $Z$ and $Q_1$ whose interior does not intersect $p_2$ (see \S\ref{sec:Jordan1SameSideDiscussion} for discussion). Since the interior of this path lies inside $p_2$, while the $p_1 - PQ$ lies outside $p_2$, it follows that the path does not intersect $p_1$. Hence, by IH5, $Q_1$ lies with the point $Z$, inside $p_1$.

We can now assume, without loss of generality, that $Q_1$ is interior to $p_1$ as depicted in Figure~\ref{fig:SubCase}. We then define the interior of $p$ to be the interior of $p_1$ minus the interior of $p_2$ and its boundary. 

Note that by IH5 and the fact that $p$ is simple, we have that all points on $p_2 - PQ$ must lie inside $p_1$, and that all points on $p_1 - PQ$ must lie outside $p_2$. We now prove IH1$'$--IH5$'$ for the step case.

Furthermore, any point interior to $p_2$ is a point interior to $p_1$. To see this, note that any interior point of $p_2$ can be connected to $Z$ by a polygonal path which does not intersect $p_2$ (see \S\ref{sec:Jordan1SameSideDiscussion} for some discussion). Thus, all points of this path must be interior to $p_2$, and so cannot intersect $p_1$. This means by IH5 that the point must be interior to $p_1$ also.

\begin{description}
\item[IH1$'$] Interior points $A$ and $B$ of $p$ can be connected by a polygonal path which does not intersect~$p$.
  \begin{proof}
    An interior point of $p$ is an interior point of $p_1$ which is not interior to $p_2$. If we apply IH1 to $p_1$, we can find a path between $A$ and $B$ through the interior of $p_1$. It is possible that this path intersects $p_2$, and so we must appeal to the same argument used for IH2$'$ in the union case and depicted in Figure~\ref{fig:Navigation1}, in order to navigate through the exterior of $p_2$.
  \end{proof}
\item[IH2$'$] Exterior points $A$ and $B$ of $p$ can be connected by a polygonal path which does not intersect $p$.
  \begin{proof}
    This argument is similar to that given for IH1$'$ in the union case. If both $A$ and $B$ lie in the exterior of $p_1$, then we know from IH1 applied to $p_1$ that they can be connected by a path exterior to $p_1$. And since $p_2 - PQ$ is interior to $p_1$, we know that this path does not intersect $p$.

    Similarly, if $A$ and $B$ lie in the interior of $p_2$, we know from IH1 that they can be connected by a path interior to $p_2$. And since $p_1$ is exterior to $p_2$, we know that this path does not intersect $p$.

    Finally, if we assume without-loss-of-generality that $A$ is exterior to $p_1$ and $B$ interior to $p_2$, then we can join $A$ to $Y$ by a path exterior to $p_1$ and thus one which does not intersect $p_2$, and join $B$ to $Z$ by a path interior to $p_2$ and thus one which does not intersect $p_1$. The segment $YZ$ then completes a path connecting $A$ to $B$ which does not intersect $p$.
    \end{proof}

\item[IH3$'$] Any path connecting an interior point $A$ of $p$ to an exterior point $B$ of $p$, must intersect $p$.
  \begin{proof}
    The proof is similar to that for the previous section. By definition, $A$ lies inside $p_1$ and outside $p_2$. The point $B$, on the other hand, might lie outside $p_1$, inside $p_2$, or it might lie on the diagonal $PQ$. If the path between them does not intersect the diagonal $PQ$, then we know from IH3 that it must intersect some side of $p$ and we are done.

So let us suppose it intersects the diagonal $PQ$. We now take the first point $R$ on the path from $A$ at which the the diagonal is intersected. Then there will be an earlier point $B'$ on the path which we can connect to one of $Y$ or $Z$ by a path which does not intersect $PQ$ (see \S\ref{sec:Jordan1SameSideDiscussion} for discussion). Thus, by IH5, $B'$ is either inside $p_2$ or outside $p_1$, and the subpath $AB'$ does not intersect $PQ$. It follows then, applying IH3, that we can find a point $S$ where the subpath intersects $p$. See Figure~\ref{fig:Shorten2} for an illustration.
  \end{proof}

\item[IH4$'$] If $Y'$ and $Z'$ are endpoints of a segment which crosses exactly one side of $p$, then they lie in different regions with respect to $p$.
  \begin{proof}
     Suppose that $Y'Z'$ intersects a side of $p_1$. If we apply IH4 to $p_1$, we find that one of these points must be interior to $p_1$ and the other exterior. Moreover, since the point of intersection of $Y'Z'$ with $p_1$ is exterior to $p_2$, we can apply IH5 to $p_2$ and conclude that $Y'$ and $Z'$ are both exterior to $p_2$. It follows by the definition of $p$ that $Y'$ and $Z'$ are then in opposite regions. A similar argument applies if $Y'Z'$ intersects a side of $p_2$.
  \end{proof}
\item[IH5$'$] If a segment $Y'Z'$ does not intersect $p$, then $Y'$ and $Z'$ lie in the same region.
  \begin{proof}
    If $Y'Z'$ does not intersect $p_1$ or $p_2$, then this reduces to the case IH5 applied to these polygons. Otherwise, we suppose that $Y'Z'$ intersects the diagonal $PQ$. 

    If $Y'$ is inside $p$, then it lies inside $p_1$, and thus $Z'$ must be outside $p_1$ by IH4 applied to $p_1$. Therefore, $Z'$ is outside $p_2$. 

Moreover, if $Y'$ lies inside $p$, then it too lies outside $p_2$. So the two points lie in the same region with respect to $p_2$. This means we have a contradiction: $Y'Z'$ could not possibly intersect $PQ$ as supposed, for otherwise, they would lie in opposite regions with respect to $p_2$ as required by IH4.

    Now what if $Y'$ is not inside $p$? Then there are three possibilities for its location:
    \begin{description}
    \item[The point $Y'$ lies inside $p_1$ and $p_2$] In this case, we can apply IH4 to $p_1$ and conclude that $Z'$ is outside $p_1$. By definition, both points are then exterior to $p$. 
    \item[The point $Y'$ lies on $PQ$] In this case, $Z'$ is a point on a ray which is cast from $Y'$ and must therefore lie either in the exterior of $p_1$, in the interior $PQ$ or in the interior of $p_2$. In any case, both $Y'$ and $Z'$ are exterior to $p$.
    \item[The point $Y'$ lies outside both $p_1$ and $p_2$] Here, we apply IH4 to $p_2$, and thus conclude that the point $Z'$ is inside $p_2$. By definition, both points are then exterior to $p$.
    \end{description}
  \end{proof}
\end{description}

\subsection{Discussion}
There are a few definite gaps in this first proof attempt. As said, we neglected to prove the base case for triangles, and we have not shown that the recursively defined interiors and exteriors are always non-empty. Finally, the without-loss-of-generality assumption made in finding the diagonal of a polygon needs to be carefully justified (it amounts to showing that a polygon cannot have a spiral shape).
 We consider these to be fairly straightforward matters, however. 

This is not the proof we chose to verify, though the one we \emph{have} verified shares many of its inferences. When we come to verify those inferences, we shall see that many details above have been glossed over. We hope, then, to convince the reader to be quite wary of the above proof as it stands. The details that we have glossed over will be put on a more solid footing when we discuss our formal verification.

\label{sec:Jordan1NavigationDiscussion}For instance, in Figure~\ref{fig:Navigation1}, we claim to be able to navigate around the exterior of a simple polygon. This intuitive idea will be needed again in our verified proof, where it needs to be carefully formulated, and the formulation needs to be carefully tied back to the premises and the desired conclusion. The details are given in \S\ref{sec:NavigationVerification}.

\label{sec:Jordan1SameSideDiscussion}At several places in the above proof, we also assumed we could always find a path connecting the points $Y$ and $Z$ in Figures~\ref{fig:UnionCase} and~\ref{fig:SubCase} to various other points. A careful formulation of what characterises such points, together with a proof that the required path can be exhibited, is given in \S\ref{sec:SameSideWallConnected}.

We have retained the above proof because of its computational properties. It recursively \emph{builds} the interior of a polygon from the interiors of triangles, and thus describes an algorithm for a point-in-polygon test by reducing the problem to point-in-triangle tests, which, in turn, reduce to point in half-plane tests. Our \emph{verified} proof does not have this feature. It was inspired by Veblen's proof, and its basic structure and focus are very different. Nevertheless, it is interesting that a number of the key steps overlap, and we shall refer back to this proof when we discuss the verification of the polygonal Jordan Curve Theorem in chapters~\ref{chapter:JordanVerification1} and~\ref{chapter:JordanVerification2}.

\section{Veblen's Proof}\label{sec:VeblenProof}
In his 1903 doctoral thesis~\cite{Veblenphd}, Veblen set out a basic set of axioms for Euclidean Geometry. His incidence and order axioms are very similar to Hilbert's own, and it should not be difficult to verify their equivalence. Unlike Hilbert, he provided a complete proof for the polygonal Jordan Curve Theorem, and we shall discuss that proof in this section, comparing it to the one we sketched in \S\ref{sec:JordanCurveFirstProof}.

Veblen does not explicitly define the interior or exterior of a polygon. Like most other proofs, he tries to show that the two regions exist implicitly by dividing the theorem into two claims: the first states that a simple polygon divides its plane into at least two regions; the second states that the polygon divides its plane into at most two regions. Both assertions can be formalised in terms of polygonal segments:
\begin{enumerate}
\item there are at least two points in the plane not on the polygon which cannot be connected by a polygonal segment without crossing the polygon\label{list:VeblenLemma1};
\item of any three points in the plane not on polygon, at least two of them can be connected by a polygonal segment without crossing the polygon.
\end{enumerate}

Veblen has a two page proof for this and it is one of the most detailed in his thesis, but it is questionable whether the proof is even correct. According to Reeken and Kanovei~\cite{HahnInconclusiveIndirect}, the proof was deemed ``inconclusive'' by Hahn, while Guggenheimer claims, citing Lennes and Hahn, that Veblen's proof assumes that the polygon can be triangulated and is thus only valid for convex polygons~\cite{GuggenheimerJordanCurve}. We could not find this criticism in Lennes~\cite{LennesPolygon}, and were initially satisfied by Veblen's proof, and so we attempted a verification of assertion (\ref{list:VeblenLemma1}) above. Eventually, as one might expect, we hit an obstacle. In the next few sections, we will explain the difficulty.

\subsection{Veblen's Lemma 1}\label{sec:VeblenLemma1}
The first half of Veblen's proof is given as the corollary to a very general theorem about polygons, expressed in terms of ``multiple points''. These are points, should they exist, where a polygon self-intersects:
\begin{quote}
Lemma 1. \emph{If a side of a polygon $q$ intersects a side of a polygon $p_n$ in a single point $O$ not a multiple point of $p_n$ or $q$, then $p_n$ and $q$, whether simple or not, have at least one other point in common.} 
\end{quote}

The lemma can be used to obtain the first half of the polygonal Jordan Curve Theorem, as we explain in \S\ref{sec:FinalProofJordan1}, but this is already a very general theorem, and one we draw special attention to in Chapter~\ref{chapter:JordanVerification1}. We can see that it holds even for very degenerate versions of polygons. Consider the example in Figure~\ref{fig:jordanDegenerate1}. Here, we have two polygons $P_1P_2P_3$ and $Q_1Q_2Q_3$. The points of these polygons are obviously collinear and so cannot divide the plane into multiple regions. But neither are they a counterexample to Veblen's lemma, since their point of intersection $X$ is a multiple point of both, lying simultaneously on the segments $P_1P_2$ and $P_2P_3$, and also lying on the segments $Q_1Q_2$ and $Q_2Q_3$.

\begin{figure}
\centering
\includegraphics{jordan/jordanDegenerate1}
\caption{Degenerate polygons intersecting at a multiple point}
\label{fig:jordanDegenerate1}
\end{figure}

We have formally verified that this particular lemma follows from Hilbert's axioms (see \S\ref{sec:PathTheorem}), but we gave up trying to reproduce Veblen's argument, and believe it to be invalid. We do not have a counterexample, as such, since we do not think Veblen's proof is sufficiently detailed to say exactly where it fails. Instead, we have tried to illustrate the difficulties we faced with the example in Figure~\ref{fig:VeblenCounter1}. This example shows a simple maze polygon $p_n$ being intersected at a point $O$ by another polygon $q$ shown in red. The goal is to identify one of the other seven points of intersection. Our labelling in the diagram is consistent with the set-up to Veblen's proof:

\begin{quote}
If $n=3$ ($q$ having any number of sides, $m$) the theorem reduces to [the case for triangles]. We assume without loss of generality that no three vertices $P_{i-1}$, $P_i$, $P_{i+1}$ are collinear and prove the lemma for every $n$ by reducing to the case $n=3$. Let $p_n$ have $n$ vertices with the notation such that the side $P_1P_2$ meets $q$ in the side $Q_1'Q_2'$ where the segment $Q_2'O$ contains no interior point of the triangle $P_1P_2P_3$.\end{quote}

\begin{figure}
\centering
\includegraphics[scale=0.6]{jordan/veblenCounter1.pdf}
\caption{Intersections on a simple maze}
\label{fig:VeblenCounter1}
\end{figure}

Veblen's basic strategy is to consider each of the triangles $P_1P_2P_3$, $P_1P_3P_4$, $P_1P_4P_5$, $P_1P_5P_6$, $\ldots$. Of these triangles, all but the first and last share exactly one side with the polygon $p_n$, with the other sides being diagonals of the polygon. Veblen tries to show that if $q$ intersects one of these triangles in one diagonal, then it intersects the next triangle. In this way, intersections can be found, one after the other, down the list of triangles, until we eventually find a second point of intersection with the polygon $p_n$.

\subsection{Finding a subset of $q$}\label{sec:SubsetOfQ}
In Figure~\ref{fig:VeblenCounter2}, we show a polygonal segment  (or ``broken line'') which is a subset of the polygon $q$, with vertices $O_kQ_2'Q_3'Q_4'Q_5'Q_6'O_j$. This segment makes contact with the diagonal $P_1P_3$ exactly twice, once from the outside of the triangle $P_1P_2P_3$, and once from the inside. The polygonal segment always exists, and can be proven to intersect $P_1P_3$ in just this way. To find it, we start from the segment $Q_2'Q_3'$ and progressively add neighbouring segments until we eventually reach a segment which intersects the line $P_1P_3$. As Veblen puts it:

\begin{figure}
\centering
\includegraphics[scale=0.6]{jordan/veblenCounter2.pdf}
\caption{A chosen subset of $q$: $k=2$ and $j=6$}
\label{fig:VeblenCounter2}
\end{figure}

\begin{quote}By the case $n=3$, $q$ meets the boundary of the triangle $P_1P_2P_3$ in at least one point other than $O$. If this point is on the broken line $P_1P_2P_3$ the lemma is verified. If not, $q$ has at least one point on $P_1P_3$, and at least one of the segments $Q_1'Q_2'$, $Q_2'Q_3'$ has no point or end-point on $P_1P_3$. Let this segment be one segment of a broken line $Q_kQ_{k+1}\cdots Q_{j-1}Q_j$ of segments of $q$ not meeting $P_1P_3$ but such that $Q_{k-1}Q_k$ and $Q_jQ_{j+1}$ do each have a point or endpoint in common with $P_1P_3$ ($1 \leq k < j \leq m$; if $k = 1$, $Q_{k-1} = Q_m$; if $j = m$, $Q_{j+1} = Q_1$). If $O_j$ is the point common to $P_1P_3$ and $Q_jQ_{j+1}$ or $Q_{j+1}$, and $O_k$ is the point common to $P_1P_3$ and $Q_{k-1}Q_k$ or $Q_{k-1}$, the broken line $O_kQ_kQ_{k+1}\cdots Q_{j-1}Q_jO_j$, has a point inside and also a point outside the triangle $P_1P_2P_3$ and cuts the broken line $P_1P_2P_3$ only once. \end{quote}

Now there is nothing particularly informative about Veblen's last remark. Notice that the segment $Q_1Q_2$ in Figure~\ref{fig:VeblenCounter1} \emph{also} has a point inside and a point outside the triangle $P_1P_2P_3$. It also cuts the polygonal segment $P_1P_2P_3$ exactly once. Something is missing here. There must be some other property had by the polygonal segment $O_kQ_2'Q_3'Q_4'Q_5'Q_6'O_j$, in order for Veblen to get to his very next claim, that ``it has a point inside and a point outside any triangle of which $P_1P_3$ is a side''. The missing property is an important detail, because as we mentioned, part of the argument is supposed to be repeated down the list of triangles $\triangle P_1P_2P_3$, $\triangle P_1P_3P_4$, $\triangle P_1P_4P_5$, $\triangle P_1P_5P_6$, $\ldots$. If we are going to repeat this part of the argument, we need to know that the missing properties are \emph{invariants}.

For now, we just note that Veblen's claim certainly follows. We believe the crucial point is that, as we remarked earlier, the polygonal segment $O_kQ_2'Q_3'Q_4'Q_5'Q_6'O_j$ touches $P_1P_3$ exactly twice, once from the outside and once from the inside. More precisely, we just note that the interior of $Q_6'O_j$ is outside the triangle, while the interior of $Q_2'O_k$ is inside. To establish this, we note that the points inside the segment $O_kO$ are all inside the triangle $P_1P_2P_3$.\footnote{For Veblen, this is a matter of definition. See~\ref{sec:TriangleInteriorDefinition}} Thus, by the Jordan Curve Theorem applied to triangles, all points of the polygonal segment $O\cdots Q_2'Q_3'Q_4'Q_5'Q_6'O_j$ must be outside the triangle $P_1P_2P_3$. It is then possible to show that the points inside the segment $O_kO$ are on the opposite side of the line $P_1P_3$ as the points inside the segment $Q_6'O_j$, and so must be in different regions of any triangle of which $P_1P_3$ is a side. Veblen's claim then follows: the polygonal segment $O_kQ_2'Q_3'Q_4'Q_5'Q_6'O_j$ has a point inside and a point outside any triangle of which $P_1P_3$ is a side. We just needed a bit of extra work to get there.

\subsection{Veblen's Conclusion}
\begin{quote}On this account if $P_1P_3P_4$ are not collinear, and obviously, if $P_1P_3P_4$ are collinear, $q$ must meet either $P_3P_4$ or $P_4$ or $P_4P_1$. If $q$ does not meet $P_3P_4$ or $P_4$, we proceed with $P_1P_4P_5$ as we did with $P_1P_3P_4$. Continuing this process, we either verify the lemma or come by $n-2$ steps to the triangle $P_1P_{n-1}P_n$ and find that $q$ must intersect the broken line $P_{n-1}P_nP_1$, which also verifies the lemma.\end{quote}

This is Veblen's conclusion, and the situation described is illustrated in Figure~\ref{fig:VeblenCounter3}. The first step follows by the Jordan Curve Theorem applied to triangles: we know that $O_kQ_2'Q_3'Q_4'Q_5'O_6'O_j$ does not strictly\footnote{The use of phrases such as ``strictly cut'', among many other details, will be clarified in our verification. See Chapter~\ref{chapter:JordanVerification1}.} \emph{cut} the line $P_1P_3$, so it must instead cut either $P_1P_4$ or $P_3P_4$. We can assume that $q$ does not meet $P_3P_4$, and so we proceed with $P_1P_4P_5$. But Veblen is not clear on exactly \emph{how} the argument repeats. The reason the first step follows in the above conclusion is because $O_kQ_2'Q_3'Q_4'Q_5'O_6'O_j$ does not cut $P_1P_3$, but we have now assumed that it \emph{does} cut $P_1P_4$, so we are not simply repeating the conclusion. 

\begin{figure}
\centering
\includegraphics[scale=0.6]{jordan/veblenCounter3.pdf}
\caption{Intersections with $P_1P_3P_4$}
\label{fig:VeblenCounter3}
\end{figure}

Notice that in this conclusion, Veblen has slipped to talking about the original polygon $q$ rather than the polygonal segment $O_kQ_2'Q_3'Q_4'Q_5'O_6'O_j$ which is a subset of $q$. Presumably then, we need to find another subset of $q$ which has the same properties as the original. And here we encounter the problem of detail mentioned at the end of \S\ref{sec:SubsetOfQ}: we are just not sure what the crucial properties of the subset \emph{are}. Nevertheless, we tried earlier to fill in those details, and based on our attempt, we might assume that the desired polygonal segment for our example is the one shown in Figure~\ref{fig:VeblenCounter4}, namely $O_k'Q_6'Q_7'Q_2'Q_3'Q_4'O_j'$. This polygonal segment can be identified by starting at the point $O_k$ and following Veblen's procedure as we did with the point $O$. 

\begin{figure}
\centering
\includegraphics[scale=0.6]{jordan/veblenCounter4.pdf}
\caption{Desired subset of $q$?}
\label{fig:VeblenCounter4}
\end{figure}

Remember that Veblen's next claim is that the polygonal segment $O_k'O_kQ_2'Q_3'Q_4'O_j'$ has a point inside and outside any triangle of which $P_1P_3$ is a side. Again, this is certainly true. But earlier, we justified it by noting that the polygonal segment touches $P_1P_4$ exactly twice, once from inside the triangle and once from outside. More formally, we can notice that the points inside the segment $O_k'Q$ lie inside the triangle $P_1P_3P_4$, while the points inside the segment $Q_4'O_j'$ lie outside the triangle. We could establish this earlier by relying on Veblen's assertion that the polygonal segment $O_k'O_kQ_2'Q_3'Q_4'O_j'$ crosses $P_1P_3$ \emph{exactly once}, but this is no longer true. Instead, the required observation is that it crosses $P_1P_3$ an \emph{odd} number of times, and so it has a non-empty suffix which lies entirely outside the polygon. 

We therefore need a proof that the number of crossings must always be odd, but we cannot see anything in Veblen's argument that would require this. We presumably need additional lemmas about the parity of crossings on a triangle's side. But as soon as we start to do \emph{that}, we can immediately identify an argument which is conceptually much more straightforward. We give this argument in full in \S\ref{sec:ParityProofInformal}.

\section{Conclusion}
In this chapter, we have looked at several proofs of the Jordan Curve Theorem for simple polygons which might be applicable to Hilbert's two groups of axioms. Our first proof started with a recursive definition of an arbitrary polygon's interior and exterior. It explained how to add and subtract smaller polygons at each recursive step until we arrive at the base-case in triangles. The rest of the proof then shows that the recursive definition always yields two distinct regions which cover the plane, and such that any two points in the regions can be joined by a polygonal segment.

We then reviewed Veblen's 1903 proof of the theorem, from axioms equivalent to Hilbert's, and which, on the face of it, is far simpler to our first proof. However, it has been claimed elsewhere~\cite{GuggenheimerJordanCurve,HahnInconclusiveIndirect} to be invalid. Based on a verification attempt, we believe we have identified the crucial problem, and have suggested that its fix leads immediately to a conceptually simple parity proof. In the next chapter, where we shall review our verification of the parity proof, we shall be able to constrast its conceptual simplicity with the complexity of its details.

There are at least three other published proofs of the Polygonal Jordan Curve Theorem from axioms equivalent to Hilbert's. Feigl's proof is cited by Bernays in the \emph{Grundlagen der Geomtetrie}~\cite{FeiglJordan}. A second proof was provided by Main~\cite{MainHilbertGeometry}, whose doctoral work was supervised by the same Professor Goheen who wrote the foreword to the 10th edition of the \emph{Grundlagen der Geometrie}. Both Feigl and Main's proof were written some decades after Veblen's proof, and both exploit a parity argument based on angles (defined as a pair of rays emanating from a single point). 

Finally, Guggenheimer supplied a proof nine years after Main's~\cite{GuggenheimerJordanProof}. He introduces the proof lamenting the fact that the general theorem is so rarely proven in the literature:
\begin{quotation}It is astonishing that none of the textbooks of elementary axiomatic geometry gives a proof [of the Polygonal Jordan Curve and Schoenflies's Theorems].
\end{quotation}
Guggenheimer's proof is conceptually more sophisticated than Feigl's and Main's, exploiting a theorem by Dehn~\cite{GuggenheimerJordanCurve} and showing that there exists a suitable homeomorphism from the plane to a half-plane which maps polygonal Jordan curves onto triangles. A consequence of this is that the theorem reduces to Pasch's Axiom~\ref{eq:g24}.

Veblen's proof, which we follow most closely, has historical precedent though, and our verification is an additional contribution in that it makes clear how Veblen's proof can be fixed. With the verification, we have now unquestionably shown that despite the seeming theoretical poverty of Hilbert's first two groups, we have sufficient primitives and axioms to prove the Jordan Curve Theorem for polygons. We also hope that our analysis of the proof shows that Hilbert was mistaken to think the theorem could be obtained ``without serious difficulty.''

%%% Local Variables: 
%%% TeX-master: "../thesis"
%%% End: 

\chapter{Formalising the Polygonal Jordan Curve Theorem}\label{chapter:JordanFormalisation}
% In this chapter, we will analyse our HOL~Light~\cite{HOLLight} verification of the Jordan Curve Theorem for Polygons. This theorem appears as Theorem~9 in the 10th edition of Hilbert's \emph{Grundlagen der Geometrie}. The verification might stand on its own: the theorem's formulation does not appeal to earlier definitions or theorems, and of all the proofs we have considered, its verification is the most complicated by an order of magnitude. But the theorem also serves as a useful case-study for the tools that we have developed up to this point: the linear reasoning tactic from Chapter~\ref{chapter:LinearOrder} was used pervasively in verifying the theorem, becoming the crucial work-horse for solving and exploring complex linear ordering problems. We leveraged the theory of Half-Planes from Chapter~\ref{chapter:HalfPlanes} as the basis for key lemmas on two-dimensional ordering. Lastly, the discovery algebra described in Chapter~\ref{chapter:Automation} was needed throughout to verify incidence problems and to identify case-splits and suggest proof strategies based on discovered facts. 

In this chapter, we introduce our verification of the Polygonal Jordan Curve Theorem, and we present its formalisation in full. We will set forth, unambiguously, \emph{what} we have verified, and thus, much of this chapter can be read independently of the verification. The formal definitions and formalisations are not as simple as those in earlier chapters, and so they must be carefully checked. But once we are convinced that the formalisations correctly capture the statement of the theorem, we have an absolute guarantee that it is derivable from Hilbert's axioms.

\section{Organisation}
The verification is divided neatly into two parts, corresponding to the verification of two theorems, which we formalise in \S\ref{sec:JordanFormulation}:
\begin{enumerate}
  \item there are two points in the plane of a simple polygon such that any polygonal path connecting them must cross the simple polygon (verified in Chapter~\ref{chapter:JordanVerification1});
  \item given three points in the plane of a simple polygon, there is a polygonal path connecting two of those points (verified in Chapter~\ref{chapter:JordanVerification2}).
\end{enumerate}

When we discuss the verifications, we will use the same basic structure. First, we shall give an overview of the theory structure, effectively amounting to a sketch proof of the theorem. From this, we shall give formalisations of any key concepts mentioned in the proof. In the case of the first part of the Polygonal Jordan Theorem, these are the concepts of triangle interiors and exteriors and the concept of a ``crossing'' of a triangle by a polygonal path. For the second part of the theorem, we have concepts of ``lines-of-sight'' and polygon rotations. 

These concepts are supported by a suite of lemmas. Consistent with the synthetic style of geometric proof, these lemmas generally break down into one of two forms. We have lemmas which introduce points and other objects in a geometrical configuration. We also have lemmas which allow us to infer properties of these configurations. The two sorts of lemmas are used in tandem: we build up complex figures using the ``introduction'' rules, and then use the other rules to satisfy the hypotheses of yet more ``introduction'' rules, and so on. Many of these lemmas are interesting in their own right, and we shall pick a few for closer examination.

We present some extracts of what we hope are interesting verifications, ones which highlight the benefits and drawbacks of our representations in terms of the resulting proof mechanics. We hope to use these extracts to convince the reader that we have stayed faithful to the synthetic style of proof, even as the verifications grow significantly in complexity. The complexity is easy enough to measure. The verifications from Chapter~\ref{chapter:Group2Eval} take only a dozen or so steps. The verifications in the next few chapters frequently run to hundreds, and we suspect this really is due to the complexity of the synthetic reasoning involved.

\section{Related Work}
The Jordan Curve Theorem has some notoriety within the formal verification community, and has long been regarded as a major milestone, one which  demonstrates the feasibility of formal verification on non-trivial mathematics problems. The MIZAR~\cite{MizarMathematicalVernacular} community first began a verification in 1991, and completed the special case for polygons in 1996. The full proof was completed in 2005. In the same year, Hales completed an independent proof in HOL~Light~\cite{HalesJordanCurve}. 

Both proofs use the special case for polygons as a lemma, though in a restricted form: in the case of the MIZAR proof, only polygons with edges parallel to axes are considered. In Hales' proof, the polygon is restricted to lie on a grid. Moreover, the formulations are algebraic rather than synthetic, and so are outside the scope of Hilbert's and Veblen's formulations.

In 2009, Dufourd~\cite{DufourdJordanCurve} formally verified a constructive proof of the Discrete Jordan Curve Theorem. This theorem characterises Jordan curves in terms of a ``ring'' of faces in a subdivision of a plane or sphere. This is a more practical achievement, since we can expect this sort of characterisation to be more useful to computer scientists trying to express and reason about topological properties computationally. However, our interest here is in Hilbert's formulation and a synthetic proof of the Jordan Curve Theorem in ordered geometry.

%\section{Foundational Issues}
% Many of Hilbert's definitions and theorems are first-order logic, and indeed, much of our formalisation makes no essential use of higher-order or number theoretical notions. But as with Hilbert`s Theorem~6 which we discussed in \S\ref{sec:Theorem6}, things are not so clear for Hilbert's definition of polygons.

% It seems easy to excuse Hilbert's lack of clarity on these matters, since he was working before symbolic first-order logic had been invented. But then, when we consider the reviews by Veblen and Poincar\'{e}~\cite{VeblenHilbertReview,PoincareReview}, we realise that the community was certainly interested in the mechanisation of logic. Both reviews anticipate the mechanical verification of Hilbert's geometry, and both mention the work of Peano, whose formulations of logic and mathematics were higher-order, and even admitted recursive definitions. Peano's ideas would later be developed in Russell's \emph{Principles of Mathematics}~\cite{PrinciplesOfMathematics}, which led directly to Russell and Whitehead's \emph{Principia Mathematica}.

% We realise there is some controversy about how strong the logic should   , we have not been shy about using 

% \begin{quote}[The axioms] presuppose only the validity of the operations of logic and counting (ordinal number).\end{quote}

% As with Hilbert's Theorem~6 \S\ref{sec:Theorem6}, a formulation of polygon and polygonal path raises a number of questions relating to our logical foundations. As we have previously stated, Hilbert's work predates the systematic development of first-order logic, and 

\section{Formulation}\label{sec:JordanFormulation}
\begin{quotation}
  DEFINITION. A set of segments $AB$, $BC$, $CD$, $\ldots$, $KL$ is called a \emph{polygonal segment} that connects the points $A$ and $L$. Such a polygonal segment will also be briefly denoted by $ABCD\ldots KL$. The points inside the segments $AB$, $BC$, $CD$, $\ldots$, $KL$ as well as the points $A$, $B$, $C$, $D$, $\ldots$, $K$, as well as the points $A$, $B$, $C$, $D$, $\ldots$, $K$, $L$ are collectively called the \emph{points of the polygonal segment}. 

In addition, for a polygonal segment $A$, $B$, $C$, $\ldots$, $L$, we shall call the segments $AB$, $BC$, $CD$, $\dots$, $KL$ the \emph{sides} of the polygonal segment.
\flushright{\emph{Foundations of Geometry}~\cite{FoundationsOfGeometry} (page 8)}
\end{quotation}
This is Hilbert's first key definition, defining what Veblen calls \emph{broken lines}, and what we shall prefer to call \emph{polygonal paths}. Contrary to his conventions~(see \S\ref{sec:DistinctVars}), Hilbert does not insist that the points involved here are distinct. Indeed, he adds an explicit distinctness clause when he defines simple polygons (see \S\ref{sec:polygonFormalisation}).

One thing which is clear to us is that Hilbert's notation is doing the heavy lifting in this definition. He uses it to take care of the otherwise clumsy constraint that all but two segments in the set share their endpoints with at least two other segments. The remaining two elements must share exactly one of their endpoints with another segment.

Had Hilbert given a proof of the polygonal Jordan Curve Theorem, he would have probably used his notation to do more heavy lifting, as Veblen did in his own proof. Veblen's argument ends up running over symbols and numerical subscripts in a way which seems to take us out of the world of synthetic geometry, and back into the sort of computational metatheory we saw in \ref{eq:six} from Chapter~\ref{chapter:LinearOrder}.
\begin{quotation}
  By the case $n=3$, $q$ meets the boundary of the triangle $P_1P_2P_3$ in at least one point other than $O$. If this point is on the broken line $P_1P_2P_3$ the lemma is verified. If not, $q$ has at least one point on $P_1P_3$, and at least one of the segments $Q_1'Q_2'$, $Q_2'Q_3'$ has no point or end-point on $P_1P_3$. Let this segment be one segment of a broken line $Q_kQ_{k+1}\cdots Q_{j-1}Q_j$ of segments of $q$ not meeting $P_1P_3$ but such that $Q_{k-1}Q_k$ and $Q_jQ_{j+1}$ do each have a point or endpoint in common with $P_1P_3$ ($1 \leq k < j \leq m$; if $k = 1$, $Q_{k-1} = Q_m$; if $j = m$, $Q_{j+1} = Q_1$). If $O_j$ is the point common to $P_1P_3$ and $Q_jQ_{j+1}$ or $Q_{j+1}$, and $O_k$ is the point common to $P_1P_3$ and $Q_{k-1}Q_k$ or $Q_{k-1}$, the broken line $O_kQ_kQ_{k+1}\cdots Q_{j-1}Q_jO_j$, has a point inside and also a point outside the triangle $P_1P_2P_3$ and cuts the broken line $P_1P_2P_3$ only once.
\flushright{Veblen~\cite{Veblenphd} (page~365)}
\end{quotation}
Looking at Veblen's proof attempt, one might expect the argument to be dominated by computations and lemmas about point sequences, and not by synthetic constructions of diagrams, but this is not entirely the case. Our verification shows that there are a \emph{lot} of useful lemmas very similar to those we saw in Chapter~\ref{chapter:Group2Eval}, whose synthetic proofs obtain and then reason about properties of diagrams.

% Metatheoretical arguments like this go at least back to Euclid. Consider Euclid's classic Proposition 20 from Book IX:
% \begin{quote}
% Prime numbers are more than any assigned multitude of prime numbers.
% Let $A$, $B$, and $C$ be the assigned prime numbers.

% I say that there are more prime numbers than $A$, $B$, and $C$.
% \end{quote}
% This is not a proof, but more like a proof \emph{template}. Euclid is assuming his reader knows how to repeat the argument for other assignments of prime numbers besides $A$, $B$ and $C$!

% This is fairly typical of prose mathematics, where the notation itself because the means to establish constraints and the mechanism of reasoning. Consider again, Veblen's proof from the last chapter, to see how mathematical proofs can involve complicated arguments, not about the mathematical objects of the theory, but the \emph{symbols} used to denote them:

% \begin{quote}Let this segment be one segment of a broken line $Q_kQ_{k+1}\cdots Q_{j-1}Q_j$ of segments of $q$ not meeting $P_1P_3$ but such that $Q_{k-1}Q_k$ and $Q_jQ_{j+1}$ do each have a point or endpoint in common with $P_1P_3$ ($1 \leq k < j \leq m$; if $k = 1$, $Q_{k-1} = Q_m$; if $j = m$, $Q_{j+1} = Q_1$). If $O_j$ is the point common to $P_1P_3$ and $Q_jQ_{j+1}$ or $Q_{j+1}$, and $O_k$ is the point common to $P_1P_3$ and $Q_{k-1}Q_k$ or $Q_{k-1}$, the broken line $O_kQ_kQ_{k+1}\cdots Q_{j-1}Q_jO_j$, has a point inside and also a point outside the triangle $P_1P_2P_3$ and cuts the broken line $P_1P_2P_3$ only once.\end{quote}

%As we described Hilbert's Theorem~6 (see \S\ref{sec:Theorem6}) as a metatheorem, we would describe Veblen's argument here as a \emph{metaproof}. The use of such metatheoretical arguments in axiomatic geometry probably goes back to Euclid, who 

With the notation doing such heavy lifting, we will formalise it directly. We opted to represent polygonal paths as their finite sequence of points, from which the original polygonal paths can be recovered. With this definition, the clumsy constraint about segments sharing endpoints is handled implicitly. The heavy lifting will be carried out as computations on lists, which are available to us in the logic, ultimately being defined in terms of natural numbers. As we saw in Chapter~\ref{chapter:LinearOrder}, this means they are still ultimately represented by geometric figures. 

%TODO: Discuss or refer to an earlier discussion about using lists to handle ``meta-level'' definitions, and why lists are somewhat justified by the fact that all infinite structures are now abstractions over geometric objects since we have derived the axiom of infinity.

Now the list library in HOL Light is slightly impoverished (at least compared with, say, that of Isabelle/HOL), and so our verification had to take a detour as we added new function definitions and theorems for lists. For instance, in order to recover the edges of a polygonal path, we must take adjacent pairs of the points in its vertex list. We can do this by following a standard pattern in functional programming, shown in Figure~\ref{fig:AdjacentSpec}. The function \code{adjacent} zips all but the last element of a list with its tail. However, it is generally easier and requires much less unfolding to compute directly with the recursive specification of the function. We give this along with the definitions and specifications of other auxiliary functions in Figure~\ref{fig:ListDefinitions}.

\begin{figure}
\begin{align*}
  &\code{adjacent}\; : \; [\alpha] \rightarrow [(\alpha,\alpha)]\\
  \vdash_{def}\;&\code{adjacent}\ [P_0,P_1,P_2,\ldots,P_n]\\
  &\quad=\code{zip}\ (\code{butlast}\ [P_0,P_1,P_2,\ldots,P_n])\ (\code{tail}\ [P_0,P_1,P_,2,\ldots,P_n])\\
  &\quad= \code{zip}\ \begin{array}[t]{rcccccl}\lbrack & P_0,&P_1,&P_2,&\ldots,&P_{n-1}&\rbrack\\
    \lbrack & P_1,&P_2,&P_3,&\ldots,&P_n&\rbrack
  \end{array}\\
  &\quad= [(P_0,P_1),(P_1,P_2),(P_2,P_3),\ldots,(P_{n-1},P_n)]\\\\
  &\vdash\code{adjacent}\ [] = []\\
  &\vdash\code{adjacent}\ [x] = []\\
  &\vdash\code{adjacent}\ (\cons{x}{\cons{y}{xs}}) = \cons{(x,y)}{\code{adjacent}\ (\cons{y}{xs})}
\end{align*}
\caption{Specifications for $\code{adjacent}$}
\label{fig:AdjacentSpec}
\end{figure}

\begin{figure}
\begin{alignat*}{3}
  & \code{head}\,:\,[\alpha] \rightarrow \alpha &\qquad
  & \code{tail}\,:\,[\alpha] \rightarrow [\alpha]\dag &\qquad
  & \code{length}\ [\alpha]\rightarrow \mathbb{N}\\
  \vdash_{def}\;& \code{head}\ (\cons{x}{xs}) = x &\qquad
  \vdash_{def}\;& \code{tail}\ [] = [] &\qquad
  \vdash_{def}\;& \code{length}\ [] = 0\\
  & &\qquad
  \vdash_{def}\;&\code{tail}\ (\cons{x}{xs}) = xs &\qquad
  \vdash_{def}\;& \code{length}\ (\cons{x}{xs})\\
  & & & & &\quad = \code{length}\ xs + 1
\end{alignat*}
\begin{align*}
  &\code{butlast}\,:\,[\alpha] \rightarrow [\alpha]\\
  \vdash_{def}\;&\code{butlast}\ [] = []\\
  \vdash_{def}\;&\code{butlast}\ (\cons{x}{xs}) = \code{if}\ xs=[]\ \code{then}\ []\ \code{else}\ \cons{x}{(\code{butlast}\ xs)}
\end{align*}
\begin{align*}
  &\code{el}\,:\,\code{int} \rightarrow [\alpha] \rightarrow \alpha\\
  \vdash_{def}\;&\code{el}\ 0\ xs = \code{head}\ xs\\
  \vdash_{def}\;&\code{el}\ (\code{suc}\ n)\ xs = \code{el}\ n\ (\code{tail}\ xs)
\end{align*}
\begin{align*}
  & \code{mem}\,:\,\alpha \rightarrow [\alpha] \rightarrow \code{bool} &\qquad
  & \code{all}\,:\,(\alpha\rightarrow\code{bool}) \rightarrow [\alpha] \rightarrow \code{bool}\\
  \vdash_{def}\;& \code{mem}\ x\ [] = \bot &\qquad
  \vdash_{def}\;& \code{all}\ p\ [] = \top\\
  \vdash_{def}\;& \code{mem}\ x\ (\cons{y}{ys}) = x = y \vee \code{mem}\ x\ ys &\qquad
  \vdash_{def}\;& \code{all}\ p\ (\cons{x}{xs}) = p\ x \wedge \code{all}\ p\ xs   
\end{align*}
\begin{align*}
  & \code{pairwise}\,:\,(\alpha \rightarrow \alpha \rightarrow \code{bool}) \rightarrow [\alpha] \rightarrow \code{bool}\\
  \vdash_{def}\;& \code{pairwise}\ R\ [] = \top\\
  \vdash_{def}\;& \code{pairwise}\ R\ (\cons{x}{xs}) = \code{all}\ (R\ x)\ xs \wedge \code{pairwise}\ R\ xs
\end{align*}
\begin{align*}
  & \code{disjoint}\,:\,(\alpha\rightarrow\code{bool})\rightarrow(\alpha\rightarrow\code{bool})\rightarrow\code{bool}\\
  \vdash_{def}\;& \code{disjoint}\ S\ T = S \cap T = \emptyset
\end{align*}
$\dag$ This function is a more well-defined version of the existing function in HOL~Light. The original version is undefined for the empty list.
\caption{List definitions and specifications}
\label{fig:ListDefinitions}
\end{figure}

With the function $\code{adjacent}$, we can define the points of a polygonal path. As per Hilbert's definition, these are the points of the vertex list and the points inside each individual segment $(x,y)$. We can test for each using the list-membership predicate~$\code{mem}$.
\begin{equation}\label{eq:OnPolyPath}
  \begin{split}
    &\code{on\_polypath}\,:\,[\code{point}] \rightarrow \code{point} \rightarrow \code{bool}\\
    \vdash_{def}\;&\code{on\_polypath}\ Ps\ P \iff \\
    &\quad\code{mem}\ P\ Ps\vee\,\exists x.\;\exists y.\; \code{mem}\ (x,y)\ (\code{adjacent}\ Ps) \wedge \between{x}{P}{y}.
  \end{split}
\end{equation}

Next, we need to formalise the notion of region as used in the Polygonal Jordan Curve Theorem. We shall follow our approach when formalising the notions of \emph{half-plane} and \emph{rays} (see \S\ref{sec:RayQuotienting}), and understand these regions as equivalence classes under a suitable relation, namely one which requires there to be a polygonal path between two given points. When two points satisfy this relation, we shall say that they are \emph{polygonal path-connected.}
\begin{align*}
  \vdash_{def}&\code{polypath\_connected}\,:\,\code{plane}\rightarrow(\code{point} \rightarrow \code{bool}) \rightarrow \code{point} \rightarrow \code{point} \rightarrow \code{bool}\\
  &\code{polypath\_connected}\ \alpha\ figure\ P\ Q \iff\\
  &\quad\exists path.\; path \neq []\\
  &\qquad\wedge\,(\forall R.\;\code{mem}\ R\ path \implies \onplane{R}{\alpha})\\
  &\qquad\wedge\,\code{head}\ path = P \wedge \code{last}\ path = Q\\
  &\qquad\wedge\,\code{disjoint}\ (\code{on\_polypath}\ path)\ figure.
\end{align*}

Note that, when formalised, this relation is defined on the set of all points in space, parameterised on a plane $\alpha$ and a $figure$. Instead of a plane parameter, we could have added constraints to the figure and path, such as requiring that they all lie in exactly one plane. The trade-offs are in the number of constraints in the definition compared with the number of constraints on later theorems. Here, we have opted for the simpler definition.

A figure here is represented by a predicate-set of all the points on the figure. Thus, the relation on points in the plane $\alpha$ which are polygonal path-connected with respect to a polygonal path $Ps$ can be cleanly expressed by 
\begin{displaymath}
\code{polypath\_connected}\ \alpha\ (\code{on\_polypath}\ Ps).
\end{displaymath}
This is obviously an equivalence relation.

\subsection{Verifying Equivalence}\label{sec:segConnectEquivalence}
The proof that we have an equivalence relation boils down entirely to properties of lists. To prove reflexivity, we use the one-element polygonal path (excluded in Hilbert's definitions). Our witness for symmetry is obtained by reversing the supplied polygonal path. Our witness for transitivity is obtained by appending the two supplied polygonal paths. To reason about the resulting lists, we first verified some extra simplification rules for our list functions:
\begin{align*}
  \vdash&xs \neq [] \wedge ys \neq []\\ 
  &\quad\implies\code{adjacent}(\append{xs}{ys})\\
  &\qquad\qquad = \append{\append{\code{adjacent}\ xs}{(\code{last}\ xs,\ \code{hd}\ ys)}}{\code{adjacent}\ ys}.\\
  \vdash&n + 1 < \code{length}\ xs\\
  &\quad\implies \el{n}{(\code{adjacent}\ xs)} = (\el{n}{xs}, \el{(n+1)}{xs}).\\
  \vdash&\code{length}\ xs = \code{length}\ ys\\
  &\quad\implies\code{reverse}\ (\code{zip}\ xs\ ys) = \code{zip}\ (\code{reverse}\ xs)\ (\code{reverse}\ ys).\\
  \vdash&\code{butlast}\ (\code{reverse}\ xs) = \code{reverse}\ (\code{tail}\ xs).\\
  \vdash&\code{tail}\ (\code{reverse}\ xs) = \code{reverse}\ (\code{butlast}\ xs).
\end{align*}

With these, we can verify three theorems showing that polygonal path-connectedness defines an equivalence relation. The domain of the relation is the set of points in the plane which are not points of the figure. The constraint appears as an assumption on reflexivity. 

Obviously, we will need the three theorems \emph{somewhere} in our later verification, though it turns out that they are not used very often. They can be seen, at least, as a sanity check on our notion of polygonal path-connectedness.
\begin{align*}
  \vdash&\code{on\_plane}\ P\ \alpha \wedge \neg figure\ P \implies \code{polypath\_connected}\ \alpha\ figure\ P\ P.\\
  \vdash&\code{polypath\_connected}\ \alpha\ figure\ P\ Q \implies \code{polypath\_connected}\ \alpha\ figure\ Q\ P.\\
  \vdash&\code{polypath\_connected}\ \alpha\ figure\ P\ Q \wedge \code{polypath\_connected}\ \alpha\ figure\ Q\ R\\
  &\qquad\implies \code{polypath\_connected}\ \alpha\ figure\ P\ R.
\end{align*}

\subsection{Polygons}\label{sec:polygonFormalisation}
Hilbert continues his definitions as follows:
\begin{quote}
``If the points $A$, $B$, $C$, $D$, $\ldots$, $K$, $L$ all lie in a plane and the point $A$ coincides with the point $L$ then the polygonal path is called a \emph{polygon} and is denoted as the polygon $ABCD\ldots K$. The segments $AB$, $BC$, $CD$, \ldots, $KA$ are also called the \emph{sides of the polygon}. The points $A$, $B$, $C$, $D$, $\ldots$, $K$ are called the \emph{vertices of the polygon.} Polygons of $3$, $4$, $\ldots$, $n$ \emph{vertices} are called \emph{triangles}, \emph{quadrilaterals}, $\ldots$, \emph{n-gons}.'' \emph{Foundations of Geometry}~\cite{FoundationsOfGeometry} (pages~8--9)
\end{quote}

The term ``side'' is ambiguous between the segments of a polygon and the half-planes of a given line. As such, we shall refer to the segments defining both a polygonal path and a polygon as \emph{edges}, and reserve \emph{side} for half-planes. These definitions will help us orientate ourselves within the familiar world of geometry, but we do not believe they correspond to any useful abstractions for verification. As such, we shall only use them informally to explain parts of the verification. The important definition for the verification is the one for \emph{simple polygons}:
\begin{quote}
  ``DEFINITION. If the vertices of a polygon are all distinct, none of them falls on [an edge] and no two of its nonadjacent [edges] have a point in common, the polygon is called \emph{simple}.'' \emph{Foundations of Geometry}~\cite{FoundationsOfGeometry}, (page~9)
\end{quote}

Combining this with the definition of polygon gives us the following formalisation:
\begin{equation}\label{eq:SimplePolygonDef}
  \begin{split}
  &\code{simple\_polygon}\,:\,\code{plane} \rightarrow [\code{point}] \rightarrow \code{bool}\\
  &\vdash_{def}\code{simple\_polygon}\ \alpha\ Ps \iff \\
  &\qquad 3 \leq \code{length}\ Ps\\
  &\qquad\wedge \code{head}\ ps = \code{last}\ Ps\\
  &\qquad\wedge (\forall P.\;\code{mem}\ P\ Ps \implies \code{on\_plane}\ P\ \alpha)\\
  &\qquad\wedge \code{pairwise}\ (\neq)\ (\code{butlast}\ Ps)\\
  &\qquad\wedge \neg(\exists P.\;\exists Q.\;\exists X.\; \code{mem}\;X\;Ps\wedge\code{mem}\;(P,Q)\;(\code{adjacent}\;Ps)\wedge\between{P}{X}{Q})\\
  &\qquad\wedge \code{pairwise}\ (\lambda(P,Q)\;(P',Q').\\
  &\qquad\qquad\neg(\exists X.\;\between{P}{X}{P'} \wedge \between{Q}{X}{Q'})\ (\code{adjacent}\;Ps)).
  \end{split}
\end{equation}

In this definition, we have introduced the polygon's plane as a parameter, but we are aware that this could be refactored. Any simple polygon uniquely determines the plane on which it lies, and so it might have been more elegant to hide the plane witness by an existential in the body of the definition. A function could then be defined to extract the unique witness from the list $Ps$ when $Ps$ is known to be a simple polygon. 

The definition would still be rather unwieldy. First, we have a ``magic'' number 3,\footnote{The number $4$ would do just as well!} which is needed to rule out the degenerate case of a point polygon $[P,P]$ which slips past Hilbert's constraints. But the real complexity is in the last three clauses which define the polygon as \emph{simple}. 

Given the unwieldiness of this formalisation, we must be wary of subtle mistakes. These could lead either to some figures being classed as simple polygons when they should not be, or some figures not being classed as simple polygons when they should. The first sort of error must show up in the verification. The second sort of error is more insidious, since it will only cause our verification to become all too easy. In the worst case, the definition will be unsatisfiable and all theorems of simple polygons will become trivial. This might happen if, say, we had removed the use of \code{butlast} above.

Of particular concern is the behaviour of the function $\code{pairwise}$. One might think that $\code{pairwise}\ R$ would check whether the relation $R$ holds across all pairs of elements drawn from its argument list, which would be the case if $\code{pairwise}$ were equivalent to $\code{all} \circ \code{fmap2}\ R$ for the usual list monad. If this were the case, the definition would be unsatisfiable, since it is always possible to draw some pair $(P,P)$ from a non-empty list, and this pair cannot satisfy $(\neq)$. But in fact, \code{pairwise} only checks half the pairs, and so can be satisfied even when the supplied relation is irreflexive (such as $(\neq)$ above). It even holds for anti-symmetric relations. Consider $\code{pairwise}\ (<)\ [1,2,3,4]$:
\begin{align*}
  \vdash\code{pairwise}\ (<)\ [1,2,3,4] &= 1 < 2 < 3 < 4 \\
  &\quad\wedge\; 2 < 3 < 4 \\
  &\quad\wedge\; 3 < 4\\
  & = \top.
\end{align*}

For now, inspection is the best way to inspire confidence in the definition, but there are two other small reassurances. Firstly, the definition does not appear until the final hurdles of our formal verification. Our verification of Veblen's lemma, for instance, makes no reference to simple polygons. Thus, there is plenty of verified theory which can be understood independently of the above definition. Secondly, we have the following simple sanity check, verifying that a triangle is a simple polygon.
\begin{equation*}
  \begin{split}
    \vdash&\Triangle{a}{A}{B}{C}\\
    &\code{on\_plane}\ A\ \alpha\wedge\code{on\_plane}\ B\ \alpha\wedge\code{on\_plane}\ C\ \alpha\\
    &\implies \code{simple\_polygon}\ \alpha\ [A,B,C,A].
  \end{split}
\end{equation*}

Note that for Chapters~\ref{chapter:JordanVerification1} and~\ref{chapter:JordanVerification2}, we shall elide all terms involving $\code{on\_plane}$, and thus present this verification as if from planar rather than spatial axioms. This is merely for clarity. Had we been working from planar axioms, all of these terms would be absent, since they serve only to relativise formulas to a single plane (see \S\ref{sec:PlanarProofs} for some discussion).

\subsection{Goal Theorems}
In \S\ref{sec:segConnectEquivalence}, we said that we would understand the regions defined by the Polygonal Jordan Curve Theorem as equivalence classes under polygonal path-connectedness. If we were to follow the style of Chapter~\ref{chapter:HalfPlanes}, we would quotient an appropriate data-type under this relation and regions would be abstract.
 
We shall not do this, however. We are not planning as of now to build on the Polygonal Jordan Curve Theorem, and so the abstraction can wait. We hope, though, that some of the example verifications in the next two chapters show that there \emph{was} a pay-off when talking abstractly of half-planes, while there was no need to talk abstractly of rays. This gives us some perspective on both Hilbert and Veblen's remarks that the Polygonal Jordan Curve Theorem is principally founded on the theory of half-planes.

Abstractly then, the Polygonal Jordan Curve Theorem tells us that a simple polygon separates the plane into exactly two regions, but we will talk concretely in terms of the underlying representation and polygonal path-connectedness. We say firstly that there are at least two points in the plane and not on the polygon which are not polygonal path-connected. Secondly, we say that of any three points in the plane and not on the polygon, at least two of them can be polygonal path-connected.

There is a final claim: one of the regions is unbounded. This theorem does not appear in Veblen's thesis, and we have left its verification to future work. Hilbert formulates the claim by saying that exactly one of the regions contains straight lines. Since we are avoiding talk of regions directly, we shall generalise and formulate it as follows:
\begin{itemize}
\item there exists a line in the plane of a polygonal path which does not intersect the polygonal path;
\item given two lines $a$ and $b$ in the plane of a polygonal path which does not intersect the polygonal path, all points of $a$ are polygonal path-connected to all points of $b$ with respect to the polygonal path.
\end{itemize}

In the first part, the rough idea is that there is at least one region containing a straight line, while in the second, that there is at most one region containing a straight line. The formulation needs some discussion. 

We have dropped the unnecessary condition that the polygonal path is a simple polygon from both parts. We have also dropped any mention of polygonal path-connectedness from the first part. A condition of polygonal path connectedness appears in the second part, and can be obtained for the first part simply by setting $a$ and $b$ to the first part's witness.

With the breakdown considered, we turn to the formalisation of all four clauses. Again, we must pay careful attention to the details. Our later verification demonstrates that the first two formalisations yield verified theorems, but we have left the matter of whether they are \emph{too easily provable} to inspection.
\begin{equation}\label{eq:jordanFormal1}
  \begin{split}
\vdash &\code{simple\_polygon}\ \alpha\ Ps\\
       &\implies \exists P.\;\exists Q.\; \code{on\_plane}\ P\ \alpha \wedge \code{on\_plane}\ Q\ \alpha\\
       &\qquad\wedge \neg\code{on\_polypath}\ Ps\ P \wedge \neg\code{on\_polypath}\ Ps\ Q\\
       &\qquad\wedge \neg\code{polypath\_connected}\ \alpha\ (\code{on\_polypath}\ Ps)\ P\ Q
  \end{split}
\end{equation}

\begin{equation}\label{eq:jordanFormal2}
  \begin{split}
\vdash &\code{simple\_polygon}\ \alpha\ Ps\\
       &\wedge \code{on\_plane}\ P\ \alpha \wedge \code{on\_plane}\ Q\ \alpha \wedge \code{on\_plane}\ R\ \alpha\\
       &\wedge \neg\code{on\_polypath}\ Ps\ P\wedge \neg\code{on\_polypath}\ Ps\ Q\wedge \neg\code{on\_polypath}\ Ps\ R\\
       &\implies \code{polypath\_connected}\ \alpha\ (\code{on\_polypath}\ Ps)\ P\ Q\\
       &\qquad\quad\vee \code{polypath\_connected}\ \alpha\ (\code{on\_polypath}\ Ps)\ P\ R\\
       &\qquad\quad\vee \code{polypath\_connected}\ \alpha\ (\code{on\_polypath}\ Ps)\ Q\ R
     \end{split}
\end{equation}

\begin{equation}\label{eq:jordanFormal3}
  \begin{split}
    &\vdash
    (\forall P.\; \code{on\_polypath}\ Ps\ P \implies \code{on\_plane}\ P\ \alpha)
    \\&\implies\exists a.\; \forall P.\; \code{on\_line}\ P\ a \implies \code{on\_plane}\ P\ \alpha\wedge\neg\code{on\_polypath}\ Ps\ P.
    \end{split}
\end{equation}

\begin{equation}\label{eq:jordanFormal4}
  \begin{split}
    &(\forall P.\;\code{on\_line}\ P\ a \vee \code{on\_line}\ P\ b\implies \neg\code{on\_polypath}\ Ps\ P \wedge \code{on\_plane}\ P\ \alpha)\\
    &\qquad\wedge\code{on\_plane}\ A\ \alpha \wedge\code{on\_plane}\ B\ \alpha\\
    &\qquad\wedge\code{on\_line}\ A\ a \wedge\code{on\_line}\ B\ b\\
    &\qquad\implies \code{polypath\_connected}\ \alpha\ (\code{on\_polypath}\ Ps)\ A\ B.
  \end{split}
\end{equation}

Here, Theorems~\ref{eq:jordanFormal1} and~\ref{eq:jordanFormal2} formalise the fact that there are respectively at least two and at most two polygonal path-connected regions, while formulas~\ref{eq:jordanFormal3} and~\ref{eq:jordanFormal4} formalise the fact that exactly one region is unbounded.

We would like to draw the reader's attention to the side-condition in the conclusion of Theorem~\ref{eq:jordanFormal1}. We must assert that $P$ and $Q$ do not lie on the polygon. \emph{Any} two points not satisfying this condition are such that they cannot be polygonal path-connected. Without the condition, the theorem is trivial.

\section{Conclusion}
In this brief chapter, we have presented the verification goals for the following two chapters, namely Theorems~\ref{eq:jordanFormal1} and~\ref{eq:jordanFormal2}. For the formalisations of these two theorems, we have chosen to identify polygons and polygonal paths with their vertex lists. This means that the definitions of simple polygons and the formalisations of the two theorems rely on a number of functions and theorems about lists.

The full formalisation can be presented in just a few pages of higher-order logic, and should be studied carefully to ensure that our verifications correspond to a proof of the Polygonal Jordan Curve Theorem. Otherwise, the suite of formal definitions is self-contained. Any new definitions and theorems we introduce from here on are just the scaffolding needed to support the verification.

%%% Local Variables: 
%%% mode: latex
%%% TeX-master: "../thesis"
%%% End: 

\chapter{Verifying the Polygonal JCT: Part I}\label{chapter:JordanVerification1}
We now come to the first of our two main contributions. We must verify Theorem~\ref{eq:jordanFormal1} from Hilbert's axioms of ordered geometry. We assume a simple polygon, and must find two points in the plane with the following property: given an arbitrary polygonal path connecting the two points, we can find another point at which the path intersects the simple polygon. The overall idea of this proof is very similar to Veblen's 1904 proof~\cite{Veblenphd} which we described in detail in \S\ref{sec:VeblenProof}. We give the correct version of this proof now.

\section{Sketch Proof}\label{sec:ParityProofInformal}
Consider the polygon $Ps$ shown in Figure~\ref{fig:rayCast1}. We pick an arbitrary point $O$ between $P_1$ and $P_2$  and then cast an arbitrary ray $h$ from $O$ to a segment of the polygon other than $P_1P_2$. Of all the intersections that $h$ makes with the polygon, we pick the one closest to $O$ and label it $H$. We then pick an arbitrary point $A$ between $O$ and $H$. For this point, we have that the segment $AX$ does not intersect the polygon $Ps$. Finally, we consider the ray emanating from $O$ in the other direction. Applying the same reasoning as above, we find a point $B$ such that $BX$ does not intersect $Ps$. We end up with a segment $AB$ which intersects the polygon $Ps$ exactly once between $P_1$ and $P_2$, namely at the point $O$.

\begin{figure}
\centering\includegraphics{jordanVerification1/rayCast1}
\caption{The witnesses ($A$ and $B$) for Theorem~\ref{eq:jordanFormal1}}
\label{fig:rayCast1}
\end{figure}

\newcommand{\insideoutsideclaim}{every time the edge of a polygon crosses an edge of a triangle, it changes from being inside to outside the triangle and \emph{vice versa}}

Now consider any polygonal path which connects $A$ and $B$. Together with the segment $AB$, this yields another polygon $Qs$ (possibly non-simple) that intersects $Ps$ at least once at the segment $P_1P_2$. We now proceed by considering the exact same sequence of triangles that appear in Veblen's proof. However, the observation we shall carry through the argument is that the closed polygon $Qs$ must cross the edges of any triangle an even number of times. This should be intuitively obvious. Indeed, \insideoutsideclaim. The total crossings must therefore be even in number, since we end in the same region we started.

\begin{figure}
\subfigure[Step 1: Four crossings]{\includegraphics[scale=0.6]{jordanVerification1/ParityProof1.pdf}\label{fig:ParityProof1}}
\subfigure[Step 2: Six crossings]{\includegraphics[scale=0.6]{jordanVerification1/ParityProof2.pdf}\label{fig:ParityProof2}}
\caption{Parity proof}
\label{fig:ParityProof}
\end{figure}

In Figures~\ref{fig:ParityProof} and~\ref{fig:ParityProofCont}, we illustrate a run of this parity argument through the triangles $P_1P_2P_3$, $P_1P_3P_4$, $P_1P_5P_6$, $P_1P_7P_8$ (the steps for the triangles $P_1P_4P_5$, $P_1P_6P_7$ have been omitted for clarity). At the start of the proof, we assume that the polygon $Qs$ crosses the edge $P_1P_2$ exactly once and at the point $O$. Note however, that for the purposes of the argument, we only need the more general fact that it crosses an odd number of times. 

Now, if $Qs$ intersects the edge $P_2P_3$, we are done. Thus, we can assume that it does not cross this edge. In that case, the polygon $Qs$ must cross the edge $P_1P_3$, also an odd number of times, to ensure that the total crossings are even. Indeed, there are 3 such crossings shown in Figure~\ref{fig:ParityProof1}. Hence, we can continue with the triangle $P_1P_3P_4$. 

We then have that the polygon $Qs$ must cross the triangle $P_1P_3P_4$ an even number of times. We know that it crosses $P_1P_3$ an odd number of times, and we can assume that it does not cross $P_3P_4$ (otherwise, we are done). Hence, it must cross $P_1P_4$ an odd number of times, in order that the total be even. Indeed, there are another three crossings shown in Figure~\ref{fig:ParityProof2}. Again, we continue with the next triangle. Eventually, we shall find a point of intersection with the polygon, shown as point $Y$ in Figure~\ref{fig:ParityProofCont}.

\begin{figure}
\subfigure[Step 4: Four crossings]{\includegraphics[scale=0.6]{jordanVerification1/ParityProof4.pdf}}
\subfigure[Step 6: Two crossings]{\includegraphics[scale=0.6]{jordanVerification1/ParityProof6.pdf}}
\caption{Parity proof (continued)}
\label{fig:ParityProofCont}
\end{figure}

This is a deceptively simple proof. Most of the work involved hinges on the informal notion of ``crossing''. In the next section, we shall show how this notion is formulated.

% \begin{enumerate}
% \item A polygon which does not intersect a triangles' vertex must cross that triangle an even number of times.
% \item The number of crossings of a polygon with the triangles $ABC$ and $ABD$ at the edge $AB$ is the same.
% \end{enumerate}

\section{Formulation: Crossings}
We hope that our use of ``crossing'' in the above is intuitively clear. The basic idea is that a polygonal path crosses a segment $AB$ when it intersects $AB$ and moves from one side to the other. While intuitive, we found the idea resisted a nice formulation.

\subsection{Context}
In our formulation, a polygonal path is a vertex list, and from this vertex list it is trivial to recover an edge list using the function \code{adjacent}. Our plan then is to use this edge list to compute the number of times the path crosses the segment $AB$ by reducing it to the problem of computing the number of times a single edge of the path crosses $AB$. We can then define the crossings of the full polygonal path by summing the crossings at each of its edges. 

There is an annoying problem. Suppose we have an edge $P_iP_{i+1}$ of a polygonal path, and a triangle $ABC$, and suppose we are interested in whether $P_iP_{i+1}$ crosses the triangle at $AB$. If there is a point on $AB$ which is strictly between $P_i$ and $P_{i+1}$, then we know there is a crossing at $P_iP_{i+1}$. But if one of the endpoints $P_{i}$ or $P_{i+1}$ are points on $AB$, then it is not the segment $P_iP_{i+1}$ which crosses the triangle, but a potentially larger polygonal path $P_{i-m}\ldots P_{i-1}P_iP_{i+1}\ldots...P_{i+p}$ with $m \geq 0, p > 0$. 

\begin{figure}
\centering\includegraphics[scale=0.85]{jordanVerification1/ContextChange}
\caption{Assignment of context on a segment}
\label{fig:ContextChanges}
\end{figure}

We can preserve the idea that the presence or absence of a crossing is nevertheless defined for each edge of a polygonal path by introducing a \emph{context} variable $\Gamma\;:\;\code{bool}$, and assign a value of this variable to each edge $P_iP_{i+1}$ of the polygonal path. The value will tell us on which side of $AB$ the endpoint $P_{i+1}$ lies. In the peculiar case that $P_{i+1}$ lies on $AB$ itself, we can just propagate the preceding context.

In Figure~\ref{fig:ContextChanges}, we show a context value assigned in this way to the edges of $P_1\ldots P_{10}$. A value of $\top$ indicates the side which is the top half of the diagram, while $\bot$ indicates the bottom half. There are two places where the context switches truth value, which indicates that the polygonal path crosses the segment $AB$ twice.

\subsection{Combined Context for Triangles}
\begin{figure}
\centering\includegraphics{jordanVerification1/ContextChangeTriangle}
\caption{Assignment of context in a triangle}
\label{fig:ContextChangesTriangle}
\end{figure}

We will be counting crossings on the edges of a triangle, which will require three context variables, one for each edge. Also, the assignment of $\top$ and $\bot$ to the sides of each edge of the triangle cannot be arbitrary as it was in Figure~\ref{fig:ContextChanges}. The problem with the triangle case is that we want to reason about the total crossings on all three sides and thus consider the way the variables interplay.

For each edge, we shall therefore declare the $\top$ side for the corresponding context to be the side (or half-plane) containing the triangle's interior. We will then simply combine the contexts by taking their conjunction. One way to think about this is that we are using a single context variable which tracks whether a segment ends inside or outside the triangle.

In Figure~\ref{fig:ContextChangesTriangle}, we show the assignment of the combined context value to each vertex of a polygon intersecting a triangle $ABC$. Here, if an edge $P_iP_{i+1}$ is such that $P_{i+1}$ lies outside the triangle, then the edge is assigned $\bot$. If $P_{i+1}$ lies inside the triangle, then the edge is assigned $\top$. The only complication is how to set the context when the point $P_{i+1}$ lands on an edge. For this, we think about the three component contexts for each edge of the triangle in terms of the ideas from the previous subsection.

Take the segment $P_5P_6$. The point $P_6$ lands on the edge $AB$. Notice that the point $P_5$ lies on the side of $AB$ which contains the triangle's interior, and so, at $P_5$, the component of the context for $AB$ is $\top$. Following the ideas from the previous section, it should stay at $\top$, since $P_6$ lies on $AB$.

Now for the contexts of the other two edges. The component of the context for the side $BC$ switches from $\bot$ to $\top$, since $P_6$ is on the side of $BC$ containing the triangle's interior while $P_5$ is on the opposite side. Finally, the component of the context for $AC$ stays at $\top$, since both $P_5$ and $P_6$ lie on the side of $AC$ containing the triangle's interior. Since the three component contexts for $P_6$ are all $\top$, so is the combined context.

To bring these ideas together, we will compute the number of crossings at each edge of the triangle as follows: first, we count a crossing for $P_iP_{i+1}$ and the edge $AB$ (or $AC$ or $BC$) every time there is a point on $AB$ which is strictly between $P_i$ and $P_{i+1}$. The only other crossings occur when $P_i$ lies on the segment $AB$. Here, we count a crossing in two circumstances:
\begin{itemize}
\item the context was last $\bot$ and $P_iP_{i+1}$ has a point in the interior of $\triangle ABC$ (and thus moves from outside to inside);
\item the context was last $\top$ and $P_iP_{i+1}$ has a point in the exterior of $\triangle ABC$ (and thus moves from inside to outside).
\end{itemize}

Thus, in Figure~\ref{fig:ContextChangesTriangle}, there is one crossing on $AC$, two crossings on $BC$, and one crossing on $AB.$

Now that we are always counting crossings at an edge relative to a triangle, it might appear that we have rendered our notion too specific. We will still need to be able to count crossings at an arbitrary segment $AB$ without mentioning triangles. To facilitate this, we show in \S\ref{sec:CrossingsWellDefined} that once we have fixed the vertices $AB$ in a triangle, our count of crossings at $AB$ is independent of the choice of the vertex $C$. In other words, the expression ``crossings of $P_iP_{i+1}$ at the edge $AB$'' is still well-defined, subject to a constraint on vertices detailed below. This fact, together with other key theorems (see Figure~\ref{fig:CrossingsSpecification} later), should fully clarify the intended semantics of a ``crossing.''

\subsection{Avoiding Vertices}\label{sec:EdgeCases}
% We have decided that a polygon crosses a triangle when it intersects the triangle and moves from inside the triangle to outside the triangle. We need to take care, though. Consider Figure~\ref{fig:crossingDifficult3}. Here, we probably want to say that there are two crossings in total, but it is not clear how to count the crossings at the three sides. Clearly, there is a crossing along the side $AB$, but where is the second crossing? We cannot say that it occurs between $A$ and $C$, because then the crossing vanishes for the triangle $AB'C$ which shares this side. Similarly, we cannot say that it occurs between $B$ and $C$, because then it vanishes for the triangle $A'BC$.

\begin{figure}
\centering\includegraphics{jordanVerification1/CrossingVertex}
\caption{Context with vertex crossings}
\label{fig:CrossingVertex}
\end{figure}

If we want the count of crossings at a particular edge of a triangle to be invariant of the position of the third vertex, we have a problem. Consider the scenario in Figure~\ref{fig:CrossingVertex}(a). Here, we have drawn a polygon $P_1P_2P_3P_4P_5P_6P_7P_8$ intersecting a triangle $ABC$. We have assigned our context appropriately to each segment, and concluded, quite reasonably, that the polygon does not cross $ABC$. 

However, when we assign context values for the triangle $BCD$ in Figure~\ref{fig:CrossingVertex}(b), we find that there suddenly appears a crossing on the shared edge $BC$ at the point $P_4$. In other words, the number of crossings at the segment $BC$ is not well-defined on our scheme.

 % is true precisely when the last segment not on the side $AB$ emerged into the \emph{interior}  of the triangle $ABC$. When we come to compute the total number of crossings of the triangle $ABC$ from the individual segments of the polygon $Ps$, we thread this value through, updating it as necessary. In \S~\ref{sec:wasInsideThreading}, we will see some interesting consequences of this choice of formulation.


% We have said that a polygon only crosses a triangle at a point of intersection, and in Figure~\ref{fig:crossingDifficult1}, the points of intersection are precisely these crossing points. But things are not always so clear. For instance, the polygon will sometimes intersect the triangle but ``bounce off''. In these cases, we do not want to count the intersection as a crossing (Figure~\ref{fig:crossingDifficult2}). 

% One thing we glossed over in the sketch proof is the fact that we count both the \emph{total} crossings of a polygon with a triangle, and also the crossings at a particular edge. We need to consider both since we are actually applying two ideas:

% \begin{enumerate}
% \item A polygon crosses a triangle an even number of times.
% \item The number of crossings of a polygon with the triangles $ABC$ and $ABD$ at the side $AB$ is the same.\label{item:crossingChange2}
% \end{enumerate}

% \begin{figure}
% \centering\subfigure[One crossing on $AB$; one crossing on $AC$; two crossings on $BC$.]{\includegraphics{jordanVerification1/crossingDifficult1}\label{fig:crossingDifficult1}}
% \qquad\centering\subfigure[No crossings.]{\includegraphics{jordanVerification1/crossingDifficult4}\label{fig:crossingDifficult2}}
% \end{figure}

% Perhaps we could say that the crossing occurs at the vertex $C$, and distinguish this from a crossing at a side. It seems we would have to do this in the situation depicted in \ref{fig:crossingDifficult4}. But then, how do we relate vertex crossings to side crossings and total crossings?

This difficulty can be eliminated quite simply by assuming that the polygonal path does not intersect any vertex of the triangle. We can get away with this because the vertices of the triangles we consider in our sketch proof are all vertices of the original polygon $Ps$. If at any time we found a point of intersection between the polygonal path and a vertex of one of the triangles, we will have found the desired point of intersection between the polygonal path and $Ps$.\label{sec:NoVertexAssumption}

% By ruling out all configurations where a polygon intersects a triangle's vertex, we eliminate the cases in Figures~\ref{fig:crossingDifficult3} and~\ref{fig:crossingDifficult4}. It makes life much easier.

\subsection{Formalisation}
% If we can first show how to compute the number of crossings between a triangle and an individual segment, then we should be able to compute the total number of crossings between the triangle and a polygon simply by summing the values for each of the polygon's sides. 

% There is a minor technical issue here, since whether an individual side of a polygon crosses a triangle in a particular segment is not generally a local property of that segment. Consider Figure~\ref{fig:crossingContext}. Here, we depict the same segment $P_iP_{i+1}$ intersecting the same triangle $ABC$. In each case, we want to know whether the endpoint $P_n$ is a crossing point of the triangle. This happens in cases $(a)$ and $(c)$ but not in $(b)$. Since the position of the triangle and $P_iP_{i+1}$ is the same in each case, it is clear that whether or not $P_n$ is a crossing depends on additional information. Here, we can look back through the history of previous vertices to see whether the last segment not on the side $AB$ emerged into the exterior of $\triangle ABC$ as in case $(a)$ and $(c)$, or whether it emerged into the interior as in case $(b)$.

% We decided to capture this dependency with a ``context'' variable which we shall denote by $\Gamma$. This is a \code{boolean} value, which is true precisely when the last segment not on the side $AB$ emerged into the \emph{interior}  of the triangle $ABC$. When we come to compute the total number of crossings of the triangle $ABC$ from the individual segments of the polygon $Ps$, we thread this value through, updating it as necessary. In \S~\ref{sec:wasInsideThreading}, we will see some interesting consequences of this choice of formulation.

% Let us turn to the task of computing the number of crossings from the list of vertices which define a polygon. Even with the edge cases and issues of context dealt with, there are still many possible configurations to consider concerning how an individual segment might or might not fall on a given side of a triangle (see Figure~\ref{fig:CrossingCases}). It took some effort and a lot of care to work these out on paper, since a mistake at this stage might not be spotted until well into the verification. We finally settled on a formulation. The details are not particularly important. Our verification itself shows that we have correctly cover all cases, and the key lemmas in \S\ref{sec:CrossingVerification} effectively give a clearer semantics to this notion of ``crossing.''

We now introduce the functions with which we shall calculate the number of crossings against an edge of a triangle. They are supplied for the curious reader to eliminate any potential ambiguity in our explanation of contexts, and to give an indication of the distance between the intuitive and the formal idea of a crossing. 

Our first function computes the crossings at an edge of a triangle based on a context. 
\begin{multline}\label{eq:oneCrossingDef}
% let crossing = new_definition
%   `crossing (A,B,C) was_inside P Q =
%     if between A P B /\ between A Q B then 0
%     else if (?R. between P R Q /\ between A R B) then 1
%     else if between A P B
%             /\ ((?R. between P R Q /\ in_triangle (A,B,C) R)
%                 <=> ~was_inside) then 1
%     else 0`;;
  \vdash_{def}\;\code{crossing}\ (A,B,C)\ \Gamma\ P_i\ P_{i+1}\\
  = 
  \begin{cases}
    0, \qquad\text{if }\between{A}{P_i}{B}\wedge\between{A}{P_{i+1}}{B}\\
    1, \qquad\text{else if }\exists R.\; \between{P_i}{R}{P_{i+1}}\wedge\between{A}{R}{B}\\
    1, \qquad\text{else if }\between{A}{P_i}{B}\\
    \qquad\qquad\wedge\; (\exists R.\; \between{P_i}{R}{P_{i+1}}\\
    \qquad\qquad\qquad\wedge\;\code{in\_triangle}\ (A,B,C)\ R \iff \neg\Gamma)\\
    0, \qquad\text{otherwise.}
  \end{cases}
\end{multline}

The first argument gives the three points defining the triangle we are interested in as a triple. We arbitrarily declare the first two components of this triple to be the edge of the triangle against which we want to compute crossings. The next argument is the context value $\Gamma$. The final two arguments are the endpoints of the polygonal path's edge against which we compute crossings.

Thus, to express the number of crossings for the edges $AC$ and $BC$, we just use the terms $\code{crossing}\ (A,C,B)$ and $\code{crossing}\ (B,A,C)$, and to express the total crossings of the segment $P_iP_{i+1}$ on the triangle, we use the term
\begin{multline*}
\code{crossing}\ (A,B,C)\ \Gamma\ P_i\ P_{i+1} + \code{crossing}\ (A,C,B)\ \Gamma\ P_i\ P_{i+1}\\ + \code{crossing}\ (B,A,C)\ \Gamma\ P_i\ P_{i+1}.
\end{multline*}

Our next function computes the context value for a segment $P_iP_{i+1}$ based on the last context. The arguments are the same, but here, the output does not depend on any particular ordering of the triple $(A,B,C)$.
\begin{equation*}
% `new_was_inside (A,B,C) was_inside P Q <=>
%   in_triangle (A,B,C) Q
%   \/ (on_triangle (A,B,C) Q /\ 
%         ((?R. between P R Q /\ in_triangle (A,B,C) R)
%             \/ on_triangle (A,B,C) P /\ was_inside))`
\begin{split}
\vdash_{def}\;\Gamma_{next}\ &(A,B,C)\ \Gamma\ P_i\ P_{i+1} \\
\iff &\code{in\_triangle}\ (A,B,C)\ P_{i+1}\\
& \vee \left(\begin{aligned}&\code{on\_triangle}\ (A,B,C)\ P_{i+1}\\
& \quad \wedge \left(\begin{aligned}&(\exists R.\; \between{P_i}{R}{P_{i+1}} \wedge \code{in\_triangle}\ (A,B,C)\ R)\\
& \quad\vee \code{on\_triangle}\ (A,B,C)\ P_i \wedge \Gamma)\end{aligned}\right)\end{aligned}\right).
\end{split}
\end{equation*}

Finally, we define the function which will compute the total number of crossings of an arbitrary polygonal path against the edge $AB$ for the triangle $ABC$. We do this recursively over the list of edges of the polygonal path, summing the values of $\code{polypath\_crossing}\ (A,B,C)$ for each segment and updating the context. Note that this function still requires an initial context $\Gamma$. We show where to get it from in \S\ref{sec:ContextInitialisation}.
\begin{equation*}
% let polypath_crossings = define
%   `polypath_crossings (A,B,C) was_inside [] = 0
%    /\ polypath_crossings (A,B,C) was_inside (CONS seg Ps)
%         = crossing (A,B,C) was_inside (FST seg) (SND seg)
%           + polypath_crossings (A,B,C)
%               (new_was_inside (A,B,C) was_inside (FST seg) (SND seg)) Ps`;;
\begin{aligned}
\vdash_{def}\;&\code{polypath\_crossings}\ (A,B,C)\ \Gamma\ [] = 0\\
\vdash_{def}\;&\code{polypath\_crossings}\ (A,B,C)\ \Gamma\ (\cons{(P_i,P_{i+1})}{segments})\\
 &\quad= \code{crossing}\ (A,B,C)\ \Gamma\ P_i\ P_{i+1}\\
 &\qquad+ \code{polypath\_crossings}\ (A,B,C)\ (\Gamma_{next}\ (A,B,C)\ \Gamma\ P_i\ P_{i+1})\ segments.
\end{aligned}
\end{equation*}

% \includegraphics{jordanVerification1/Crossing3}
% \includegraphics{jordanVerification1/Crossing4}
% \includegraphics{jordanVerification1/Crossing5}
% \includegraphics{jordanVerification1/Crossing6}
% \includegraphics{jordanVerification1/Crossing7}

\section{Triangle Interiors}\label{sec:TriangleInteriorDefinition}
% One way or another, our verification reduces to the base case in triangles. Our proof from \S\label{sec:JordanCurveFirstProof} defines the interior and exterior of a polygon by adding and subtracting triangles until the desired figure is obtained. Veblen's Proof considers triples of vertices going around the polygon, and applies the polygonal Jordan Curve Theorem to the resulting triangles (which appears as Corollary~2 of Theorem~27 in his 1904 thesis~\cite{Veblenphd}.) Our revision, on the other hand, does not appeal directly to the base case of the theorem, but instead counts ``crossings'' along a triangle's edge. As we shall see, the theoretical details of crossings ultimately exploit the base-case of the Polygonal Jordan Curve Theorem. It is therefore worth having a look at this base case, especially because it makes non-trivial use of our ordering tactic and the theory of half-planes.

% \section{Triangle Interiors}\label{sec:TriangleInteriorDefinition}
The above formulations and formalisation assume that we know how to express the interior, exterior and boundary of a triangle (respectively $\code{in\_triangle}$, $\code{on\_triangle}$ and $\code{out\_triangle}$). This we can do directly. Veblen, for instance, in his 1903 thesis~\cite{Veblenphd}, defined the interior of the triangle $ABC$ as the set of points $P$ such that there is a point $X$ on the segment $AB$ and a point $Y$ on $AC$ with $X$ between $Y$ and $Z$ (see Figure~\ref{fig:triangleDefs}). Here is another definition: the interior of $\triangle ABC$ is the set of all points on the same side of $AB$ as $C$, on the same side of $AC$ as $B$ and on the same side of $BC$ as $A$. In other words, the interior of a triangle is the intersection of three half-planes. The two formulations can be formalised as:
\begin{equation*}
\begin{split}
  &\code{in\_triangle\_veblen}\ (A,B,C)\ P \iff\\
  &\qquad\exists X\;\exists Y.\; \between{A}{X}{B} \wedge \between{A}{Y}{C} \wedge \between{X}{P}{Y}.
\end{split}
\end{equation*}

\begin{equation}\label{eq:inTriangleDef}
\begin{split}
  \vdash_{def}\;&\code{in\_triangle}\ (A,B,C)\ P \iff\\
    &\qquad\exists hp\;\exists hq\;\exists hr.\;\code{on\_line}\ A\ (\code{line\_of\_half\_plane}\ hp)\\
    &\qquad\qquad\wedge \code{on\_line}\ B\ (\code{line\_of\_half\_plane}\  hp)\\
    &\qquad\qquad\wedge \code{on\_line}\ A\ (\code{line\_of\_half\_plane}\  hq)\\
    &\qquad\qquad\wedge \code{on\_line}\ C\ (\code{line\_of\_half\_plane}\  hq)\\
    &\qquad\qquad\wedge \code{on\_line}\ B\ (\code{line\_of\_half\_plane}\  hr)\\
    &\qquad\qquad\wedge \code{on\_line}\ C\ (\code{line\_of\_half\_plane}\  hr)\\
    &\qquad\qquad\wedge \code{on\_half\_plane}\ hp\ C \wedge \code{on\_half\_plane}\ hq\ B \wedge \code{on\_half\_plane}\ hr\ A\\
    &\qquad\qquad\wedge \code{on\_half\_plane}\ hp\ P \wedge \code{on\_half\_plane}\ hq\ P \wedge \code{on\_half\_plane}\ hr\ P.
\end{split}
\end{equation}

\begin{figure}
\centering\includegraphics{jordanVerification1/triangleDefs}
\caption{Two definitions of a triangle's interior}
\label{fig:triangleDefs}
\end{figure}

Veblen's definition is significantly shorter, but we wanted to try leveraging our theory of half-planes as much as possible in our verification of the Polygonal Jordan Curve Theorem, and the second definition gives us direct information about these. Besides which, the existentials in Veblen's definition do not have unique witnesses, while ours do, making Veblen's definition more complicated to reason with. Consider that it is not immediately clear that his definition is symmetric up to permutations of $A$, $B$ and $C$. To prove this, we would need to figure out how to move from the arbitrary $X$ and $Y$ on $AB$ and $AC$ satisfying the given condition (and there are infinitely many possible choices), to another $X'$ and $Y'$ on another choice of segments. With the second definition, the symmetry is almost immediate. In fact, HOL~Light`s $\code{MESON}$ can easily prove the rewrites needed to normalise expressions of the form $\code{in\_triangle}\ (A,B,C)$. 
\begin{equation}\label{eq:triSyms}
  \begin{split}
    \vdash&\code{in\_triangle}\ (A,B,C)\ P \iff \code{on\_triangle}\ (A,C,B)\ P\\
&    \wedge \code{in\_triangle}\ (A,B,C)\ P \iff \code{on\_triangle}\ (B,A,C)\ P.
  \end{split}
\end{equation}

That said, with our definition, our early verifications about triangles always became bloated in the same clumsy way. When we started from a hypothesis that a point lies inside a triangle, we found ourselves having to extract all three witnessed half-planes in the definition and all twelve conjuncts they satisfy. In many cases, we found that the the main body of the proof was shorter than the bloated statement of the assumptions. It might be suggested that this ugliness could have been avoided had we persevered instead with Veblen's definition, and tried to avoid reasoning about half-planes. We have some circumstantial against this: if Veblen's definition is a more useful starting point for reasoning about the interiors of triangle's, we should expect that his formulation would appear as an immediate step in our proofs. But this never happened. There was no need obtain the two point witnesses given in Veblen's definition. Conversely, there were twelve places in our verifications where we had two points that could satisfy Veblen's definition, and where we appealed to a lemma which shows Veblen's definition implies our own. This suggests that our formulation is the more useful starting point.

% These two observations can be understood by again dividing geometric lemmas into those which introduce geometric entities and those which allow us to infer properties from a figure. As a theorem, Veblen's definition is weak for introducing entities, but useful for inferring properties of figures.

We make one final remark about our definition, which is important to keep in mind for some of the later verification. Whenever we have $\code{in\_triangle}\ (A,B,C)\ P$, we know that all triples chosen from $\{A,B,C,P\}$ are non-collinear. This means that explicit assumptions about non-collinearity can be suppressed in many of our verified theorems. It also means we can implement a discoverer \code{add\_in\_triangle} to derive these non-collinear triples automatically and make them available to the $\code{obviously}$ primitive. 

We end this subsection by considering the formalisation of a triangle's boundary and exterior. Since the boundary is just a polygonal path, it suffices to define:
\begin{equation}\label{eq:onTriangleDef}
\vdash_{def}\;\code{on\_triangle}\ (A,B,C)\ P \iff \code{on\_polypath}\ [A,B,C,A]\ P.
\end{equation}

In fact, it is useful to reuse \code{on\_polypath} in this way in other places. For instance, we can refer to the set of points of a segment $AB$ with $\code{on\_polypath}\ [A,B]$, and given a triangle $ABC$, we can write $\code{on\_polypath}\ [A,B,C]$ to refer to the points on just two sides of the boundary. Such formulas will prove convenient later on in our verification.

Finally, the exterior of a triangle can be defined simply as the set of points not on the triangle and not on the boundary. This definition classifies all points which are not on the plane as part of the exterior, but since we shall be relativising all of our theorems against a single plane, this does not matter.
\begin{multline*}
  \vdash_{def}\;\code{out\_triangle}\ (A,B,C)\ P\\
  \iff \neg\code{in\_triangle}\ (A,B,C)\ P\wedge\neg\code{on\_triangle}\ (A,B,C)\ P.
\end{multline*}

\section{Some Preliminary Theorems}
We recall the basic approach to synthetic axiomatic geometry as divided into two kinds of reasoning step: ones which introduce geometrical entities and ones which identify salient properties of the resulting figures. These properties allow us to introduce new geometrical entities, establish facts about them from which we can introduce new entities, and so on, until we have verified our goal theorem. 

We have six theorems for triangle interiors, two to introduce points and four to reason about such points with respect to triangle interiors. In this section, we shall look in detail at a verification of one of the introduction theorems, and then summarise the remaining ones.

\subsection{The Base Case}
Our main goal in this chapter is to show that a simple polygon divides the plane into at least two regions. In the simplest case, we take the polygon to be a \emph{triangle}, and we find that a triangle divides the plane much as a line divides the plane into half-planes. Specifically, given two points in different half-planes, we know there is a point between them which lies on the boundary, meaning we have an introduction theorem. There is an analogous introduction theorem for triangles, which we use frequently. It is even needed to prove our other introduction theorem~\eqref{eq:triCutHalfPlane} in \S\ref{sec:AdditionalTheorems}. In turn, this second introduction theorem is crucial to the verification of the well-definedness of crossings at a triangle's side (see \S\ref{sec:CrossingsWellDefined}):
\begin{multline}\label{eq:baseCase}
% |- on_plane A 'a /\
%      on_plane B 'a /\
%      on_plane C 'a /\
%      in_triangle (A,B,C) P /\
%      out_triangle (A,B,C) Q /\
%      on_plane Q 'a
%      ==> (?R. on_triangle (A,B,C) R /\ between P R Q)
\code{in\_triangle}\ (A,B,C)\ P \wedge \code{out\_triangle}\ (A,B,C)\ Q \\\implies \exists R.\; \code{on\_triangle}\ (A,B,C)\ R \wedge \between{P}{R}{Q}.
\end{multline}

We initially hoped the proof of Theorem~\ref{eq:baseCase} would be trivial. After all, an almost identical theorem holds for half-planes, and a triangle is defined as the intersection of three of these. Instead, we found ourselves needing a point introduction lemma.

\subsection{An ``Inner Pasch'' Lemma}
Initially, our only means to introduce points relative to triangles was by Pasch's axiom~\eqref{eq:g24}. But this axiom is often difficult to apply because it has a disjunctive conclusion. Luckily, there are easier versions to apply, namely the inner and outer variations, which we have derived as Theorems~\ref{eq:OuterPasch} and~\ref{eq:InnerPasch}. Our point introduction lemma can be thought of as a variation of Theorem~\ref{eq:InnerPasch}. It says that, given an interior point $P$ of a triangle $ABC$, and a point $Q$ outside the triangle on the ray $AB$, we can introduce the point $X$ at which the line $PQ$ intersects $BC$ (we could then use the Outer Pasch Axiom to find the point at which $PQ$ intersects $AC$). See Figure~\ref{fig:tricut1}.

\begin{figure}
\centering\includegraphics[scale=1.0]{jordanVerification1/tricut1.pdf}
\begin{equation}\label{eq:tricut1}
  \begin{split}
  % "!'a A B C P Q. on_plane A 'a /\ on_plane B 'a /\ on_plane C 'a
  %    /\ in_triangle (A,B,C) P
  %    /\ between A B Q
  %    ==> ?X. between P X Q /\ between B X C"
    &\code{in\_triangle}\ (A,B,C)\ P \wedge \between{A}{B}{Q} \\
    &\implies\exists X.\; \between{P}{X}{Q} \wedge \between{B}{X}{C}
  \end{split}
\end{equation}
\caption{``Inner Pasch'' for an interior point}
\label{fig:tricut1}
\end{figure}

The verification of this lemma illustrates some common patterns of reasoning with half-planes, and some of the pros and cons of our representation. The first half of the verification is shown in Figure~\ref{fig:tricut11}, where we obtain the three half planes defining the triangle. We must explicate the verbose constraints on these half-planes, before showing that the lines of each lie in the plane $\alpha$. These facts are needed in order to infer the defining property of each half-plane, namely that two points in the plane $\alpha$ are in the same half-plane precisely when their segment does not cross the line of the half-plane. The annoyance here is that we really do not care about such details, since all our assumptions should constrain the figure to the plane $\alpha$ anyway. If we transcribed these proofs to planar geometry, these details could be omitted, but for now, they show up as a weakness in our representation.

\begin{boxedfigure}[h]
\small
\begin{align*}
&\texttt{theorem }\code{on\_plane}\ A\ \alpha \wedge \code{on\_plane}\ B\ \alpha \wedge \code{on\_plane}\ C\ \alpha\\
&\qquad\qquad\code{in\_triangle}\ (A,B,C)\ P \wedge \between{A}{B}{Q}\\
&\qquad\qquad\implies\exists X.\; \between{P}{X}{Q} \wedge \between{B}{X}{C}\\
&\texttt{assume } \Triangle{a}{A}{B}{C}A \texttt{ by } \eqref{eq:inTriangleNcol}&0\\
&\texttt{assume } \code{on\_plane}\ A\ \alpha \wedge \code{on\_plane}\ B\ \alpha \wedge \code{on\_plane}\ C\ \alpha&1,2,3\\
  &\texttt{assume } \code{in\_triangle}\ (A,B,C)\ P\\
  &\texttt{so consider } hp, hq \text{ and } hr \texttt{ such that }\\
&\qquad \code{on\_line}\ A\ (\code{line\_of\_half\_plane}\ hp) \wedge \code{on\_line}\ B\ (\code{line\_of\_half\_plane}\ hp)& 4,5\\ 
% &\qquad \code{on\_line}\ A\ (\code{line\_of\_half\_plane}\ hp) & 6\\
% &\qquad \code{on\_line}\ C\ (\code{line\_of\_half\_plane}\ hp) & 7\\
% &\qquad \code{on\_line}\ B\ (\code{line\_of\_half\_plane}\ hp) & 8\\
% &\qquad \code{on\_line}\ C\ (\code{line\_of\_half\_plane}\ hp) & 9\\
&\qquad \code{on\_half\_plane}\ C\ hp\ldots \wedge \code{on\_half\_plane}\ P\ hr & 6,7\\
% &\qquad \code{on\_half\_plane}\ B\ hq & 11\\
% &\qquad \code{on\_half\_plane}\ A\ hr & 12\\
% &\qquad \code{on\_half\_plane}\ P\ hp & 13\\
% &\qquad \code{on\_half\_plane}\ P\ hq & 14\\
%&\qquad \code{on\_half\_plane}\ P\ hr 
&\ldots \texttt{ by } \eqref{eq:inTriangleDef} & 8..15\\
&\texttt{assume } \between{A}{B}{Q} & 16\\
&\texttt{obviously by\_neqs have } \forall X.\; \code{on\_half\_plane }\ hp\ X \implies \code{on\_plane}\ X\ \alpha\\ &\qquad\texttt{from 0,1,2,3,4,5,6 by } \eqref{eq:g16}, \eqref{eq:halfPlaneOnPlane}& 17\\
&\ldots & 18,19
\end{align*}
\caption{Proof of ``Inner Pasch'' for an interior point (part 1)}
\label{fig:tricut11}
\end{boxedfigure}

The rest of the proof is shown in Figure~\ref{fig:tricut12}. In contrast to the first part of the proof, the steps here are succinct, readable and geometrically interesting. With the necessary assumptions laid out, we see how easily the theory of half-planes has been leveraged via Theorems~\ref{eq:betOnHalfPlane1} and~\ref{eq:betOnHalfPlane2}. In contrast to the first part of the proof, this puts the use of half-planes in a much more positive light.
\begin{equation*}
%!hp P Q R. on_line P (line_of_half_plane hp)
%      /\ on_half_plane hp Q
%      /\ (between P Q R \/ between P R Q) ==> on_half_plane hp R
\tag{\ref{eq:betOnHalfPlane1}}
  \begin{split}
    \vdash\;&\code{on\_line}\ P\ (\code{line\_of\_half\_plane}\ hp) \wedge \code{on\_half\_plane}\ hp\ Q\\
    &\implies \between{P}{Q}{R} \vee \between{P}{R}{Q} \implies \code{on\_half\_plane}\ hp
  \end{split}
\end{equation*}
\begin{equation*}\tag{\ref{eq:betOnHalfPlane2}}
\begin{split}
  % "!P Q R hp. on_half_plane hp P /\ on_half_plane hp R /\ between P Q R 
  % ==> on_half_plane hp Q"
    \vdash\;&\code{on\_half\_plane}\ hp\ P \wedge \code{on\_half\_plane}\ hp\ R\\
    &\implies \between{P}{Q}{R} \implies \code{on\_half\_plane}\ hp\ Q
  \end{split}
\end{equation*}

\begin{boxedfigure}[h]
\small
\begin{align*}
&\texttt{obviously by\_ncols hence } \code{on\_plane}\ Q\ \alpha& 20\\
&\qquad\neg\code{on\_line}\ Q\ (\code{line\_of\_half\_plane}\ hr) & 21\\
&\qquad\neg\code{on\_half\_plane}\ hr\ Q \texttt{ from 0,1,2,12,13,14,16 by } \eqref{eq:g16}, \eqref{eq:g21}, \eqref{eq:onHalfPlaneNotBet}& 22\\
&\texttt{consider } X \texttt{ such that } \code{on\_line}\ X\ (\code{line\_of\_half\_plane}\ hr) \wedge \between{P}{X}{Q}\\
&\texttt{ from 15,19,20,21,22 by } \eqref{eq:onHalfPlaneNotBet} & 23 \\
&\texttt{have } \code{on\_line}\ Q\ (\code{line\_of\_half\_plane}\ hp)\texttt{ by } \eqref{eq:g12}, \eqref{eq:g21} \texttt{ from 4,5,16}\\
&\texttt{hence } \code{on\_half\_plane}\ hp\ X\texttt{ by } \eqref{eq:g21}, \eqref{eq:betOnHalfPlane1} \texttt{ from 7,23} & 24\\
&\texttt{hence } \neg\between{C}{B}{X} \texttt{ from 5,6,17 by } \eqref{eq:onHalfPlaneNotBet} & 25\\
&\texttt{hence } \code{on\_half\_plane}\ hq\ Q \texttt{ from 8,10,16 by } \eqref{eq:betOnHalfPlane1}\\
&\texttt{hence } \code{on\_half\_plane}\ hq\ X \texttt{ from 11,23 by } \eqref{eq:betOnHalfPlane2}\\
&\texttt{hence } \neg\between{B}{C}{X} \wedge B \neq X \wedge C \neq X\texttt{ from 5,9,10,18,24 by } \eqref{eq:onHalfPlaneNotBet}, \eqref{eq:halfPlaneNotOnLine}\\
&\texttt{obviously by\_neqs qed from 0,12,13,23,25 by } \eqref{eq:four}
\end{align*}
\caption{Proof of ``Inner Pasch'' for an interior point (part 2)}
\label{fig:tricut12}
\end{boxedfigure}

%\begin{figure}
% \caption{Betweenness and half-planes}
% \label{fig:BetOnHalfPlaneTheorems}
% \end{figure}

The basic strategy of the verification is to note that because $A$ and $Q$ lie on opposite sides of $BC$, so too must $P$ and $Q$. Thus, we can find a point $X$ between $P$ and $Q$ which is on the line $BC$. We just need to show that this point $X$ lies more specifically between $B$ and $C$.

To do this, we note that $P$ and $X$ lie on a ray emerging from the point $Q$ on the line $AB$, and so they must be on the same side of this line. Thus $P$, $C$ and $X$ must all lie on the same side of $AB$ which means that the point $B$ cannot possibly lie between any of them. Similar considerations apply if we look at the line $AC$. We can thus conclude that $X$ can only lie between $B$ and $C$.

The verification captures the \emph{structure} of this line of argument almost exactly. However, the terms are almost completely different. To begin with, we do not introduce anonymous rays. This would only add extra $\code{consider}$ steps, which is unnecessary when we can talk directly in terms of betweenness. We also avoid talking in terms of sides of a line by talking instead in terms of half-planes. We effectively have the translation

\label{sec:HalfPlaneTranslations}
\begin{tabular}{ccc}
  $\code{on\_half\_plane}\ hp$ & becomes & on the same side of $AB$ as $C$ and $P$;\\
  $\code{on\_half\_plane}\ hq$ & becomes & on the same side of $AC$ as $B$ and $P$;\\
  $\code{on\_half\_plane}\ hr$ & becomes & on the same side of $BC$ as $A$ and $P$;\\
  $\code{line\_of\_half\_plane}\ hp$ & becomes & the line $AB$; \\
  $\code{line\_of\_half\_plane}\ hq$ & becomes & the line $AC$; \\ 
  $\code{line\_of\_half\_plane}\ hr$ & becomes & the line $BC$. \\
\end{tabular}\linebreak

With these translations in mind, we hope the reader is convinced that the informal argument and a model synthetic proof can be recovered systematically from the verification. We can try to excuse the translation by drawing an analogy between synthetic proofs and their accompanying diagrams. The diagram is strictly unnecessary, but can  easily be recovered by carefully following the prose, and it is often helpful to reconstruct it. Similarly, our informal argument can be easily recovered from our formal verification, substituting intuitive phrases such as ``a ray emerging from the line'' so that it is easier to follow, even if such phrases do not point to interesting abstractions that would help the theorem prover.

\subsection{From ``Inner Pasch'' to the Base Case}\label{sec:JordanBaseCase1}
We can now give an informal proof of Theorem~\ref{eq:baseCase}. In Figure~\ref{fig:BaseCasePasch}, we suppose that $P$ is inside a triangle $ABC$ and $Q$ is outside. Then $P$ is on the same side of one of the triangle's edges and opposite vertex, while $Q$ is not. Let us suppose, without loss of generality, that $P$ is on the same side of $AB$ as $C$ while $Q$ is not. Then $PQ$ must intersect the line $AB$. If $PQ$ intersects the \emph{segment} $AB$, we have found the required point on the triangle's boundary. Otherwise, there is a point $R$ on the segment $PQ$ which also lies on either the ray emanating from $B$ in the direction $\overrightarrow{AB}$ or on the ray emanating from $A$ in the direction $\overrightarrow{BA}$. By applying \eqref{eq:tricut1} to each case, we can then find a point on the side $BC$ or the side $AC$ respectively, and we are done.

\begin{figure}
\centering\includegraphics{jordanVerification1/baseCase}
\caption{``Inner Pasch'' to the base case}
\label{fig:BaseCasePasch}
\end{figure}

We have made a without-loss-of-generality assumption in this argument, namely in our choice of $AB$ and the point $C$. As Harrison has shown~\cite{HarrisonWLOG}, such assumptions can often be handled elegantly using without-loss-of-generality tactics, particularly in geometry. However, these tactics typically exploit a \emph{Kleinian View} of geometry. This view of geometry can be described as ``subtractive''~\cite{SubtractiveKlein}: we start from a rich mathematical structure such as $\mathbb{R}^n$, and then ignore details by working only with invariants under a transformation group. Axiomatic geometry, on the other add, is additive, starting with only the most primitive machinery. As such, it is not clear how to build a theory of invariants which could capture our without-loss-of-generality cases.

Instead, we formalised the above argument as a lemma, and then wrote an \emph{ad hoc} procedural script to manually apply the symmetries. In ordered geometry, we only need to consider six symmetries and so the procedural boilerplate is hardly a bottle-neck compared to our use of MESON in declarative proofs, but this is still somewhat inelegant compared to doing proper without-loss-of-generality reasoning.

\subsection{Additional Theorems}\label{sec:AdditionalTheorems}
\begin{figure}
\centering
\includegraphics{jordanVerification1/triCutHalfPlane}
\caption{Another point introduction theorem}
\label{fig:triCutHalfPlane}
\end{figure}

We have one more theorem to introduce points. Here, we suppose that we have a point $P$ on the edge $AB$ of a triangle $ABC$ and a point $Q$ outside the triangle but on the same side of $AB$ as $C$. In this case, the segment $PQ$ must intersect the polygonal path $[A,B,C]$ at a point $X$ (see Figure~\ref{fig:triCutHalfPlane}). The half-plane $hp$ in this theorem is used to signify the side of $AB$ on which the point $C$ lies.
\begin{equation}\label{eq:triCutHalfPlane}
\begin{split}
\vdash&\between{A}{P}{B}\\
&\wedge \code{on\_line}\ A\ (\code{line\_of\_half\_plane}\ hp) \wedge \code{on\_line}\ B\ (\code{line\_of\_half\_plane}\ hp)\\
&\wedge \code{on\_half\_plane}\ C\ hp \wedge \code{on\_half\_plane}\ Q\ hp\\
&\wedge \code{out\_triangle}\ (A,B,C)\ Q\implies \exists X.\; \between{P}{X}{Q} \wedge \code{on\_polypath}\ [A,B,C]\ X.
\end{split}
\end{equation}

\begin{figure}
\centering
\subfigure[Points between the sides of a triangle are interior \eqref{eq:inTriangle1}]{\includegraphics[scale=1.2]{jordanVerification1/inTriangle1}}
\qquad\subfigure[Points between an interior point and a side are interior \eqref{eq:inTriangle2}]{\includegraphics[scale=1.2]{jordanVerification1/inTriangle2}}
\caption{Triangle interior theorems}
\label{fig:inTriangleTheorems}
\end{figure}

The remaining four theorems assume we have a configuration of points in relation to a triangle, and conclude that one of the points is interior or exterior. These theorems are used routinely throughout the first half of our main verification, particularly when we come to counting how many times a polygonal path crosses the sides of a triangle (see \S\ref{sec:CrossingVerification}). Their proofs are similar to the one given in the previous section, and always reduce to reasoning about the interaction between rays and half planes.

We give diagrams and a short description for each theorem in Figures~\ref{fig:inTriangleTheorems} and~\ref{fig:outTriangleTheorems}. These theorems all have reasonably clear synthetic verifications, and together require 82 verification steps. Roughly two fifths of these steps are assisted by our incidence discoverer via the \code{obviously} and \code{clearly} primitives.

Note that Theorem~\ref{eq:inTriangle1} is one direction of the equivalence between our definition of triangles and Veblen's (see \S\ref{sec:TriangleInteriorDefinition}).

\begin{figure}
\centering\subfigure[A ray through an opposite side leaves the triangle \eqref{eq:outTriangle1}]{\includegraphics[scale=1.2]{jordanVerification1/outTriangle1}}
\qquad\subfigure[A ray from inside the triangle to a side emerges outside the triangle \eqref{eq:outTriangle2}]{\includegraphics[scale=1.2]{jordanVerification1/outTriangle2}}
\caption{Triangle exterior theorems}
\label{fig:outTriangleTheorems}
\end{figure}

\begin{equation}\label{eq:inTriangle1}
  % "!A B C X Y P.
  %    ~(?a. on_line A a /\ on_line B a /\ on_line C a)
  %    /\ between A X B /\ between A Y C /\ between X P Y
  %    ==> in_triangle (A,B,C) P"
  \begin{split}
    \vdash\;&\Triangle{a}{A}{B}{C}\\
    &\wedge \between{A}{X}{B} \wedge (\between{A}{Y}{C} \vee C = Y)\\
    &\implies \between{X}{P}{Y} \implies \code{in\_triangle}\ (A,B,C)\ P.
  \end{split}
\end{equation}

\begin{equation}\label{eq:inTriangle2}
  % "!A B C X P.
  %    ~(?a. on_line A a /\ on_line B a /\ on_line C a)
  %    /\ between A X B /\ between C P X
  %    ==> in_triangle (A,B,C) P"
  \begin{split}
    \vdash&\code{in\_triangle}\ (A,B,C)\ X \wedge \code{on\_triangle}\ (A,B,C)\ Y\\
    &\implies \between{X}{P}{Y} \implies \code{in\_triangle}\ (A,B,C)\ P.
  \end{split}
\end{equation}

\begin{equation}\label{eq:outTriangle1}
  % "!A B C X P.
  %    ~(?a. on_line A a /\ on_line B a /\ on_line C a)
  %    /\ between A X B /\ between C P X
  %    ==> in_triangle (A,B,C) P"
  \begin{split}
    \vdash&\Triangle{a}{A}{B}{C}\\
    &\wedge \between{A}{X}{B} \wedge \between{A}{Y}{C}\\
    &\implies \between{X}{Y}{P} \implies \code{out\_triangle}\ (A,B,C)\ P.
  \end{split}
\end{equation}

\begin{equation}\label{eq:outTriangle2}
  % "!A B C X P.
  %    ~(?a. on_line A a /\ on_line B a /\ on_line C a)
  %    /\ between A X B /\ between C P X
  %    ==> in_triangle (A,B,C) P"
  \begin{split}
    \vdash&\code{in\_triangle}\ (A,B,C) X \wedge \code{on\_triangle}\ (A,B,C)\ Y\\
    &\implies \between{X}{Y}{P} \implies \code{out\_triangle}\ (A,B,C)\ P.
  \end{split}
\end{equation}

\section{Key Theorems of Crossings}\label{sec:CrossingVerification}
The formal definition of crossings as the threading of a context variable through a sequence of conditionals takes us a \emph{long way} from the intuitive idea. The intuition only reappears in our key theorems governing the definition, and the distance between the intuition and the formalisation can be measured by the thousand or so lines of mostly declarative proof and the enormous number of case-splits we consider to bridge the gap.

\subsection{Numbers of Crossings}
First, a relatively simple matter: a single segment crosses a triangle at most twice. Our verification of this takes the form of a crisp declarative proof based on Bernays' supplement~\eqref{eq:SupplementI} that we discussed in \S\ref{sec:SupplementI}. We do not need any messy case-splits, only a short piece of procedural script to eliminate without-loss-of-generality assumptions. We end up with this:
\begin{equation*}
  \begin{split}
    \vdash&\Triangle{a}{A}{B}{C}\\
    &\implies \code{crossing}\ (A,B,C)\ \Gamma\ P_i\ P_{i+1} + \code{crossing}\ (A,C,B)\ \Gamma\ P_i\ P_{i+1}\\
    &\qquad\qquad+ \code{crossing}\ (B,C,A)\ \Gamma\ P_i\ P_{i+1} \leq 2.
  \end{split}
\end{equation*}

The only unpleasantness comes from unfolding the definition of $\code{crossing}$, which requires that we face the mess of case-splits from Definition~\ref{eq:oneCrossingDef}. For this, we use a \code{tactics} step and a tactic \code{unfold\_crossing\_tac} which unfolds the definition of $\code{crossing}$ and then sweeps through the goal term eliminating the cases. Again, this tactic does not modify any assumptions, and it is typically only applied at the very start of a verification. With the cases converted, the $\code{assume}$ steps allow us to make more meaningful assumptions, as in the verification extract in Figure~\ref{fig:UnfoldingCrossings}.

\begin{boxedfigure}
\small
\begin{align*}
&\code{theorem } \Triangle{a}{A}{B}{C} &\\
&\qquad\wedge \code{crossing}\ (A,B,C)\ X\ P_i\ P_{i+1} = 1&\\
&\qquad\wedge \code{crossing}\ (A,C,B)\ X\ P_i\ P_{i+1} = 1&\\
&\qquad\implies \code{crossing}\ (B,C,A)\ X\ P_i\ P_{i+1} = 0&\\
&\code{assume } \Triangle{a}{A}{B}{C}&\\
&\code{tactics } \code{unfold\_crossing\_tac}&\\
&\code{assume } \between{A}{P_i}{B} \vee \exists R.\; \between{P_i}{R}{P_{i+1}} \wedge \between{A}{R}{B}
\end{align*}
\caption{Unfolding crossings}
\label{fig:UnfoldingCrossings}
\end{boxedfigure}

Things get \emph{really} hairy for our next theorem, which clearly explains how the values of $\code{crossing}$ compare when evaluated for a single segment at the various sides of a triangle. We give an impression of the cases involved in Figure~\ref{fig:CrossingCases}.

\begin{figure}
\centering\includegraphics{jordanVerification1/CrossingCases}
\caption{Cases of crossings}
\label{fig:CrossingCases}
\end{figure}

\begin{multline*}
% `!A B C P Q.
%       on_plane A 'a
%       /\ on_plane B 'a
%       /\ on_plane C 'a
%       /\ on_plane P 'a
%       /\ on_plane Q 'a
%       /\ ~(?a. on_line A a /\ on_line B a /\ on_line C a)
%       /\ ~on_polypath [P; Q] A
%       /\ ~on_polypath [P; Q] B
%       /\ ~on_polypath [P; Q] C
%       /\ (~on_triangle (A,B,C) P ==> (in_triangle (A,B,C) P <=> was_inside))
%       ==> (crossing (A,B,C) was_inside P Q
%            + crossing (A,C,B) was_inside P Q
%            + crossing (B,C,A) was_inside P Q = 1
%            <=> (was_inside =
%                   ~new_was_inside (A,B,C) was_inside P Q))`  
  \begin{aligned}
    \vdash&\Triangle{a}{A}{B}{C}\\
    &\wedge \neg\code{on\_polypath}\ [P_i,P_{i+1}]\ A\wedge \neg\code{on\_polypath}\ [P_i,P_{i+1}]\ B\\&\wedge \neg\code{on\_polypath}\ [P_i,P_{i+1}]\ C\\
    &\wedge(\neg\code{on\_triangle}\ (A,B,C)\ P_i \implies (\code{in\_triangle}\ (A,B,C)\ P_i \iff \Gamma))\\
  \end{aligned}\\
    \implies\left(\begin{aligned}[]
        &\code{crossing}\ (A,B,C)\ \Gamma\ P_i\ P_{i+1} + \code{crossing}\ (A,C,B)\ \Gamma\ P_i\ P_{i+1}\\
          &\qquad+ \code{crossing}\ (B,C,A)\ \Gamma\ P_i\ P_{i+1} = 1 \\
          &\iff \Gamma = \neg \Gamma_{next}\ (A,B,C)\ \Gamma\ P_i\ P_{i+1}
        \end{aligned}\right)
\end{multline*}

The first hypothesis just requires that $ABC$ is a triangle. The second requires that the segment $P_iP_{i+1}$ does not intersect any of the vertices, as per our discussion in \S\ref{sec:NoVertexAssumption}. 

The rest of the theorem then clarifies both the idea behind a crossing and the idea behind the context variable $\Gamma$. The conclusion says that the sum of crossings at the three sides is 1 precisely when the context variable switches truth value. The formalisation almost transparently captures a claim made in the sketch proof: ``\insideoutsideclaim.''

There is one more thing we should say about the context $\Gamma$. The theorem hypothesises that when $P_i$ is not on the sides of a triangle then $\Gamma$ tracks whether the point is inside or outside. Since $P_i$ is intended to be a vertex of a polygonal path and $P_iP_{i+1}$ an edge, we want to make sure that this hypothesis on $\Gamma$ is preserved as it threads through the remaining edges. 

Because a vertex of the polygon $P_{i+1}$ is the successor of $P_i$, what we are saying here is that, just as $\Gamma$ tracks whether $P_i$ is inside or outside the triangle, so too must $\Gamma_{next}$ track whether $P_{i+1}$ is inside or outside. This matter is settled trivially from the definition using the simplifier.
\begin{multline*}
%`!A B C P Q. ~on_triangle (A,B,C) Q
%   ==> (in_triangle (A,B,C) Q <=> new_was_inside (A,B,C) was_inside P Q)`
\vdash\neg\code{on\_triangle}\ (A,B,C)\ P_{i+1}\\
\implies (\code{in\_triangle}\ (A,B,C)\ P_{i+1} \iff \Gamma_{next}\ (A,B,C)\ \Gamma\ P_i\ P_{i+1}).
\end{multline*}

\subsection{Overview of Some Verification}
Rather than go into all the details of the verification, we will give a typical extract of a specific case, showing how in these proofs we are still relying on our discovery algebra from Chapter~\ref{chapter:Automation} and our linear ordering tactic from Chapter~\ref{chapter:LinearOrder}. We also see how we leverage our lemmas for this section, and thus avoid having to deal directly with half-planes.

The case we consider says that if a segment $P_iP_{i+1}$ crosses a triangle $ABC$ exactly once between $P_i$ and $P_{i+1}$, and not at a vertex, then its endpoints are in different regions with respect to the triangle. 
\begin{multline}\label{eq:IH3}
% crossing (A,B,C) was\_inside P Q = 1
%          ==> crossing (A,C,B) was_inside P Q = 0
%          ==> crossing (B,C,A) was_inside P Q = 0
%          ==> on_plane A 'a /\ on_plane B 'a /\ on_plane C 'a
%              /\ on_plane P 'a /\ on_plane Q 'a
%              /\ ~(?a. on_line A a /\ on_line B a /\ on_line C a)
%              /\ ~on_polypath [P;Q] A /\ ~on_polypath [P;Q] B /\ ~on_polypath [P;Q] C
%              /\ ~on_triangle (A,B,C) Q
%              ==> (in_triangle (A,B,C) P \/ on_triangle (A,B,C) P /\ was_inside
%                   <=> ~(in_triangle (A,B,C) Q))
  \begin{aligned}
    &\Triangle{a}{A}{B}{C}\\
    &\wedge\between{P_i}{R}{P_{i+1}} \wedge \between{A}{R}{B} \\
    &\quad\wedge\code{crossing}\ (A,C,B)\ \Gamma\ P_i\ P_{i+1} = 0\wedge\code{crossing}\ (B,C,A)\ \Gamma\ P_i\ P_{i+1} = 0\\
    &\wedge\neg\code{on\_polypath}\ [P_i,P_{i+1}]\ A \wedge \neg\code{on\_polypath}\ [P_i,P_{i+1}]\ B \\
    &\quad\wedge \neg\code{on\_polypath}\ [P_i,P_{i+1}]\ C\\
    &\wedge\neg\code{on\_triangle}\ (A,B,C)\ P_i \wedge\neg\code{on\_triangle}\ (A,B,C)\ P_{i+1}
  \end{aligned}\\
  \implies\left(\code{in\_triangle}\ (A,B,C)\ P_i \iff \code{out\_triangle}\ (A,B,C)\ P_{i+1}\right).
\end{multline}

We divide the verification into the three cases shown in Figure~\ref{fig:IH3CaseSplit}. In case (a), we have assumed that $P_i$ is interior. It then follows immediately from Theorem~\ref{eq:outTriangle2} that $P_{i+1}$ is exterior. In case (b), we have assumed that $P_i$ is exterior and that $P_i$ and $P_{i+1}$ are in line with the vertex $C$. In this case, we just apply Theorem~\ref{eq:inTriangle1}. Finally, in case (c), we have assumed that $P_i$ is again exterior but that the line of $P_iP_{i+1}$ does not intersect $C$. Under these circumstances, we can apply Pasch's Axiom \eqref{eq:g24} to the triangle and the line of $P_iP_{i+1}$ using our discoverer \code{by\_pasch} and thus obtain a point $S$ either on $AC$ or $BC$. It then follows from Theorem~\ref{eq:inTriangle1} that $P_{i+1}$ is interior.

\begin{figure}
\centering\includegraphics{jordanVerification1/IH3CaseSplit}
\caption{Main case-split}
\label{fig:IH3CaseSplit}
\end{figure}

Actually, things are not \emph{quite} so simple for cases (b) and (c). In order to apply Theorem~\ref{eq:inTriangle1} in case (b), we first have to prove that $P_{i+1}$ is between $C$ and $R$. To do this, we want to apply our linear ordering tactic, but for this to work, the tactic will need some facts about the existing order relations among the points $P_i$, $P_{i+1}$, $R$ and $C$. These facts come from various places.

First off, the incidence discoverer tells us that $C \neq R$. Next, from \eqref{eq:inTriangle2} and the fact that $P_i$ is exterior, we conclude that $P_i$ does not lie between $C$ and $R$. Finally, since $P_iP_{i+1}$ does not intersect $C$, we know that all three points are distinct and that $C$ does not lie between $P_i$ and $P_{i+1}$. Each of these inferences corresponds to a single declarative step, and once in place, the linear reasoning tactic can be applied to the four points $C$, $P_i$, $P_{i+1}$ and $R$, where it is able to show that $P_{i+1}$ lies between $C$ and $R$. We finish by applying~\eqref{eq:inTriangle1} to show that $P_{i+1}$ is interior to the triangle.

Case (c) is more involved, but the most interesting part is probably that which establishes that neither $A$ nor $B$ lie on the line of $P_iP_{i+1}$. Here, we proceed by contradiction, once for $A$ and once for $B$. We can solve the goal in one step with the linear reasoning tactic, provided we again obtain the necessary information for it to do its work.

For instance, assuming that $A$, $P_i$ and $P_{i+1}$ lie on a line, the linear reasoning tactic will first infer that $A$, $B$, $P_i$, $P_{i+1}$ and $R$ are all collinear. If it knew further than $P_{i}$ does not lie on the segment $AB$, it would conclude that one of $A$ or $B$ lies on the segment $P_iP_{i+1}$, which \emph{we} know to be impossible. This suggests that we should seed the tactic with the following facts, with which it solves the goal by reasoning about the ordering of $A$, $B$, $P_i$, $P_{i+1}$ and $R$.
\begin{multline*}
A \neq P_i \wedge A \neq P_{i+1} \wedge B \neq P_i \wedge B \neq P_{i+1}\\ \wedge \neg\between{P_i}{A}{P_{i+1}} \wedge \neg\between{P_i}{B}{P_{i+1}} \wedge \neg\between{A}{P_i}{B}.
\end{multline*}

%This deals with the three cases from Figure~\ref{fig:IH3CaseSplit}, which cover a crossing strictly between $P_i$ $P_{i+1}$. When we include the case of a crossing at one of the endpoints, for which we make use of the context, we have a verification which runs to 70 steps. We are putting Mizar~Light through its paces, and for the most part, it copes very well with the complexity. In the places where the prover struggled, it was usually simple enough to apply the \code{using} combinator to inject pieces of procedural proof into the script. We shall say more on this in \S\ref{sec:InjectingProcedural}.

% \begin{multline}\label{eq:IH3}
% % crossing (A,B,C) was\_inside P Q = 1
% %          ==> crossing (A,C,B) was_inside P Q = 0
% %          ==> crossing (B,C,A) was_inside P Q = 0
% %          ==> on_plane A 'a /\ on_plane B 'a /\ on_plane C 'a
% %              /\ on_plane P 'a /\ on_plane Q 'a
% %              /\ ~(?a. on_line A a /\ on_line B a /\ on_line C a)
% %              /\ ~on_polypath [P;Q] A /\ ~on_polypath [P;Q] B /\ ~on_polypath [P;Q] C
% %              /\ ~on_triangle (A,B,C) Q
% %              ==> (in_triangle (A,B,C) P \/ on_triangle (A,B,C) P /\ was_inside
% %                   <=> ~(in_triangle (A,B,C) Q))
%   \begin{aligned}
%     \vdash&\Triangle{a}{A}{B}{C}\\
%     &\wedge\code{crossing}\ (A,B,C)\ \Gamma\ P_i\ P_{i+1} = 1\\
%     &\quad\wedge\code{crossing}\ (A,C,B)\ \Gamma\ P_i\ P_{i+1} = 0\wedge\code{crossing}\ (B,C,A)\ \Gamma\ P_i\ P_{i+1} = 0\\
%     &\wedge\neg\code{on\_polypath}\ [P_i,P_{i+1}]\ A \wedge \neg\code{on\_polypath}\ [P_i,P_{i+1}]\ B \\
%     &\quad\wedge \neg\code{on\_polypath}\ [P_i,P_{i+1}]\ C\\
%     &\wedge\neg\code{on\_triangle}\ (A,B,C)\ P_{i+1}
%   \end{aligned}\\
%   \implies\left(\begin{aligned}&\code{in\_triangle}\ (A,B,C)\ P_i \vee \code{on\_triangle}\ (A,B,C)\ P_i \wedge \Gamma\\
%     &\qquad\iff \code{out\_triangle}\ (A,B,C)\ P_{i+1}\end{aligned}\right).
% \end{multline}

%In doing this, we generally tried to respect the aims of declarative proof. For instance, at no point do we destructively modify the proof context, and we always insist that whenever our tactics apply an assumption, we include the assumption label so that a reader can at least track the dependencies.

%For instance, the automatic prover often struggles to apply complex conditionals with many hypotheses. For these situations, we add an initial \code{MATCH\_MP\_TAC} in a \code{using} clause. This leaves the default prover \code{MESON} with just the theorem's assumptions as goals. This tactic still keeps things largely declarative, since we name the theorem we are matching against and we do not modify any assumptions. 

%In other places, we found it helpful to introduce a function \code{mutual\_simp} which takes a list of theorems and simplifies each with respect to the others. This function is used to process the justifying theorems of a declarative step before they are handed to \code{MESON}. The need for such processing comes from the fact that a declarative proof is based on accumulating theorems in a proof context. These theorems are expected to influence one another without any destructive modification (consider, as a simple example, an equation inferred late in a verification which could rewrite all previous assumptions). It is often helpful to have the theorems modify one another as they are applied at a step, using a function such as \code{mutual\_simp}, so that they are more useful to \code{MESON}.

% Mizar~Light's prover struggles to apply large theorems such as our ray-casting theorem. Luckily, the actual tactic used to justify a step can be customised inline. For theorems such as this, where we have a conditional with many assumptions, we often use code of the form

% \begin{displaymath}
% \code{K (MATCH\_MP\_TAC } \eqref{eq:changeTriangle}) \code{ THEN } (\code{MESON\_TAC } \circ \code{mutual\_simp}
% \end{displaymath}

% This tactic tells us to first match the goal with the conclusion \eqref{eq:changeTriangle}. This leaves us to prove the assumptions of \eqref{eq:changeTriangle}, which can be done by mutually simplifying the justifying theorems and feeding the results to $\code{MESON\_TAC}$. This preparatory step using \code{mutual\_simp} is often needed. In forward declarative proofs, we expect facts that we add to the proof context to affect facts already there.

\subsection{Crossings are Well-defined}\label{sec:CrossingsWellDefined}
In our sketch proof from \S\ref{sec:ParityProofInformal}, we implicitly assume that when we have two triangles $ABC$ and $ABC'$, then the number of crossings made by a polygonal path against the shared edge $AB$ is always the same. This is not obvious from our formulation, because the number of crossings at $AB$ is dependent on a choice of triangle with edge $AB$. We need to show that this choice is arbitrary.

Now the definition of a crossing makes use of a triangle's interior, and different triangles sharing the edge $AB$ will have interiors which may be disjoint, may overlap, or may contain one another. We will need to verify that, nevertheless, the values of the function \code{crossing} are always consistent. In other words, we must show that the expression ``crossings at $AB$'' is well-defined, without reference to the vertex $C$.

\begin{figure}
\centering\includegraphics{jordanVerification1/CrossingWellDefined}
\caption{Triangles sharing an edge}
\label{fig:CrossingWellDefined}
\end{figure}

There is key case-split here, shown in Figure~\ref{fig:CrossingWellDefined}. If $C$ and $C'$ are on the same side of $AB$ as in case (a), then the triangle interiors will overlap. Here, as we cross the edge $AB$, we enter or leave the interiors of both triangles together, a fact we verify as Theorem~\ref{eq:crossChangeLemma1} in Figure~\ref{fig:crossChangeLemma1}. On the other hand, if $C$ and $C'$ are on opposite sides of $AB$ as in (b), then the interiors of the two triangles are disjoint. In this case, as we move from the interior of $ABC$ across the edge $AB$ to the exterior, we simultaneously move from the exterior of $ABC'$ to the interior, a fact we verify as Theorem~\ref{eq:crossChangeLemma2} in Figure~\ref{fig:crossChangeLemma2}.

\begin{figure}
\begin{multline}\label{eq:crossChangeLemma1}
  \begin{aligned}
    \vdash&\code{on\_line}\ A\ (\code{line\_of\_half\_plane}\ hp) \wedge \code{on\_line}\ B\ (\code{line\_of\_half\_plane}\ hp)\\
    &\wedge\code{on\_half\_plane}\ hp\ C \wedge \code{on\_half\_plane}\ hp\ C'\\
    &\wedge\between{A}{P_i}{B} \wedge \neg\between{A}{P_{i+1}}{B}
  \end{aligned}\\
  \implies\left(\begin{aligned} 
      &(\exists R.\; \between{P_i}{R}{P_{i+1}} \wedge \code{in\_triangle}\ (A,B,C)\ R)\\
      & \iff (\exists R.\; \between{P_i}{R}{P_{i+1}} \wedge \code{in\_triangle}\ (A,B,C')\ R)
  \end{aligned}\right).
\end{multline}
\caption{Moving across $AB$ when $C$ and $C'$ are on the same side.}
\label{fig:crossChangeLemma1}
\end{figure}

\begin{figure}
\begin{multline}\label{eq:crossChangeLemma2}
  \begin{aligned}
    \vdash&\Triangle{a}{A}{B}{C'}\\
    &\wedge \neg\code{on\_polypath}\ [P_i, P_{i+1}]\ A \wedge \neg\code{on\_polypath}\ [P_i, P_{i+1}]\ B\\
    &\wedge\code{on\_line}\ A\ (\code{line\_of\_half\_plane}\ hp) \wedge \code{on\_line}\ B\ (\code{line\_of\_half\_plane}\ hp)\\
    &\wedge\code{on\_half\_plane}\ hp\ C \wedge \neg\code{on\_half\_plane}\ hp\ C'\\
    &\wedge \between{A}{P_i}{B} \wedge \neg\between{A}{P_{i+1}}{B}
  \end{aligned}\\
  \implies\left(\begin{aligned}
      &(\exists R.\; \between{P_i}{R}{P_{i+1}} \wedge \code{in\_triangle}\ (A,B,C)\ R)\\
      &\iff\neg \exists R.\; \between{P_i}{R}{P_{i+1}} \wedge \code{in\_triangle}\ (A,B,C')\ R
  \end{aligned}\right).
\end{multline}
\caption{Moving across $AB$ when $C$ and $C'$ are on opposite sides.}
\label{fig:crossChangeLemma2}
\end{figure}

These two theorems are proven by reasoning about half-planes and, in both cases, applying Theorem~\ref{eq:triCutHalfPlane}. The assumptions on half-planes in Theorem~\ref{eq:crossChangeLemma1} require that the points $C$ and $C'$ lie on the same side of $AB$. In Theorem~\ref{eq:crossChangeLemma2}, they require that $C$ and $C'$ lie on opposite sides. Theorem~\ref{eq:crossChangeLemma2} needs some extra assumptions since the negations make for very weak claims. For instance, the fact that $C'$ does not lie on $hp$ might just mean that it lies on the line $AB$, so we have to add in a condition that the points $A$, $B$ and $C'$ are non-collinear.

The important assumption to note in both theorems is $\between{A}{P_i}{B}$. This reflects the fact that the case-split is only pertinent when we come to update and make use of the context variable $\Gamma$, which happens when the edge $P_iP_{i+1}$ has exactly one endpoint on the segment $AB$. So when the edge \emph{lands} on $AB$, the context variable must be correctly updated to say that we were last inside or outside the triangle. When it \emph{emerges} from $AB$, the context variable must be correctly utilised to say whether the edge $P_iP_{i+1}$ counts as a crossing. These matters can be formally clarified by the corollaries in Figures~\ref{fig:CrossChange1} and~\ref{fig:CrossChange2} (we do not reproduce the assumptions in full).

\begin{figure}
  \begin{gather*}
    \begin{split}
      \vdash\neg\between{A}{P_i}{B}&\wedge\between{A}{P_{i+1}}{B}\\
      &\implies \Gamma_{next}\ (A,B,C)\ \Gamma\ P_i\ P_{i+1} = \Gamma_{next}\ (A,B,C)\ \Gamma\ P_i\ P_{i+1}
    \end{split}\\
    \begin{split}
      \vdash&(\between{A}{P_i}{B}\implies \Gamma = \Gamma')\\
      &\implies\code{crossing}\ (A,B,C)\ \Gamma\ P_i\ P_{i+1} = \code{crossing}\ (A,B,C')\ \Gamma'\ P_i\ P_{i+1}
    \end{split}
  \end{gather*}
  \caption{Well-definedness theorems when $C$ and $C'$ are on the same side of $AB$}
  \label{fig:CrossChange1}
\end{figure}

\begin{figure}
  \begin{gather*}
    \begin{split}
      \vdash\neg\between{A}{P_i}{B}&\wedge\between{A}{P_{i+1}}{B}\\
      &\implies \Gamma_{next}\ (A,B,C)\ \Gamma\ P_i\ P_{i+1} = \neg\Gamma_{next}\ (A,B,C)\ \Gamma\ P_i\ P_{i+1}
    \end{split}\\
    \begin{split}
      \vdash&(\between{A}{P_i}{B}\implies \Gamma = \neg\Gamma')\\
      &\implies\code{crossing}\ (A,B,C)\ \Gamma\ P_i\ P_{i+1} = \code{crossing}\ (A,B,C')\ \Gamma'\ P_i\ P_{i+1}
    \end{split}
  \end{gather*}
  \caption{Well-definedness theorems when $C$ and $C'$ are on opposite sides of $AB$}
  \label{fig:CrossChange2}
\end{figure}

Thus, when $C$ and $C'$ are on the same side, we expect the context values to be the same when we hit the segment $AB$, and we expect them to compute the same crossing value when we leave $AB$. When $C$ and $C'$ are on opposite sides, we expect the context values to be opposite when we hit $AB$, and, as before, we expect them to compute the same crossing value when we leave the segment.

In any case, we know that the vertex $C$ in the expression $\code{crossing}\ (A,B,C)\ \Gamma\ P_i\ P_{i+1}$ can be varied about the sides of $AB$ (so long as we change $\Gamma$ appropriately), and so we can generalise our notion of crossing. Instead of saying that a polygonal path crosses the side of a  \emph{triangle} by moving from interior to exterior and \emph{vice versa}, we can say that a polygonal path crosses an arbitrary \emph{segment} precisely when it moves from one side of the segment to the other. In other words, we can abstract away the vertex $C$. The mechanics of this will become clear in our final proof in \S\ref{sec:InductionProof}. Before we get to that, we must consider how we initialise $\Gamma$.

% These cases are factored in when we consider the update and use of the context variable $\Gamma$. The update is pertinent when a segment $P_iP_{i+1}$ lands on the edge $AB$. The use of the context is pertinent when a segment $P'Q'$ leaves the edge $AB$. 

% More generally, we should show that the number of crossings of a polygon at a side $AB$ of triangle $ABC$ is independent of the point $C$. Because of this, we can abstract over all possible choices of $C$ with a universal quantifier and thus recover the notion of crossings on an arbitrary segment. 

%  we consider a triangle sequence where adjacent elements share an edge. Implicit in the sketch proof is that the number of crossings at an edge is well-defined. This is obvious intuitively, but not at all clear from our formalisation of crossings. This is the last gap between our formalisation and the intuitive idea of a crossing. 

% The problem is that our definition of a crossing of $AB$ by a segment $P_iP_{i+1}$ \eqref{eq:oneCrossingDef} is given with respect to a triangle, since it must make use of the 

%  we have formally defined  of crossing at the segment $AB$ in terms of the interior of a triangle $ABC$ and a context $\Gamma$. We must show that its value is well-defined as as we move both the vertex $C$ around the plane and thus consider a different triangle sharing the edge $AB$. We also need to consider the dependencies on the value of $\Gamma$.

% There are three cases to consider here, and they arise in the context of our sketch proof with little difficulty. In Figure~\ref{fig:ChangeTriangle}, we show parts of two intersecting polygons $Ps$ and $Qs$, and we consider three transitions from the triangle $ABC$ to the next triangle $ABC'$ in the sequence defined by the sketch proof. In each case, we are interested in crossings of the segment $P_iP_{i+1}$ at the shared edge $AB$. Our aim is to show that the number of crossings at $AB$ by the segment $P_iP_{i+1}$ with respect to $ABC$ is the same as it is with respect to $ABC'$

% What distinguishes these three cases are the positions of $C$ and $C'$ relative to the edge $AB$, and the consequence this has on the respective interiors of the triangles. In (a), we see that $C$ and $C'$ are on opposite sides of $AB$, and the interiors of the two triangles are therefore disjoint. Therefore, as $P_iP_{i+1}$ crosses the side $AB$, it leaves the interior of $ABC$ and enters the interior of $ABC'$. More generally, it leaves the interior of one triangle and enters the other.

% In case (b), we see that $C$ and $C'$ are on the same side of $AB$, and the interiors overlap. Here, as $P_iP_{i+1}$ crosses $AB$, it leaves the interior of both triangles. Generally, it leaves the interior of both triangles or enters the interior of both.

% In case (c), we see that $C$ and $C'$ are neither on opposite sides nor on the same side of $AB$. Instead, $C'$ is on the line of $AB$, and $ABC'$ is a degenerate case of a triangle and has \emph{no} interior. This last possibility can be factored into the previous two. Instead of allowing the point $C$ to move to any other point in the plane, we insist that it moves strictly between the two \emph{sides} of $AB$. To capture the case of (c), we will also allow the point $B$ to move along the ray $\overrightarrow{AB}$ to a point $B'$. The third case can then be recovered by setting $C=C'$ and moving $B$ to $B'$.

% However, we will add a constraint on $B'$, namely that the segment $BB'$ is not intersected by $P_iP_{i+1}$. This makes sense in context if we consider \ref{fig:ChangeTriangleOnLine}. In general, the segment $BB'$ will be a segment of the original polygon $Ps$, and thus, any intersection here is the witness we need for the main theorem. The upshot is that the intersections of $P_iP_{i+1}$ with $AB$ are precisely the intersections of $P_iP_{i+1}$ with $AB'$. In effect, we can treat the two segments as identical.

%\subsection{Crossings are Well-defined: Some Verification}
%In this section, we shall give an overview of how our verification breaks down. 

\subsection{Initialising the Context}\label{sec:ContextInitialisation}
Recall that, in general, when a segment $P_iP_{i+1}$ emerges from the boundary of a triangle $ABC$, the question of whether there is a crossing depends on additional information provided by the context. 

When computing the total crossings for a polygonal path, this context is threaded through the calculations for each individual edge, starting from some initial context. The question is: how do we choose this initial context?

Sometimes, the answer is straightforward. If the endpoint $P_i$ lies in the interior of the triangle, then the value of the context used to compute crossings for $P_iP_{i+1}$ \emph{must} be $\top$. If $P_i$ lies in the exterior of $ABC$, then the value \emph{must} be $\bot$. These give us some solutions for the initial context, and it would be convenient if we could rely on this simple case. But what happens if $P_i$ lies \emph{on} the triangle?

We considered two possible ways to answer this question, one of which turned out to have a surprising difficulty which left us favouring the other for the final verification. The first approach has us avoid the question, by always counting crossings from a vertex which does not lie on the triangle. The second approach has us compute a starting context based on the points of the polygon.

The first approach could work because if a closed polygon crosses a triangle, it must have at least one vertex outside the triangle. We just need to verify this and then rotate the polygon's vertex list so we can start counting crossings from there, and we will show how to perform such a rotation in \S\ref{sec:PolygonRotation}.

However, consider what happens as we vary the point $C$ of the triangle, something we need to do during the sketch proof. An example is shown in Figure~\ref{fig:ContextInitialiseCounter}, where we have two triangles $ABC$ and $ABC'$ sharing an edge $AB$, each crossed by the polygonal path $P_1P_2P_3P_1$. As we claimed, this polygonal path has a vertex outside of each triangle ($P_3$ for $\triangle ABC$ and $P_1$ for $\triangle ABC'$), but there is no vertex off both boundaries. This means that after we move the point $C$ to $C'$, we would need to perform another polygon rotation. Doing this continually during our proof would complicate the basic inductive argument.

\begin{figure}
\centering\includegraphics{jordanVerification1/ContextInitialiseCounter}
\caption{No well-defined initial context}
\label{fig:ContextInitialiseCounter}
\end{figure}

For the second approach, we need to compute a single consistent value for the initial context of any polygon. Fortunately, such a value exists. To spot it, we just realise that the initial value of the context for a polygonal path is related to the value of the context at the polygonal path's final edge. We can compute this final value with a recursive function:
\begin{equation*}
% let polypath_new_was_inside = define
%   `polypath_new_was_inside (A,B,C) was_inside [] = was_inside
%    /\ polypath_new_was_inside (A,B,C) was_inside (CONS seg Ps)
%         = polypath_new_was_inside (A,B,C)
%             (new_was_inside (A,B,C) was_inside (FST seg) (SND seg))
%             Ps`;;
\begin{aligned}
\vdash_{def}\;&\Gamma_{final}\ (A,B,C)\ \Gamma\ [] = \Gamma.\\
\vdash_{def}\;&\Gamma_{final}\ (A,B,C)\ \Gamma\ (\cons{(P_i,P_{i+1})}{segments}) = \\
&\qquad\qquad\Gamma_{final}\ (A,B,C)\ (\Gamma_{next}\ (A,B,C)\ \Gamma\ \ P_i\ P_{i+1})\ segments.
\end{aligned}
\end{equation*}

Now it turns out that if we push the final context back through the above function, we end up with the same value. Formally, $(\Gamma_{final}\ (A,B,C)\ \Gamma\ segments)$ is a fixpoint of the function $(\lambda \Gamma'.\; \Gamma_{final}\ (A,B,C)\ \Gamma'\ segments)$ for arbitrary $\Gamma$. No matter what our starting choice of $\Gamma$, the computed final context $\Gamma_{final}$ can be consistently taken as the initial context from then on. This expression therefore gives us a suitable starting context.

With this second approach, we do not have to rotate our polygon during the proof, and this means we can opt for a simple structural inductive verification.

The computation of the initial context appears in our specification of crossings, and the fact that this expression denotes a fixpoint is a lemma used in the verification of Theorem~\ref{eq:polypathCrossingsEven}.

\subsection{The Specification of Crossings}
\begin{figure}
  \begin{multline}\label{eq:crossNZIntersect}
    \vdash\code{polypath\_crossings}\ (A,B,C)\ \Gamma\ (\code{adjacent}\ Ps) > 0\\
    \implies \exists Q.\; \code{on\_polypath}\ Ps\ Q \wedge\ \between{A}{Q}{B}
  \end{multline}

  \begin{multline}\label{eq:polypathCrossingsEven}
  \begin{aligned}
% !A B C was_inside Qs 'a P Ps.
%          Qs = CONS P (APPEND Ps [P])
%          ==> on_plane A 'a /\
%              on_plane B 'a /\
%              on_plane C 'a /\
%              (!X. MEM X Qs ==> on_plane X 'a) /\
%              ~on_polypath Qs A /\
%              ~on_polypath Qs B /\
%              ~on_polypath Qs C /\
%              ~(?a. on_line A a /\ on_line B a /\ on_line C a)
%          ==> EVEN
%              (polypath_crossings (A,B,C)
%               (polypath_new_was_inside (A,B,C) was_inside (ADJACENT Qs))
%               (ADJACENT Qs) +
%               polypath_crossings (A,C,B)
%               (polypath_new_was_inside (A,B,C) was_inside (ADJACENT Qs))
%               (ADJACENT Qs) +
%               polypath_crossings (B,C,A)
%               (polypath_new_was_inside (A,B,C) was_inside (ADJACENT Qs))
%               (ADJACENT Qs))  
    \vdash&Qs = \append{[P]}{\append{Ps}{[P]}}\\
    &\wedge \Gamma_{initial} = \Gamma_{final}\ (A,B,C)\ \Gamma\ (\code{adjacent}\ Qs)\\
    &\wedge \neg\code{on\_polypath}\ Qs\ A \wedge \neg\code{on\_polypath}\ Qs\ B \wedge \neg\code{on\_polypath}\ Qs\ C\\
    &\wedge \Triangle{a}{A}{B}{C}
\end{aligned}\\
\implies\code{even}\left(\begin{aligned}& \code{polypath\_crossings}\ (A,B,C)\ \Gamma_{initial}\ (\code{adjacent}\ Qs)\\
    &+\;\code{polypath\_crossings}\ (A,C,B)\ \Gamma_{initial}\ (\code{adjacent}\ Qs)\\
    &+\;\code{polypath\_crossings}\ (B,C,A)\ \Gamma_{initial}\ (\code{adjacent}\ Qs)
  \end{aligned}\right)
\end{multline}

\begin{multline}\label{eq:changeTriangle}
  %        Qs = CONS P (APPEND Ps [P]) /\
  %        on_plane A 'a /\
  %        on_plane B 'a /\
  %        on_plane C 'a /\
  %        on_plane C' 'a /\
  %        (!X. MEM X Qs ==> on_plane X 'a) /\
  %        ~(?a. on_line A a /\ on_line B a /\ on_line C a) /\
  %        ~(?a. on_line A a /\ on_line B' a /\ on_line C' a) /\
  %        ~on_polypath Qs A /\
  %        (between A B B' \/ between A B' B \/ ~(A = B) /\ B = B') /\
  %        ~(?X. on_polypath [B; B'] X /\ on_polypath Qs X)
  %        ==> (?was_inside'. polypath_crossings (A,B,C)
  %                           (polypath_new_was_inside (A,B,C) was_inside
  %                           (ADJACENT Qs))
  %                           (ADJACENT Qs) =
  %                           polypath_crossings (A,B',C')
  %                           (polypath_new_was_inside (A,B',C') was_inside'
  %                           (ADJACENT Qs))
  %                           (ADJACENT Qs))
  \begin{aligned}
    \vdash&Qs = \append{[P]}{\append{Ps}{[P]}}\\
    &\wedge \neg\code{on\_polypath}\ Qs\ A \wedge \neg\code{on\_polypath}\ Qs\ B\\
    &\wedge \Triangle{a}{A}{B}{C}\\
    &\wedge \Triangle{a}{A}{B}{C'}\\
    &\implies \exists \Gamma'.\; \code{polypath\_crossings}\ (A,B,C)\ (\Gamma_{final}\ (A,B,C)\ \Gamma\ (\code{adjacent}\ Qs))\\
    &\qquad\qquad\qquad(\code{adjacent}\ Qs)\\
    &\qquad = \code{polypath\_crossings}\ (A,B,C')\ (\Gamma_{final}\ (A,B,C')\ \Gamma'\ (\code{adjacent}\ Qs))\\
    &\qquad\qquad\qquad(\code{adjacent}\ Qs)\\
  \end{aligned}
\end{multline}
\caption{Final specification of crossings}
\label{fig:CrossingsSpecification}
\end{figure}

At last, we will recover the intuitive idea behind crossings from the mess of case-analyses and implementation detail of the previous sections. In Figure~\ref{fig:CrossingsSpecification} we give the key theorems which subsume the important details of the other theorems considered thus far. It is these theorems which we shall appeal to exclusively in \S\ref{sec:InductionProof}, where we verify Veblen's Lemma from~\S\ref{sec:VeblenLemma1}.

The first theorem~\eqref{eq:crossNZIntersect} is mostly a convenience. It simply relates crossings to intersections, telling us that if there are crossings at $AB$ by a polygonal path $Ps$, then $Ps$ really does intersect $AB$. The converse does not hold, since the polygonal path might merely intersect and then ``bounce off'', thus staying on the same side of $AB$.

Theorem~\ref{eq:polypathCrossingsEven} assumes that we have a polygon $Qs$ and sets an initial context as described in the previous section. It also assumes we have a triangle $ABC$ and that $Qs$ does not intersect any of its vertices, as per our discussion in \S\ref{sec:NoVertexAssumption}. Under these conditions, the total number of crossings against the three sides is always even.

Theorem~\ref{eq:changeTriangle} tells us that the choice of $C$ when counting crossings is arbitrary, so long as it is not on the line $AB$. There is a slight complication, in that the theorem tells us to reset the initial context using the supplied $\Gamma'$ given in the conclusion, but since Theorem~\ref{eq:polypathCrossingsEven} holds for arbitrary choices of $\Gamma$, we can ignore this constraint when we apply the two theorems.

The upshot of Theorem~\ref{eq:changeTriangle} is that we can understand a crossing without reference to a triangle, but instead only with reference to the points $A$ and $B$. In the next section, we shall see how this more general understanding plays out in our verification.

\section{Verifying the Sketch Proof}
In this section, we shall review our verification of the parity proof that we sketched in \S\ref{sec:ParityProofInformal}. There are interesting details in the verification relating to our use of the theorems of the previous section. But more importantly, we can obtain a beautiful theorem from which the first half of the Polygonal Jordan Curve Theorem arises as a corollary. Unlike the Polygonal Jordan Curve Theorem, this theorem makes no reference to \emph{simple polygons}. It is a general theorem about arbitrary polygonal paths, one which does not hinge on complex definitions such as those that appear for the Polygonal Jordan Curve Theorem.

\subsection{The Induction Proof}\label{sec:InductionProof}
The parity proof assumes that we have two polygons $Ps$ and $Qs$ intersecting at an edge. Based on this, we consider a sequence of triangles formed from the vertices of the polygon $Ps$, and repeat a parity argument over the number of crossings.

This argument readily formalises as a proof by induction, which gives us a nice reinterpretation. Rather than considering triangles with vertices drawn from $Ps$, we continually reduce the problem to smaller polygons. This inductive proof yields the (somewhat ugly) lemma:
\begin{equation}\label{eq:oddCrossClosedPolypath}
  % "!Ps Qs A B.
  %    LENGTH Qs >= 2
  %    /\ HD Qs = LAST Qs
  %    /\ on_plane A 'a /\ on_plane B 'a
  %    /\ (!X. MEM X Ps ==> on_plane X 'a)
  %    /\ (!X. MEM X Qs ==> on_plane X 'a)
  %    /\ ~(A = B) 
  %    /\ (!C. on_plane C 'a
  %            /\ ~(?a. on_line A a /\ on_line B a /\ on_line C a)
  %            ==> ?was_inside. 
  %              ODD
  %                (polypath_crossings (A,B,C)
  %                  (polypath_new_was_inside (A,B,C) was_inside (ADJACENT Qs))
  %               (ADJACENT Qs)))
  %  ==> ?X. on_polypath (CONS B (APPEND Ps [A])) X
  %          /\ on_polypath Qs X"
  \begin{aligned}
    \vdash&\code{length}\ Qs \geq 2 \wedge \code{head}\ Qs = \code{last}\ Qs\wedge P_1\neq P_2\\
    &\wedge \left(\begin{aligned}&\forall C.\; \Triangle{a}{P_1}{P_2}{C}\\
    &\qquad\quad \implies \exists \Gamma.\; \code{odd}\ (\code{polypath\_crossings}\ (P_1,P_2,C)\\
    &\qquad\quad\qquad\qquad\qquad(\Gamma_{final}\ (P_1,P_2,C)\ \Gamma\ (\code{adjacent}\ Qs))\ (\code{adjacent}\ Qs))\end{aligned}\right)\\
    &\implies\exists X.\; \code{on\_polypath}\ (\append{[P_2]}{\append{Ps}{[P_1]}})\ X \wedge \code{on\_polypath}\ Qs\ X.
  \end{aligned}
\end{equation}

Here, we assume two polygons of length at least two, namely $\append{[P_1,P_2]}{\append{Ps}{[P_1]}}$ and $Qs$. The polygon $Qs$ is assumed to cross the edge $P_1P_2$ an odd number of times. We then conclude that $Qs$ intersects the tail of $\append{[P_1,P_2]}{\append{Ps}{[P_1]}}$, exactly as we require in the sketch proof.

Of particular note is how we formalise the idea that $Qs$ crosses the edge $P_1P_2$ in terms of the function \code{polypath\_crossings}. To do this, we abstract away the $C$ and the $\Gamma$ variables with universal and existential quantifiers, knowing it is valid to do so based on our well-definedness theorems.

Our aim will be to show that there is an odd number of crossings on $P_1P_3$, after which we can apply the inductive hypothesis. We get to assume that there is no crossing at $P_2P_3$, since otherwise we are done. In Mizar~Light, this assumption is made quite literally:

\begin{center}\boxed{\code{assume}\ \neg(\exists X.\; \code{on\_polypath}\ [P_2, P_3]\ X \wedge \code{on\_polypath}\ Qs\ X)}\end{center}

According to our treatment of the idea of crossings from \S\ref{sec:CrossingsWellDefined}, our goal is therefore formalised as
\begin{equation*}
% !C'. on_plane C' 'a
%                    /\ ~(?a. on_line A a /\ on_line C a /\ on_line C' a)
%                    ==> (?was_inside. ODD
%                              (polypath_crossings (A,C,C')
%                                 (polypath_new_was_inside (A,C,C') was_inside
%                                    (ADJACENT Qs))
%                                (ADJACENT Qs)))
  \begin{aligned}
    &\forall C.\; \Triangle{a}{P_1}{P_3}{C}\\
    &\quad \implies \exists \Gamma.\; \code{odd}\ (\code{polypath\_crossings}\ (P_1,P_3,C)\\
    &\qquad\qquad\qquad(\Gamma_{final}\ (P_1,P_3,C)\ \Gamma\ (\code{adjacent}\ Qs))\ (\code{adjacent}\ Qs)).
  \end{aligned}
\end{equation*}

There is actually a case-split to consider here. It is possible that $P_3$ lies on the line of $P_1P_2$, or, more specifically, on the \emph{ray} $\overrightarrow{P_1P_2}$\footnote{For this inference, we use our linear reasoning tactic.}. We shall not cover the details of this case. Suffice to say, it requires a complication of Theorem~\ref{eq:changeTriangle}, which we give in Appendix~\ref{app:JordanVerificationExtra}. Explaining it here would just obscure the basic ideas of the verification.

Thus, we shall assume that $P_3$ forms a triangle with $P_1P_2$. This means we can apply our assumption that there are an odd number of crossings at $P_1P_2$, namely
\begin{align*}&\forall C.\; \Triangle{a}{P_1}{P_2}{C}\\
    &\quad\implies \exists \Gamma.\; \code{odd}\ (\code{polypath\_crossings}\ (P_1,P_2,C)\\
    &\qquad\qquad\qquad(\Gamma_{final}\ (P_1,P_2,C)\ \Gamma\ (\code{adjacent}\ Qs))\ (\code{adjacent}\ Qs))).
\end{align*}

We now have an idea of how the quantifiers in this formula are to be used. We need $C$ to be arbitrary, because we must instantiate it with the particular vertex $P_3$ that we have obtained from the list $Ps$. We must then obtain an appropriate starting context $\Gamma$ depending on $C$, which is chosen according to which side of the edge $P_1P_2$ the vertex $P_3$ lies. 

By applying Theorems~\ref{eq:crossNZIntersect} and~\ref{eq:polypathCrossingsEven}, we can then conclude that there must be an odd number of crossings at $P_1P_3$ \emph{with respect to the triangle $P_1P_2P_3$}. All we need now to complete the inductive step is to generalise this claim by quantifying over the variable $P_2$. For this final step, we just use our well-definedness theorem~\eqref{eq:changeTriangle}.

\subsection{A Theorem of Polygonal Paths}\label{sec:PathTheorem}
Theorem~\ref{eq:oddCrossClosedPolypath} is all we need to show that a simple polygon divides the plane into two regions. It is interesting to note, however, that we have not mentioned path connectedness in this theorem, nor have we supposed that the polygons concerned are simple. This suggests there is a more general corollary to be had, one whose formalisation does not mention \emph{crossings} at all. Indeed, this ugly concept serves only as the crucial scaffolding of the verification. It can be stripped away when presenting the final theorem.

In a fairly short verification (42 steps), we apply Theorem~\ref{eq:oddCrossClosedPolypath} to obtain a beautifully symmetric theorem concerning arbitrary intersecting polygons:
\begin{equation}\label{eq:intersectPolypathClosed}
   % on_plane P1 'a /\ on_plane P2 'a /\ on_plane Q1 'a /\ on_plane Q2 'a 
   % /\ (!X. MEM X Ps ==> on_plane X 'a) /\ (!X. MEM X Qs ==> on_plane X 'a) 
   % /\ ~(?a. on_line P1 a /\ on_line P2 a /\ on_line Q1 a)
   % /\ ~(?a. on_line Q1 a /\ on_line Q2 a /\ on_line P1 a)
   % /\ between P1 X P2 /\ between Q1 X Q2
   % /\ P1 = LAST (CONS P2 Ps) /\ Q1 = LAST (CONS Q2 Qs)
   % ==> ?Y. on_polypath (CONS P2 Ps) Y /\ on_polypath (CONS Q1 (CONS Q2 Qs)) Y
   %         \/ on_polypath (CONS P1 (CONS P2 Ps)) Y /\ on_polypath (CONS Q2 Qs) Y"
  \begin{aligned}
    \vdash&\Triangle{a}{P_1}{P_2}{Q_1}\\
    &\wedge\Triangle{a}{Q_1}{Q_2}{P_1}\\
    &\wedge\between{P_1}{X}{P_2} \wedge \between{Q_1}{X}{Q_2}\\
    &\wedge P_1 = \code{last}\ Ps \wedge Q_1 = \code{last}\ Qs\\
    &\implies \exists Y.\; \code{on\_polypath}\ (\cons{P_2}{Ps})\ Y \wedge \code{on\_polypath}\ (\cons{Q_1}{\cons{Q_2}{Qs}})\ Y\\
    &\qquad\qquad\vee \code{on\_polypath}\ (\cons{P_1}{\cons{P_2}{Ps}})\ Y \wedge \code{on\_polypath}\ (\cons{Q_2}{Qs})\ Y.
  \end{aligned}
\end{equation}

In words, if we have polygons $P_1P_2\ldots P_1$ and $Q_1Q_2\ldots Q_1$ such that the segment $P_1P_2$ and $Q_1Q_2$ intersect, then one of the polygons intersects a non-trivial suffix of the other.

\begin{figure}
\centering\includegraphics[scale=0.5]{jordanVerification1/Theorem}
\caption{Arbitrary intersecting polygons}
\label{fig:IntersectingPolygons}
\end{figure}

The theorem places no constraints on the polygons other than that they cross at their first edges. They can have repeated vertices; they can self-intersect; they could even be the trivial polygons $P_1P_2P_1$ and $Q_1Q_2Q_1$. The point to visualise is that if two segments $P_1P_2$ and $Q_1Q_2$ cross one another, and we attempt to connect $P_2$ back to $P_1$ whilst attempting to connect $Q_2$ back to $Q_1$, we will find another point of intersection. See Figure~\ref{fig:IntersectingPolygons}.

\section{The Plane Divides into at Least Two Regions}\label{sec:FinalProofJordan1}
We can now verify the main theorem for this chapter. We assume a simple polygon $P_1P_2\ldots P_n$, and we must find two points off this polygon which cannot be connected by a polygonal path without crossing the simple polygon. Equivalently, any polygonal path connecting the two chosen points must intersect the simple polygon.

The strategy we use to prove this has already been covered in \S\ref{sec:ParityProofInformal}, and the verification respects the structure. In the sketch proof, we consider two rays emerging on either side of the edge $P_1P_2$. We find the points where these rays intersect $Ps$, and pick the point of intersection closest to $AB$. In our verification, this step is handled by a ``ray-casting'' theorem which we discuss in \S\ref{sec:RayCasting}. 

Ray-casting is the one and only place where we need to assume the simplicity of the simple polygon. In fact, we do not need to assume that much. All we really need to know is that there is \emph{some} edge of a polygon, and \emph{some} point $P$ inside that edge, such that $P$ does not lie on the rest of the polygon. Under these circumstances, we know that the polygon divides the plane into at least two regions.

This is to be contrasted with the verification in the next section. There, the assumption of a polygon's simplicity will feature heavily. The reason is that there are many ways for a polygon to divide the plane into multiple regions, but fewer ways for a polygon to restrict the number of regions to \emph{two}.

We have come this far by building a large tower of abstractions, complicated and unwieldy definitions, and theorems containing an irritatingly large number of hypotheses. The pay-off from this sort of verification is a result which throws out the scaffolding and brings us to a neat and easily grasped theorem.

\begin{equation}
  \begin{split}
    \vdash&\code{simple\_polygon}\ \alpha\ Ps\\
    &\implies \exists P\;\exists Q.\; \code{on\_plane}\ P\ \alpha \wedge \code{on\_plane}\ Q\ \alpha\\
    &\qquad\wedge \neg\code{on\_polypath}\ Ps\ P \wedge \neg\code{on\_polypath}\ Ps\ Q\\
    &\qquad\wedge \neg\code{polypath\_connected}\ \alpha\ (\code{on\_polypath}\ Ps)\ P\ Q.
  \end{split}\tag{\ref{eq:jordanFormal1}}
\end{equation}

%%% Local Variables: 
%%% mode: latex
%%% TeX-master: "../thesis"
%%% End: 

\chapter{Verifying the Polygonal JCT: Part II}\label{chapter:JordanVerification2}
We are nearing the end of our verifications. All that remains is to verify the second half of the Jordan Curve Theorem for polygons based on the axioms of Hilbert's ordered geometry. In this half of the verification, we must prove that a simple polygon separates its plane into at most two regions. As discussed when we gave the formalisation of this theorem in Chapter~\ref{chapter:JordanFormalisation}, it amounts to proving that given three points in the plane and not on the polygon, at least two of them are connected by a polygonal path.

This is effectively a maze navigation problem, lively and visual, with lots of geometrically interesting lemmas. Unlike the ``crossings'' of the last chapter, the basic concepts we appeal to are reflected cleanly in the low-level details of the verification, rather than being obscured by case-analyses and edge cases.

In discussing our verification, we will cover the same basic ground as we did in the last chapter. In \S\ref{sec:SketchProofJordan2}, we shall lay out the general approach of the proof. In \S\ref{sec:Jordan2Formulation}, we will look more closely at some of the basic machinery we will need, and formalise the key concepts in higher-order logic. Then, in \S\ref{sec:Jordan2Lemmas}, we shall cover the key lemmas that support our basic formalised concepts. As in the last chapter, these lemmas can be divided into those which introduce points in a geometrical configuration, and those which allow us to infer properties of the resulting configurations. 

In the rest of the chapter, we shall look in more detail at how the lemmas are applied to recover all the details of the sketch proof. We provide a few readable extracts of interesting verifications, demonstrating how faithfully we can formalise the intuitive synthetic arguments.

\section{Strategy}\label{sec:SketchProofJordan2}
The basic intuition behind the proof is similar to the ones presented by Veblen~\cite{Veblenphd} and Feigl~\cite{FeiglJordan}. We follow Veblen's proof the most closely. Contrary to Guggenheimer's~\cite{GuggenheimerJordanCurve} claim that Veblen's proof only holds for convex polygons, we believe the evidence of this chapter shows that Veblen was basically correct. That said, our verification is based on a more thorough analysis than presented by Veblen.

We are required to show that, given three points in the plane and not on a polygon, at least two of them are connected by a polygonal path. Let us reinterpret this and understand the three points as three players trying to navigate a polygonal maze.

Our basic goal is to get the three players ``next to'' the same edge. Then we just need to find a path between whichever of the two players are on the same side of that edge. We will find that the difficulty here lies in getting the players through potentially very tight corridors, and around difficult corners. We must show how to obtain paths for the players without recourse to notions such as comparable directions, parallel lines or distances. This will rule out common approaches to the theorem, such as the one given by Tverberg~\cite{TverbergJordan}. In Tverberg's proof, we just need to consider a sufficiently small region around the edges of the maze (an ``offset curve'' ), which we know to be polygonal path-connected. Without notions of distance, this description is out of scope of Hilbert's ordered geometry.

One way to formulate the idea that the players are ``next to'' the same edge of the maze is to assert that all three have line-of-sight to that edge. This metaphor does not appear explicitly in Veblen or Feigl's proof, but it can be read into both, and we found it extremely helpful in providing an intuitive grasp of the formalisation.

To make clear the idea about lines-of-sight, we will depict our players as \emph{eyes}, with a dashed line-of-sight to a point of the maze. In Figure~\ref{fig:SketchProofJordan2}, we show players $Red$, $Black$ and $Blue$ situated and staring at various points of a maze. Players $Red$ and $Black$ are inside the maze, while $Blue$ is on the outside. In the figure, we depict the paths they follow as they traverse the maze so that they have line-of-sight to the edge $P_iP_{i+1}$. Since $Red$ and $Black$ end up on the same side of $P_iP_{i+1}$, we can connect them by a polygonal path.

\begin{figure}
  \centering\includegraphics[scale=0.75]{jordanVerification2/SketchProof}
  \caption{Navigating a maze}
  \label{fig:SketchProofJordan2}
\end{figure}

The paths we have drawn through the maze are potential witnesses to the paths we consider in our verification. In fact, we can take the line-extension axiom 
\begin{equation}
  \tag{\ref{eq:g22}}
  \vdash A \neq B \implies \exists C.\; \between{A}{B}{C}
\end{equation}
and suppose that the witness $C$ in the conclusion is always chosen so that $B$ is half-way between $A$ and $C$. In this case, the paths sketched in Figure~\ref{fig:SketchProofJordan2} are precisely those that we witness in our formal verification.

\section{Formulation and Formalisation}\label{sec:Jordan2Formulation}
Compared to the last chapter, where we introduce the complex idea of a crossing, the basic ideas needed in the verification for this chapter are relatively straightforward. Firstly, given a simple polygon $Ps$, we will say that a point $X$ has line-of-sight to a point $X'$ if there is no point of $Ps$ which lies strictly between $X$ and $X'$. We shall say that when the point $X'$ lies between vertices $P_i$ and $P_{i+1}$ of $Ps$, then the point $X$ has line-of-sight to the \emph{edge} $P_iP_{i+1}$. The situation is formalised as
\begin{displaymath}
  \begin{aligned}
    &\neg\code{on\_polypath}\ Ps\ X \wedge \between{P_i}{X'}{P_{i+1}}\\
    &\wedge \neg\exists Z.\; \between{X}{Z}{X'} \wedge \code{on\_polypath}\ Ps\ Z.
  \end{aligned}
\end{displaymath}

Our verification breaks down into three parts. Firstly, we must show how every point not on a simple polygon has a line-of-sight to some edge of the maze. Next, we must show how, if a point $X$ has line-of-sight to an edge $P_iP_{i+1}$, then there is a polygonal path to a point $Y$ which has line-of-sight to the next edge $P_{i+1}P_{i+2}$. As such, for any edge and any point $X$, there is a polygonal path from $X$ to another point which has line-of-sight to that edge. Finally, we must show that if two players have line-of-sight to the same edge, and lie on the same side of that edge, then there is a polygonal path between them.

We shall describe the informal proofs and verifications of each part in \S\ref{sec:NavigationVerification}. First, we consider the crucial supporting lemmas.

\section{Obtaining Lines-of-Sight}\label{sec:Jordan2Lemmas}
We have two key theorems: a ray-casting theorem which obtains a new line-of-sight, and a theorem dubbed ``squeeze'' which handles narrow cracks in corridors. In proving the squeeze theorem, we shall need recourse to many of our earlier theorems about triangles and their interiors, and a new theorem about a triangle containing another triangle.

\subsection{Ray-casting}\label{sec:RayCasting}
Our ray-casting theorem gives us a line-of-sight to a polygonal path, aimed in an arbitrary direction towards that path. To achieve this, we must find the first point of intersection that the ray makes with the polygonal path. Ray-casting is actually needed in the first half of the Polygonal Jordan Curve Theorem (see \S\ref{sec:FinalProofJordan1}), but it makes more sense to explain it here where we are appealing to metaphors from computer graphics.

Ray-casting appears in a weakened form in Veblen's proof, but there he only considers casting a ray which does not intersect any \emph{vertex} of the polygonal path. This can be generalised by considering a few additional cases, after which we have a much more useful theorem.

This ray-casting theorem relies almost exclusively on linear reasoning and we found it particularly tricky to verify. As with our verification of Theorem~\ref{eq:IH3} in the last chapter, our linear reasoning tactic came through as a powerful tool for dealing with these problems, but first it has to be fed the right starting hypotheses. The trouble we had in the verification was deciding which hypotheses were needed. 

It often ended up being a matter of trial and error, but fortunately, the linear reasoning tactic is based on a decision procedure. When there were not enough facts available, it would promptly terminate and announce that the goal was not solvable. We could then go back through the problem and try to identify additional facts to feed the tactic and then retry. 

The need for this sort of manual labour is not particularly worthy of an automated proof assistant, and in future, some feedback on why the tactic failed would be useful and reasonably straightforward to implement. We are only working here with small sets of points (and the tactic struggles anyway when larger sets are considered), so it is feasible to enumerate all their permutations, including permutations when some combination of the points are equal. For instance, in the case of 5 points, there are only 431 possible arrangements. These serve as models, which can be filtered down by finding those which satisfy both the current hypotheses and the negation of the conclusion. After filtering, we are left with just the counterexamples, which would help if we could identify features in them which we can recognise as impossible given our diagrams and general intuition about the proof. We leave such a counterexample-checker for future work.

Sometimes, such counterexamples mean that a case-split must be considered. Sometimes, this reflected two different linear reasoning problems, which meant providing two subproofs. But sometimes we got lucky. If the case-split led to a single linear reasoning problem, we could let our incidence-discoverer handle the case-analyses automatically using its internal representation of proof trees.

The verification applies structural induction on the vertex list, and reduces the problem to that of ray-casting to a single edge. The basic case-analyses are shown in Figure~\ref{fig:RayCast}. We cast rays from the points $A$, $B$ and $C$ to the polygonal path $P_1P_2\ldots P_5$. The salient differences between the three lines-of-sight are as follows: point $A$ has line-of-sight to an endpoint of an edge, but the edge itself is not on the line-of-sight. Point $B$ has line-of-sight to an endpoint of an edge, and the edge itself \emph{is} on the line-of-sight. Finally, point $C$ has line-of-sight to the interior of an edge $P_2P_3$. 

\begin{figure}
\centering\includegraphics[scale=1]{jordanVerification2/RayCast}
\caption{Ray-casting}
\label{fig:RayCast}
\end{figure}

We now give the formalisation of the theorem. From a point $X$, we fire a ray to an arbitrary point $P$ on the polygonal path, and then obtain a point $Y$ to which $X$ has line-of-sight. Though it does not prove necessary in our verifications, we provide some extra information about the point $Y$, namely that it is either strictly between $X$ and $P$, or else we already had line-of-sight to $P$. 
\begin{multline}\label{eq:RayCast}
% !Ps X Y. ~(on_polyseg Ps X) /\ on_polyseg Ps Y
%                     ==> ?Z. on_polyseg Ps Z /\ (between X Z Y \/ Y = Z)
%                     /\ ~(?R. between X R Z /\ on_polyseg Ps R)
  \vdash\neg\code{on\_polypath}\ Ps\ X \wedge \code{on\_polypath}\ Ps\ P\\
  \implies \left(
    \begin{aligned}&\exists Y.\; \code{on\_polypath}\ Ps\ Y \wedge (\between{P}{Y}{X} \vee P = Y)\\
      &\qquad\qquad\wedge \neg(\exists Q.\; \between{X}{Q}{Y} \wedge \code{on\_polypath}\ Ps\ Q)
    \end{aligned}\right).
\end{multline}

Note the first conjunct in the hypothesis of this theorem. We can only cast rays to a polygonal path if we are not on that path. This should clarify a point made in \S\ref{sec:FinalProofJordan1} of the last chapter. There, we said that the final part of the verification showing that there are at least two regions of a simple polygon is based on ray-casting from some point $X$ on that polygon. In particular, we are ray-casting to the rest of the polygon, and thus, if we are to ray-cast, we need to assume that the rest of the polygon does not have a self-intersection at $X$. This we guarantee based on the fact that the polygon is assumed to be simple.

\subsection{Squeeze}\label{sec:Squeeze}
The most powerful theorem in our arsenal is one we dubbed ``squeeze'', since the intuition is that it allows us to find segments which squeeze through arbitrarily narrow gaps of a maze. What counts as a narrow gap in the abstract world of ordered geometry is  determined by the betweenness relation, and so the basic axioms governing this relation limit our powers in navigating such gaps. We can get some idea of the challenge by realising that, on some interpretations, these gaps are \emph{infinitesimally} narrow. Hilbert's axioms are independent of Archimedes' axiom.

Abstractly, our \emph{squeeze} theorem, Theorem~\ref{eq:Squeeze}, tells us that if we have a polygonal path $[A,B,C]$ which is not intersected by the polygonal path $Ps$, then we can introduce a point $A'$ between $A$ and $B$ such that $Ps$ intersects $A'C$ in at most one place. We can apply and interpret this theorem in a number of ways. In \S\ref{sec:SqueezeEye}, we will show how to interpret it in terms of lines-of-sight. Here, we interpret it in terms of finding diagonals that divide a polygon into two simple polygons.
\begin{equation}\label{eq:Squeeze}
  \begin{aligned}
      % "!P start goal 'a polyseg.
      % ~(?a. on_line P a /\ on_line start a /\ on_line goal a)
      % /\ on_plane P 'a /\ on_plane start 'a /\ on_plane goal 'a
      % /\ (!X. MEM X polyseg ==> on_plane X 'a)
      % /\ ~on_polyseg polyseg P
      % /\ (!X. between P X start ==> ~on_polyseg polyseg X)
      % /\ (!X. between P X goal ==> ~on_polyseg polyseg X)
      % ==> ?s. between P s start
      %         /\ !X. in_triangle (P,s,goal) X ==> ~on_polyseg polyseg X"
    \vdash&\neg\code{on\_polypath}\ Ps\ B\\
    &\wedge \neg(\exists X.\; \between{A}{X}{B} \wedge \code{on\_polypath}\ Ps\ X)\\
    &\wedge \neg(\exists X.\; \between{B}{X}{C} \wedge \code{on\_polypath}\ Ps\ X)\\
    &\implies \exists A'.\; \between{A}{A'}{B} \wedge \neg\exists X.\; \code{in\_triangle}\ (A',B,C)\ X\wedge \code{on\_polypath}\ Ps\ X.
  \end{aligned}
\end{equation}

\begin{figure}
\centering\includegraphics[scale=0.75]{jordanVerification2/SqueezeDiagonal}
\caption{Squeezing a diagonal}
\label{fig:SqueezeDiagonal}
\end{figure}

In Figure~\ref{fig:SqueezeDiagonal}, we use squeeze to find a diagonal of the polygon $P_1P_2\ldots P_{11}$. Here, we have set $A = P_1$, $B = P_{11}$, and $C = P_{10}$, while we set $path$ to be the rest of the polygon $P_1P_2P_3\ldots P_{10}$. 

The form of the conclusion in Theorem~\ref{eq:Squeeze} reflects its verification. We make a slightly different claim than declaring the existence of a diagonal. We say instead that the polygonal path does not lie in the interior of $\triangle A'BC$, which means that any point between $A'$ and $B$ yields a segment with the point $C$ which does not intersect $Ps$. We prove this starting with the triangle $ABC$, and find the first vertex in $Ps$ which lies inside this triangle. In the case shown, this would be the vertex $P_3$. We draw a line through $P_{10}$ and $P_3$ to the point $A_3$, and continue the argument with this new triangle. Eventually, we will be left with a triangle whose interior contains no point of $Ps$. In a sense, the triangle $ABC$ has been ``squeezed'' by $Ps$ into the triangle $A'BC$.

Unhappily, proving that $\triangle A'BC$ contains no point of $Ps$ has us boiled down in case-splits similar to those needed to analyse triangle crossings in the last chapter (\S\ref{sec:CrossingVerification}). Rather than go into the details of these, we shall focus on the more illuminating verification that $\triangle A'BC$ contains no \emph{vertex} of $Ps$. This only boils down to two more lemmas for triangle interiors.

\subsubsection{Another ``Inner Pasch'' Rule}
Our first supporting theorem allows us to introduce the intersection points $A_3$, $A_4$, $A_6$ and $A'$ in Figure~\ref{fig:SqueezeDiagonal}. This theorem is similar in spirit to the Pasch axioms (\ref{eq:OuterPasch}, \ref{eq:InnerPasch}) and its variant for triangle interiors \eqref{eq:tricut1}. It has a very succinct formalisation, but a non-trivial verification:
\begin{equation}\label{eq:TriCut3}
\vdash\code{in\_triangle}\ (A,B,C)\ P \implies \exists X.\; \between{B}{X}{C} \wedge \between{A}{P}{X}.
\end{equation}

The verification is the most interesting for these point introduction theorems. Unlike the verification of Theorem~\ref{eq:tricut1} from the last chapter, we find ourselves back employing Pasch's axiom \eqref{eq:g24} rather than exploiting our theorems of half-planes. First, we use \ref{eq:three} to find a point $X$ on $AB$. Next, we apply \eqref{eq:g24} to the triangle $ABC$ and the line $PX$. This gives us a point $Y$ which is either between $B$ and $C$ or between $A$ and $C$. Here is one of the rare times where both cases are possible. Normally, we would be able to refute one of the cases based on incidence reasoning. Here, we need two subproofs.

\begin{figure}
\centering\includegraphics[scale=1.2]{jordanVerification2/TriCut3}
\caption{Drawing a line from a vertex to the opposite side}
\label{fig:TriCut3}
\end{figure}

If $Y$ lies between $B$ and $C$ as in case (a) of Figure~\ref{fig:TriCut3}, then we apply \eqref{eq:g24} to the triangle $BXY$ and the line $AP$ to find a point $Z$ between $B$ and $Y$. This point is then between $B$ and $C$ (via linear reasoning on $B$, $C$, $Y$ and $Z$). Furthermore, $P$ is between $A$ and $Z$ by~\eqref{eq:inTriangle2}.

In case (b) of Figure~\ref{fig:TriCut3}, we find that $P$ is now an interior point according to Veblen's definition and the position of the points $X$ and $Y$. Here, we apply \eqref{eq:g24} to the triangle $CXY$ and the line $AP$ to find a point $P'$ on $CX$. By the same axiom applied to $\triangle BCX$ and the line $AP'$, we find the desired point $Z$ on $BC$.

\subsubsection{Subtriangles}\label{sec:Subtriangles}
In our verification of Theorem~\ref{eq:Squeeze}, we consider a sequence of triangles such as those of Figure~\ref{fig:SqueezeDiagonal}, namely $\triangle ABC$, $\triangle A_3BC$, $\triangle A_4BC$, $\triangle A_6BC$ and $\triangle A'BC$. Each of these is intended to exclude one vertex of $Ps$ from its interior. By the time we reach the last triangle, we can conclude that all vertices of $Ps$ will lie outside the interior. This conclusion assumes a transitivity property, which is given by our second lemma. It shows that the ordering of $A$, $A_3$, $A_4$, $A_6$ and $A'$ along the segment $AB$ yields a sequence of triangles with \emph{nested} interiors. 

The verification, given in Figure~\ref{fig:SubTriangleProof}, is extremely short and readable, making use of a number of lemmas we have previously considered, including Theorem~\ref{eq:TriCut3} from the last subsection. We show that any point $P$ interior to a triangle $AA'C$ is interior to any triangle $ABC$ where $A'$ is between $A$ and $B$. That is, the interior of $AA'C$ is nested in $ABC$. To prove it, we put a point $Q$ on the line $A'C$, and show that it must be interior to $\triangle ABC$ on the basis of Theorem~\ref{eq:inTriangle1}. It then follows by Theorem~\ref{eq:inTriangle2} that $P$ must also be interior. See Figure~\ref{fig:SubTriangle}.

\begin{figure}
\centering\includegraphics[scale=1.2]{jordanVerification2/SubTriangle}
\caption{Any interior point $P$ of $\triangle AA'C$ is an interior point of $\triangle ABC$.}
\label{fig:SubTriangle}
\end{figure}

\begin{boxedfigure}
\small
\begin{align*}
&\code{theorem }\code{in\_triangle}\ (A,A',C)\ P \wedge \between{A}{A'}{B} 
                  \implies \code{in\_triangle}\ (A,B,C)\ P\\
&\code{assume}\ \code{in\_triangle}\ (A,A',C)\ P \wedge \between{A}{A'}{B} & 0 \\
&\code{so consider}\ Q\ \code{such that}\ \between{A'}{Q}{C} \wedge \between{A}{P}{Q} \ \code{by}\ \eqref{eq:TriCut3} & 1\\
&\code{obviously}\ \code{(by\_ncols}\ \circ\ \code{add\_in\_triangle}\ \circ\ \code{conjuncts})\\
&\qquad \code{hence}\ \code{in\_triangle}\ (A,B,C)\ Q\ \code{by}\ \eqref{eq:triSyms},\eqref{eq:inTriangle1},\eqref{eq:g21}\ \code{from}\ 0\\
&\code{qed from}\ 1\ \code{by}\ \eqref{eq:onTriangleDef},\eqref{eq:inTriangle2}
\end{align*}
\caption{Subtriangles}
\label{fig:SubTriangleProof}
\end{boxedfigure}

\subsection{Obtaining lines-of-sight via Squeeze}\label{sec:SqueezeEye}
As we suggested earlier, Theorem~\ref{eq:Squeeze} is quite powerful. In this section, we will show two applications of the theorem in terms of lines-of-sight, both of which we shall need for the core verifications of this chapter. 

\subsubsection{Moving to a new line-of-sight}\label{sec:MoveToNew}
\begin{figure}
\centering\includegraphics[scale=0.75]{jordanVerification2/Squeeze1}
\caption{Obtaining line-of-sight to $C$}
\label{fig:Squeeze1}
\end{figure}

Consider Figure~\ref{fig:Squeeze1}. Here, we have slightly modified and relabelled the diagram from Figure~\ref{fig:SqueezeDiagonal} so that we can interpret the point $A$ as our player's starting location, with line-of-sight to a point $B$ on the edge $P_{11}P_{12}$ of a simple polygon. The goal is to obtain line-of-sight to a given point $C$ on that same edge.

The idea is that if the player proceeds far enough down their initial line-of-sight, we can easily rotate them to the new line-of-sight. We do this by applying Theorem~\ref{eq:Squeeze} and taking $path$ to be the fragment of the polygon $P_{12}P_1P_2\ldots P_{11}$. Let us think about how to fulfil the hypotheses. We are basically being asked to rule out the possibility that the polygonal path $[A,B,C]$ is not intersected by this fragment, save for possible intersections at $A$ and at $C$. 

Well, we know that the segment $AB$ does not lie on the fragment $P_{12}P_1P_2\ldots P_{11}$, because $AB$ is supposed to be a line-of-sight. We also know that $BC$ does not intersect the fragment, because we are assuming \emph{simple} polygons: the polygon should not self-intersect the edge $P_{11}P_{12}$, and this edge contains $BC$. We leave open the possibility that $P_{12}P_1P_2\ldots P_{11}$ intersects the lone point $C$, since, as we shall see in \S\ref{sec:ConcaveMove}, we will sometimes need to obtain line-of-sight to a vertex, where we will set $C=P_{11}$.

Applying Theorem~\ref{eq:Squeeze} is not quite enough. We now have a point $A'$ which \emph{almost} has line-of-sight to $C$. We just need to use \ref{eq:three}, obtaining a point $A''$ and a segment $A''C$ which lies properly in the triangle $A'BC$, and, as per the conclusion of Theorem~\ref{eq:Squeeze}, a segment which is a line-of-sight to the point $C$. We verify the argument here as a corollary of Theorem~\ref{eq:Squeeze}:
\begin{equation}\label{eq:Squeeze2}
  \begin{aligned}
  % "!P start goal 'a polyseg.
  %     ~(?a. on_line P a /\ on_line start a /\ on_line goal a)
  %     /\ on_plane P 'a /\ on_plane start 'a /\ on_plane goal 'a
  %     /\ (!X. MEM X polyseg ==> on_plane X 'a)
  %     /\ ~on_polyseg polyseg P
  %     /\ (!X. between P X start ==> ~on_polyseg polyseg X)
  %     /\ (!X. between P X goal ==> ~on_polyseg polyseg X)
  %     ==> (?s. between P s start
  %              /\ !X. between s X goal  ==> ~on_polyseg polyseg X)"
\vdash    &\neg\code{on\_polypath}\ path\ B\\
    &\wedge \neg(\exists X.\; \between{A}{X}{B} \wedge \code{on\_polypath}\ path\ X)\\
    &\wedge \neg(\exists X.\; \between{B}{X}{C} \wedge \code{on\_polypath}\ path\ X)\\
    &\implies \exists A'.\; \between{A}{A'}{B} \wedge \neg\exists X.\; \between{A'}{X}{C} \wedge \code{on\_polypath}\ path\ X.
  \end{aligned}
\end{equation}

Now because $AB$ is a line-of-sight, we know that $AA''$ is part of a polygonal path which does not intersect $P_1P_2\ldots P_{11}P_{12}P_{1}$. What we have exploited here is the fact that we can always move along our lines-of-sight without intersecting the polygon. The use of squeeze \eqref{eq:Squeeze} in this section is therefore telling us how to build up paths through a maze, by getting the player to move far enough along their line-of-sight that they will be able to see a point further down the current edge. 

The next application of squeeze will be just as critical for our verifications.

\subsubsection{Rotating to a new line-of-sight}\label{sec:RotateToNew}
\begin{figure}
\centering\includegraphics[scale=0.75]{jordanVerification2/Squeeze2}
\caption{Obtaining line-of-sight to the interior of $AB$}
\label{fig:Squeeze2}
\end{figure}
In Figure~\ref{fig:Squeeze2}, we have again modified and relabelled the diagram. The idea here is that our player is situated at a point $C$ and initially has line-of-sight to the vertex $B$. The problem with this scenario is that, in our verifications, we are more concerned about having lines-of-sight to points on a polygon which are \emph{not} vertices (for a start, this forces the player off the line of the edge).  So we want our player to rotate their line-of-sight slightly. 

Therefore, we apply Theorem~\ref{eq:Squeeze} by reversing the order of $A$, $B$ and $C$. The polygonal fragment we are concerned with here is again $P_{12}P_1P_2\ldots P_{11}$, and as before, we have to be sure that the hypotheses of Theorem~\ref{eq:Squeeze} have been met. That is, we must show that the polygonal path $[A,B,C]$ is not intersected by the fragment $P_{12}P_1P_2\ldots P_{11}$. 

Again, we know that $AB$ is not intersected by the fragment, because we assume that the polygon is simple and so does not have such self-intersections, and we know that $BC$ is not intersected by the fragment because $BC$ is assumed to be a line-of-sight. This means we have met the hypotheses. We now obtain the point $A'$ as shown, and as before, we use \ref{eq:three} to obtain a point $A''$ between $B$ and $A'$. Our player will now have line-of-sight to a non-vertex point of the segment $AB$.

The reader may have noticed that we have not mentioned the segment $BP_{12}$ shown in the diagram. In fact, this segment could prove a problem. If it were chosen to lie on the other side of $BC$, then it might intersect the segment $A''C$, and if this happens, then $A''C$ will not count as a line-of-sight. Fortunately, we shall be able to rule out this circumstance when we come to apply Theorem~\ref{eq:Squeeze2}. We shall always be assuming that, when we rotate our line-of-sight, the points $P_{11}$ and $P_{12}$ are on opposite sides of $BC$, as shown.

To summarise, we can say that the first use of squeeze allows us to reduce the \emph{distance} to the point $B$ sufficiently and thus obtain a new line-of-sight, while this second use of squeeze is telling us that we can reduce the \emph{angle} $ABC$ sufficiently. What the theorem gives us is the ability to reduce distances and angles even without a general theory for comparing them, and without any sort of arithmetic for them. We ``squeeze'' our distances and angles by working exclusively with a weak order relation and properties of incidence.

\section{Edge-to-Edge}\label{sec:NavigationVerification}
Since a simple polygon is just a list of adjacent edges, we can navigate every single edge just by moving between one edge and the next in the adjacency list. In Section~\ref{sec:Jordan2Formulation}, we explained that the movement from an edge $P_{i}P_{i+1}$ to an edge $P_{i+1}P_{i+2}$ amounts to showing three things: (1) that there is a point $X$ with line-of-sight to $P_{i}P_{i+1}$; (2) that there is a point $Y$ with line-of-sight to $P_{i+1}P_{i+2}$; and (3), that there is a polygonal path between $X$ and $Y$.

Now when inside a convex polygon, this matter is completely trivial. Indeed, \emph{every} interior point of a convex polygon has line-of-sight to \emph{every} edge, as in Figure~\ref{fig:ConvexEasy}. The salient fact to notice is that, in a convex polygon, interior points are on the same side of every edge as every other vertex. 

\begin{figure}
\centering\includegraphics[scale=0.75]{jordanVerification2/ConvexEasy}
\caption{Edge-to-edge in a convex polygon}
\label{fig:ConvexEasy}
\end{figure}

Generalising this, we can say that two edges $P_{i}P_{i+1}$ and $P_{i+1}P_{i+2}$ appear ``locally convex'' from the perspective of a point $P$ if the point $P$ is on the same side of $P_{i}P_{i+1}$ as $P_{i+2}$. Otherwise, we can say that the edges appear ``locally concave''. These provide our two cases for how we navigate the edges of a simple polygon.

\subsection{Locally Convex Edges}\label{sec:ConcaveMove}
To explain the case for locally convex edges, we shall assume we have a polygon $P_1P_2P_3\ldots P_n$, and we shall assume that we are moving from edge $P_1P_2$ to edge $P_2P_3$ (we can use the polygon rotations described in \S\ref{sec:PolygonRotation} to consider the other pairs of edges).

\begin{figure}
\centering\includegraphics[scale=0.75]{jordanVerification2/Convex1}
\caption{Edge-to-edge in a locally convex polygon}
\label{fig:Convex1}
\end{figure}

In Figure~\ref{fig:Convex1}, we start at a point $X$ which has line-of-sight to $P_1P_2$. Our goal is to reach a point $Y$ with line-of-sight to $P_2P_3$. We start by moving towards $P_1P_2$ by applying Theorem~\ref{eq:Squeeze2} as described in \S\ref{sec:MoveToNew}, seeking a line-of-sight with the vertex~$P_2$. 

\label{sec:InjectingProcedural}Applying squeeze in this way is usually a tricky business, because there are a number of hypotheses which must be fulfilled and it is not always immediately clear how. Even when we had found all the necessary hypotheses, the default \code{MESON} prover would struggle to apply them all, and so we would have to inject some procedural code as justification. In doing this, we tried to respect the aims of declarative verification. For instance, at no point do we destructively modify the proof context, and we always insist that whenever our tactics apply an assumption, we include the assumption label so that a reader can at least track the dependencies. Here, we use \code{EXISTS\_TAC} followed by \code{MATCH\_MP\_TAC}, which leaves \code{MESON} just having to discharge the assumptions of \eqref{eq:Squeeze2}.

%We then use $\code{MATCH\_MP}$ to retrieve our hypotheses as a big conjunction, and then split each off into its own subgoal. We then tackle these one at a time. Afterwards, we can gather up all the dependent assumptions and prove the original goal in one step. We just need a bit of tactic script to help \code{MESON}, but as usual, we make sure not to destructively modify the goal state, and we reference our justifying theorem \eqref{eq:Squeeze2}. The step is still largely declarative.

\fbox{\begin{minipage}{\boxwidth}\setlength\abovedisplayskip{0cm}
\begin{align*}
\small
&\code{obviously by\_incidence so consider}\ Y\ \code{such that}\ \between{X}{Y}{X'}\\
&\qquad\wedge \forall Z.\; \between{Y}{Z}{P_2} \implies \neg\code{on\_polypath}\ (\cons{P_2}{\cons{P_3}{Ps}})\ Y\\ &\qquad\code{from}\ \ldots\ \code{by}\ \code{on\_polypath},\eqref{eq:g22}\\
&\qquad\code{using K (MATCH\_MP\_TAC}\ \eqref{eq:Squeeze2}\ \code{THEN EXISTS\_TAC}\ \alpha) & 16,17
\end{align*}\end{minipage}}\linebreak

This done, we have our desired path $XY$. We know that this path does not intersect any part of the simple polygon, since it is part of our original line-of-sight. All that remains then is to rotate ourselves slightly to obtain a line-of-sight $YY'$ with some point on $P_2P_3$. This we do in Figure~\ref{fig:Convex2}, using Theorem~\ref{eq:Squeeze2} as described in \S\ref{sec:RotateToNew}. 

\begin{figure}
\centering\includegraphics[scale=0.75]{jordanVerification2/Convex2}
\caption{Edge-to-edge in a locally convex polygon (continued)}
\label{fig:Convex2}
\end{figure}

\begin{boxedfigure}
\small
% so consider ["Y:point"]
%          st "between s Y hand'' /\ on_polyseg [P1;P2;P3] Y"
%          from [22] at [24] by [on_polyseg_CONS2;on_polyseg_pair;BET_SYM]
%        ;obviously (by_ncols o Di.conjuncts)
%          (hence "~on_polyseg [P2;P3] Y"
%             from [4;5;20;23]
%             by [on_polyseg_pair;g11_weak;g21])
%        ;hence "on_polyseg [P1;P2] Y" from [24] at [25]
%          by [on_polyseg_pair;on_polyseg_CONS2]
%        ;have "on_half_plane hp hand''"
%          from [12;14;23] by [bet_on_half_plane]
%        ;hence "on_half_plane hp Y"
%          from [19;24] by [bet_on_half_plane2] at [26]
%        ;hence "between P1 Y P2"
%          from [3;12;25] by [on_polyseg_pair;half_plane_not_on_line]
%        ;qed from [3;12;26] by [BET_NEQS;g12;g21
%                               ;half_plane_not_on_line]]
\begin{align*}
&\code{so consider}\ Z\ \code{such that}\ \between{Y}{Z}{Y'} \wedge \code{on\_polypath}\ [P_1,P_2,P_3]\ Z\\\
&\qquad\qquad\code{from}\ \ldots\ \code{by}\ \eqref{eq:OnPolyPath} & 24\\
&\code{obviously}\ (\code{by\_ncols}\circ\code{conjuncts})\ \code{hence}\ \neg\code{on\_polypath}\ [P_2,P_3]\ Z\ \code{from}\ \ldots\\
&\qquad\code{by}\ \eqref{eq:OnPolyPath},\eqref{eq:g11},\eqref{eq:g21}\\
&\code{hence}\ \code{on\_polypath}\ [P_1,P_2]\ Z\ \code{from}\ 24\ \code{by}\ \eqref{eq:OnPolyPath}&25\\
&\code{have on\_half\_plane}\ hp\ Y'\ \code{from}\ 12,14,23\ \code{by}\ \eqref{eq:betOnHalfPlane1}\\
&\code{hence on\_half\_plane}\ hp\ Z\ \code{from}\ \ldots,24\ \code{by}\ \eqref{eq:betOnHalfPlane2} & 26\\
&\code{hence}\ \between{P_1}{Z}{P_2}\ \code{from}\ \ldots,25\ \code{by}\ \eqref{eq:OnPolyPath},\eqref{eq:halfPlaneNotOnLine}\\
&\code{qed from}\ \ldots,26\ \code{by}\ \eqref{eq:g12},\eqref{eq:g21},\eqref{eq:halfPlaneNotOnLine}
\end{align*}
\caption{Verification extract for the convex case of theorem~\ref{eq:PolygonMove}}
\label{fig:ConvexVerification}
\end{boxedfigure}

This is not the whole story. What is missing is any mention of the local convexity. This is needed to show that our segment $YY'$ does not intersect the polygon, and thus is a genuine line-of-sight. Our second application of \eqref{eq:Squeeze2} only tells us that it does not intersect the fragment of the polygon $P_3P_4\ldots P_1$. Our incidence discoverer tells us further that any point $Z$ between $Y$ and $Y'$ cannot intersect $P_2P_3$. What is left to rule out is a possible intersection with $P_1P_2$. For this, we need to think about half-planes. 

We reason as follows. We know that $Y$ and $P_3$ are on the same side of $P_1P_2$ (since $P_1P_2$ and $P_2P_3$ are locally convex from the perspective of $Y$), and $Y'$, being on $P_2P_3$, must also be on the same side of $P_1P_2$. That means that $Z$ is also on the same side, because it lies on $YY'$. But that means it cannot intersect the line of $P_1P_2$, and, in particular, it cannot be on the segment $P_1P_2$.

The verification shown in Figure~\ref{fig:ConvexVerification} matches this argument's structure.  We start by assuming there is some point $Z$ on $YY'$ which intersects the polygonal path $[P_1,P_2,P_3]$, and proceed by contradiction (we have removed some of the references to earlier steps and inserted ellipses for readability). 

\subsection{Locally Concave Edges}
When the edges are (strictly) locally concave, we have it that our starting point $X$ is on the opposite side of the line $P_1P_2$ as the point $P_3$. In this case, we will generally need to ``round the corner'' $P_1P_2P_3$ to get line-of-sight with the next edge. 

Before we apply Theorem~\ref{eq:Squeeze2}, we will pick our destination. This will be the point $Y'$ shown in Figure~\ref{fig:Concave1}. This point is ``just off'' the vertex $P_2$, and can be found by ray-casting (Theorem~\ref{eq:RayCast}) from $P_2$ in the direction $\overrightarrow{P_1P_2}$. We then use Theorem~\ref{eq:Squeeze2} as described in \S\ref{sec:MoveToNew}, moving along our initial line-of-sight $XX'$ to obtain a new line-of-sight with $Y'$. The segment $XYY'$ will then be the required path.

\begin{figure}
\centering\includegraphics[scale=0.75]{jordanVerification2/Concave1}
\caption{Edge-to-edge in a locally concave polygon}
\label{fig:Concave1}
\end{figure}

Now since $Y'$ was found by ray-casting, we know it has line-of-sight to $P_2$. So all we need to do is rotate this line-of-sight to point at $P_2P_3$, which we know we can do using Theorem~\ref{eq:Squeeze2} as described in \S\ref{sec:RotateToNew}. We thus have the required path and line-of-sight to $Y''$ as shown in Figure~\ref{fig:Concave2}.

We just have to confirm that the path $XYY'$ does not intersect the polygon, and that $Y'Y''$ is a genuine line-of-sight. We know immediately that $XY$ is off the polygon, since it is part of our original line-of-sight. This leaves us having to show that $YY'$ and $Y'Y''$ do not intersect the polygon. 

The way these segments were obtained via Theorem~\ref{eq:Squeeze2} guarantees that they do not intersect the fragment $P_3P_4\ldots P_n$, so that leaves us to consider whether either of the segments $YY'$ and $Y'Y''$ intersect the fragment $P_1P_2P_3$. 

It might seem that these cases would boil down to reasoning with half-planes again, since $YY'$ and $Y'Y''$ are part of two rays emanating in opposite directions from the line of $P_1P_2$. But we were surprised to find that, when we set our discoverer at the problem, it showed that $Y'Y''$ is off the fragment $P_1P_2P_3$ and thus a line-of-sight according to incidence reasoning alone.

The discoverer also showed that $YY'$ does not intersect the edge $P_1P_2$. We only need to use half-planes to show that it does not intersect the edge $P_2P_3$, using the same sort of tidy declarative verification that we saw in the last subsection: since $Y$ lies between $X$ and $P_1P_2$, and $X$ and $P_3$ lie on opposite sides of $P_1P_2$, it must be the case that $Y$ and $P_3$ lie on opposite sides of this line as well. And then, since $Y'$ is on the line of $P_1P_2$, it follows from Theorem~\ref{eq:betOnHalfPlane1} that every point on $YY'$ lies on the opposite side to $P_3$, and thus, does not lie on the line of $P_1P_2$. 

\begin{figure}
\centering\includegraphics[scale=0.75]{jordanVerification2/Concave2}
\caption{Edge-to-edge in a locally concave polygon (continued)}
\label{fig:Concave2}
\end{figure}

This almost completes the verification. We have not talked about the case that $P_1$, $P_2$ and $P_3$ are collinear, which requires one application of Theorem~\ref{eq:Squeeze2}, but we hope by this stage the reader is convinced we can achieve this, and is further convinced of how powerful our squeeze theorem is in allowing us to move between edges of a maze in a geometrically intuitive way.

\subsection{Putting it all Together}
We have packaged all the details discussed so far in this section into a single declarative verification consisting of 119 steps. That we can manage such long verifications shows how easily Mizar~Light, with our additional automated tools, can scale.

We would like to draw special attention to the hypotheses in the verified theorem \eqref{eq:PolygonMove}. Note that we do not need to assume that we are dealing with simple polygons. We just have to assume that $P_1P_2$ does not intersect the rest of the path $P_2P_3\ldots P_n$, and that $P_2P_3$ does not intersect the rest of the path $P_3P_4\ldots P_n$. These are the minimal hypotheses we need to apply Theorem~\ref{eq:Squeeze2} as described in this section. We can think of them as saying that the polygon appears \emph{locally} simple.

\begin{equation}\label{eq:PolygonMove}
  \begin{split}
  % "!P1 P2 P3 Ps X hand 'a.
  %    on_plane P1 'a /\ on_plane P2 'a /\ on_plane P3 'a
  %    /\ (!P. MEM P Ps ==> on_plane P 'a)
  %    /\ on_plane X 'a
  %    /\ between P1 hand P2
  %    /\ ~(P2 = P3)
  %    /\ ~on_polyseg (CONS P1 (CONS P2 (CONS P3 Ps))) X
  %    /\ ~on_polyseg (CONS P3 Ps) P2
  %    /\ ~(?Y. between X Y hand /\ on_polyseg (CONS P1 (CONS P2 (CONS P3 Ps))) Y)
  %    /\ ~(?Y. between P1 Y P2 /\ on_polyseg (CONS P2 (CONS P3 Ps)) Y)
  %    /\ ~(?Y. between P2 Y P3 /\ on_polyseg (CONS P3 Ps) Y)
  %    ==> ?hand' X'.
  %           seg_connected 'a
  %             (on_polyseg (CONS P1 (CONS P2 (CONS P3 Ps)))) X X'
  %           /\ between P2 hand' P3
  %           /\ ~on_polyseg (CONS P1 (CONS P2 (CONS P3 Ps))) X'
  %           /\ ~(?Y. between X' Y hand'
  %           /\ on_polyseg (CONS P1 (CONS P2 (CONS P3 Ps))) Y)"
\vdash    &\between{P_1}{X'}{P_2} \wedge P_2 \neq P_3\\
    &\wedge \neg\code{on\_polypath}\ (\cons{P_1}{\cons{P_2}{\cons{P_3}{Ps}}})\ X \wedge \neg\code{on\_polypath}\ (\cons{P_3}{Ps})\ P_2\\
    &\wedge \neg(\exists Z.\; \between{X}{Z}{X'} \wedge \code{on\_polypath}\ (\cons{P_1}{\cons{P_2}{\cons{P_3}{Ps}}})\ Z)\\
    &\wedge \neg(\exists Z.\; \between{P_1}{Z}{P_2} \wedge \code{on\_polypath}\ (\cons{P_2}{\cons{P_3}{Ps}})\ Z)\\
    &\wedge \neg(\exists Z.\; \between{P_2}{Z}{P_3} \wedge \code{on\_polypath}\ (\cons{P_3}{Ps})\ Z)\\
    &\implies \exists Y.\;\exists Y'.\; \code{polypath\_connected}\ (\code{on\_polypath}\ (\cons{P_1}{\cons{P_2}{\cons{P_3}{Ps}}}))\ X\ Y\\
    &\qquad\wedge \between{P_2}{Y'}{P_3}\wedge \neg\code{on\_polypath}\ (\cons{P_1}{\cons{P_2}{\cons{P_3}{Ps}}})\ Y)\\
    &\qquad\wedge \neg\exists Z.\; \between{Y}{Z}{Y'} \wedge (\cons{P_1}{\cons{P_2}{\cons{P_3}{Ps}}})\ Z).
  \end{split}
\end{equation}

\section{Without-Loss-of-Generality}
In the last section, we described how to move edge-to-edge in a polygon. However, we effectively made a without-loss-of-generality assumption when stating the theorem, since we assumed that the edges in question were defined by the first three elements of the vertex list. More generally, they could be any two edges defined by  three adjacent elements of the list, or, if we are considering moving forward from the last edge of a polygon back to the first, then the vertices defining the edges come from the last two elements and the first two elements.

We can follow Harrison's approach to dealing with without-loss-of-generality assumptions~\cite{HarrisonWLOG} if we can find an equivalence relation which preserves the properties of interest to us. In effect, we say that before we move a player to the next edge, we must be able to transform the polygon's vertex list into the player's perspective, much as we would perform a coordinate transform.

\subsection{Polygon Rotations: Formulation}\label{sec:PolygonRotation}
To make the idea work formally, we will need a way to represent a polygon rotation. Since our polygons are being represented by vertex lists, we must briefly leave the pleasant world of synthetic geometry and instead look at computing with lists. On the upside, some of these ideas are a general contribution to the theory of lists, and can be applied in other contexts. 

It is not uncommon, for instance, to need to rotate a list. This can be understood in terms of splitting the list at some point and then switching the two halves. Formally:
\begin{displaymath}
  \begin{split}
&\code{rotation\_of}\; :\; [\alpha] \rightarrow [\alpha] \rightarrow \code{bool}\\
&\code{rotation\_of}\ ps\ qs \iff \exists xs.\;\exists ys.\; ps = \append{xs}{ys} \wedge qs = \append{ys}{xs}.\\
  \end{split}
\end{displaymath}
This defines an equivalence relation, but our use-case for list rotations gives us a complication. We represent a polygon by a vertex list where we assume that the first and last elements are duplicated. We cannot simply rotate this vertex list, since this will generally give us duplicate vertices and endpoints which do not match. Instead, we should only rotate the largest non-trivial prefix of the list, and then duplicate the new head at the end. 

The definition of this sort of rotation is slightly more complex, but assuming the rotation is not the identity, we can understand it as taking some vertex inside the list and swapping it with the first and last elements. The sublists in between are then exchanged. In other words:
\begin{align*}
    \vdash_{def}\;&\code{rotation\_of}\ Ps\ Qs \iff\\
    &\qquad\begin{aligned}
    \exists P.\;\exists Q.\;\exists Ps.\;\exists Qs.\; Ps &= (\append{[P]}{\append{Ps'}{\append{[Q]}{\append{Ps''}{[P]}}}})\\
    \wedge\; Qs &= (\append{[Q]}{\append{Ps''}{\append{[P]}{\append{Ps'}{[Q]}}}})\\
    \vee\; Ps &= Qs.
  \end{aligned}
\end{align*}

We verified that this is an equivalence relation.

\subsection{Invariance}
In order to use these list rotations to justify without-loss-of-generality, we need to know that a rotation preserves the properties of interest. In our case, we need to know that a rotated polygon is still the same figure as a set of points, and we need to know that it is still \emph{simple} as a set of segments. 

The proofs are not exactly straightforward, since the set of points of a polygon and the simplicity of a polygon are defined in terms of list functions such as $\code{adjacent}$, $\code{mem}$ and $\code{pairwise}$. The reasoning involves the interplay with these functions, for which we have had to contribute many new lemmas. We found it despairing to be doing this at such a late stage, when so much of the earlier verification was focused on synthetic declarative geometry. But with perseverance, we obtained the necessary theorem:
\begin{equation}\label{eq:PolygonRotateInvariant}
\begin{aligned}
\vdash      &\code{rotation\_of}\ Ps\ Qs\\
    &\qquad\implies (\code{simple\_polygon}\ Ps \implies \code{simple\_polygon}\ Qs)\\
    &\qquad\qquad\quad\wedge \code{on\_polypath}\ Ps = \code{on\_polypath}\ Qs.
  \end{aligned}
\end{equation}

\subsection{Example}
We end this section by explaining some of the reasoning that goes into using polygon rotations in the context of the main proof for this chapter. Suppose we have a polygon $Ps$ as a vertex list, and we know we have line-of-sight to a point $X'$ on an edge of the polygon. We can formally describe the position of $X'$ with:
\begin{displaymath}
  \exists P_i.\;\exists P_{i+1}.\; \code{mem}\ (P_i,P_{i+1})\ (\code{adjacent}\ Ps) \wedge \between{P_i}{X'}{P_{i+1}}.
\end{displaymath}

The presence of $\code{adjacent}$ is troublesome, but we have verified a theorem to help us out here:
\begin{equation*}
  % (`!x:a y xs. MEM (x,y) (ADJACENT xs)
  %      <=> ?ys zs. xs = APPEND ys (CONS x (CONS y zs))`,
\vdash  \code{mem}\ (x,y)\ (\code{adjacent}\ xs) \iff 
\exists ws.\;\exists zs.\; xs = \append{ws}{\append{[x,y]}{zs}}.
\end{equation*}

This gives us just about everything we need to perform a rotation. We know that if the witnessed $ys$ is empty, then our edge $[x,y]$ must already be at the front of the list. Otherwise, we know that the list $xs$ is of the form
\begin{displaymath}
  xs = \append{[w]}{\append{ws}{\append{[x,y]}{\append{zs}{[w]}}}}
\end{displaymath}

which is a rotation of
\begin{displaymath}
  xs = \append{[x,y]}{\append{zs}{\append{[w]}{\append{ws}{[x]}}}}.
\end{displaymath}

This puts the edge at the front of the list, as we require to apply Theorem~\ref{eq:PolygonMove}. Our invariance theorem \eqref{eq:PolygonRotateInvariant} assures us that the rotation will not change the set of points defined by the vertex list, nor affect the simplicity of the polygon.

\section{Moving to Any Edge}
We are now able to move between any two edges of a polygon. This takes us a significant way towards a verification of this last half of the Polygonal Jordan Curve Theorem. 

In particular, we have shown that, given a point $X$ with line-of-sight to a point $X'$ on some edge, there must be a polygonal path-connected point $Y$ which has line-of-sight to a point $Y'$ on the \emph{first} edge. Formally, we verify:
\begin{equation*}
  % "!'a Ps X P Q hand.
  %  simple_polygon 'a Ps
  %  /\ on_plane X 'a /\ ~on_polyseg Ps X
  %  /\ MEM (P,Q) (ADJACENT Ps) /\ between P hand Q
  %  /\ ~(?Y. between hand Y X /\ on_polyseg Ps Y)
  %  ==> ?hand' X'. seg_connected 'a (on_polyseg Ps) X X'
  %                 /\ between (HD Ps) hand' (HD (TL Ps))
  %                 /\ ~on_polyseg Ps X'
  %                 /\ ~(?Y. between hand' Y X' /\ on_polyseg Ps Y)"
  \begin{split}
\vdash    &\code{simple\_polygon}\ Ps \wedge \neg\code{on\_polypath}\ Ps\ X\\
    &\wedge \code{mem}\ (P,Q)\ (\code{adjacent}\ Ps) \wedge \between{P}{X'}{Q}\\
    &\wedge\neg(\exists Z.\; \between{X}{Z}{X'} \wedge \code{on\_polypath}\ Ps\ Z)\\
    &\qquad\implies \exists Y.\;\exists Y'.\; \code{polypath\_connected}\ (\code{on\_polypath}\ Ps)\ X\ Y\\
    &\qquad\qquad\qquad\wedge\neg\code{on\_polypath}\ Ps\ Y\\
    &\qquad\qquad\qquad\wedge \between{(\code{head}\ Ps)}{Y'}{(\code{head}\ (\code{tail}\ Ps))}\\
    &\qquad\qquad\qquad\wedge \neg\exists Z.\; \between{Y}{Z}{Y'} \wedge \code{on\_polypath}\ Ps\ Z.
  \end{split}
\end{equation*}

Note that for the very first time we are working on the hypothesis that our vertex list $Ps$ defines a simple polygon. We can see why this hypothesis is needed by considering two of the hypotheses from Theorem~\ref{eq:PolygonMove}, which we have said assumes that a polygon is only ``locally'' simple:
\begin{enumerate}
\item $\neg\code{on\_polypath}\ (\cons{P_3}{Ps})\ P_2$;
\item $\neg(\exists Z.\; \between{P_1}{Z}{P_2} \wedge \code{on\_polypath}\ (\cons{P_2}{\cons{P_3}{Ps}})\ Z)$.
\end{enumerate}

By allowing $Ps$ to be an arbitrarily rotated polygon, the first of these assumptions will require that all vertices are distinct and that no vertex appears on another edge. The second will require that the edges themselves do not intersect. This is just to say that the polygonal segment $Ps$ must be simple.

To conclude this section, we will give some perspective on what is involved in the proof so far. In Figure~\ref{fig:SketchProofJordan2Full}, we have reproduced the example from \S\ref{sec:SketchProofJordan2}, showing three players navigating a maze in order to have line-of-sight to the edge $P_1P_2$. However, we can now explain the peculiar shape of the red and blue paths, understanding that they have been obtained by precise applications of Theorem~\ref{eq:Squeeze2} using the auxiliary dashed lines. Each dashed line marks a point on a player's line-of-sight that they move towards, or a point on an edge towards which they rotate. Thus, we see each player navigating without compass or ruler, using only incidence and ordering.

\begin{figure}
  \centering\includegraphics[scale=0.9]{jordanVerification2/SketchProofFull}
  \caption{Navigating a maze with Theorem~\ref{eq:Squeeze2}}
  \label{fig:SketchProofJordan2Full}
\end{figure}

\section{Final Steps}
We are almost there. To finish the proof, we will need to show that when two points are on the same side and looking at the same edge, then they are polygonal path-connected. We shall deal with this matter in \S\ref{sec:SameSideEdgeConnected}. First, we shall deal with the neglected matter of how we are able to obtain line-of-sight to an edge in the first place. Both problems will have us reusing our near ubiquitous squeeze theorem~\eqref{eq:Squeeze2}.

\subsection{Getting onto the Maze}
If we take an arbitrary point $X$ in the plane and an arbitrary point $X'$ on a maze, then a simple ray-cast gives us line-of-sight to some other point $X''$ of the maze. If this point $X''$ lies between two other vertices, we have a line-of-sight to an edge. 

However, if $X''$ coincides with a vertex as shown in Figure~\ref{fig:SightToEdge}, we will need to rotate our line-of-sight slightly. We will first assume, without loss-of-generality (or more accurately, we will assume we have rotated the polygon's vertex list) so that the vertex $X''$ coincides with the vertex $P_2$. We will want to obtain line-of-sight to either the edge $P_1P_2$ or $P_2P_3$.

\begin{figure}
  \centering\includegraphics[scale=0.75]{jordanVerification2/SightToWall}
  \caption{Obtaining line-of-sight with an edge}
  \label{fig:SightToEdge}
\end{figure}

We can do this by applying Theorem~\ref{eq:Squeeze2} to the edge $P_1P_2$ and the polygonal fragment $P_3P_4\ldots P_n$ as described in \S\ref{sec:RotateToNew}. This will not generally give us our required point of intersection. In fact, it might be that there is \emph{no} line-of-sight from $X$ to the edge $P_1P_2$, because the edge $P_2P_3$ is in the way. 

This does not pose much of a problem. If the segment $XY$ intersects $P_3X''$ at the point $Y'$, then we shall just take $Y'$ as our line-of-sight. We just need two steps for this, one exploiting our linear reasoning tactic and the other our incidence discoverer, to show that the segment $XY'$ is our required line-of-sight. On the other hand, if $XY$ does not intersect $P_3X''$, then one step with our incidence discoverer verifies that it is the required line-of-sight. 

We have formalised the result in Figure~\ref{fig:RotateToEdge}, retaining the without-loss-of-generality assumption. This allows us to review which hypotheses are actually needed for this particular theorem, and is one of the great benefits of formal verification. The labour involved in teasing out the necessary hypotheses means we usually end up with very tight lemmas with which to easily trace dependencies. Consider that this theorem appears in our theory file before simple polygons are even \emph{defined}.

\begin{figure}
\begin{equation}\label{eq:RotateToEdge}
  \begin{split}
  % "!P1 P2 P3 Ps X 'a.
  %  on_plane P1 'a /\ on_plane P2 'a /\ on_plane P3 'a /\ on_plane X 'a
  %  /\ (!P. MEM P Ps ==> on_plane P 'a)
  %  /\ ~between P1 P3 P2 /\ ~between P2 P1 P3
  %  /\ ~(P1 = P2) /\ ~(P1 = P3)
  %  /\ ~on_polyseg (CONS P3 Ps) P2
  %  /\ ~(?Z. between P1 Z P2 /\ on_polyseg (CONS P2 (CONS P3 Ps)) Z)
  %  /\ ~(?Z. between P2 Z P3 /\ on_polyseg (CONS P3 Ps) Z)
  %  /\ ~on_polyseg (CONS P1 (CONS P2 (CONS P3 Ps))) X
  %  /\ ~(?Z. between P2 Z X /\ on_polyseg (CONS P1 (CONS P2 (CONS P3 Ps))) Z)
  %  ==> ?Y. (between P1 Y P2 \/ between P2 Y P3)
  %          /\ ~(?Z. between X Z Y
  %                   /\ on_polyseg (CONS P1 (CONS P2 (CONS P3 Ps))) Z)"
\vdash&\neg\between{P_1}{P_3}{P_2} \wedge \neg\between{P_2}{P_1}{P_3} \wedge P_1 \neq P_2 \wedge P_1 \neq P_3\\
    &\wedge\neg\code{on\_polypath}\ (\cons{P_3}{Ps})\ P_2\\
    &\wedge\neg(\exists Z.\; \between{P_1}{Z}{P_2} \wedge \code{on\_polypath}\ (\cons{P_2}{\cons{P_3}{Ps}})\ Z)\\
    &\wedge\neg(\exists Z.\; \between{P_2}{Z}{P_3} \wedge \code{on\_polypath}\ (\cons{P_3}{Ps})\ Z)\\
    &\wedge\neg\code{on\_polypath}\ (\cons{P_1}{\cons{P_2}{\cons{P_3}{Ps}}})\ X\\
    &\wedge\neg(\exists Z.\; \between{P_2}{Z}{X} \wedge \code{on\_polypath}\ (\cons{P_1}{\cons{P_2}{\cons{P_3}{Ps}}})\ Z)\\
    &\implies\exists Y.\; (\between{P_1}{Y}{P_2} \vee \between{P_2}{Y}{P_3})\\
    &\qquad \wedge \neg(\exists Z.\; \between{X}{Z}{Y} \wedge \code{on\_polypath}\ (\cons{P_1}{\cons{P_2}{\cons{P_3}{Ps}}})\ Z).
  \end{split}
\end{equation}
\caption{Rotating a line-of-sight to an edge}
\label{fig:RotateToEdge}
\end{figure}

We can now get any point to have line-of-sight to the maze, and thus we can connect every point in the plane to another point with line-of-sight to any edge. This means that, if we have three points in the plane, we can find three paths which bring them together at the same edge. It is enough now to verify that two of the three points are polygonal path-connected.

\subsection{There are at Most Two Regions}\label{sec:SameSideEdgeConnected}
Here is where we are: we have three points with line-of-sight to an edge $P_1P_2$ (without loss of generality). We know that two of these points $X$ and $Y$ are on the same side of $P_1P_2$. We will show that these two points can be polygonal path-connected.

First off, we have to consider that $X$ and $Y$ have line-of-sight to \emph{different} points $X'$ and $Y'$. Our diagrams so far in this chapter have given a different impression, but we must remember that the points obtained in our proofs ultimately rely on Axiom~\ref{eq:g22}. This axiom allows us to extend a line-segment, but there are many possible witnesses for its existential conclusion, and we have generally allowed the abstraction to proliferate in our theory, rather than eliminating it with the epsilon-operator. 

\begin{figure}
\centering\includegraphics[scale=0.6]{jordanVerification2/SameSideWallConnected1}
\caption{Connecting the final points}
\label{fig:SameSideEdgeConnected}
\end{figure}

Consider the scenario depicted in Figure~\ref{fig:SameSideEdgeConnected}(a). Here, our points $X$ and $Y$ have line-of-sight to different points on the same edge, so our first order of business is to get $X$ and $Y$ to have line-of-sight to the same point. Again, we use Theorem~\ref{eq:Squeeze2} against the fragment $P_2P_3\ldots P_n$, moving $X$ along its line-of-sight to a point $X''$ so that it now has line-of-sight with $Y'$. A second application of the same theorem, moving along our new line-of-sight sees us facing the point $Y$. With the two points in each other's sights, we have the last piece of our path.

We are almost home free. We still need to factor in the fact that $X$ and $Y$ are on the same side of $P_1P_2$. This is needed because we only applied Theorem~\ref{eq:Squeeze2} to the fragment $P_2P_3\ldots P_n$. The resulting lines-of-sight will not intersect this fragment, but we will need to account for the segment $P_1P_2$ separately.

Yet again, this is settled with our theorems for half-planes. Since $X$ lies on the same side of $P_1P_2$ as $Y$, and $X'$ lies on $P_1P_2$, all points on the segment $XX''$ must also lie on the same side. The same argument applies to the segment $X''Y'$, and finally to $Y'Y$. Since all of these points are on the same side of $P_1P_2$, they must lie off the line $P_1P_2$. 

The final extract of the verified proof, witnessing the final path and showing this line of argument, is reproduced in Figure~\ref{fig:SameSideEdgeConnectedExtract}. We have omitted references to earlier steps.

\begin{equation}
  \label{eq:SameSideEdgeConnected}
  \begin{split}
  % "!P1 P2 Ps X Y hand hand' hp 'a. 
  %    on_plane P1 'a /\ on_plane P2 'a
  %    /\ (!P. MEM P Ps ==> on_plane P 'a)
  %    /\ on_plane X 'a /\ on_plane Y 'a
  %    /\ between P1 hand P2 /\ between P1 hand' P2
  %    /\ ~on_polyseg (CONS P1 (CONS P2 Ps)) X
  %    /\ ~on_polyseg (CONS P1 (CONS P2 Ps)) Y
  %    /\ ~(?Z. between X Z hand /\ on_polyseg (CONS P1 (CONS P2 Ps)) Z)
  %    /\ ~(?Z. between Y Z hand' /\ on_polyseg (CONS P1 (CONS P2 Ps)) Z)
  %    /\ ~(?Z. between P1 Z P2 /\ on_polyseg (CONS P2 Ps) Z)
  %    /\ on_line P1 (line_of_half_plane hp) /\ on_line P2 (line_of_half_plane hp)
  %    /\ on_half_plane hp X /\ on_half_plane hp Y
  %    ==> seg_connected 'a (on_polyseg (CONS P1 (CONS P2 Ps))) X Y"
\vdash &\neg\code{on\_polypath}\ (\cons{P_1}{\cons{P_2}{Ps}})\ X \wedge\neg\code{on\_polypath}\ (\cons{P_1}{\cons{P_2}{Ps}})\ Y\\
    &\wedge\between{P_1}{X'}{P_2}\wedge\between{P_1}{Y'}{P_2}\\
    &\wedge\neg(\exists Z.\; \between{X}{Z}{X'} \wedge \code{on\_polypath}\ (\cons{P_1}{\cons{P_2}{Ps}})\ Z)\\
    &\wedge\neg(\exists Z.\; \between{Y}{Z}{Y'} \wedge \code{on\_polypath}\ (\cons{P_1}{\cons{P_2}{Ps}})\ Z)\\
    &\wedge\neg(\exists Z.\; \between{P_1}{Z}{P_2} \wedge \code{on\_polypath}\ (\cons{P_2}{Ps})\ Z)\\
    &\wedge \code{on\_line}\ P1\ (\code{line\_of\_half\_plane}\ hp) \wedge \code{on\_line}\ P2\ (\code{line\_of\_half\_plane}\ hp)\\
    &\wedge \code{on\_half\_plane}\ hp\ X \wedge \code{on\_half\_plane}\ hp\ Y\\
    &\implies \code{polypath\_connected}\ (\code{on\_polypath}\ (\cons{P_1}{\cons{P_2}{Ps}}))\ X\ Y.
  \end{split}
\end{equation}

\begin{boxedfigure}
\small
\begin{align*}
  % ;take ["[X:point;s:point;s':point;Y:point]"]
  % ;tactics [REWRITE_TAC [NOT_CONS_NIL;GSYM IN_SET_OF_LIST;HD;LAST
  % ;DISJOINT_IMP;set_of_list]
  % THEN REWRITE_TAC [FORALL_IN_INSERT;NOT_IN_EMPTY]]
  % ;obviously (by_planes o Di.conjuncts)
  % (thus "on_plane s 'a /\ on_plane s' 'a 
  % /\ on_plane X 'a /\ on_plane Y 'a"
  % from [0;2;3;11;12;18])
  % ;have "on_line hand (line_of_half_plane hp)
  % /\ on_line hand' (line_of_half_plane hp)"
  % from [3;8] by [g12;g21]
  % ;hence "on_half_plane hp s /\ on_half_plane hp s'
  % /\ !Z. between s Z s' \/ between s' Z Y
  % ==> on_half_plane hp Z"
  % from [8;12;16;18] by [bet_on_half_plane;bet_on_half_plane2]
  % at [20]
  % ;have "!Z. on_half_plane hp Z ==> ~on_polyseg [P1;P2] Z"
  % from [3;8] by [on_polyseg_pair;g12;g21;half_plane_not_on_line]
  % ;hence "!Z. on_polyseg [s;s';Y] Z ==> ~on_polyseg [P1;P2] Z"
  % from [8;20] by [on_polyseg_CONS2;on_polyseg_sing] at [21]
  % ;have "!Z. between s Z s' ==> ~on_polyseg (CONS P2 Ps) Z" proof
  % [fix ["Z:point"]
  % ;assume "between s Z s'"
  % ;hence "between s Z hand'" from [18]
  % using ORDER_TAC `{hand':point,s,s',Z}`
  % ;qed from [13]]
  % ;qed from [4;13;14;18;19;21] by
  % [IN;on_polyseg_CONS2;on_polyseg_sing;BET_SYM]]]];;
  &\code{take}\ [X,X'',X''',Y]\\
  &\code{have}\ \code{on\_line}\ X'\ (\code{line\_of\_half\_plane}\ hp)\\
  &\qquad\wedge \code{on\_line}\ Y'\ (\code{line\_of\_half\_plane}\ hp)\\
  &\qquad\code{from}\ \ldots\ \code{by}\ \eqref{eq:g12},\eqref{eq:g21}\\
  &\code{hence}\ \code{on\_half\_plane}\ hp\ X'' \wedge \code{on\_half\_plane}\ hp\ X'''\\
  &\qquad\wedge\forall Z.\; \between{X''}{Z}{X'''} \vee \between{X'''}{Z}{Y}\ \code{from} \ldots\ \code{by}\ \eqref{eq:betOnHalfPlane1},\eqref{eq:betOnHalfPlane2}\\
  &\code{have}\ \forall Z.\; \code{on\_half\_plane}\ hp\ Z \implies \neg\code{on\_polypath}\ [P_1,P_2]\ Z\
  \code{from}\ 3,8\\
  &\qquad\code{by}\ \eqref{eq:OnPolyPath},\eqref{eq:g12},\eqref{eq:g21},\eqref{eq:halfPlaneNotOnLine}\\
  &\code{hence}\ \forall Z.\; \code{on\_polypath}\ [X'',X'',Y]\ Z \implies \neg\code{on\_polypath}\ [P_1,P_2]\ Z\\
  &\qquad\code{from} \ldots\ \code{by}\ \eqref{eq:OnPolyPath}\ & 21\\
  &\code{have}\ \forall Z.\; \between{X''}{Z}{X'''} \implies \neg\code{on\_polypath}\ (\cons{P_2}{Ps})\ Z\\\
  &\code{proof:}\ \code{fix}\ Z\\
  &\qquad \code{assume}\ \between{X''}{Z}{X'''}\\
  &\qquad \code{hence}\ \between{X''}{Z}{Y'}\ \code{from}\ \ldots\ \code{using}\ \code{ORDER\_TAC}\ \{X'',X''',Y',Z\}\\
  &\qquad \code{qed}\ \code{from}\ \ldots\\
  &\code{qed}\ \code{from}\ \ldots,21\ \code{by}\ \eqref{eq:OnPolyPath},\eqref{eq:g21}
\end{align*}
\caption{Verification Extract for Theorem~\ref{eq:SameSideEdgeConnected}}
\label{fig:SameSideEdgeConnectedExtract}
\end{boxedfigure}

\section{Conclusion}
The proof of the final result, Theorem~\ref{eq:jordanFormal2}, puts all of the pieces together. We start from three arbitrary points in the plane not on the polygonal segment, have each obtain a line-of-sight to an edge of the maze, and then use polygon rotations and Theorem~\ref{eq:PolygonMove} to find three paths to three points with line-of-sight to the first edge of the maze. We then apply  Theorem~\ref{eq:HalfPlaneCover} which says that two of these points must be on the same side of the first edge, and thus, from Theorem~\ref{eq:SameSideEdgeConnected}, we know that these two points are polygonal path-connected.

%As with the final theorem in the last chapter, the final theorem here is much more readable than the lemmas considered up to now. The theory so far builds theorems with large numbers of complicated hypotheses, such as those for Theorem~\ref{eq:SameSideEdgeConnected} (and remember that for brevity, we have omitted all the planar hypotheses needed for every one of these theorems). When developing a theory such as this, where theorems are almost made opaque by their hypotheses, we have to pay particularly close attention to be confident that we are making progress. Fortunately, success in formal verification is unambiguous.
\begin{equation}\tag{\ref{eq:jordanFormal2}}
\begin{aligned}
\vdash &\code{simple\_polygon}\ \alpha\ Ps\\
       &\wedge \code{on\_plane}\ P\ \alpha \wedge \code{on\_plane}\ Q\ \alpha \wedge \code{on\_plane}\ R\ \alpha\\
       &\wedge \neg\code{on\_polypath}\ Ps\ P\wedge \neg\code{on\_polypath}\ Ps\ Q\wedge \neg\code{on\_polypath}\ Ps\ R\\
       &\implies \code{polypath\_connected}\ \alpha\ (\code{on\_polypath}\ Ps)\ P\ Q\\
       &\qquad\quad\vee \code{polypath\_connected}\ \alpha\ (\code{on\_polypath}\ Ps)\ P\ R\\
       &\qquad\quad\vee \code{polypath\_connected}\ \alpha\ (\code{on\_polypath}\ Ps)\ Q\ R
\end{aligned}
\end{equation}

%%% Local Variables: 
%%% TeX-master: "../thesis"
%%% End: 


\bibliographystyle{plain}
\bibliography{../formalising}

\appendix
\chapter{Group~I Axioms and Elementary Consequences}\label{app:group1}

\section{Axioms}
\begin{equation}\label{eq:g11}
  \tag{I, 1}
    A \neq B \implies \exists a. \code{on\_line}\ A\ a \wedge \code{on\_line}\ B\ a
\end{equation}

\begin{equation}\label{eq:g12}
  \tag{I, 2}
  \begin{split}
    A \neq B &\wedge \code{on\_line}\ A\ a \wedge \code{on\_line}\ B\ a\\
    &\wedge \code{on\_line}\ A\ b \wedge \code{on\_line}\ B\ b\\
    &\implies a = b
  \end{split}
\end{equation}

\begin{equation}\label{eq:g16}
  \tag{I, 6}
  \begin{split}
    A \neq B &\wedge \code{on\_plane}\ A\ \alpha \wedge \code{on\_plane}\ B\ \alpha\\
    &\wedge \code{on\_line}\ A\ a \wedge \code{on\_line}\ B\ a\\
    &\implies \code{on\_line}\ P\ a \implies \code{on\_plane}\ P\ \alpha
  \end{split}
\end{equation}

\section{Elementary Consequences}
\begin{multline}\label{eq:PlaneThree}
  \exists A\;B\;C. \code{on\_plane}\ A\ \alpha\wedge\code{on\_plane}\ B\ \alpha\wedge\code{on\_plane}\ C\ \alpha\\
  \wedge \Triangle{a}{A}{B}{C}
\end{multline}

%%% Local Variables: 
%%% mode: latex
%%% TeX-master: "thesis"
%%% End: 

\chapter{Elementary Consequences of Group~II}\label{app:Group2}

\begin{equation}
    \tag{\ref{eq:PaschPointSet}}
  \begin{aligned}
    &\neg\code{collinear}\ \{A,B,C\}\wedge\neg\code{collinear}\ \{A,D,E\}\wedge\neg\code{collinear}\ \{C,D,E\}\\
    &\wedge\code{planar}\ \{A,B,C,D,E\}\wedge\between{A}{D}{B}\\
    &\implies\exists F.\; \code{collinear}\ \{D,E,F\} \wedge (\between{A}{F}{C}\vee\between{B}{F}{C})
  \end{aligned} 
\end{equation}

\begin{equation}\label{eq:PaschPointSetUnfold}
  \begin{split}
  % |- !A B C D E.
  %        (~(?a. on_line A a /\ on_line B a /\ on_line C a) /\
  %         ~(?a. on_line A a /\ on_line D a /\ on_line E a) /\
  %         ~(?a. on_line C a /\ on_line D a /\ on_line E a) /\
  %         (?'a. on_plane A 'a /\
  %               on_plane B 'a /\
  %               on_plane C 'a /\
  %               on_plane D 'a /\
  %               on_plane E 'a)) /\
  %        between A D B
  %        ==> (?Fa. (?a. on_line D a /\ on_line E a /\ on_line Fa a) /\
  %                  (between A Fa C \/ between B Fa C))
    &\Triangle{a}{A}{B}{C}\\
    &\wedge\Triangle{a}{A}{D}{E}\\
    &\wedge\Triangle{a}{C}{D}{E}\\
    &\wedge(\exists\alpha.\; \code{on\_plane}\ A\ \alpha\wedge \code{on\_plane}\ B\ \alpha\wedge \code{on\_plane}\ C\ \alpha\\
    &\qquad\wedge \code{on\_plane}\ D\ \alpha\wedge \code{on\_plane}\ E\ \alpha)\\
    &\wedge\between{A}{D}{B}\\
    &\implies\exists F.\; (\exists a.\; \code{on\_line}\ D\ a\wedge \code{on\_line}\ E\ a\wedge \code{on\_line}\ F\ a)\\
    &\qquad\qquad (\between{A}{F}{C}\vee\between{B}{F}{C})
  \end{split}
\end{equation}

\begin{equation}
  \tag{\ref{eq:OuterPasch}}
  % "!A B C D X. 
  %  ~(?a. on_line A a /\ on_line B a /\ on_line C a)
  %  /\ between A B D /\ between A X C
  %  ==> ?Y. between D Y X /\ between B Y C"
  \begin{split}
    &\Triangle{a}{A}{B}{C}\\ & \wedge \between{B}{C}{D} \wedge \between{A}{E}{C}\\ 
    &\qquad\implies \exists F.\; \between{D}{E}{F} \wedge \between{A}{F}{B}
  \end{split}
\end{equation}

\begin{equation}
  \tag{\ref{eq:InnerPasch}}
  % "!A B C D X. 
  %    ~(?a. on_line A a /\ on_line B a /\ on_line C a)
  %    /\ between A B D /\ between B X C
  %    ==> ?Y. between D X Y /\ between A Y C"
  \begin{split}
    &\Triangle{a}{A}{B}{C}\\ & \wedge \between{B}{C}{D} \wedge \between{A}{E}{B}\\ 
    &\qquad\implies \exists F.\; \between{D}{F}{E} \wedge \between{A}{F}{C}
  \end{split}
\end{equation}

\begin{equation}\label{eq:three}\tag{THEOREM~3}
% "!A C. ~(A = C) ==> ?D. between A D C"
A \neq C \implies \exists D.\; \between{A}{D}{C}
\end{equation}

\begin{equation}\label{eq:four}\tag{THEOREM~4}
  \begin{split}
%    "!A B C a. on_line A a /\ on_line B a /\ on_line C a " ^
%      "==> ~(A = B) /\ ~(A = C) /\ ~(B = C) " ^
%      "==> between A B C \/ between B A C \/ between A C B"
    &\code{on\_line}\ A\ a \wedge \code{on\_line}\ B\ a \wedge \code{on\_line}\ C\ a \\
    &\wedge\; A \neq B \wedge A \neq C \wedge B \neq C\\
    &\implies \between{A}{B}{C} \vee \between{B}{A}{C} \vee \between{A}{C}{B}
  \end{split}
\end{equation}

\begin{equation}\label{eq:five}\tag{THEOREM~5}
  \begin{split}
% |- !A B C D.
%          between A B C /\ between B C D ==> between A B D /\ between A C D
% |- !A B C D.
%          between A B C /\ between A C D ==> between A B D /\ between B C D
    &(\between{A}{B}{C} \wedge \between{B}{C}{D} \implies \between{A}{B}{D} \wedge \between{A}{C}{D})\\
    &\wedge(\between{A}{B}{C} \wedge \between{A}{C}{D} \implies \between{A}{B}{D} \wedge \between{B}{C}{D})
  \end{split}
\end{equation}

\begin{equation*}
  A \neq B \implies \code{infinite}\ \{ P \vert \between{A}{P}{B} \}
\end{equation*}

\begin{equation}\tag{\ref{eq:SupplementI}}
  % |- !A B C D E Fa.
  %        ~on_line A a /\
  %        ~on_line B a /\
  %        ~on_line C a /\
  %        on_line D a /\
  %        on_line E a /\
  %        on_line Fa a
  %        ==> ~between A D B \/ ~between A E C \/ ~between B Fa C
  \begin{split}
    &\neg\code{on\_line}\ A\ a \wedge \neg\code{on\_line}\ B\ a \wedge \neg\code{on\_line}\ C\ a\\
    &\wedge \code{on\_line}\ D\ a \wedge \code{on\_line}\ E\ a \wedge \code{on\_line}\ F\ a\\
    &\implies \neg\between{A}{D}{B} \vee \neg\between{A}{E}{C} \vee \neg\between{B}{F}{C}
  \end{split}
\end{equation}

\section{Half-Planes}
\begin{equation*}
  \exists P.\; \code{on\_half\_plane}\ hp\ P
\end{equation*}

\begin{equation*}
  % |- !P p q.
  %        on_half_plane p P
  %        ==> on_half_plane q P
  %        ==> line_of_half_plane p = line_of_half_plane q
  %        ==> p = q
  \begin{split}
    &\code{on\_half\_plane}\ hp\ P \wedge \code{on\_half\_plane}\ hq\ P\\
    &\wedge\code{line\_of\_half\_plane}\ hp = \code{line\_of\_half\_plane}\ hq\\
    &\implies hp = hq
  \end{split}
\end{equation*}

\begin{equation*}
  \begin{split}
    &\Triangle{a}{A}{B}{C}\\
    &\implies \exists! hp.\; \code{on\_line}\ A\ (\code{line\_of\_half\_plane}\ hp)\\
    &\qquad\qquad\wedge\code{on\_line}\ B\ (\code{line\_of\_half\_plane}\ hp)\\
    &\qquad\qquad\wedge\code{on\_half\_plane}\ hp\ C
\end{split}
\end{equation*}

\begin{equation}\tag{\ref{eq:HalfPlaneCover}}
  \begin{split}
    &(\forall P.\; \code{on\_line}\ P\ a \implies \code{on\_plane}\ P\ \alpha)\\
    &\implies \exists hp\; hq.\; hp \neq hq\\
    &\qquad \wedge a = \code{line\_of\_half\_plane}\ hp \wedge a = \code{line\_of\_half\_plane}\ hq\\
    &\qquad \wedge (\forall P.\; \code{on\_plane}\ P\ \alpha\\
    &\qquad\qquad \iff \code{on\_line}\ P\ a \vee \code{on\_half\_plane}\ hp\ P \vee \code{on\_half\_plane}\ hq\ P)
  \end{split}
\end{equation}

\begin{equation}\tag{\ref{eq:onHalfPlaneNotBet}}
  \begin{split}
    &(\forall R.\; \code{on\_half\_plane}\ hp\ R \implies \code{on\_plane}\ R\ \alpha) \wedge \code{on\_half\_plane}\ hp\ P\\
    &\implies (\code{on\_half\_plane}\ hp\ Q\\
    &\qquad \iff \neg(\exists R.\; \code{on\_line}\ R\ (\code{line\_of\_half\_plane}\ hp) \wedge \between{P}{R}{Q})\\
    &\qquad\qquad\quad \wedge \code{on\_plane}\ Q\ \alpha \wedge \neg\code{on\_line}\ Q\ (\code{line\_of\_half\_plane hp}))
  \end{split}
\end{equation}

\begin{equation}\tag{\ref{eq:betOnHalfPlane1}}
%!hp P Q R. on_line P (line_of_half_plane hp)
%      /\ on_half_plane hp Q
%      /\ (between P Q R \/ between P R Q) ==> on_half_plane hp R
  \begin{split}
    &\code{on\_line}\ P\ (\code{line\_of\_half\_plane}\ hp) \wedge \code{on\_half\_plane}\ hp\ Q\\
    &\implies \between{P}{Q}{R} \vee \between{P}{R}{Q} \implies \code{on\_half\_plane}\ hp
  \end{split}
\end{equation}

\begin{equation}\tag{\ref{eq:betOnHalfPlane2}}
  \begin{split}
  % "!P Q R hp. on_half_plane hp P /\ on_half_plane hp R /\ between P Q R 
  %             ==> on_half_plane hp Q"
    &\code{on\_half\_plane}\ hp\ P \wedge \code{on\_half\_plane}\ hp\ R\\
    &\implies \between{P}{Q}{R} \implies \code{on\_half\_plane}\ hp\ Q
  \end{split}
\end{equation}

\section{Rays}
\begin{equation*}
  \exists P.\; \code{on\_ray}\ r\ P
\end{equation*}

\begin{equation*}
% |- !r s.
%          on_ray r P ==> on_ray s P ==> ray_origin r = ray_origin s ==> r = s
    \code{on\_ray}\ r\ P \wedge \code{on\_ray}\ s\ P\wedge\code{ray\_origin}\ r = \code{ray\_origin}\ s\implies r = s
\end{equation*}

\begin{equation*}
% |- !a P r Q.
%          on_ray r P /\ on_line P a /\ on_line (ray_origin r) a
%          ==> on_ray r Q
%          ==> on_line Q a
  \begin{split}
    &\code{on\_ray}\ r\ P \wedge \code{on\_line}\ P\ a \wedge \code{on\_line}\ (\code{ray\_origin}\ r)\ a\\
    &\implies \code{on\_ray}\ r\ Q \implies \code{on\_line}\ Q\ a
  \end{split}
\end{equation*}

\begin{equation*}
% |- !r. ~on_ray r (ray_origin r)
\neg\code{on\_ray}\ r\ (\code{ray\_origin}\ r)
\end{equation*}

\begin{equation*}
         % on_line P a
         % ==> (?r s.
         %          ~(r = s) /\
         %          P = ray_origin r /\
         %          P = ray_origin s /\
         %          (!X. on_line X a <=>
         %               X = ray_origin r \/ on_ray r X \/ on_ray s X))
  \begin{split}
    &\code{on\_line}\ P\ a\\
    &\qquad\implies \exists r\ s.\; r \neq s\wedge P = \code{ray\_origin}\ r \wedge Q = \code{ray\_origin}\ s\\
    &\qquad\qquad \wedge (\forall X.\; \code{on\_line}\ X\ a \iff X = \code{ray\_origin}\ r \vee \code{on\_ray}\ r\ X \vee \code{on\_ray}\ s\ X)
    \end{split}
\end{equation*}

\begin{equation*}
  \begin{split}
% |- !a r P Q.
%          (!R. on_ray r R ==> on_line R a) /\ on_ray r P
%          ==> (on_ray r Q <=>
%               ~(Q = ray_origin r) /\
%               ~between P (ray_origin r) Q /\
%               on_line Q a)
    &(\forall R.\; \code{on\_ray}\ r\ R \implies \code{on\_line}\ R\ a) \wedge \code{on\_ray}\ r\ P\\
    &\qquad\implies (\code{on\_ray}\ r\ Q\\
    &\qquad\qquad\iff Q \neq \code{ray\_origin}\ r \wedge \neg\between{P}{(\code{ray\_origin}\ r)}{Q} \wedge \code{on\_line}\ Q\ a
  \end{split}
\end{equation*}
%%% Local Variables: 
%%% TeX-master: "thesis"
%%% End: 

\chapter{Polygonal Jordan Curve Theorem: Full Specification}
\label{app:JordanVerification}

\section{HOL Light List and Set Library}
Function specifications marked with $\dag$ have been contributed.
\begin{alignat*}{3}
  & \code{head}\,:\,[\alpha] \rightarrow \alpha &\qquad
  & \code{tail}\,:\,[\alpha] \rightarrow [\alpha]\dag &\qquad
  & \code{length}\ [\alpha]\rightarrow \mathbb{N}\\
  & \vdash_{def}\code{head}\ (\cons{x}{xs}) = x &\qquad
  & \vdash_{def}\code{tail}\ [] = [] &\qquad
  & \vdash_{def}\code{length}\ [] = 0\\
  & &\qquad
  &\vdash_{def}\code{tail}\ (\cons{x}{xs}) = xs &\qquad
  & \vdash_{def}\code{length}\ (\cons{x}{xs}) = \code{length}\ xs + 1
\end{alignat*}
\begin{align*}
  &\code{butlast}\,:\,[\alpha] \rightarrow [\alpha]\\
  &\vdash_{def}\code{butlast}\ [] = []\\
  &\vdash_{def}\code{butlast}\ (\cons{x}{xs}) = \code{if}\ xs=[]\ \code{then}\ []\ \code{else}\ \cons{x}{(\code{butlast}\ xs)}
\end{align*}
\begin{align*}
  &\code{el}\,:\,\code{int} \rightarrow [\alpha] \rightarrow \alpha\\
  &\code{el}\vdash_{def}\ 0\ xs = \code{head}\ xs\\
  &\code{el}\vdash_{def}\ (\code{suc}\ n)\ xs = \code{el}\ n\ (\code{tail}\ xs)
\end{align*}
\begin{align*}
  & \code{mem}\,:\,\alpha \rightarrow [\alpha] \rightarrow \code{bool} &\qquad\qquad
  & \code{all}\,:\,(\alpha\rightarrow\code{bool}) \rightarrow [\alpha] \rightarrow \code{bool}\\
  & \code{mem}\vdash_{def}\ x\ [] = \bot &\qquad\qquad
  & \code{all}\vdash_{def}\ p\ [] = \bot\\
  & \code{mem}\vdash_{def}\ x\ (\cons{y}{ys}) = x = y \vee \code{mem}\ x\ ys &\qquad\qquad
  & \code{all}\vdash_{def}\ p\ (\cons{x}{xs}) = p\ x \wedge \code{all}\ p\ xs   
\end{align*}
\begin{align*}
  & \code{pairwise}\,:\,(\alpha \rightarrow \alpha \rightarrow \code{bool}) \rightarrow [\alpha] \rightarrow \code{bool}\\
  & \code{pairwise}\vdash_{def}\ R\ [] = \top\\
  & \code{pairwise}\vdash_{def}\ R\ (\cons{x}{xs}) = \code{all}\ (R x)\ xs \wedge \code{pairwise}\ R\ xs
\end{align*}
\begin{align*}
  & \code{zip}\,:\,[\alpha]\rightarrow[\beta]\rightarrow[(\alpha,\beta)]\\
  & \code{zip}\vdash_{def}\,[]\,[] = []\\
  & \code{zip}\vdash_{def}\,(\cons{x}{xs})(\cons{y}{ys}) = \cons{(x,y)}{\code{zip}\ xs\ ys}
\end{align*}
\begin{align*}
  & \code{adjacent}\,:\,[\alpha]\rightarrow[(\alpha,\alpha)]\dag\\
  & \code{adjacent}\vdash_{def}\,xs = \code{zip}\ (\code{butlast}\ xs)\ (\code{tail}\ xs)
\end{align*}
\begin{align*}
  & \code{disjoint}\,:\,(\alpha\rightarrow\code{bool})\rightarrow(\alpha\rightarrow\code{bool})\rightarrow\code{bool}\\
  & \code{disjoint}\vdash_{def}\ S\ T = S \cap T = \emptyset
\end{align*}

\section{Polygon Definitions}
\begin{align*}
  &\code{on\_polypath}\,:\,[\code{point}] \rightarrow \code{point} \rightarrow \code{bool}\\
  &\code{on\_polypath}\vdash_{def}\ Ps\ P \iff\\
  &\quad\code{mem}\ P\ Ps\vee\,\exists x\ y.\; \code{mem}\ (x,y)\ (\code{adjacent}\ Ps) \wedge \between{x}{P}{y}.
\end{align*}

\begin{align*}
  &\code{polypath\_connected}\,:\,\code{plane}\rightarrow(\code{point} \rightarrow \code{bool}) \rightarrow \code{point} \rightarrow \code{point} \rightarrow \code{bool}\\
  &\code{polypath\_connected}\vdash_{def}\ \alpha\ figure\ P\ Q \iff\\
  &\quad\exists path.\; path \neq []\\
  &\qquad\wedge\,(\forall R.\ \code{mem}\ R\ path \implies \onplane{R}{\alpha})\\
  &\qquad\wedge\,\code{head}\ path = P \wedge \code{last}\ path = Q\\
  &\qquad\wedge\,\code{disjoint}\ (\code{on\_polypath}\ path)\ figure.
\end{align*}

\begin{align*}
  &\code{simple\_polygon}\,:\,[\code{point}] \rightarrow \code{bool}\\
  &\code{simple\_polygon}\vdash_{def}\ Ps \iff \\
  &\qquad 3 \leq \code{length}\ Ps\\
  &\qquad\wedge \code{head}\ ps = \code{last}\ Ps\\
  &\qquad\wedge (\forall P.\;\code{mem}\ P\ Ps \implies \code{on\_plane}\ P\ \alpha)\\
  &\qquad\wedge \code{pairwise}\ (\neq)\ (\code{butlast}\ Ps)\\
  &\qquad\wedge \neg(\exists P\;Q\;X.\; \code{mem}\;X\;Ps\wedge\code{mem}\;(P,Q)\;(\code{adjacent}\;Ps)\wedge\between{P}{X}{Q})\\
  &\qquad\wedge \code{pairwise}\ (\lambda(P,Q)\;(P',Q').\\
  &\qquad\qquad\neg(\exists X. \between{P}{X}{P'} \wedge \between{Q}{X}{Q'})\ (\code{adjacent}\;Ps)).
\end{align*}

\section{Theorems}
\begin{align*}
\vdash &\code{on\_plane}\ P1\ \alpha \wedge \code{on\_plane}\ P2\ \alpha \wedge \code{on\_plane}\ Q1\ \alpha \wedge \code{on\_plane}\ Q2\ \alpha\\
       &\wedge(\forall X. \code{mem}\ X\ Ps \implies \code{on\_plane}\ X\ \alpha) \wedge (\forall X. \code{mem}\ X\ Qs \implies \code{on\_plane}\ X\ \alpha)\\
       &\wedge\Triangle{a}{P1}{P2}{Q1}\\
       &\wedge\Triangle{a}{Q1}{Q2}{P1}\\
       &\wedge\code{between}\ P1\ X\ P2 \wedge \code{between}\ Q1\ X\ Q2\\
       &\wedge P1 = \code{last}\ (\cons{P2}{Ps}) \wedge Q1 = \code{last}\ (\cons{Q2}{Qs})\\
       &\implies \exists Y. \code{on\_polyseg}\ (\cons{P2}{Ps}\ Y) \wedge \code{on\_polyseg}\ (\cons{Q1}{\cons{Q2}{Qs}})\ Y\\
       &\qquad\vee \code{on\_polyseg}\ (\cons{P1}{\cons{P2}{Ps}})\ Y) \wedge \code{on\_polyseg}\ (\cons{Q2}{Qs})\ Y
\end{align*}

\begin{align*}
\vdash &\code{simple\_polygon}\ \alpha\ Ps\\
       &\implies \exists P\ Q.\; \code{on\_plane}\ P\ \alpha \wedge \code{on\_plane}\ Q\ \alpha\\
       &\qquad\wedge \neg\code{on\_polyseg}\ Ps\ P \wedge \neg\code{on\_polyseg}\ Ps\ Q\\
       &\qquad\wedge \neg\code{seg\_connected}\ \alpha\ (\code{on\_polyseg}\ Ps)\ P\ Q
\end{align*}

\begin{align*}
\vdash &\code{simple\_polygon}\ \alpha\ Ps\\
       &\wedge \code{on\_plane}\ P\ \alpha \wedge \code{on\_plane}\ Q\ \alpha \wedge \code{on\_plane}\ R\ \alpha\\
       &\wedge \neg\code{on\_polyseg}\ Ps\ P\wedge \neg\code{on\_polyseg}\ Ps\ Q\wedge \neg\code{on\_polyseg}\ Ps\ R\\
       &\implies \code{seg\_connected}\ \alpha\ (\code{on\_polyseg}\ Ps)\ P\ Q\\
       &\qquad\quad\vee \code{seg\_connected}\ \alpha\ (\code{on\_polyseg}\ Ps)\ P\ R\\
       &\qquad\quad\vee \code{seg\_connected}\ \alpha\ (\code{on\_polyseg}\ Ps)\ Q\ R
\end{align*}


%    & &\quad
%    & &\quad
%    & &\quad
%    &\quad\code{else}\ \cons{x}{(\code{butlast}\ xs)}
  % \begin{alignedat}{2}
  %   &\code{butlast}\ [] = [] &\qquad & \code{}\ 
  %   &\code{init}

%  &\code{head}\ [] = [] & & & 
%  &\code{head}\ (\cons{x}{xs})
  
%   &\code{tail}\,:\,[\alpha] \rightarrow \alpha\\
%   &\code{tail}\ []             = []\dag\\
%   &\code{tail}\ (\cons{x}{xs}) = xs\\
%   &\code{butlast}\ []             = []\\
%   &\code{butlast}\ (\cons{x}{xs}) = \code{if}\ xs=[]\ \code{then}\ []\\
%                                  &\qquad\code{else}\ \cons{x}{(\code{butlast}\ xs)}\\
%   &\code{length}\ [] = 0\\
%   &\code{length}\ (\cons{x}{xs}) = 1 + \code{length}\ xs\\
%   &\code{last}\ xs = \code{el}\ (\code{length} - 1)\ xs\\
%   &\append{[]}{ys} = ys\\
%   &\append{(\cons{x}{xs})}{ys} = \cons{x}{(\append{xs}{ys})}\\
%   &\code{zip}\ []\  []                = []\\
%   &\code{zip}\ (x:xs)\ (y:ys) = \cons{(x,y)}{\code{zip}\ xs\ ys}\\
%   &\code{all}\ P\ [] \iff \top\\
%   &\code{all}\ P\ (\cons{x}{xs}) \iff P\ h \wedge \code{all}\ xs\\
%   &\code{pairwise}\ P\ [] \iff \top\\
%   &\code{pairwise}\ P\ (\cons{x}{xs}) \iff \code{all}\ (P\ x)\ xs \wedge \code{pairwise}\ xs\\
%   &\code{mem}\ P\ xs \iff \code{set\_of\_list} xs \subseteq P\\
%   &\code{adjacent}\ xs = \code{zip} (\code{butlast}\ xs) (\code{tail}\ xs)\\
%   \\
%   &\code{disjoint}\ s\ t \iff s \cap t = \emptyset
% \end{align*}

% \section{Definition of Polygons}
% \begin{align*}
%   \code{on\_polyseg}\ p\ P \iff \code{mem}\ P\ p \vee \exists x\ y. \code{mem}\ (x,y)\ (\code{adjacent}\ p) \wedge \between{x}{P}{y}
%   \code{head}\ (\cons{x}{xs}) &= x\\
%   \code{tail}\ []             &= []\dag\\
%   \code{tail}\ (\cons{x}{xs}) &= xs\\
%   \code{butlast}\ []             &= []\\
%   \code{butlast}\ (\cons{x}{xs}) &= \code{if}\ xs=[]\ \code{then}\ []\\
%                                  &\qquad\code{else}\ \cons{x}{(\code{butlast}\ xs)}\\
%   \code{length}\ [] &= 0\\
%   \code{length}\ (\cons{x}{xs}) &= 1 + \code{length}\ xs\\
%   \code{last}\ xs &= \code{el}\ (\code{length} - 1)\ xs\\
%   \append{[]}{ys} &= ys\\
%   \append{(\cons{x}{xs})}{ys} &= \cons{x}{(\append{xs}{ys})}\\
%   \code{zip}\ []\  []                &= []\\
%   \code{zip}\ (x:xs)\ (y:ys) &= \cons{(x,y)}{\code{zip}\ xs\ ys}\\
%   \code{all}\ P\ [] &\iff \top\\
%   \code{all}\ P\ (\cons{x}{xs}) &\iff P\ h \wedge \code{all}\ xs\\
%   \code{pairwise}\ P\ [] &\iff \top\\
%   \code{pairwise}\ P\ (\cons{x}{xs}) &\iff \code{all}\ (P\ x)\ xs \wedge \code{pairwise}\ xs\\
%   \code{mem}\ P\ xs &\iff \code{set\_of\_list} xs \subseteq P\\
%   \code{adjacent}\ xs = \code{zip} (\code{butlast}\ xs) (\code{tail}\ xs)\\
%   \\
%   \code{disjoint}\ s\ t \iff& s \cap t = \emptyset
% \end{align*}

%%% Local Variables: 
%%% TeX-master: "../thesis"
%%% End: 

\label{app:JordanVerificationExtra}
\chapter{Polygonal Jordan Curve Theorem: Supporting Theorems}

\begin{equation}\label{eq:inTriangleNcol}
\code{in\_triangle}\ (A,B,C)\ P \implies \Triangle{a}{A}{B}{C}
\end{equation}

\begin{equation}
\begin{aligned}
% !was_inside A B B' C C' P Ps Qs.
%       Qs = CONS P (APPEND Ps [P])
%       /\ on_plane A 'a /\ on_plane B 'a /\ on_plane C 'a /\ on_plane C' 'a
%       /\ (!X. MEM X Qs ==> on_plane X 'a)
%       /\ ~(?a. on_line A a /\ on_line B a /\ on_line C a)
%       /\ ~(?a. on_line A a /\ on_line B' a /\ on_line C' a)
%       /\ ~on_polyseg Qs A
%       /\ (between A B B' \/ between A B' B \/ ~(A = B) /\ B = B')
%       /\ ~(?X. on_polyseg [B; B'] X /\ on_polyseg Qs X)
%       ==> ?was_inside'.
%             polyseg_crossings (A,B,C) (polyseg_new_was_inside (A,B,C)
%                                           was_inside (ADJACENT Qs))
%             (ADJACENT Qs)
%           = polyseg_crossings (A,B',C') (polyseg_new_was_inside (A,B',C')
%                                            was_inside' (ADJACENT Qs))
%             (ADJACENT Qs)
    &Qs = \append{[P]}{\append{Ps}{[P]}}\\
    &\wedge \code{on\_plane}\ A\ \alpha \wedge \code{on\_plane}\ B\ \alpha \wedge \code{on\_plane}\ C\ \alpha \wedge \code{on\_plane}\ C'\ \alpha\\
    &(\forall X. \code{mem}\ X\ Qs \implies \code{on\_plane}\ X\ \alpha)\\
    &\neg\code{on\_polyseg} Qs A\\
    &\wedge (\between{A}{B}{B'} \vee \between{A}{B'}{B} \vee A\neq B \wedge B\neq B')\\
    &\wedge \neg(\exists. \code{on\_polyseg}\ [B,B']\ X \wedge \code{on\_polyseg}\ Qs\ X)\\
    &\implies \forall C\;C'. \Triangle{a}{A}{B}{C}\\
    &\qquad\qquad\wedge \Triangle{a}{A}{B'}{C'}\\
    &\qquad\implies \exists \Gamma'. \code{polyseg\_crossings}\ (A,B,C)\ (\Gamma_{final}\ (A,B,C)\ \Gamma\ (\code{adjacent}\ Qs))\\
    &\qquad\qquad\qquad\qquad(\code{adjacent}\ Qs)\\
    &\qquad\qquad = \code{polyseg\_crossings}\ (A,B',C')\ (\Gamma_{final}\ (A,B',C')\ \Gamma'\ (\code{adjacent}\ Qs))\\
    &\qquad\qquad\qquad\qquad(\code{adjacent}\ Qs)\\
\end{aligned}\tag{\ref{eq:changeTriangle}}
\end{equation}

%%% Local Variables: 
%%% mode: latex
%%% TeX-master: "thesis"
%%% End: 

\chapter{Translations}\label{app:Translations}

Throughout the thesis, we have tried to use conventional notation from mathematics (including type theory). This is not quite the syntax used by HOL~Light, which lacks unicode support to render the symbols correctly. We also use various pieces of sugar to keep things more elegant.

\begin{center}
  \begin{tabular}{|c|c|}
    \hline
    Notation   & Translation \\
    \hline
    $\neg$     & \code{\tt\char`~} \\
    $\wedge$   & \code{/{\tt\char`\\}} \\
    $\vee$     & \code{{\tt\char`\\}/} \\
    $\implies$ & \code{==>} \\
    $\iff$     & \code{<=>} \\
    $\forall$  & \code{!} \\
    $\exists$  & \code{?} \\
    $\lambda x.$  & \code{\tt\char`\\ x.} \\
  $P \neq Q$ & \code{\tt\char`~(P=Q)}\\
  $\alpha$, $\beta$, $\gamma$ & \code{'a}, \code{'b}, \code{'c}\\
  $x,y \in A,B$ & \code{x IN A /{\tt\char`\\} y IN A /{\tt\char`\\} x IN B /{\tt\char`\\} y IN A}\\
  \hline
  \end{tabular}
\end{center}
%%% Local Variables: 
%%% mode: latex
%%% TeX-master: "../thesis"
%%% End: 


\end{document}

