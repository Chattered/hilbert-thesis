\chapter{Introduction}\label{chapter:Introduction}
This project started as an attempt to bring computer aided verification to bear on a classic text of modern mathematics: David Hilbert's \emph{Foundations of Geometry}. When we began, the idea had already gathered some interest in the formal verification community. Dehlinger et al~\cite{DehlingerFOG} had tried to reformulate Hilbert's axiomatics along constructivist lines, and we were initially building on work by Meikle and Fleuriot~\cite{MeikleFleuriotFormalizingHilbert}, who tried to address questions concerning the logical accuracy of Hilbert's deductions. 

During the research, our broad aims gradually shrank as the depth grew. By the end, we had unearthed unexpected findings and made fine-grained observations in just the first two of Hilbert's five groups of axioms, groups which Hilbert had largely neglected. These axioms characterise, not the full Euclidean geometry which Hilbert developed, but a much weaker, non-metrical \emph{ordered geometry}.

In this chapter, we give a brief introduction to the various historical strands that can be thought of as making the verification of Hilbert's axiomatics a natural aim for the theorem proving community. The details discussed in this section are elaborated by Kleiner~\cite{RigourProof} and Nagel~\cite{NagelModernGeometry}.

\section{The Early Axiomatic Method in Geometry}
We motivated our earlier work~\cite{ScottMScThesis} verifying Hilbert's \emph{Foundations of Geometry} as concluding a story that began with what is probably the most influential~\cite{BoyerEuclidInfluence} mathematical text ever written: Euclid's \emph{Elements}~\cite{HeathElements}. 

The \emph{Elements} was probably an introductory textbook to Greek geometry. It described in meticulous detail how one can construct and reason about simple geometric diagrams using a ruler and compass, how one can do algebra where numbers are understood to refer to line segments, how one can develop a theory of proportions or ratios in geometrical terms, and prove theorems in what we would nowadays regard as number theory. 

The results contained in Euclid's text were probably already well-known at the time, and the book was not even the first of its kind. Its legacy is presumably due to the extent to which Euclid was able to systematise so much of his contemporary knowledge, and perhaps to place it on an extremely secure footing where it was safe from the foundational crisis in the discovery of incommensurable magnitudes. Euclid was able to recover, from readily accepted axioms, a wealth of interesting geometrical results by systematically deriving them from just ten postulates or axioms~\footnote{We have grouped Euclid's ``postulates'' and ``common notions'' together here.}. This feat has inspired generations since. Aubrey gives a delightful account of Thomas Hobbes expressing scepticism at a mathematical proposition in Euclid's \emph{Elements}, and only becoming convinced as he traced the cross-referenced hypotheses through logical derivations all the way back to Euclid's original axioms~\cite{ElementaryGeometryRoe}. Now, over two millenia later, the structure of Euclid's text, broken down as it is into axioms, definitions, theorems and proofs, still characterises the structure of contemporary pure mathematics.

The simplicity of Euclid's axioms is still remarkable. Euclid permits himself only the simplest geometric toolkit, captured by five geometric axioms, with which he must then obtain and reason about all other constructions and geometric theory. He effectively imposes a severe handicap on himself, not even permitting himself the means to draw a line segment of given length emanating from a given point. The permissibility of even this simple technique must be carefully demonstrated using the much more primitive toolkit.

This shall be a theme running through the current thesis. Hilbert himself imposes an even stricter handicap. As we shall see in later chapters, it is not clear how far Hilbert respects this handicap, since he spends undue effort deriving fairly obvious results, while glossing over results that turn out to be fiendishly complicated. The thesis ends with a detailed account of and formalisation of a complex theorem which Hilbert declared was easily proved.

\subsection{Rigorisation}
Modern pure mathematics is partly characterised by a commitment to logical correctness and Euclid's \emph{Elements} went largely uncontested in this respect until the 19th century. Here, the tide was generally turning against the sloppy approaches in mathematics, such as those which seemed to plague the calculus, the foundations of which, as Berkeley had argued~\cite{BerkeleyNewton}, were hardly clear. In the 19th century, Cauchy's ideas, refined by Weierstrass, partly settled the matter by introducing the modern epsilon-delta definitions~\cite{RigorousCalculus}, with Dedekind closing the final gaps by giving a model for the elusive \emph{continuum}~\cite{DedekindsCuts}. 

At the same time, the priority of geometry was being contested by the successes of algebraic as opposed to synthetic approaches based on Cartesian coordinates, while the whole subject itself was taking an abstract turn in both synthetic geometry, with the introduction of ideal points in projective geometry, and the growing acceptance of imaginary and negative numbers in algebra. Here, the axiomatic method was able to serve a new purpose in capturing the emerging abstractions, which would eventually be refined in the form of axiomatisations for abstract algebra, topology and measure theory to name a few.

Synthetic geometry was revitalised along rigorous lines by Pasch, who has been called the ``father of rigour in geometry''~\cite{PaschToPeano}. He developed his own axiomatic system for geometry, and in so doing, closed a major logical gap in Euclid's \emph{Elements} by introducing axioms for a \emph{betweenness} relation. He then secured the logical correctness by insisting that all logical deductions must take place without reference to the intuitive meaning of terms such as ``point'', ``line'' and ``between.'' 

Partly embracing the abstract turn in the 19th century, Pasch initially justified his axioms as idealisations of empirical hypotheses, but as his treatise progresses, he embraces a more abstract view by allowing the meaning of his primitive notions to be generalised. This is needed, in part, to explain the theory of duality, wherein the terms ``point'' and ``line'' can be swapped in theorems without changing their validity.

Finally, we mention Peano, who recast Pasch's axiomatisation along modern symbolic lines and further embraced the abstract turn by regarding his postulates as merely characterising equivalent classes of possible structures. The importance of Peano for this story is that he was not only part of the strand following the rigorisation of geometry, but was an important inspiration to Russell~\cite{PrinciplesOfMathematics}, who would complete the first major formal verification effort in his monumental \emph{Principia Mathematica}. 

\section{The \emph{Foundations of Geometry}}
Hilbert published his text following a lecture series in the foundations of geometry. The book was so successful that it had ten editions in both German and English, and was hailed as the most influential book in a hundred years~\cite{BirkhoffHilbertInfluence}. 

In his famous remark that all references to ``point'', ``line'' and ``plane'' is his thesis should be substitutable with ``mug'', ``table'' and ``chair'', Hilbert was reinforcing Pasch's idea that all deductions must proceed without reference to the intuitive meaning of the terms involved. 

Probably the most notable achievement of Hilbert was not in his axiomatics, but in his metamathematics. He made axiom systems an object of mathematical scrutiny in their own right, and carefully analysed which axioms were independent of others. To do this, he took abstraction a stage further, by assigning interpretations to his axioms where points were now identified with real algebraic functions. He carefully analysed the minimal sets of axioms could be used to define an arithmetic in geometrical terms, and made some effort to establish the categoricity of his full set of axioms.

\subsection{Verification}
The computer assisted verification of the axiomatics and elementary consequences of Hilbert's \emph{Foundations of Geometry} is anticipated in two reviews of the text by Veblen~\cite{VeblenHilbertReview} and Poincar\'{e}~\cite{PoincareReview}, both of whom mention logic computers that had been developed in the 19th century~\cite{LogicMachines}, and the latter of whom mentions Peano's investigation of a symbolism for modern formal logic. This verification is the subject of the present thesis.

The possibility of a formal verification of mathematical theories such as geometry was made possible by the the highly expressive formal systems developed by Frege, Russell and Whitehead. Frege abandoned his project when Russell pointed out the famous Russell paradox~\cite{RussellsParadox} in his system, contributing to the growing sense of crisis in the foundations of logic itself. Whitehead and Russell revised Frege's system, but the heavy labour and seeming endlessness involved in verifying even the simple theory given in the \emph{Principia Mathematica} nearly drove an already melancholy Russell to suicide~\cite{RussellSuicide} and its completion left him completely exhausted.

Poincar\'{e} was punishingly dismissive of the project, regarding the programme as offering mathematicians nothing but ``shackles''~\cite{PoincareShackles}, and even going so far as to call Peano's projects and the aims of verification ``peurile''~\cite{PoincareReview}. However, with new crises appearing in 20th century mathematics with proofs that are so complicated they can no longer be verified by individuals as discussed by Davies~\cite{WhitherMathematics}, it would seem that Poincar\'{e} cast his dismissal too broadly. Besides, the shackles in Russell and Whitehead's research programme can be illeviated with the help of machines.

\subsubsection{Computer Assistance}
By the mid-1950s, as programmable computers were coming to prominence, Herbert Simon build a logic machine which could automatically prove all the theorems in Russell's \emph{Principia}, thereby paving the way for computer assisted verification which could relieve the poor human of the Herculean task of manually deriving the theorems by hand. Sadly, Russell's felt forced to declare in correspondence with Simon that the manual verification in the \emph{Principia} had been a ``wasted'' effort~\cite{SimonObituary}.

DeBruijn's AUTOMATH project represented the first computer \emph{assistant} for formally verifying mathematical theories, and was successfully used to verify Landau's classic text on real analysis~\cite{LandauGrundlagen,LandauAUTOMATH}. It became quite clear though that, even with a powerful computer assistant, there was still a wide gulf between mathematical proofs as they appear in the literature and the formal proof steps used in computer aided verification. A single logical inference might be multiplied to many logical steps in the verification, and the term ``DeBruijn'' factor was coined for the multiplier. Besides the DeBruijn factor, there is also the time factor involved. Hales estimates that a single page of mathematics requires a week for an expert to formally verify~\cite{HalesFormalisingCost}.

Difficulties aside, computer assisted verification has had some notable successes. The Four Colour Theorem, whose 1976 computer assisted proof was rightly viewed with suspicion, has now been meticulously verified by Gonthier in an extremely robust verifier~\cite{GonthierFCT} and according to Hales is now one of the most well-established results in all of mathematics. Gonthier has also led the project to verify the Feit-Thompson Theorem, a milestone in verifying the troublesome classification of all finite simple groups discussed by Davies. Finally, as of writing, the verification of the Kepler Conjecture is approaching completion~\cite{flyspeck}.

\subsubsection{Reducing the DeBruijn Factor}
The gulf between formal verifications and practiced mathematics concerns us. Like Meikle and Fleuriot, we wanted our verification of Hilbert's axiomatics to allow us to explore potential redundancies, logical gaps, and unbalanced presentation. Verification was to be our analytical tool, as it had been conceived of by Frege~\cite{ProofsAboutProofs}. 

However, we initially felt there was little hope of analysing Hilbert's prose with verifications so long as there was such a gulf between the two. We needed to close that gap, to bring our verification as close as possible to Hilbert's intentions. Only then could we hope to validate his approach or identify possible cracks in his presentation.

To that end, we have emphasised a particular style of verification which aims to be close to the ``mathematical vernacular''~\cite{MizarMathematicalVernacular}, and a style which respects the logical progressions typical of synthetic geometry. Our theorems and proofs are intended to obtain \emph{diagrams}, configurations of points, lines and planes, and our axioms are intended to allow us to derive interesting properties of these configurations.

\section{Aims}
As we stated at the beginning of this introduction, our initial aims for the project evolved over time. We had initially hoped to verify all five groups of Hilbert's axioms and their elementary consequences. Instead, we found there were many interesting sights to be found in just the first two groups. 

In focusing on these two groups, we have produced a verification of \emph{ordered geometry}~\cite{AxiomaticsOrderedGeometry}. This geometry, which lacks any notion of angle, parallels of distance, is a good deal more general than full Euclidean geometry, and to some extent, it has been investigated independently of other geometries~\cite{AxiomaticsOrderedGeometry}.

It turns out that the two groups actually contain most of Hilbert's axioms, but they are largely left unexplored by Hilbert. Our efforts to bring the DeBruijn factor of our verifications as close to 1 as possible had us carrying out this exploration ourselves. The investigation culminated in a verification of the polygonal Jordan Curve Theorem whose complexity dwarves all our other verifications put together.

The contributions of this thesis are thus:
\begin{enumerate}
\item a verification of the axioms for ordered geometry in Hilbert's \emph{Foundations of Geometry};
\item a fully verified synthetic proof of the Polygonal Jordan Curve Theorem using only axioms for incidence and order;
\item a combinator language allowing user's to define complex search strategies to assist declarative proof;
\item a stress test of declarative proof using the simple Mizar~Light~\cite{MizarLight} language against non-trivial verifications of synthetic geometry;
\item a case-study in developing readable verifications with a low DeBruijn factor;
\item a detailed investigation of the logical character of Hilbert's text.
\end{enumerate}

We should stress that the main contributions of this thesis are the first three items: the verifications themselves. These can be checked by downloading the code at \url{https://github.com/Chattered} and running in the HOL~Light theorem prover~\cite{HOLLight}. Note that since we have preferred proofs optimised for human readability over proofs optimised for the computer, the full verification takes some time to run (roughly 30 minutes on an Intel Core 2 2.53GHz machine.)

Our last contribution should be tempered somewhat. We are very much outside the scholarly community investigating the history of mathematics, and we cannot claim authority on some of our historical and philosophical remarks. we will simply say that in formal verifying Hilbert's proofs in a \emph{faithful} way, we have had to investigate every line in pedantic detail. We hope that any insights and observations we have made in doing so might be useful to those in the scholarly community.

Hilbert's elementary theorems, those contained in the first chapter of the text, where then to be formally verified with the aid of a computer verifier. 

At around the time of its publication, there were programmes for codifying the laws of thought in a way which could recover mathematical arguments, being developed by the Italian school under Peano. There were also British engineers such as De Morgan and Stanley Jevons designing machines to automate the application of logical laws such as those of George Boole. It is small wonder then that we find our modern goals of mechanising the \emph{Foundations of Geometry} anticipated in reviews by Veblen and Poincar\'{e}~\cite{VeblenHilbertReview,PoincareReview}, where we read casual claims that Hilbert's proofs could be derived by such simple machines.

Since Russell and Whitehead's Herculean efforts to derive mathematics from axiomatic foundations, which we are told left Russell so exhausted that he never fully recovered, we have known that there is a gap between the practices of rigorous mathematics and the requirements of mechanisation, and that computer assistance is needed in order for mechanisation to be feasible~\cite{FormalizedMathematics}. Since DeBruijn's first mechanisation, those working in formalised mathematics have assigned a numerical value to this gap in the form of the ``DeBruijn factor''. For a perspicious example of the gap, consider Lamport's recent comments about the proof that the sum of the first $n$ triangle is $\tfrac{1}{2}n(n+1)$~\cite{ProofMessageCertificate}.

When they began verifying the \emph{Foundations of Geometry}, Meikle and Fleuriot~\cite{MeikleFleuriotFormalizingHilbert} emphasised the gaps between verification and prose proof, but we wanted to close it down as much as possible, to reach a DeBruijn factor near to 1. The aim then is to produce a \emph{faithful} mechanisation of Hilbert's text. In the present thesis, we try to make the case that the main contribution of a faithful mechanisation lies in the critical appraisal it allows us to make of a prose text.

\section{Organisation}
The thesis is organised as follows. In Chapter~\ref{chapter:System} we deal with some necessary preliminaries concerning our theorem prover, its associated proof tools and its logical foundations. In Chapter~\ref{chapter:Axiomatics}, we introduce Hilbert's first two groups of axioms and discuss the various problems they introduce in terms of staying faithful to Hilbert in verification. In the next chapter, we discuss how to deal with these problems by introducing new automation in the form of a combinator language for theorem discovery. We then look in Chapter~\ref{chapter:Group2Eval} at Hilbert's first three prose proofs, using each as a case-study for our automated tool, and drawing out numerous observations about the proofs themselves. 

In Chapters~\ref{chapter:LinearOrder} and \ref{chapter:HalfPlanes}, we get into the meat of ordered geometry, dealing with the matter of linear and two-dimensional ordering in the plane, and finalising the concepts and additional automation that sees us through our final verification effort. This last effort is dealt with across four chapters. In Chapter~\ref{chapter:JordanInformal}, we lay out the problem and describe some high-level proofs and their weaknesses. In Chapter~\ref{chapter:JordanFormalisation}, we give the full formalisation of the theorem in a self-contained form which explains exactly what is to be verified. The full verification itself is divided into two parts across Chapters~\ref{chapter:JordanVerification1} and~\ref{chapter:JordanVerification2}.

We conclude in Chapter~\ref{chapter:Conclusion}. 

%%% Local Variables: 
%%% mode: latex
%%% TeX-master: "../thesis"
%%% End: 
