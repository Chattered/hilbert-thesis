\chapter{Foundations of Geometry}
\section{Tarski's Geometry}\label{OtherSystems}
Tarksi had developed a formal system for elementary geometry stated in modern formal logic \cite{TarskiGeometrySystem}. He showed how all formulas in this system could be interpreted in terms of polynomial equations and inequalities over real closed fields for which he could apply a decision procedure. This decidable fragment of geometry is therefore recursively axiomatisable, and indeed, Tarski's system is complete. This means, as per the results of G\"{o}del, that it cannot contain arithmetic. So, unlike Hilbert, Tarski could not state the Archimedean axiom for his geometry which we need for integral proofs. Finally, his continuity axiom which is supposed to be a second-order axiom over \emph{sets} of points, is weakened to range over only those sets which are definable by first-order formulas. 

Tarski's system is arguably simpler and more elegant: firstly, unlike Hilbert, he used only one primitive sort, namely points, and only two primitive notions, namely a ternary relation of betweeness and a congruence relation on pairs of points. Secondly, by using only points and a clever plane axiom, his theory is easily generalised to arbitrary dimension. Thirdly, Tarski's axioms did not need to exclude as many degenerate cases as Hilbert's own, which make his proofs simpler. Finally, every one of his axioms is given directly in terms of his primitives, while many of Hilbert's axioms rely on concepts defined \emph{in terms} of primitives. 

Art Quaife mechanised Tarski's axioms and used OTTER to automatically generate a selection of theorems which were regarded as challenge problems for automatic geometry theorem proving at the time \cite{QuaifeTarski}, while Narboux used Coq to mechanise a significant part of Tarski's \emph{Metamathematische Methoden in der Geometrie} \cite{NarbouxTarski}.

Despite the conceptual advantages of Tarski's system, we have chosen to mechanise Hilbert's. Of the two, Hilbert's can claim historical precedence as the first rigorous treatise of elementary geometry. It is also the more influential with ten published editions, and yet the question of its rigour is still open. As mentioned in \S\ref{Background}, Meikle and Fleuriot have evidence that Hilbert's proofs are not strictly valid, and that his definitions are not as precise as we would demand in a modern treatise. Our work so far has identified further weak points, and we suspect more to be revealed as we mechanise the later axioms and proofs. 