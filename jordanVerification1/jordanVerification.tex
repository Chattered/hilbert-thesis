\chapter{Verifying the Polygonal JCT: Part I}\label{chapter:JordanVerification1}
We now come to the first of our two main contributions. We must verify Theorem~\ref{eq:jordanFormal1} from Hilbert's axioms of ordered geometry. We assume a simple polygon, and must find two points in the plane with the following property: given an arbitrary polygonal path connecting the two points, we can find another point at which the path intersects the simple polygon. The overall idea of this proof is very similar to Veblen's 1904 proof~\cite{Veblenphd} which we described in detail in \S\ref{sec:VeblenProof}. We give the correct version of this proof now.

\section{Sketch Proof}\label{sec:ParityProofInformal}
Consider the polygon $Ps$ shown in Figure~\ref{fig:rayCast1}. We pick an arbitrary point $O$ between $P_1$ and $P_2$  and then cast an arbitrary ray $h$ from $O$ to a segment of the polygon other than $P_1P_2$. Of all the intersections that $h$ makes with the polygon, we pick the one closest to $O$ and label it $H$. We then pick an arbitrary point $A$ between $O$ and $H$. For this point, we have that the segment $AO$ does not intersect the polygon $Ps$. Finally, we consider the ray emanating from $O$ in the other direction. Applying the same reasoning as above, we find a point $B$ such that $BO$ does not intersect $Ps$. We end up with a segment $AB$ which intersects the polygon $Ps$ exactly once between $P_1$ and $P_2$, namely at the point $O$.

\begin{figure}
\centering\includegraphics{jordanVerification1/rayCast1}
\caption{The witnesses ($A$ and $B$) for Theorem~\ref{eq:jordanFormal1}}
\label{fig:rayCast1}
\end{figure}

\newcommand{\insideoutsideclaim}{every time the edge of a polygon crosses an edge of a triangle, it changes from being inside to outside the triangle and \emph{vice versa}}

Now consider any polygonal path which connects $A$ and $B$. Together with the segment $AB$, this yields another polygon $Qs$ (possibly non-simple) that intersects $Ps$ at least once at the segment $P_1P_2$. We now proceed by considering the exact same sequence of triangles that appear in Veblen's proof. However, the observation we shall carry through the argument is that the closed polygon $Qs$ must cross the edges of any triangle an even number of times. This should be intuitively obvious. Indeed, \insideoutsideclaim. The total crossings must therefore be even in number, since we end in the same region we started.

\begin{figure}
\subfigure[Step 1: Four crossings]{\includegraphics[scale=0.6]{jordanVerification1/ParityProof1.pdf}\label{fig:ParityProof1}}
\subfigure[Step 2: Six crossings]{\includegraphics[scale=0.6]{jordanVerification1/ParityProof2.pdf}\label{fig:ParityProof2}}
\caption{Parity proof}
\label{fig:ParityProof}
\end{figure}

In Figures~\ref{fig:ParityProof} and~\ref{fig:ParityProofCont}, we illustrate a run of this parity argument through the triangles $P_1P_2P_3$, $P_1P_3P_4$, $P_1P_5P_6$, $P_1P_7P_8$ (the steps for the triangles $P_1P_4P_5$, $P_1P_6P_7$ have been omitted for clarity). At the start of the proof, we assume that the polygon $Qs$ crosses the edge $P_1P_2$ exactly once and at the point $O$. Note however, that for the purposes of the argument, we only need the more general fact that it crosses an odd number of times. 

Now, if $Qs$ intersects the edge $P_2P_3$, we are done. Thus, we can assume that it does not cross this edge. In that case, the polygon $Qs$ must cross the edge $P_1P_3$, also an odd number of times, to ensure that the total crossings are even. Indeed, there are 3 such crossings shown in Figure~\ref{fig:ParityProof1}. Hence, we can continue with the triangle $P_1P_3P_4$. 

We then have that the polygon $Qs$ must cross the triangle $P_1P_3P_4$ an even number of times. We know that it crosses $P_1P_3$ an odd number of times, and we can assume that it does not cross $P_3P_4$ (otherwise, we are done). Hence, it must cross $P_1P_4$ an odd number of times, in order that the total be even. Indeed, there are another three crossings shown in Figure~\ref{fig:ParityProof2}. Again, we continue with the next triangle. Eventually, we shall find a point of intersection with the polygon, shown as point $Y$ in Figure~\ref{fig:ParityProofCont}.

\begin{figure}
\subfigure[Step 4: Four crossings]{\includegraphics[scale=0.6]{jordanVerification1/ParityProof4.pdf}}
\subfigure[Step 6: Two crossings]{\includegraphics[scale=0.6]{jordanVerification1/ParityProof6.pdf}}
\caption{Parity proof (continued)}
\label{fig:ParityProofCont}
\end{figure}

This is a deceptively simple proof. Most of the work involved hinges on the informal notion of ``crossing''. In the next section, we shall show how this notion is formulated.

% \begin{enumerate}
% \item A polygon which does not intersect a triangles' vertex must cross that triangle an even number of times.
% \item The number of crossings of a polygon with the triangles $ABC$ and $ABD$ at the edge $AB$ is the same.
% \end{enumerate}

\section{Formulation: Crossings}
We hope that our use of ``crossing'' in the above is intuitively clear. The basic idea is that a polygonal path crosses a segment $AB$ when it intersects $AB$ and moves from one side to the other. While intuitive, we found the idea resisted a nice formulation.

\subsection{Context}
In our formulation, a polygonal path is a vertex list, and from this vertex list it is trivial to recover an edge list using the function \code{adjacent}. Our plan then is to use this edge list to compute the number of times the path crosses the segment $AB$ by reducing it to the problem of computing the number of times a single edge of the path crosses $AB$. We can then define the crossings of the full polygonal path by summing the crossings at each of its edges. 

There is an annoying problem. Suppose we have an edge $P_iP_{i+1}$ of a polygonal path, and a triangle $ABC$, and suppose we are interested in whether $P_iP_{i+1}$ crosses the triangle at $AB$. If there is a point on $AB$ which is strictly between $P_i$ and $P_{i+1}$, then we know there is a crossing at $P_iP_{i+1}$. But if one of the endpoints $P_{i}$ or $P_{i+1}$ are points on $AB$, then it is not the segment $P_iP_{i+1}$ which crosses the triangle, but a potentially larger polygonal path $P_{i-m}\ldots P_{i-1}P_iP_{i+1}\ldots...P_{i+p}$ with $m \geq 0, p > 0$. 

\begin{figure}
\centering\includegraphics[scale=0.85]{jordanVerification1/ContextChange}
\caption{Assignment of context on a segment}
\label{fig:ContextChanges}
\end{figure}

We can preserve the idea that the presence or absence of a crossing is nevertheless defined for each edge of a polygonal path by introducing a \emph{context} variable $\Gamma\;:\;\code{bool}$, and assign a value of this variable to each edge $P_iP_{i+1}$ of the polygonal path. The value will tell us on which side of $AB$ the endpoint $P_{i+1}$ lies. In the peculiar case that $P_{i+1}$ lies on $AB$ itself, we can just propagate the preceding context.

In Figure~\ref{fig:ContextChanges}, we show a context value assigned in this way to the edges of $P_1\ldots P_{10}$. A value of $\top$ indicates the side which is the top half of the diagram, while $\bot$ indicates the bottom half. There are two places where the context switches truth value, which indicates that the polygonal path crosses the segment $AB$ twice.

\subsection{Combined Context for Triangles}
\begin{figure}
\centering\includegraphics{jordanVerification1/ContextChangeTriangle}
\caption{Assignment of context in a triangle}
\label{fig:ContextChangesTriangle}
\end{figure}

We will be counting crossings on the edges of a triangle, which will require three context variables, one for each edge. Also, the assignment of $\top$ and $\bot$ to the sides of each edge of the triangle cannot be arbitrary as it was in Figure~\ref{fig:ContextChanges}. The problem with the triangle case is that we want to reason about the total crossings on all three sides and thus consider the way the variables interplay.

For each edge, we shall therefore declare the $\top$ side for the corresponding context to be the side (or half-plane) containing the triangle's interior. We will then simply combine the contexts by taking their conjunction. One way to think about this is that we are using a single context variable which tracks whether a segment ends inside or outside the triangle.

In Figure~\ref{fig:ContextChangesTriangle}, we show the assignment of the combined context value to each vertex of a polygon intersecting a triangle $ABC$. Here, if an edge $P_iP_{i+1}$ is such that $P_{i+1}$ lies outside the triangle, then the edge is assigned $\bot$. If $P_{i+1}$ lies inside the triangle, then the edge is assigned $\top$. The only complication is how to set the context when the point $P_{i+1}$ lands on an edge. For this, we think about the three component contexts for each edge of the triangle in terms of the ideas from the previous subsection.

Take the segment $P_5P_6$. The point $P_6$ lands on the edge $AB$, so the component context for this edge is propagated from the previous segment. In particular, since $P_5$ lies on the side of $AB$ which contains the triangle's interior, we propagate the value $\top$.

The component of the context for the side $BC$ is simply $\top$, since $P_6$ is on the side of $BC$ containing the triangle's interior. The component of the context for $AC$ is also $\top$, since $P_6$ lies on the side of $AC$ containing the triangle's interior. Since the three component contexts for $P_6$ are all $\top$, so is the combined context.

To bring these ideas together, we will compute the number of crossings at each edge of the triangle as follows: first, we count a crossing for $P_iP_{i+1}$ and the edge $AB$ (or $AC$ or $BC$) every time there is a point on $AB$ which is strictly between $P_i$ and $P_{i+1}$. The only other crossings occur when $P_i$ lies on the segment $AB$. Here, we count a crossing in two circumstances:
\begin{itemize}
\item the context was last $\bot$ and $P_iP_{i+1}$ has a point in the interior of $\triangle ABC$ (and thus moves from outside to inside);
\item the context was last $\top$ and $P_iP_{i+1}$ has a point in the exterior of $\triangle ABC$ (and thus moves from inside to outside).
\end{itemize}

Thus, in Figure~\ref{fig:ContextChangesTriangle}, there is one crossing on $AC$, two crossings on $BC$, and one crossing on $AB.$

Now that we are always counting crossings at an edge relative to a triangle, it might appear that we have rendered our notion too specific. We will still need to be able to count crossings at an arbitrary segment $AB$ without mentioning triangles. To facilitate this, we show in \S\ref{sec:CrossingsWellDefined} that once we have fixed the vertices $AB$ in a triangle, our count of crossings at $AB$ is independent of the choice of the vertex $C$. In other words, the expression ``crossings of $P_iP_{i+1}$ at the edge $AB$'' is still well-defined, subject to a constraint on vertices detailed below. This fact, together with other key theorems (see Figure~\ref{fig:CrossingsSpecification} later), should fully clarify the intended semantics of a ``crossing.''

\subsection{Avoiding Vertices}\label{sec:EdgeCases}
% We have decided that a polygon crosses a triangle when it intersects the triangle and moves from inside the triangle to outside the triangle. We need to take care, though. Consider Figure~\ref{fig:crossingDifficult3}. Here, we probably want to say that there are two crossings in total, but it is not clear how to count the crossings at the three sides. Clearly, there is a crossing along the side $AB$, but where is the second crossing? We cannot say that it occurs between $A$ and $C$, because then the crossing vanishes for the triangle $AB'C$ which shares this side. Similarly, we cannot say that it occurs between $B$ and $C$, because then it vanishes for the triangle $A'BC$.

\begin{figure}
\centering\includegraphics{jordanVerification1/CrossingVertex}
\caption{Context with vertex crossings}
\label{fig:CrossingVertex}
\end{figure}

If we want the count of crossings at a particular edge of a triangle to be invariant of the position of the third vertex, we have a problem. Consider the scenario in Figure~\ref{fig:CrossingVertex}(a). Here, we have drawn a polygon $P_1P_2P_3P_4P_5P_6P_7P_8$ intersecting a triangle $ABC$. We have assigned our context appropriately to each segment, and concluded, quite reasonably, that the polygon does not cross $ABC$. 

However, when we assign context values for the triangle $BCD$ in Figure~\ref{fig:CrossingVertex}(b), we find that there suddenly appears a crossing on the shared edge $BC$ at the point $P_4$. In other words, the number of crossings at the segment $BC$ is not well-defined on our scheme.

 % is true precisely when the last segment not on the side $AB$ emerged into the \emph{interior}  of the triangle $ABC$. When we come to compute the total number of crossings of the triangle $ABC$ from the individual segments of the polygon $Ps$, we thread this value through, updating it as necessary. In \S~\ref{sec:wasInsideThreading}, we will see some interesting consequences of this choice of formulation.


% We have said that a polygon only crosses a triangle at a point of intersection, and in Figure~\ref{fig:crossingDifficult1}, the points of intersection are precisely these crossing points. But things are not always so clear. For instance, the polygon will sometimes intersect the triangle but ``bounce off''. In these cases, we do not want to count the intersection as a crossing (Figure~\ref{fig:crossingDifficult2}). 

% One thing we glossed over in the sketch proof is the fact that we count both the \emph{total} crossings of a polygon with a triangle, and also the crossings at a particular edge. We need to consider both since we are actually applying two ideas:

% \begin{enumerate}
% \item A polygon crosses a triangle an even number of times.
% \item The number of crossings of a polygon with the triangles $ABC$ and $ABD$ at the side $AB$ is the same.\label{item:crossingChange2}
% \end{enumerate}

% \begin{figure}
% \centering\subfigure[One crossing on $AB$; one crossing on $AC$; two crossings on $BC$.]{\includegraphics{jordanVerification1/crossingDifficult1}\label{fig:crossingDifficult1}}
% \qquad\centering\subfigure[No crossings.]{\includegraphics{jordanVerification1/crossingDifficult4}\label{fig:crossingDifficult2}}
% \end{figure}

% Perhaps we could say that the crossing occurs at the vertex $C$, and distinguish this from a crossing at a side. It seems we would have to do this in the situation depicted in \ref{fig:crossingDifficult4}. But then, how do we relate vertex crossings to side crossings and total crossings?

This difficulty can be eliminated quite simply by assuming that the polygonal path does not intersect any vertex of the triangle. We can get away with this because the vertices of the triangles we consider in our sketch proof are all vertices of the original polygon $Ps$. If at any time we found a point of intersection between the polygonal path and a vertex of one of the triangles, we will have found the desired point of intersection between the polygonal path and $Ps$.\label{sec:NoVertexAssumption}

% By ruling out all configurations where a polygon intersects a triangle's vertex, we eliminate the cases in Figures~\ref{fig:crossingDifficult3} and~\ref{fig:crossingDifficult4}. It makes life much easier.

\subsection{Formalisation}
% If we can first show how to compute the number of crossings between a triangle and an individual segment, then we should be able to compute the total number of crossings between the triangle and a polygon simply by summing the values for each of the polygon's sides. 

% There is a minor technical issue here, since whether an individual side of a polygon crosses a triangle in a particular segment is not generally a local property of that segment. Consider Figure~\ref{fig:crossingContext}. Here, we depict the same segment $P_iP_{i+1}$ intersecting the same triangle $ABC$. In each case, we want to know whether the endpoint $P_n$ is a crossing point of the triangle. This happens in cases $(a)$ and $(c)$ but not in $(b)$. Since the position of the triangle and $P_iP_{i+1}$ is the same in each case, it is clear that whether or not $P_n$ is a crossing depends on additional information. Here, we can look back through the history of previous vertices to see whether the last segment not on the side $AB$ emerged into the exterior of $\triangle ABC$ as in case $(a)$ and $(c)$, or whether it emerged into the interior as in case $(b)$.

% We decided to capture this dependency with a ``context'' variable which we shall denote by $\Gamma$. This is a \code{boolean} value, which is true precisely when the last segment not on the side $AB$ emerged into the \emph{interior}  of the triangle $ABC$. When we come to compute the total number of crossings of the triangle $ABC$ from the individual segments of the polygon $Ps$, we thread this value through, updating it as necessary. In \S~\ref{sec:wasInsideThreading}, we will see some interesting consequences of this choice of formulation.

% Let us turn to the task of computing the number of crossings from the list of vertices which define a polygon. Even with the edge cases and issues of context dealt with, there are still many possible configurations to consider concerning how an individual segment might or might not fall on a given side of a triangle (see Figure~\ref{fig:CrossingCases}). It took some effort and a lot of care to work these out on paper, since a mistake at this stage might not be spotted until well into the verification. We finally settled on a formulation. The details are not particularly important. Our verification itself shows that we have correctly cover all cases, and the key lemmas in \S\ref{sec:CrossingVerification} effectively give a clearer semantics to this notion of ``crossing.''

We now introduce the functions with which we shall calculate the number of crossings against an edge of a triangle. They are supplied for the curious reader to eliminate any potential ambiguity in our explanation of contexts, and to give an indication of the distance between the intuitive and the formal idea of a crossing. 

Our first function computes the crossings at an edge of a triangle based on a context. 
\begin{multline}\label{eq:oneCrossingDef}
% let crossing = new_definition
%   `crossing (A,B,C) was_inside P Q =
%     if between A P B /\ between A Q B then 0
%     else if (?R. between P R Q /\ between A R B) then 1
%     else if between A P B
%             /\ ((?R. between P R Q /\ in_triangle (A,B,C) R)
%                 <=> ~was_inside) then 1
%     else 0`;;
  \vdash_{def}\;\code{crossing}\ (A,B,C)\ \Gamma\ P_i\ P_{i+1}\\
  = 
  \begin{cases}
    0, \qquad\text{if }\between{A}{P_i}{B}\wedge\between{A}{P_{i+1}}{B}\\
    1, \qquad\text{else if }\exists R.\; \between{P_i}{R}{P_{i+1}}\wedge\between{A}{R}{B}\\
    1, \qquad\text{else if }\between{A}{P_i}{B}\\
    \qquad\qquad\wedge\; (\exists R.\; \between{P_i}{R}{P_{i+1}}\\
    \qquad\qquad\qquad\wedge\;\code{in\_triangle}\ (A,B,C)\ R \iff \neg\Gamma)\\
    0, \qquad\text{otherwise.}
  \end{cases}
\end{multline}

The first argument gives the three points defining the triangle we are interested in as a triple. We arbitrarily declare the first two components of this triple to be the edge of the triangle against which we want to compute crossings. The next argument is the context value $\Gamma$. The final two arguments are the endpoints of the polygonal path's edge against which we compute crossings.

Thus, to express the number of crossings for the edges $AC$ and $BC$, we just use the terms $\code{crossing}\ (A,C,B)$ and $\code{crossing}\ (B,A,C)$, and to express the total crossings of the segment $P_iP_{i+1}$ on the triangle, we use the term
\begin{multline*}
\code{crossing}\ (A,B,C)\ \Gamma\ P_i\ P_{i+1} + \code{crossing}\ (A,C,B)\ \Gamma\ P_i\ P_{i+1}\\ + \code{crossing}\ (B,A,C)\ \Gamma\ P_i\ P_{i+1}.
\end{multline*}

Our next function computes the context value for a segment $P_iP_{i+1}$ based on the last context. The arguments are the same, but here, the output does not depend on any particular ordering of the triple $(A,B,C)$.
\begin{equation*}
% `new_was_inside (A,B,C) was_inside P Q <=>
%   in_triangle (A,B,C) Q
%   \/ (on_triangle (A,B,C) Q /\ 
%         ((?R. between P R Q /\ in_triangle (A,B,C) R)
%             \/ on_triangle (A,B,C) P /\ was_inside))`
\begin{split}
\vdash_{def}\;\Gamma_{next}\ &(A,B,C)\ \Gamma\ P_i\ P_{i+1} \\
\iff &\code{in\_triangle}\ (A,B,C)\ P_{i+1}\\
& \vee \left(\begin{aligned}&\code{on\_triangle}\ (A,B,C)\ P_{i+1}\\
& \quad \wedge \left(\begin{aligned}&(\exists R.\; \between{P_i}{R}{P_{i+1}} \wedge \code{in\_triangle}\ (A,B,C)\ R)\\
& \quad\vee \code{on\_triangle}\ (A,B,C)\ P_i \wedge \Gamma)\end{aligned}\right)\end{aligned}\right).
\end{split}
\end{equation*}

Finally, we define the function which will compute the total number of crossings of an arbitrary polygonal path against the edge $AB$ for the triangle $ABC$. We do this recursively over the list of edges of the polygonal path, summing the values of $\code{polypath\_crossing}\ (A,B,C)$ for each segment and updating the context. Note that this function still requires an initial context $\Gamma$. We show where to get it from in \S\ref{sec:ContextInitialisation}.
\begin{equation*}
% let polypath_crossings = define
%   `polypath_crossings (A,B,C) was_inside [] = 0
%    /\ polypath_crossings (A,B,C) was_inside (CONS seg Ps)
%         = crossing (A,B,C) was_inside (FST seg) (SND seg)
%           + polypath_crossings (A,B,C)
%               (new_was_inside (A,B,C) was_inside (FST seg) (SND seg)) Ps`;;
\begin{aligned}
\vdash_{def}\;&\code{polypath\_crossings}\ (A,B,C)\ \Gamma\ [] = 0\\
\vdash_{def}\;&\code{polypath\_crossings}\ (A,B,C)\ \Gamma\ (\cons{(P_i,P_{i+1})}{segments})\\
 &\quad= \code{crossing}\ (A,B,C)\ \Gamma\ P_i\ P_{i+1}\\
 &\qquad+ \code{polypath\_crossings}\ (A,B,C)\ (\Gamma_{next}\ (A,B,C)\ \Gamma\ P_i\ P_{i+1})\ segments.
\end{aligned}
\end{equation*}

% \includegraphics{jordanVerification1/Crossing3}
% \includegraphics{jordanVerification1/Crossing4}
% \includegraphics{jordanVerification1/Crossing5}
% \includegraphics{jordanVerification1/Crossing6}
% \includegraphics{jordanVerification1/Crossing7}

\section{Triangle Interiors}\label{sec:TriangleInteriorDefinition}
% One way or another, our verification reduces to the base case in triangles. Our proof from \S\label{sec:JordanCurveFirstProof} defines the interior and exterior of a polygon by adding and subtracting triangles until the desired figure is obtained. Veblen's Proof considers triples of vertices going around the polygon, and applies the polygonal Jordan Curve Theorem to the resulting triangles (which appears as Corollary~2 of Theorem~27 in his 1904 thesis~\cite{Veblenphd}.) Our revision, on the other hand, does not appeal directly to the base case of the theorem, but instead counts ``crossings'' along a triangle's edge. As we shall see, the theoretical details of crossings ultimately exploit the base-case of the Polygonal Jordan Curve Theorem. It is therefore worth having a look at this base case, especially because it makes non-trivial use of our ordering tactic and the theory of half-planes.

% \section{Triangle Interiors}\label{sec:TriangleInteriorDefinition}
The above formulations and formalisation assume that we know how to express the interior, exterior and boundary of a triangle (respectively $\code{in\_triangle}$, $\code{on\_triangle}$ and $\code{out\_triangle}$). This we can do directly. Veblen, for instance, in his 1903 thesis~\cite{Veblenphd}, defined the interior of the triangle $ABC$ as the set of points $P$ such that there is a point $X$ on the segment $AB$ and a point $Y$ on $AC$ with $X$ between $Y$ and $Z$ (see Figure~\ref{fig:triangleDefs}). Here is another definition: the interior of $\triangle ABC$ is the set of all points on the same side of $AB$ as $C$, on the same side of $AC$ as $B$ and on the same side of $BC$ as $A$. In other words, the interior of a triangle is the intersection of three half-planes. The two formulations can be formalised as:
\begin{equation*}
\begin{split}
  &\code{in\_triangle\_veblen}\ (A,B,C)\ P \iff\\
  &\qquad\exists X.\;\exists Y.\; \between{A}{X}{B} \wedge \between{A}{Y}{C} \wedge \between{X}{P}{Y}.
\end{split}
\end{equation*}

\begin{equation}\label{eq:inTriangleDef}
\begin{split}
  \vdash_{def}\;&\code{in\_triangle}\ (A,B,C)\ P \iff\\
    &\qquad\exists hp.\;\exists hq.\;\exists hr.\;\code{on\_line}\ A\ (\code{line\_of\_half\_plane}\ hp)\\
    &\qquad\qquad\wedge \code{on\_line}\ B\ (\code{line\_of\_half\_plane}\  hp)\\
    &\qquad\qquad\wedge \code{on\_line}\ A\ (\code{line\_of\_half\_plane}\  hq)\\
    &\qquad\qquad\wedge \code{on\_line}\ C\ (\code{line\_of\_half\_plane}\  hq)\\
    &\qquad\qquad\wedge \code{on\_line}\ B\ (\code{line\_of\_half\_plane}\  hr)\\
    &\qquad\qquad\wedge \code{on\_line}\ C\ (\code{line\_of\_half\_plane}\  hr)\\
    &\qquad\qquad\wedge \code{on\_half\_plane}\ hp\ C \wedge \code{on\_half\_plane}\ hq\ B \wedge \code{on\_half\_plane}\ hr\ A\\
    &\qquad\qquad\wedge \code{on\_half\_plane}\ hp\ P \wedge \code{on\_half\_plane}\ hq\ P \wedge \code{on\_half\_plane}\ hr\ P.
\end{split}
\end{equation}

\begin{figure}
\centering\includegraphics{jordanVerification1/triangleDefs}
\caption{Two definitions of a triangle's interior}
\label{fig:triangleDefs}
\end{figure}

Veblen's definition is significantly shorter, but we wanted to try leveraging our theory of half-planes as much as possible in our verification of the Polygonal Jordan Curve Theorem, and the second definition gives us direct information about these. Besides which, the existentials in Veblen's definition do not have unique witnesses, while ours do, making Veblen's definition more complicated to reason with. Consider that it is not immediately clear that his definition is symmetric up to permutations of $A$, $B$ and $C$. To prove this, we would need to figure out how to move from the arbitrary $X$ and $Y$ on $AB$ and $AC$ satisfying the given condition (and there are infinitely many possible choices), to another $X'$ and $Y'$ on another choice of segments. With the second definition, the symmetry is almost immediate. In fact, HOL~Light`s $\code{MESON}$ can easily prove the rewrites needed to normalise expressions of the form $\code{in\_triangle}\ (A,B,C)$. 
\begin{equation}\label{eq:triSyms}
  \begin{split}
    \vdash&\code{in\_triangle}\ (A,B,C)\ P \iff \code{on\_triangle}\ (A,C,B)\ P\\
&    \wedge \code{in\_triangle}\ (A,B,C)\ P \iff \code{on\_triangle}\ (B,A,C)\ P.
  \end{split}
\end{equation}

That said, with our definition, our early verifications about triangles always became bloated in the same clumsy way. When we started from a hypothesis that a point lies inside a triangle, we found ourselves having to extract all three witnessed half-planes in the definition and all twelve conjuncts they satisfy. In many cases, we found that the the main body of the proof was shorter than the bloated statement of the assumptions. It might be suggested that this ugliness could have been avoided had we persevered instead with Veblen's definition, and tried to avoid reasoning about half-planes. We have some circumstantial against this: if Veblen's definition were a more useful starting point for reasoning about the interiors of triangles, we would expect that his formulation would appear as an immediate step in our proofs, whereupon we can reason more effectively about interiors. But this was never the case. There was no need obtain the two point witnesses given in Veblen's definition. Conversely, there were twelve places in our verifications where we had two points that could satisfy Veblen's definition, and where we appealed to a lemma which shows Veblen's definition implies our own. This suggests that our formulation is the more useful starting point.

% These two observations can be understood by again dividing geometric lemmas into those which introduce geometric entities and those which allow us to infer properties from a figure. As a theorem, Veblen's definition is weak for introducing entities, but useful for inferring properties of figures.

We make one final remark about our definition, which is important to keep in mind for some of the later verification. Whenever we have $\code{in\_triangle}\ (A,B,C)\ P$, we know that all triples chosen from $\{A,B,C,P\}$ are non-collinear. This means that explicit assumptions about non-collinearity can be suppressed in many of our verified theorems. It also means we can implement a discoverer \code{add\_in\_triangle} to derive these non-collinear triples automatically and make them available to the $\code{obviously}$ primitive. 

We end this subsection by considering the formalisation of a triangle's boundary and exterior. Since the boundary is just a polygonal path, it suffices to define:
\begin{equation}\label{eq:onTriangleDef}
\vdash_{def}\;\code{on\_triangle}\ (A,B,C)\ P \iff \code{on\_polypath}\ [A,B,C,A]\ P.
\end{equation}

In fact, it is useful to reuse \code{on\_polypath} in this way in other places. For instance, we can refer to the set of points of a segment $AB$ with $\code{on\_polypath}\ [A,B]$, and given a triangle $ABC$, we can write $\code{on\_polypath}\ [A,B,C]$ to refer to the points on just two sides of the boundary. Such formulas will prove convenient later on in our verification.

Finally, the exterior of a triangle can be defined simply as the set of points not on the triangle and not on the boundary. This definition classifies all points which are not on the plane as part of the exterior, but since we shall be relativising all of our theorems against a single plane, this does not matter.
\begin{multline*}
  \vdash_{def}\;\code{out\_triangle}\ (A,B,C)\ P\\
  \iff \neg\code{in\_triangle}\ (A,B,C)\ P\wedge\neg\code{on\_triangle}\ (A,B,C)\ P.
\end{multline*}

\section{Some Preliminary Theorems}
We recall the basic approach to synthetic axiomatic geometry as divided into two kinds of reasoning step: ones which introduce geometrical entities and ones which identify salient properties of the resulting figures. These properties allow us to introduce new geometrical entities, establish facts about them from which we can introduce new entities, and so on, until we have verified our goal theorem. 

We have six theorems for triangle interiors, two to introduce points and four to reason about such points with respect to triangle interiors. In this section, we shall look in detail at a verification of one of the introduction theorems, and then summarise the remaining ones.

\subsection{The Base Case}
Our main goal in this chapter is to show that a simple polygon divides the plane into at least two regions. In the simplest case, we take the polygon to be a \emph{triangle}, and we find that a triangle divides the plane much as a line divides the plane into half-planes. Specifically, given two points in different half-planes, we know there is a point between them which lies on the boundary, meaning we have an introduction theorem. There is an analogous introduction theorem for triangles, which we use frequently. It is even needed to prove our other introduction theorem~\eqref{eq:triCutHalfPlane} in \S\ref{sec:AdditionalTheorems}. In turn, this second introduction theorem is crucial to the verification of the well-definedness of crossings at a triangle's side (see \S\ref{sec:CrossingsWellDefined}):
\begin{multline}\label{eq:baseCase}
% |- on_plane A 'a /\
%      on_plane B 'a /\
%      on_plane C 'a /\
%      in_triangle (A,B,C) P /\
%      out_triangle (A,B,C) Q /\
%      on_plane Q 'a
%      ==> (?R. on_triangle (A,B,C) R /\ between P R Q)
\code{in\_triangle}\ (A,B,C)\ P \wedge \code{out\_triangle}\ (A,B,C)\ Q \\\implies \exists R.\; \code{on\_triangle}\ (A,B,C)\ R \wedge \between{P}{R}{Q}.
\end{multline}

We initially hoped the proof of Theorem~\ref{eq:baseCase} would be trivial. After all, an almost identical theorem holds for half-planes, and a triangle is defined as the intersection of three of these. Instead, we found ourselves needing a point introduction lemma.

\subsection{An ``Inner Pasch'' Lemma}
Initially, our only means to introduce points relative to triangles was by Pasch's axiom~\eqref{eq:g24}. But this axiom is often difficult to apply because it has a disjunctive conclusion. Luckily, there are easier versions to apply, namely the inner and outer variations, which we have derived as Theorems~\ref{eq:OuterPasch} and~\ref{eq:InnerPasch}. Our point introduction lemma can be thought of as a variation of Theorem~\ref{eq:InnerPasch}. It says that, given an interior point $P$ of a triangle $ABC$, and a point $Q$ outside the triangle on the ray $AB$, we can introduce the point $X$ at which the line $PQ$ intersects $BC$ (we could then use the Outer Pasch Axiom to find the point at which $PQ$ intersects $AC$). See Figure~\ref{fig:tricut1}.

\begin{figure}
\centering\includegraphics[scale=1.0]{jordanVerification1/tricut1.pdf}
\begin{equation}\label{eq:tricut1}
  \begin{split}
  % "!'a A B C P Q. on_plane A 'a /\ on_plane B 'a /\ on_plane C 'a
  %    /\ in_triangle (A,B,C) P
  %    /\ between A B Q
  %    ==> ?X. between P X Q /\ between B X C"
    &\code{in\_triangle}\ (A,B,C)\ P \wedge \between{A}{B}{Q} \\
    &\implies\exists X.\; \between{P}{X}{Q} \wedge \between{B}{X}{C}
  \end{split}
\end{equation}
\caption{``Inner Pasch'' for an interior point}
\label{fig:tricut1}
\end{figure}

The verification of this lemma illustrates some common patterns of reasoning with half-planes, and some of the pros and cons of our representation. The first half of the verification is shown in Figure~\ref{fig:tricut11}, where we obtain the three half planes defining the triangle. We must explicate the verbose constraints on these half-planes, before showing that the lines of each lie in the plane $\alpha$. These facts are needed in order to infer the defining property of each half-plane, namely that two points in the plane $\alpha$ are in the same half-plane precisely when their segment does not cross the line of the half-plane. The annoyance here is that we really do not care about such details, since all our assumptions should constrain the figure to the plane $\alpha$ anyway. If we transcribed these proofs to planar geometry, these details could be omitted, but for now, they show up as a weakness in our representation.

\begin{boxedfigure}[h]
\small
\begin{align*}
&\texttt{theorem }\code{on\_plane}\ A\ \alpha \wedge \code{on\_plane}\ B\ \alpha \wedge \code{on\_plane}\ C\ \alpha\\
&\qquad\qquad\code{in\_triangle}\ (A,B,C)\ P \wedge \between{A}{B}{Q}\\
&\qquad\qquad\implies\exists X.\; \between{P}{X}{Q} \wedge \between{B}{X}{C}\\
&\texttt{assume } \Triangle{a}{A}{B}{C}A \texttt{ by } \eqref{eq:inTriangleNcol}&0\\
&\texttt{assume } \code{on\_plane}\ A\ \alpha \wedge \code{on\_plane}\ B\ \alpha \wedge \code{on\_plane}\ C\ \alpha&1,2,3\\
  &\texttt{assume } \code{in\_triangle}\ (A,B,C)\ P\\
  &\texttt{so consider } hp, hq \text{ and } hr \texttt{ such that }\\
&\qquad \code{on\_line}\ A\ (\code{line\_of\_half\_plane}\ hp) \wedge \code{on\_line}\ B\ (\code{line\_of\_half\_plane}\ hp)& 4,5\\ 
% &\qquad \code{on\_line}\ A\ (\code{line\_of\_half\_plane}\ hp) & 6\\
% &\qquad \code{on\_line}\ C\ (\code{line\_of\_half\_plane}\ hp) & 7\\
% &\qquad \code{on\_line}\ B\ (\code{line\_of\_half\_plane}\ hp) & 8\\
% &\qquad \code{on\_line}\ C\ (\code{line\_of\_half\_plane}\ hp) & 9\\
&\qquad \code{on\_half\_plane}\ C\ hp\ldots \wedge \code{on\_half\_plane}\ P\ hr & 6,7\\
% &\qquad \code{on\_half\_plane}\ B\ hq & 11\\
% &\qquad \code{on\_half\_plane}\ A\ hr & 12\\
% &\qquad \code{on\_half\_plane}\ P\ hp & 13\\
% &\qquad \code{on\_half\_plane}\ P\ hq & 14\\
%&\qquad \code{on\_half\_plane}\ P\ hr 
&\ldots \texttt{ by } \eqref{eq:inTriangleDef} & 8..15\\
&\texttt{assume } \between{A}{B}{Q} & 16\\
&\texttt{obviously by\_neqs have } \forall X.\; \code{on\_half\_plane }\ hp\ X \implies \code{on\_plane}\ X\ \alpha\\ &\qquad\texttt{from 0,1,2,3,4,5,6 by } \eqref{eq:g16}, \eqref{eq:halfPlaneOnPlane}& 17\\
&\ldots & 18,19
\end{align*}
\caption{Proof of ``Inner Pasch'' for an interior point (part 1)}
\label{fig:tricut11}
\end{boxedfigure}

The rest of the proof is shown in Figure~\ref{fig:tricut12}. In contrast to the first part of the proof, the steps here are succinct, readable and geometrically interesting. With the necessary assumptions laid out, we see how easily the theory of half-planes has been leveraged via Theorems~\ref{eq:betOnHalfPlane1} and~\ref{eq:betOnHalfPlane2}. In contrast to the first part of the proof, this puts the use of half-planes in a much more positive light.
\begin{equation*}
%!hp P Q R. on_line P (line_of_half_plane hp)
%      /\ on_half_plane hp Q
%      /\ (between P Q R \/ between P R Q) ==> on_half_plane hp R
\tag{\ref{eq:betOnHalfPlane1}}
  \begin{split}
    \vdash\;&\code{on\_line}\ P\ (\code{line\_of\_half\_plane}\ hp) \wedge \code{on\_half\_plane}\ hp\ Q\\
    &\implies \between{P}{Q}{R} \vee \between{P}{R}{Q} \implies \code{on\_half\_plane}\ hp
  \end{split}
\end{equation*}
\begin{equation*}\tag{\ref{eq:betOnHalfPlane2}}
\begin{split}
  % "!P Q R hp. on_half_plane hp P /\ on_half_plane hp R /\ between P Q R 
  % ==> on_half_plane hp Q"
    \vdash\;&\code{on\_half\_plane}\ hp\ P \wedge \code{on\_half\_plane}\ hp\ R\\
    &\implies \between{P}{Q}{R} \implies \code{on\_half\_plane}\ hp\ Q
  \end{split}
\end{equation*}

\begin{boxedfigure}[h]
\small
\begin{align*}
&\texttt{obviously by\_ncols hence } \code{on\_plane}\ Q\ \alpha& 20\\
&\qquad\neg\code{on\_line}\ Q\ (\code{line\_of\_half\_plane}\ hr) & 21\\
&\qquad\neg\code{on\_half\_plane}\ hr\ Q \texttt{ from 0,1,2,12,13,14,16 by } \eqref{eq:g16}, \eqref{eq:g21}, \eqref{eq:onHalfPlaneNotBet}& 22\\
&\texttt{consider } X \texttt{ such that } \code{on\_line}\ X\ (\code{line\_of\_half\_plane}\ hr) \wedge \between{P}{X}{Q}\\
&\texttt{ from 15,19,20,21,22 by } \eqref{eq:onHalfPlaneNotBet} & 23 \\
&\texttt{have } \code{on\_line}\ Q\ (\code{line\_of\_half\_plane}\ hp)\texttt{ by } \eqref{eq:g12}, \eqref{eq:g21} \texttt{ from 4,5,16}\\
&\texttt{hence } \code{on\_half\_plane}\ hp\ X\texttt{ by } \eqref{eq:g21}, \eqref{eq:betOnHalfPlane1} \texttt{ from 7,23} & 24\\
&\texttt{hence } \neg\between{C}{B}{X} \texttt{ from 5,6,17 by } \eqref{eq:onHalfPlaneNotBet} & 25\\
&\texttt{hence } \code{on\_half\_plane}\ hq\ Q \texttt{ from 8,10,16 by } \eqref{eq:betOnHalfPlane1}\\
&\texttt{hence } \code{on\_half\_plane}\ hq\ X \texttt{ from 11,23 by } \eqref{eq:betOnHalfPlane2}\\
&\texttt{hence } \neg\between{B}{C}{X} \wedge B \neq X \wedge C \neq X\texttt{ from 5,9,10,18,24 by } \eqref{eq:onHalfPlaneNotBet}, \eqref{eq:halfPlaneNotOnLine}\\
&\texttt{obviously by\_neqs qed from 0,12,13,23,25 by } \eqref{eq:four}
\end{align*}
\caption{Proof of ``Inner Pasch'' for an interior point (part 2)}
\label{fig:tricut12}
\end{boxedfigure}

%\begin{figure}
% \caption{Betweenness and half-planes}
% \label{fig:BetOnHalfPlaneTheorems}
% \end{figure}

The basic strategy of the verification is to note that because $A$ and $Q$ lie on opposite sides of $BC$, so too must $P$ and $Q$. Thus, we can find a point $X$ between $P$ and $Q$ which is on the line $BC$. We just need to show that this point $X$ lies more specifically between $B$ and $C$.

To do this, we note that $P$ and $X$ lie on a ray emerging from the point $Q$ on the line $AB$, and so they must be on the same side of this line. Thus $P$, $C$ and $X$ must all lie on the same side of $AB$ which means that the point $B$ cannot possibly lie between any of them. Similar considerations apply if we look at the line $AC$. We can thus conclude that $X$ can only lie between $B$ and $C$.

The verification captures the \emph{structure} of this line of argument almost exactly. However, the terms are almost completely different. To begin with, we do not introduce anonymous rays. This would only add extra $\code{consider}$ steps, which is unnecessary when we can talk directly in terms of betweenness. We also avoid talking in terms of sides of a line by talking instead in terms of half-planes. We effectively have the translation

\label{sec:HalfPlaneTranslations}
\begin{tabular}{ccc}
  $\code{on\_half\_plane}\ hp$ & becomes & on the same side of $AB$ as $C$ and $P$;\\
  $\code{on\_half\_plane}\ hq$ & becomes & on the same side of $AC$ as $B$ and $P$;\\
  $\code{on\_half\_plane}\ hr$ & becomes & on the same side of $BC$ as $A$ and $P$;\\
  $\code{line\_of\_half\_plane}\ hp$ & becomes & the line $AB$; \\
  $\code{line\_of\_half\_plane}\ hq$ & becomes & the line $AC$; \\ 
  $\code{line\_of\_half\_plane}\ hr$ & becomes & the line $BC$. \\
\end{tabular}\linebreak

With these translations in mind, we hope the reader is convinced that the informal argument and a model synthetic proof can be recovered systematically from the verification. We can try to excuse the translation by drawing an analogy between synthetic proofs and their accompanying diagrams. The diagram is strictly unnecessary, but can  easily be recovered by carefully following the prose, and it is often helpful to reconstruct it. Similarly, our informal argument can be easily recovered from our formal verification, substituting intuitive phrases such as ``a ray emerging from the line'' so that it is easier to follow, even if such phrases do not point to interesting abstractions that would help the theorem prover.

\subsection{From ``Inner Pasch'' to the Base Case}\label{sec:JordanBaseCase1}
We can now give an informal proof of Theorem~\ref{eq:baseCase}. In Figure~\ref{fig:BaseCasePasch}, we suppose that $P$ is inside a triangle $ABC$ and $Q$ is outside. Then $P$ is on the same side of one of the triangle's edges and opposite vertex, while $Q$ is not. Let us suppose, without loss of generality, that $P$ is on the same side of $AB$ as $C$ while $Q$ is not. Then $PQ$ must intersect the line $AB$. If $PQ$ intersects the \emph{segment} $AB$, we have found the required point on the triangle's boundary. Otherwise, there is a point $R$ on the segment $PQ$ which also lies on either the ray emanating from $B$ in the direction $\overrightarrow{AB}$ or on the ray emanating from $A$ in the direction $\overrightarrow{BA}$. By applying \eqref{eq:tricut1} to each case, we can then find a point on the side $BC$ or the side $AC$ respectively, and we are done.

\begin{figure}
\centering\includegraphics{jordanVerification1/baseCase}
\caption{``Inner Pasch'' to the base case}
\label{fig:BaseCasePasch}
\end{figure}

We have made a without-loss-of-generality assumption in this argument, namely in our choice of $AB$ and the point $C$. As Harrison has shown~\cite{HarrisonWLOG}, such assumptions can often be handled elegantly using without-loss-of-generality tactics, particularly in geometry. However, these tactics typically exploit a \emph{Kleinian View} of geometry. This view of geometry can be described as ``subtractive''~\cite{SubtractiveKlein}: we start from a rich mathematical structure such as $\mathbb{R}^n$, and then ignore details by working only with invariants under a transformation group. Axiomatic geometry, on the other add, is additive, starting with only the most primitive machinery. As such, it is not clear how to build a theory of invariants which could capture our without-loss-of-generality cases.

Instead, we formalised the above argument as a lemma, and then wrote an \emph{ad hoc} procedural script to manually apply the symmetries. In ordered geometry, we only need to consider six symmetries and so the procedural boilerplate is hardly a bottle-neck compared to our use of MESON in declarative proofs, but this is still somewhat inelegant compared to doing proper without-loss-of-generality reasoning.

\subsection{Additional Theorems}\label{sec:AdditionalTheorems}
\begin{figure}
\centering
\includegraphics{jordanVerification1/triCutHalfPlane}
\caption{Another point introduction theorem}
\label{fig:triCutHalfPlane}
\end{figure}

We have one more theorem to introduce points. Here, we suppose that we have a point $P$ on the edge $AB$ of a triangle $ABC$ and a point $Q$ outside the triangle but on the same side of $AB$ as $C$. In this case, the segment $PQ$ must intersect the polygonal path $[A,B,C]$ at a point $X$ (see Figure~\ref{fig:triCutHalfPlane}). The half-plane $hp$ in this theorem is used to signify the side of $AB$ on which the point $C$ lies.
\begin{equation}\label{eq:triCutHalfPlane}
\begin{split}
\vdash&\between{A}{P}{B}\\
&\wedge \code{on\_line}\ A\ (\code{line\_of\_half\_plane}\ hp) \wedge \code{on\_line}\ B\ (\code{line\_of\_half\_plane}\ hp)\\
&\wedge \code{on\_half\_plane}\ C\ hp \wedge \code{on\_half\_plane}\ Q\ hp\\
&\wedge \code{out\_triangle}\ (A,B,C)\ Q\implies \exists X.\; \between{P}{X}{Q} \wedge \code{on\_polypath}\ [A,B,C]\ X.
\end{split}
\end{equation}

\begin{figure}
\centering
\subfigure[Points between the sides of a triangle are interior \eqref{eq:inTriangle1}]{\includegraphics[scale=1.2]{jordanVerification1/inTriangle1}}
\qquad\subfigure[Points between an interior point and a side are interior \eqref{eq:inTriangle2}]{\includegraphics[scale=1.2]{jordanVerification1/inTriangle2}}
\caption{Triangle interior theorems}
\label{fig:inTriangleTheorems}
\end{figure}

The remaining four theorems assume we have a configuration of points in relation to a triangle, and conclude that one of the points is interior or exterior. These theorems are used routinely throughout the first half of our main verification, particularly when we come to counting how many times a polygonal path crosses the sides of a triangle (see \S\ref{sec:CrossingVerification}). Their proofs are similar to the one given in the previous section, and always reduce to reasoning about the interaction between rays and half planes.

We give diagrams and a short description for each theorem in Figures~\ref{fig:inTriangleTheorems} and~\ref{fig:outTriangleTheorems}. These theorems all have reasonably clear synthetic verifications, and together require 82 verification steps. Roughly two fifths of these steps are assisted by our incidence discoverer via the \code{obviously} and \code{clearly} primitives.

Note that Theorem~\ref{eq:inTriangle1} is one direction of the equivalence between our definition of triangles and Veblen's (see \S\ref{sec:TriangleInteriorDefinition}).

\begin{figure}
\centering\subfigure[A ray through an opposite side leaves the triangle \eqref{eq:outTriangle1}]{\includegraphics[scale=1.2]{jordanVerification1/outTriangle1}}
\qquad\subfigure[A ray from inside the triangle to a side emerges outside the triangle \eqref{eq:outTriangle2}]{\includegraphics[scale=1.2]{jordanVerification1/outTriangle2}}
\caption{Triangle exterior theorems}
\label{fig:outTriangleTheorems}
\end{figure}

\begin{equation}\label{eq:inTriangle1}
  % "!A B C X Y P.
  %    ~(?a. on_line A a /\ on_line B a /\ on_line C a)
  %    /\ between A X B /\ between A Y C /\ between X P Y
  %    ==> in_triangle (A,B,C) P"
  \begin{split}
    \vdash\;&\Triangle{a}{A}{B}{C}\\
    &\wedge \between{A}{X}{B} \wedge (\between{A}{Y}{C} \vee C = Y)\\
    &\implies \between{X}{P}{Y} \implies \code{in\_triangle}\ (A,B,C)\ P.
  \end{split}
\end{equation}

\begin{equation}\label{eq:inTriangle2}
  % "!A B C X P.
  %    ~(?a. on_line A a /\ on_line B a /\ on_line C a)
  %    /\ between A X B /\ between C P X
  %    ==> in_triangle (A,B,C) P"
  \begin{split}
    \vdash&\code{in\_triangle}\ (A,B,C)\ X \wedge \code{on\_triangle}\ (A,B,C)\ Y\\
    &\implies \between{X}{P}{Y} \implies \code{in\_triangle}\ (A,B,C)\ P.
  \end{split}
\end{equation}

\begin{equation}\label{eq:outTriangle1}
  % "!A B C X P.
  %    ~(?a. on_line A a /\ on_line B a /\ on_line C a)
  %    /\ between A X B /\ between C P X
  %    ==> in_triangle (A,B,C) P"
  \begin{split}
    \vdash&\Triangle{a}{A}{B}{C}\\
    &\wedge \between{A}{X}{B} \wedge \between{A}{Y}{C}\\
    &\implies \between{X}{Y}{P} \implies \code{out\_triangle}\ (A,B,C)\ P.
  \end{split}
\end{equation}

\begin{equation}\label{eq:outTriangle2}
  % "!A B C X P.
  %    ~(?a. on_line A a /\ on_line B a /\ on_line C a)
  %    /\ between A X B /\ between C P X
  %    ==> in_triangle (A,B,C) P"
  \begin{split}
    \vdash&\code{in\_triangle}\ (A,B,C) X \wedge \code{on\_triangle}\ (A,B,C)\ Y\\
    &\implies \between{X}{Y}{P} \implies \code{out\_triangle}\ (A,B,C)\ P.
  \end{split}
\end{equation}

\section{Key Theorems of Crossings}\label{sec:CrossingVerification}
The formal definition of crossings as the threading of a context variable through a sequence of conditionals takes us a \emph{long way} from the intuitive idea. The intuition only reappears in our key theorems governing the definition, and the distance between the intuition and the formalisation can be measured by the thousand or so lines of mostly declarative proof and the enormous number of case-splits we consider to bridge the gap.

\subsection{Numbers of Crossings}
First, a relatively simple matter: a single segment crosses a triangle at most twice. Our verification of this takes the form of a crisp declarative proof based on Bernays' supplement~\eqref{eq:SupplementI} that we discussed in \S\ref{sec:SupplementI}. We do not need any messy case-splits, only a short piece of procedural script to eliminate without-loss-of-generality assumptions. We end up with this:
\begin{equation*}
  \begin{split}
    \vdash&\Triangle{a}{A}{B}{C}\\
    &\implies \code{crossing}\ (A,B,C)\ \Gamma\ P_i\ P_{i+1} + \code{crossing}\ (A,C,B)\ \Gamma\ P_i\ P_{i+1}\\
    &\qquad\qquad+ \code{crossing}\ (B,C,A)\ \Gamma\ P_i\ P_{i+1} \leq 2.
  \end{split}
\end{equation*}

The only unpleasantness comes from unfolding the definition of $\code{crossing}$, which requires that we face the mess of case-splits from Definition~\ref{eq:oneCrossingDef}. For this, we use a \code{tactics} step and a tactic \code{unfold\_crossing\_tac} which unfolds the definition of $\code{crossing}$ and then sweeps through the goal term eliminating the cases. Again, this tactic does not modify any assumptions, and it is typically only applied at the very start of a verification. With the cases converted, the $\code{assume}$ steps allow us to make more meaningful assumptions, as in the verification extract in Figure~\ref{fig:UnfoldingCrossings}.

\begin{boxedfigure}
\small
\begin{align*}
&\code{theorem } \Triangle{a}{A}{B}{C} &\\
&\qquad\wedge \code{crossing}\ (A,B,C)\ X\ P_i\ P_{i+1} = 1&\\
&\qquad\wedge \code{crossing}\ (A,C,B)\ X\ P_i\ P_{i+1} = 1&\\
&\qquad\implies \code{crossing}\ (B,C,A)\ X\ P_i\ P_{i+1} = 0&\\
&\code{assume } \Triangle{a}{A}{B}{C}&\\
&\code{tactics } \code{unfold\_crossing\_tac}&\\
&\code{assume } \between{A}{P_i}{B} \vee \exists R.\; \between{P_i}{R}{P_{i+1}} \wedge \between{A}{R}{B}
\end{align*}
\caption{Unfolding crossings}
\label{fig:UnfoldingCrossings}
\end{boxedfigure}

Things get \emph{really} hairy for our next theorem, which clearly explains how the values of $\code{crossing}$ compare when evaluated for a single segment at the various sides of a triangle. We give an impression of the cases involved in Figure~\ref{fig:CrossingCases}.

\begin{figure}
\centering\includegraphics{jordanVerification1/CrossingCases}
\caption{Cases of crossings}
\label{fig:CrossingCases}
\end{figure}

\begin{multline*}
% `!A B C P Q.
%       on_plane A 'a
%       /\ on_plane B 'a
%       /\ on_plane C 'a
%       /\ on_plane P 'a
%       /\ on_plane Q 'a
%       /\ ~(?a. on_line A a /\ on_line B a /\ on_line C a)
%       /\ ~on_polypath [P; Q] A
%       /\ ~on_polypath [P; Q] B
%       /\ ~on_polypath [P; Q] C
%       /\ (~on_triangle (A,B,C) P ==> (in_triangle (A,B,C) P <=> was_inside))
%       ==> (crossing (A,B,C) was_inside P Q
%            + crossing (A,C,B) was_inside P Q
%            + crossing (B,C,A) was_inside P Q = 1
%            <=> (was_inside =
%                   ~new_was_inside (A,B,C) was_inside P Q))`  
  \begin{aligned}
    \vdash&\Triangle{a}{A}{B}{C}\\
    &\wedge \neg\code{on\_polypath}\ [P_i,P_{i+1}]\ A\wedge \neg\code{on\_polypath}\ [P_i,P_{i+1}]\ B\\&\wedge \neg\code{on\_polypath}\ [P_i,P_{i+1}]\ C\\
    &\wedge(\neg\code{on\_triangle}\ (A,B,C)\ P_i \implies (\code{in\_triangle}\ (A,B,C)\ P_i \iff \Gamma))\\
  \end{aligned}\\
    \implies\left(\begin{aligned}[]
        &\code{crossing}\ (A,B,C)\ \Gamma\ P_i\ P_{i+1} + \code{crossing}\ (A,C,B)\ \Gamma\ P_i\ P_{i+1}\\
          &\qquad+ \code{crossing}\ (B,C,A)\ \Gamma\ P_i\ P_{i+1} = 1 \\
          &\iff \Gamma = \neg \Gamma_{next}\ (A,B,C)\ \Gamma\ P_i\ P_{i+1}
        \end{aligned}\right)
\end{multline*}

The first hypothesis just requires that $ABC$ is a triangle. The second requires that the segment $P_iP_{i+1}$ does not intersect any of the vertices, as per our discussion in \S\ref{sec:NoVertexAssumption}. 

The rest of the theorem then clarifies both the idea behind a crossing and the idea behind the context variable $\Gamma$. The conclusion says that the sum of crossings at the three sides is 1 precisely when the context variable switches truth value. The formalisation almost transparently captures a claim made in the sketch proof: ``\insideoutsideclaim.''

There is one more thing we should say about the context $\Gamma$. The theorem hypothesises that when $P_i$ is not on the sides of a triangle then $\Gamma$ tracks whether the point is inside or outside. Since $P_i$ is intended to be a vertex of a polygonal path and $P_iP_{i+1}$ an edge, we want to make sure that this hypothesis on $\Gamma$ is preserved as it threads through the remaining edges. 

Because a vertex of the polygon $P_{i+1}$ is the successor of $P_i$, what we are saying here is that, just as $\Gamma$ tracks whether $P_i$ is inside or outside the triangle, so too must $\Gamma_{next}$ track whether $P_{i+1}$ is inside or outside. This matter is settled trivially from the definition using the simplifier.
\begin{multline*}
%`!A B C P Q. ~on_triangle (A,B,C) Q
%   ==> (in_triangle (A,B,C) Q <=> new_was_inside (A,B,C) was_inside P Q)`
\vdash\neg\code{on\_triangle}\ (A,B,C)\ P_{i+1}\\
\implies (\code{in\_triangle}\ (A,B,C)\ P_{i+1} \iff \Gamma_{next}\ (A,B,C)\ \Gamma\ P_i\ P_{i+1}).
\end{multline*}

\subsection{Overview of Some Verification}
Rather than go into all the details of the verification, we will give a typical extract of a specific case, showing how in these proofs we are still relying on our discovery algebra from Chapter~\ref{chapter:Automation} and our linear ordering tactic from Chapter~\ref{chapter:LinearOrder}. We also see how we leverage our lemmas for this section, and thus avoid having to deal directly with half-planes.

The case we consider is equivalent to saying that if a segment $P_iP_{i+1}$ crosses a triangle $ABC$ exactly once between $P_i$ and $P_{i+1}$ at $AB$, then one of $P_i$ and $P_{i+1}$ is interior to the triangle while the other is exterior.
\begin{multline}\label{eq:IH3}
% crossing (A,B,C) was\_inside P Q = 1
%          ==> crossing (A,C,B) was_inside P Q = 0
%          ==> crossing (B,C,A) was_inside P Q = 0
%          ==> on_plane A 'a /\ on_plane B 'a /\ on_plane C 'a
%              /\ on_plane P 'a /\ on_plane Q 'a
%              /\ ~(?a. on_line A a /\ on_line B a /\ on_line C a)
%              /\ ~on_polypath [P;Q] A /\ ~on_polypath [P;Q] B /\ ~on_polypath [P;Q] C
%              /\ ~on_triangle (A,B,C) Q
%              ==> (in_triangle (A,B,C) P \/ on_triangle (A,B,C) P /\ was_inside
%                   <=> ~(in_triangle (A,B,C) Q))
  \begin{aligned}
    &\Triangle{a}{A}{B}{C}\\
    &\wedge\between{P_i}{R}{P_{i+1}} \wedge \between{A}{R}{B} \\
    &\quad\wedge\code{crossing}\ (A,C,B)\ \Gamma\ P_i\ P_{i+1} = 0\wedge\code{crossing}\ (B,C,A)\ \Gamma\ P_i\ P_{i+1} = 0\\
    &\wedge\neg\code{on\_polypath}\ [P_i,P_{i+1}]\ A \wedge \neg\code{on\_polypath}\ [P_i,P_{i+1}]\ B \\
    &\quad\wedge \neg\code{on\_polypath}\ [P_i,P_{i+1}]\ C\\
    &\wedge\neg\code{on\_triangle}\ (A,B,C)\ P_i \wedge\neg\code{on\_triangle}\ (A,B,C)\ P_{i+1}
  \end{aligned}\\
  \implies\left(\code{in\_triangle}\ (A,B,C)\ P_i \iff \code{out\_triangle}\ (A,B,C)\ P_{i+1}\right).
\end{multline}

We divide the verification into the three cases shown in Figure~\ref{fig:IH3CaseSplit}. In case (a), we have assumed that $P_i$ is interior. It then follows immediately from Theorem~\ref{eq:outTriangle2} that $P_{i+1}$ is exterior. In case (b), we have assumed that $P_i$ is exterior and that $P_i$ and $P_{i+1}$ are in line with the vertex $C$. In this case, we just apply Theorem~\ref{eq:inTriangle1}. Finally, in case (c), we have assumed that $P_i$ is again exterior but that the line of $P_iP_{i+1}$ does not intersect $C$. Under these circumstances, we can apply Pasch's Axiom \eqref{eq:g24} to the triangle and the line of $P_iP_{i+1}$ using our discoverer \code{by\_pasch} and thus obtain a point $S$ either on $AC$ or $BC$. It then follows from Theorem~\ref{eq:inTriangle1} that $P_{i+1}$ is interior.

\begin{figure}
\centering\includegraphics{jordanVerification1/IH3CaseSplit}
\caption{Main case-split}
\label{fig:IH3CaseSplit}
\end{figure}

Actually, things are not \emph{quite} so simple for cases (b) and (c). In order to apply Theorem~\ref{eq:inTriangle1} in case (b), we first have to prove that $P_{i+1}$ is between $C$ and $R$. To do this, we want to apply our linear ordering tactic, but for this to work, the tactic will need some facts about the existing order relations among the points $P_i$, $P_{i+1}$, $R$ and $C$. These facts come from various places.

First off, the incidence discoverer tells us that $C \neq R$. Next, from \eqref{eq:inTriangle2} and the fact that $P_i$ is exterior, we conclude that $P_i$ does not lie between $C$ and $R$. Finally, since $P_iP_{i+1}$ does not intersect $C$, we know that all three points are distinct and that $C$ does not lie between $P_i$ and $P_{i+1}$. Each of these inferences corresponds to a single declarative step, and once in place, the linear reasoning tactic can be applied to the four points $C$, $P_i$, $P_{i+1}$ and $R$, where it is able to show that $P_{i+1}$ lies between $C$ and $R$. We finish by applying~\eqref{eq:inTriangle1} to show that $P_{i+1}$ is interior to the triangle.

Case (c) is more involved, but the most interesting part is probably that which establishes that neither $A$ nor $B$ lie on the line of $P_iP_{i+1}$. Here, we proceed by contradiction, once for $A$ and once for $B$. We can solve the goal in one step with the linear reasoning tactic, provided we again obtain the necessary information for it to do its work.

For instance, assuming that $A$, $P_i$ and $P_{i+1}$ lie on a line, the linear reasoning tactic will first infer that $A$, $B$, $P_i$, $P_{i+1}$ and $R$ are all collinear. If it knew further than $P_{i}$ does not lie on the segment $AB$, it would conclude that one of $A$ or $B$ lies on the segment $P_iP_{i+1}$, which \emph{we} know to be impossible. This suggests that we should seed the tactic with the following facts, with which it solves the goal by reasoning about the ordering of $A$, $B$, $P_i$, $P_{i+1}$ and $R$.
\begin{multline*}
A \neq P_i \wedge A \neq P_{i+1} \wedge B \neq P_i \wedge B \neq P_{i+1}\\ \wedge \neg\between{P_i}{A}{P_{i+1}} \wedge \neg\between{P_i}{B}{P_{i+1}} \wedge \neg\between{A}{P_i}{B}.
\end{multline*}

%This deals with the three cases from Figure~\ref{fig:IH3CaseSplit}, which cover a crossing strictly between $P_i$ $P_{i+1}$. When we include the case of a crossing at one of the endpoints, for which we make use of the context, we have a verification which runs to 70 steps. We are putting Mizar~Light through its paces, and for the most part, it copes very well with the complexity. In the places where the prover struggled, it was usually simple enough to apply the \code{using} combinator to inject pieces of procedural proof into the script. We shall say more on this in \S\ref{sec:InjectingProcedural}.

% \begin{multline}\label{eq:IH3}
% % crossing (A,B,C) was\_inside P Q = 1
% %          ==> crossing (A,C,B) was_inside P Q = 0
% %          ==> crossing (B,C,A) was_inside P Q = 0
% %          ==> on_plane A 'a /\ on_plane B 'a /\ on_plane C 'a
% %              /\ on_plane P 'a /\ on_plane Q 'a
% %              /\ ~(?a. on_line A a /\ on_line B a /\ on_line C a)
% %              /\ ~on_polypath [P;Q] A /\ ~on_polypath [P;Q] B /\ ~on_polypath [P;Q] C
% %              /\ ~on_triangle (A,B,C) Q
% %              ==> (in_triangle (A,B,C) P \/ on_triangle (A,B,C) P /\ was_inside
% %                   <=> ~(in_triangle (A,B,C) Q))
%   \begin{aligned}
%     \vdash&\Triangle{a}{A}{B}{C}\\
%     &\wedge\code{crossing}\ (A,B,C)\ \Gamma\ P_i\ P_{i+1} = 1\\
%     &\quad\wedge\code{crossing}\ (A,C,B)\ \Gamma\ P_i\ P_{i+1} = 0\wedge\code{crossing}\ (B,C,A)\ \Gamma\ P_i\ P_{i+1} = 0\\
%     &\wedge\neg\code{on\_polypath}\ [P_i,P_{i+1}]\ A \wedge \neg\code{on\_polypath}\ [P_i,P_{i+1}]\ B \\
%     &\quad\wedge \neg\code{on\_polypath}\ [P_i,P_{i+1}]\ C\\
%     &\wedge\neg\code{on\_triangle}\ (A,B,C)\ P_{i+1}
%   \end{aligned}\\
%   \implies\left(\begin{aligned}&\code{in\_triangle}\ (A,B,C)\ P_i \vee \code{on\_triangle}\ (A,B,C)\ P_i \wedge \Gamma\\
%     &\qquad\iff \code{out\_triangle}\ (A,B,C)\ P_{i+1}\end{aligned}\right).
% \end{multline}

%In doing this, we generally tried to respect the aims of declarative proof. For instance, at no point do we destructively modify the proof context, and we always insist that whenever our tactics apply an assumption, we include the assumption label so that a reader can at least track the dependencies.

%For instance, the automatic prover often struggles to apply complex conditionals with many hypotheses. For these situations, we add an initial \code{MATCH\_MP\_TAC} in a \code{using} clause. This leaves the default prover \code{MESON} with just the theorem's assumptions as goals. This tactic still keeps things largely declarative, since we name the theorem we are matching against and we do not modify any assumptions. 

%In other places, we found it helpful to introduce a function \code{mutual\_simp} which takes a list of theorems and simplifies each with respect to the others. This function is used to process the justifying theorems of a declarative step before they are handed to \code{MESON}. The need for such processing comes from the fact that a declarative proof is based on accumulating theorems in a proof context. These theorems are expected to influence one another without any destructive modification (consider, as a simple example, an equation inferred late in a verification which could rewrite all previous assumptions). It is often helpful to have the theorems modify one another as they are applied at a step, using a function such as \code{mutual\_simp}, so that they are more useful to \code{MESON}.

% Mizar~Light's prover struggles to apply large theorems such as our ray-casting theorem. Luckily, the actual tactic used to justify a step can be customised inline. For theorems such as this, where we have a conditional with many assumptions, we often use code of the form

% \begin{displaymath}
% \code{K (MATCH\_MP\_TAC } \eqref{eq:changeTriangle}) \code{ THEN } (\code{MESON\_TAC } \circ \code{mutual\_simp}
% \end{displaymath}

% This tactic tells us to first match the goal with the conclusion \eqref{eq:changeTriangle}. This leaves us to prove the assumptions of \eqref{eq:changeTriangle}, which can be done by mutually simplifying the justifying theorems and feeding the results to $\code{MESON\_TAC}$. This preparatory step using \code{mutual\_simp} is often needed. In forward declarative proofs, we expect facts that we add to the proof context to affect facts already there.

\subsection{Crossings are Well-defined}\label{sec:CrossingsWellDefined}
In our sketch proof from \S\ref{sec:ParityProofInformal}, we implicitly assume that when we have two triangles $ABC$ and $ABC'$, then the number of crossings made by a polygonal path against the shared edge $AB$ is always the same. This is not obvious from our formulation, because the number of crossings at $AB$ is dependent on a choice of triangle with edge $AB$. We need to show that this choice is arbitrary.

Now the definition of a crossing makes use of a triangle's interior, and different triangles sharing the edge $AB$ will have interiors which may be disjoint, may overlap, or may contain one another. We will need to verify that, nevertheless, the values of the function \code{crossing} are always consistent. In other words, we must show that the expression ``crossings at $AB$'' is well-defined, without reference to the vertex $C$.

\begin{figure}
\centering\includegraphics{jordanVerification1/CrossingWellDefined}
\caption{Triangles sharing an edge}
\label{fig:CrossingWellDefined}
\end{figure}

There is key case-split here, shown in Figure~\ref{fig:CrossingWellDefined}. If $C$ and $C'$ are on the same side of $AB$ as in case (a), then the triangle interiors will overlap. Here, as we cross the edge $AB$, we enter or leave the interiors of both triangles together, a fact we verify as Theorem~\ref{eq:crossChangeLemma1} in Figure~\ref{fig:crossChangeLemma1}. On the other hand, if $C$ and $C'$ are on opposite sides of $AB$ as in (b), then the interiors of the two triangles are disjoint. In this case, as we move from the interior of $ABC$ across the edge $AB$ to the exterior, we simultaneously move from the exterior of $ABC'$ to the interior, a fact we verify as Theorem~\ref{eq:crossChangeLemma2} in Figure~\ref{fig:crossChangeLemma2}.

\begin{figure}
\begin{multline}\label{eq:crossChangeLemma1}
  \begin{aligned}
    \vdash&\code{on\_line}\ A\ (\code{line\_of\_half\_plane}\ hp) \wedge \code{on\_line}\ B\ (\code{line\_of\_half\_plane}\ hp)\\
    &\wedge\code{on\_half\_plane}\ hp\ C \wedge \code{on\_half\_plane}\ hp\ C'\\
    &\wedge\between{A}{P_i}{B} \wedge \neg\between{A}{P_{i+1}}{B}
  \end{aligned}\\
  \implies\left(\begin{aligned} 
      &(\exists R.\; \between{P_i}{R}{P_{i+1}} \wedge \code{in\_triangle}\ (A,B,C)\ R)\\
      & \iff (\exists R.\; \between{P_i}{R}{P_{i+1}} \wedge \code{in\_triangle}\ (A,B,C')\ R)
  \end{aligned}\right).
\end{multline}
\caption{Moving across $AB$ when $C$ and $C'$ are on the same side.}
\label{fig:crossChangeLemma1}
\end{figure}

\begin{figure}
\begin{multline}\label{eq:crossChangeLemma2}
  \begin{aligned}
    \vdash&\Triangle{a}{A}{B}{C'}\\
    &\wedge \neg\code{on\_polypath}\ [P_i, P_{i+1}]\ A \wedge \neg\code{on\_polypath}\ [P_i, P_{i+1}]\ B\\
    &\wedge\code{on\_line}\ A\ (\code{line\_of\_half\_plane}\ hp) \wedge \code{on\_line}\ B\ (\code{line\_of\_half\_plane}\ hp)\\
    &\wedge\code{on\_half\_plane}\ hp\ C \wedge \neg\code{on\_half\_plane}\ hp\ C'\\
    &\wedge \between{A}{P_i}{B} \wedge \neg\between{A}{P_{i+1}}{B}
  \end{aligned}\\
  \implies\left(\begin{aligned}
      &(\exists R.\; \between{P_i}{R}{P_{i+1}} \wedge \code{in\_triangle}\ (A,B,C)\ R)\\
      &\iff\neg \exists R.\; \between{P_i}{R}{P_{i+1}} \wedge \code{in\_triangle}\ (A,B,C')\ R
  \end{aligned}\right).
\end{multline}
\caption{Moving across $AB$ when $C$ and $C'$ are on opposite sides.}
\label{fig:crossChangeLemma2}
\end{figure}

These two theorems are proven by reasoning about half-planes and, in both cases, applying Theorem~\ref{eq:triCutHalfPlane}. The assumptions on half-planes in Theorem~\ref{eq:crossChangeLemma1} require that the points $C$ and $C'$ lie on the same side of $AB$. In Theorem~\ref{eq:crossChangeLemma2}, they require that $C$ and $C'$ lie on opposite sides. Theorem~\ref{eq:crossChangeLemma2} needs some extra assumptions since the negations make for very weak claims. For instance, the fact that $C'$ does not lie on $hp$ might just mean that it lies on the line $AB$, so we have to add in a condition that the points $A$, $B$ and $C'$ are non-collinear.

The important assumption to note in both theorems is $\between{A}{P_i}{B}$. This reflects the fact that the case-split is only pertinent when we come to update and make use of the context variable $\Gamma$, which happens when the edge $P_iP_{i+1}$ has exactly one endpoint on the segment $AB$. So when the edge \emph{lands} on $AB$, the context variable must be correctly updated to say that we were last inside or outside the triangle. When it \emph{emerges} from $AB$, the context variable must be correctly utilised to say whether the edge $P_iP_{i+1}$ counts as a crossing. These matters can be formally clarified by the corollaries in Figures~\ref{fig:CrossChange1} and~\ref{fig:CrossChange2} (we do not reproduce the assumptions in full).

\begin{figure}
  \begin{gather*}
    \begin{split}
      \vdash\neg\between{A}{P_i}{B}&\wedge\between{A}{P_{i+1}}{B}\\
      &\implies \Gamma_{next}\ (A,B,C)\ \Gamma\ P_i\ P_{i+1} = \Gamma_{next}\ (A,B,C)\ \Gamma\ P_i\ P_{i+1}
    \end{split}\\
    \begin{split}
      \vdash&(\between{A}{P_i}{B}\implies \Gamma = \Gamma')\\
      &\implies\code{crossing}\ (A,B,C)\ \Gamma\ P_i\ P_{i+1} = \code{crossing}\ (A,B,C')\ \Gamma'\ P_i\ P_{i+1}
    \end{split}
  \end{gather*}
  \caption{Well-definedness theorems when $C$ and $C'$ are on the same side of $AB$}
  \label{fig:CrossChange1}
\end{figure}

\begin{figure}
  \begin{gather*}
    \begin{split}
      \vdash\neg\between{A}{P_i}{B}&\wedge\between{A}{P_{i+1}}{B}\\
      &\implies \Gamma_{next}\ (A,B,C)\ \Gamma\ P_i\ P_{i+1} = \neg\Gamma_{next}\ (A,B,C)\ \Gamma\ P_i\ P_{i+1}
    \end{split}\\
    \begin{split}
      \vdash&(\between{A}{P_i}{B}\implies \Gamma = \neg\Gamma')\\
      &\implies\code{crossing}\ (A,B,C)\ \Gamma\ P_i\ P_{i+1} = \code{crossing}\ (A,B,C')\ \Gamma'\ P_i\ P_{i+1}
    \end{split}
  \end{gather*}
  \caption{Well-definedness theorems when $C$ and $C'$ are on opposite sides of $AB$}
  \label{fig:CrossChange2}
\end{figure}

Thus, when $C$ and $C'$ are on the same side, we expect the context values to be the same when we hit the segment $AB$, and we expect them to compute the same crossing value when we leave $AB$. When $C$ and $C'$ are on opposite sides, we expect the context values to be opposite when we hit $AB$, and, as before, we expect them to compute the same crossing value when we leave the segment.

In any case, we know that the vertex $C$ in the expression $\code{crossing}\ (A,B,C)\ \Gamma\ P_i\ P_{i+1}$ can be varied about the sides of $AB$ (so long as we change $\Gamma$ appropriately), and so we can generalise our notion of crossing. Instead of saying that a polygonal path crosses the side of a  \emph{triangle} by moving from interior to exterior and \emph{vice versa}, we can say that a polygonal path crosses an arbitrary \emph{segment} precisely when it moves from one side of the segment to the other. In other words, we can abstract away the vertex $C$. The mechanics of this will become clear in our final proof in \S\ref{sec:InductionProof}. Before we get to that, we must consider how we initialise $\Gamma$.

% These cases are factored in when we consider the update and use of the context variable $\Gamma$. The update is pertinent when a segment $P_iP_{i+1}$ lands on the edge $AB$. The use of the context is pertinent when a segment $P'Q'$ leaves the edge $AB$. 

% More generally, we should show that the number of crossings of a polygon at a side $AB$ of triangle $ABC$ is independent of the point $C$. Because of this, we can abstract over all possible choices of $C$ with a universal quantifier and thus recover the notion of crossings on an arbitrary segment. 

%  we consider a triangle sequence where adjacent elements share an edge. Implicit in the sketch proof is that the number of crossings at an edge is well-defined. This is obvious intuitively, but not at all clear from our formalisation of crossings. This is the last gap between our formalisation and the intuitive idea of a crossing. 

% The problem is that our definition of a crossing of $AB$ by a segment $P_iP_{i+1}$ \eqref{eq:oneCrossingDef} is given with respect to a triangle, since it must make use of the 

%  we have formally defined  of crossing at the segment $AB$ in terms of the interior of a triangle $ABC$ and a context $\Gamma$. We must show that its value is well-defined as as we move both the vertex $C$ around the plane and thus consider a different triangle sharing the edge $AB$. We also need to consider the dependencies on the value of $\Gamma$.

% There are three cases to consider here, and they arise in the context of our sketch proof with little difficulty. In Figure~\ref{fig:ChangeTriangle}, we show parts of two intersecting polygons $Ps$ and $Qs$, and we consider three transitions from the triangle $ABC$ to the next triangle $ABC'$ in the sequence defined by the sketch proof. In each case, we are interested in crossings of the segment $P_iP_{i+1}$ at the shared edge $AB$. Our aim is to show that the number of crossings at $AB$ by the segment $P_iP_{i+1}$ with respect to $ABC$ is the same as it is with respect to $ABC'$

% What distinguishes these three cases are the positions of $C$ and $C'$ relative to the edge $AB$, and the consequence this has on the respective interiors of the triangles. In (a), we see that $C$ and $C'$ are on opposite sides of $AB$, and the interiors of the two triangles are therefore disjoint. Therefore, as $P_iP_{i+1}$ crosses the side $AB$, it leaves the interior of $ABC$ and enters the interior of $ABC'$. More generally, it leaves the interior of one triangle and enters the other.

% In case (b), we see that $C$ and $C'$ are on the same side of $AB$, and the interiors overlap. Here, as $P_iP_{i+1}$ crosses $AB$, it leaves the interior of both triangles. Generally, it leaves the interior of both triangles or enters the interior of both.

% In case (c), we see that $C$ and $C'$ are neither on opposite sides nor on the same side of $AB$. Instead, $C'$ is on the line of $AB$, and $ABC'$ is a degenerate case of a triangle and has \emph{no} interior. This last possibility can be factored into the previous two. Instead of allowing the point $C$ to move to any other point in the plane, we insist that it moves strictly between the two \emph{sides} of $AB$. To capture the case of (c), we will also allow the point $B$ to move along the ray $\overrightarrow{AB}$ to a point $B'$. The third case can then be recovered by setting $C=C'$ and moving $B$ to $B'$.

% However, we will add a constraint on $B'$, namely that the segment $BB'$ is not intersected by $P_iP_{i+1}$. This makes sense in context if we consider \ref{fig:ChangeTriangleOnLine}. In general, the segment $BB'$ will be a segment of the original polygon $Ps$, and thus, any intersection here is the witness we need for the main theorem. The upshot is that the intersections of $P_iP_{i+1}$ with $AB$ are precisely the intersections of $P_iP_{i+1}$ with $AB'$. In effect, we can treat the two segments as identical.

%\subsection{Crossings are Well-defined: Some Verification}
%In this section, we shall give an overview of how our verification breaks down. 

\subsection{Initialising the Context}\label{sec:ContextInitialisation}
Recall that, in general, when a segment $P_iP_{i+1}$ emerges from the boundary of a triangle $ABC$, the question of whether there is a crossing depends on additional information provided by the context. 

When computing the total crossings for a polygonal path, this context is threaded through the calculations for each individual edge, starting from some initial context. The question is: how do we choose this initial context?

Sometimes, the answer is straightforward. If the endpoint $P_i$ lies in the interior of the triangle, then the value of the context used to compute crossings for $P_iP_{i+1}$ \emph{must} be $\top$. If $P_i$ lies in the exterior of $ABC$, then the value \emph{must} be $\bot$. These give us some solutions for the initial context, and it would be convenient if we could rely on this simple case. But what happens if $P_i$ lies \emph{on} the triangle?

We considered two possible ways to answer this question, one of which turned out to have a surprising difficulty which left us favouring the other for the final verification. The first approach has us avoid the question, by always counting crossings from a vertex which does not lie on the triangle. The second approach has us compute a starting context based on the points of the polygon.

The first approach could work because if a closed polygon crosses a triangle, it must have at least one vertex outside the triangle. We just need to verify this and then rotate the polygon's vertex list so we can start counting crossings from there, and we will show how to perform such a rotation in \S\ref{sec:PolygonRotation}.

However, consider what happens as we vary the point $C$ of the triangle, something we need to do during the sketch proof. An example is shown in Figure~\ref{fig:ContextInitialiseCounter}, where we have two triangles $ABC$ and $ABC'$ sharing an edge $AB$, each crossed by the polygonal path $P_1P_2P_3P_1$. As we claimed, this polygonal path has a vertex outside of each triangle ($P_3$ for $\triangle ABC$ and $P_1$ for $\triangle ABC'$), but there is no vertex off both boundaries. This means that after we move the point $C$ to $C'$, we would need to perform another polygon rotation. Doing this continually during our proof would complicate the basic inductive argument.

\begin{figure}
\centering\includegraphics{jordanVerification1/ContextInitialiseCounter}
\caption{No well-defined initial context}
\label{fig:ContextInitialiseCounter}
\end{figure}

For the second approach, we need to compute a single consistent value for the initial context of any polygon. Fortunately, such a value exists. To spot it, we just realise that the initial value of the context for a polygonal path is related to the value of the context at the polygonal path's final edge. We can compute this final value with a recursive function:
\begin{equation*}
% let polypath_new_was_inside = define
%   `polypath_new_was_inside (A,B,C) was_inside [] = was_inside
%    /\ polypath_new_was_inside (A,B,C) was_inside (CONS seg Ps)
%         = polypath_new_was_inside (A,B,C)
%             (new_was_inside (A,B,C) was_inside (FST seg) (SND seg))
%             Ps`;;
\begin{aligned}
\vdash_{def}\;&\Gamma_{final}\ (A,B,C)\ \Gamma\ [] = \Gamma.\\
\vdash_{def}\;&\Gamma_{final}\ (A,B,C)\ \Gamma\ (\cons{(P_i,P_{i+1})}{segments}) = \\
&\qquad\qquad\Gamma_{final}\ (A,B,C)\ (\Gamma_{next}\ (A,B,C)\ \Gamma\ \ P_i\ P_{i+1})\ segments.
\end{aligned}
\end{equation*}

Now it turns out that if we push the final context back through the above function, we end up with the same value. Formally, $(\Gamma_{final}\ (A,B,C)\ \Gamma\ segments)$ is a fixpoint of the function $(\lambda \Gamma'.\; \Gamma_{final}\ (A,B,C)\ \Gamma'\ segments)$ for arbitrary $\Gamma$. No matter what our starting choice of $\Gamma$, the computed final context $\Gamma_{final}$ can be consistently taken as the initial context from then on. This expression therefore gives us a suitable starting context.

The computation of the initial context appears in our specification of crossings, and the fact that this expression denotes a fixpoint is a lemma used in the verification of Theorem~\ref{eq:polypathCrossingsEven}.

\subsection{The Specification of Crossings}
\begin{figure}
  \begin{multline}\label{eq:crossNZIntersect}
    \vdash\code{polypath\_crossings}\ (A,B,C)\ \Gamma\ (\code{adjacent}\ Ps) > 0\\
    \implies \exists Q.\; \code{on\_polypath}\ Ps\ Q \wedge\ \between{A}{Q}{B}
  \end{multline}

  \begin{multline}\label{eq:polypathCrossingsEven}
  \begin{aligned}
% !A B C was_inside Qs 'a P Ps.
%          Qs = CONS P (APPEND Ps [P])
%          ==> on_plane A 'a /\
%              on_plane B 'a /\
%              on_plane C 'a /\
%              (!X. MEM X Qs ==> on_plane X 'a) /\
%              ~on_polypath Qs A /\
%              ~on_polypath Qs B /\
%              ~on_polypath Qs C /\
%              ~(?a. on_line A a /\ on_line B a /\ on_line C a)
%          ==> EVEN
%              (polypath_crossings (A,B,C)
%               (polypath_new_was_inside (A,B,C) was_inside (ADJACENT Qs))
%               (ADJACENT Qs) +
%               polypath_crossings (A,C,B)
%               (polypath_new_was_inside (A,B,C) was_inside (ADJACENT Qs))
%               (ADJACENT Qs) +
%               polypath_crossings (B,C,A)
%               (polypath_new_was_inside (A,B,C) was_inside (ADJACENT Qs))
%               (ADJACENT Qs))  
    \vdash&Qs = \append{[P]}{\append{Ps}{[P]}}\\
    &\wedge \Gamma_{initial} = \Gamma_{final}\ (A,B,C)\ \Gamma\ (\code{adjacent}\ Qs)\\
    &\wedge \neg\code{on\_polypath}\ Qs\ A \wedge \neg\code{on\_polypath}\ Qs\ B \wedge \neg\code{on\_polypath}\ Qs\ C\\
    &\wedge \Triangle{a}{A}{B}{C}
\end{aligned}\\
\implies\code{even}\left(\begin{aligned}& \code{polypath\_crossings}\ (A,B,C)\ \Gamma_{initial}\ (\code{adjacent}\ Qs)\\
    &+\;\code{polypath\_crossings}\ (A,C,B)\ \Gamma_{initial}\ (\code{adjacent}\ Qs)\\
    &+\;\code{polypath\_crossings}\ (B,C,A)\ \Gamma_{initial}\ (\code{adjacent}\ Qs)
  \end{aligned}\right)
\end{multline}

  \begin{multline}\label{eq:changeTriangle}
  %        Qs = CONS P (APPEND Ps [P]) /\
  %        on_plane A 'a /\
  %        on_plane B 'a /\
  %        on_plane C 'a /\
  %        on_plane C' 'a /\
  %        (!X. MEM X Qs ==> on_plane X 'a) /\
  %        ~(?a. on_line A a /\ on_line B a /\ on_line C a) /\
  %        ~(?a. on_line A a /\ on_line B' a /\ on_line C' a) /\
  %        ~on_polypath Qs A /\
  %        (between A B B' \/ between A B' B \/ ~(A = B) /\ B = B') /\
  %        ~(?X. on_polypath [B; B'] X /\ on_polypath Qs X)
  %        ==> (?was_inside'. polypath_crossings (A,B,C)
  %                           (polypath_new_was_inside (A,B,C) was_inside
  %                           (ADJACENT Qs))
  %                           (ADJACENT Qs) =
  %                           polypath_crossings (A,B',C')
  %                           (polypath_new_was_inside (A,B',C') was_inside'
  %                           (ADJACENT Qs))
  %                           (ADJACENT Qs))
  \begin{aligned}
    \vdash&Qs = \append{[P]}{\append{Ps}{[P]}}\\
    &\wedge \neg\code{on\_polypath}\ Qs\ A \wedge \neg\code{on\_polypath}\ Qs\ B\\
    &\wedge \Triangle{a}{A}{B}{C}\\
    &\wedge \Triangle{a}{A}{B}{C'}\\
    &\implies \exists \Gamma'.\; \code{polypath\_crossings}\ (A,B,C)\ (\Gamma_{final}\ (A,B,C)\ \Gamma\ (\code{adjacent}\ Qs))\\
    &\qquad\qquad\qquad(\code{adjacent}\ Qs)\\
    &\qquad = \code{polypath\_crossings}\ (A,B,C')\ (\Gamma_{final}\ (A,B,C')\ \Gamma'\ (\code{adjacent}\ Qs))\\
    &\qquad\qquad\qquad(\code{adjacent}\ Qs)
  \end{aligned}
\end{multline}
\caption{Final specification of crossings}
\label{fig:CrossingsSpecification}
\end{figure}

At last, we will recover the intuitive idea behind crossings from the mess of case-analyses and implementation detail of the previous sections. In Figure~\ref{fig:CrossingsSpecification} we give the key theorems which subsume the important details of the other theorems considered thus far. It is these theorems which we shall appeal to exclusively in \S\ref{sec:InductionProof}, where we verify Veblen's Lemma from~\S\ref{sec:VeblenLemma1}.

The first theorem~\eqref{eq:crossNZIntersect} is mostly a convenience. It simply relates crossings to intersections, telling us that if there are crossings at $AB$ by a polygonal path $Ps$, then $Ps$ really does intersect $AB$. The converse does not hold, since the polygonal path might merely intersect and then ``bounce off'', thus staying on the same side of $AB$.

Theorem~\ref{eq:polypathCrossingsEven} assumes that we have a polygon $Qs$ and sets an initial context as described in the previous section. It also assumes we have a triangle $ABC$ and that $Qs$ does not intersect any of its vertices, as per our discussion in \S\ref{sec:NoVertexAssumption}. Under these conditions, the total number of crossings against the three sides is always even.

Theorem~\ref{eq:changeTriangle} tells us that the choice of $C$ when counting
crossings is arbitrary, so long as it is not on the line $AB$. There is a slight
complication, in that the theorem tells us to reset the initial context using the
supplied $\Gamma'$ given in the conclusion, but since
Theorem~\ref{eq:polypathCrossingsEven} holds for arbitrary choices of $\Gamma$, we
can ignore this constraint when we apply the two theorems.

The upshot of Theorem~\ref{eq:changeTriangle} is that we can understand a crossing without reference to a triangle, but instead only with reference to the points $A$ and $B$. In the next section, we shall see how this more general understanding plays out in our verification.

\section{Verifying the Sketch Proof}
In this section, we shall review our verification of the parity proof that we sketched in \S\ref{sec:ParityProofInformal}. There are interesting details in the verification relating to our use of the theorems of the previous section. But more importantly, we can obtain a beautiful theorem from which the first half of the Polygonal Jordan Curve Theorem arises as a corollary. Unlike the Polygonal Jordan Curve Theorem, this theorem makes no reference to \emph{simple polygons}. It is a general theorem about arbitrary polygonal paths, one which does not hinge on complex definitions such as those that appear for the Polygonal Jordan Curve Theorem.

\subsection{The Induction Proof}\label{sec:InductionProof}
The parity proof assumes that we have two polygons $Ps$ and $Qs$ intersecting at an edge. Based on this, we consider a sequence of triangles formed from the vertices of the polygon $Ps$, and repeat a parity argument over the number of crossings.

This argument readily formalises as a proof by induction, which gives us a nice reinterpretation. Rather than considering triangles with vertices drawn from $Ps$, we continually reduce the problem to smaller polygons. This inductive proof yields the (somewhat ugly) lemma:
\begin{equation}\label{eq:oddCrossClosedPolypath}
  % "!Ps Qs A B.
  %    LENGTH Qs >= 2
  %    /\ HD Qs = LAST Qs
  %    /\ on_plane A 'a /\ on_plane B 'a
  %    /\ (!X. MEM X Ps ==> on_plane X 'a)
  %    /\ (!X. MEM X Qs ==> on_plane X 'a)
  %    /\ ~(A = B) 
  %    /\ (!C. on_plane C 'a
  %            /\ ~(?a. on_line A a /\ on_line B a /\ on_line C a)
  %            ==> ?was_inside. 
  %              ODD
  %                (polypath_crossings (A,B,C)
  %                  (polypath_new_was_inside (A,B,C) was_inside (ADJACENT Qs))
  %               (ADJACENT Qs)))
  %  ==> ?X. on_polypath (CONS B (APPEND Ps [A])) X
  %          /\ on_polypath Qs X"
  \begin{aligned}
    \vdash&\code{length}\ Qs \geq 2 \wedge \code{head}\ Qs = \code{last}\ Qs\wedge P_1\neq P_2\\
    &\wedge \left(\begin{aligned}&\forall C.\; \Triangle{a}{P_1}{P_2}{C}\\
    &\qquad\quad \implies \exists \Gamma.\; \code{odd}\ (\code{polypath\_crossings}\ (P_1,P_2,C)\\
    &\qquad\quad\qquad\qquad\qquad(\Gamma_{final}\ (P_1,P_2,C)\ \Gamma\ (\code{adjacent}\ Qs))\ (\code{adjacent}\ Qs))\end{aligned}\right)\\
    &\implies\exists X.\; \code{on\_polypath}\ (\append{[P_2]}{\append{Ps}{[P_1]}})\ X \wedge \code{on\_polypath}\ Qs\ X.
  \end{aligned}
\end{equation}

Here, we assume two polygons of length at least two, namely $\append{[P_1,P_2]}{\append{Ps}{[P_1]}}$ and $Qs$. The polygon $Qs$ is assumed to cross the edge $P_1P_2$ an odd number of times. We then conclude that $Qs$ intersects the tail of $\append{[P_1,P_2]}{\append{Ps}{[P_1]}}$, exactly as we require in the sketch proof.

Of particular note is how we formalise the idea that $Qs$ crosses the edge $P_1P_2$ in terms of the function \code{polypath\_crossings}: we abstract away the $C$ and the $\Gamma$ variables with universal and existential quantifiers, knowing it is valid to do so based on our well-definedness theorems.

Using structural induction on the list $Ps$, we become tasked with showing that the polygonal path $Qs$ crosses the path $[P_2,P_3] + Ps' + [P_1]$ for any $P_3$ and $Ps'$ where $Ps'$ satisfies the inductive hypothesis. To do this, we show that the number of crossings at $P_1P_3$ is odd, and then apply the induction hypothesis, to tell us that $Qs$ crosses the path $[P_3] + Ps' + [P_1]$. The desired result then follows.

We assume here that $Qs$ does not cross at $P_2P_3$, since otherwise we are done. In Mizar~Light, this assumption is made quite literally:

\begin{center}\boxed{\code{assume}\ \neg(\exists X.\; \code{on\_polypath}\ [P_2, P_3]\ X \wedge \code{on\_polypath}\ Qs\ X)}\end{center}

Now according to our treatment of the idea of crossings from \S\ref{sec:CrossingsWellDefined}, our goal is formalised as
\begin{equation*}
% !C'. on_plane C' 'a
%                    /\ ~(?a. on_line A a /\ on_line C a /\ on_line C' a)
%                    ==> (?was_inside. ODD
%                              (polypath_crossings (A,C,C')
%                                 (polypath_new_was_inside (A,C,C') was_inside
%                                    (ADJACENT Qs))
%                                (ADJACENT Qs)))
  \begin{aligned}
    &\forall C.\; \Triangle{a}{P_1}{P_3}{C}\\
    &\quad \implies \exists \Gamma.\; \code{odd}\ (\code{polypath\_crossings}\ (P_1,P_3,C)\\
    &\qquad\qquad\qquad(\Gamma_{final}\ (P_1,P_3,C)\ \Gamma\ (\code{adjacent}\ Qs))\ (\code{adjacent}\ Qs)).
  \end{aligned}
\end{equation*}

There is actually a case-split to consider here. It is possible that $P_3$ lies on the line of $P_1P_2$, or, more specifically, on the \emph{ray} $\overrightarrow{P_1P_2}$\footnote{For this inference, we use our linear reasoning tactic.}. We shall not cover the details of this case. Suffice to say, it requires a complication of Theorem~\ref{eq:changeTriangle}, which we give in Appendix~\ref{app:JordanVerificationExtra}. Explaining it here would just obscure the basic ideas of the verification.

Thus, we shall assume that $P_3$ forms a triangle with $P_1P_2$. This means we can apply our assumption that there are an odd number of crossings at $P_1P_2$, namely
\begin{align*}&\forall C.\; \Triangle{a}{P_1}{P_2}{C}\\
    &\quad\implies \exists \Gamma.\; \code{odd}\ (\code{polypath\_crossings}\ (P_1,P_2,C)\\
    &\qquad\qquad\qquad(\Gamma_{final}\ (P_1,P_2,C)\ \Gamma\ (\code{adjacent}\ Qs))\ (\code{adjacent}\ Qs))).
\end{align*}

We now have an idea of how the quantifiers in this formula are to be used. We need $C$ to be arbitrary, because we must instantiate it with the particular vertex $P_3$ that we have obtained from the list $Ps$. We must then obtain an appropriate starting context $\Gamma$ depending on $C$, which is chosen according to which side of the edge $P_1P_2$ the vertex $P_3$ lies. 

By applying Theorems~\ref{eq:crossNZIntersect} and~\ref{eq:polypathCrossingsEven}, we can then conclude that there must be an odd number of crossings at $P_1P_3$ \emph{with respect to the triangle $P_1P_2P_3$}. All we need now to complete the inductive step is to generalise this claim by quantifying over the variable $P_2$. For this final step, we just use our well-definedness theorem~\eqref{eq:changeTriangle}.

\subsection{A Theorem of Polygonal Paths}\label{sec:PathTheorem}
Theorem~\ref{eq:oddCrossClosedPolypath} is all we need to show that a simple polygon divides the plane into two regions. It is interesting to note, however, that we have not mentioned path connectedness in this theorem, nor have we supposed that the polygons concerned are simple. This suggests there is a more general corollary to be had, one whose formalisation does not mention \emph{crossings} at all. Indeed, this ugly concept serves only as the crucial scaffolding of the verification. It appears absent when presenting the final theorem.

In a fairly short verification (42 steps), we apply Theorem~\ref{eq:oddCrossClosedPolypath} to obtain a beautifully symmetric theorem concerning arbitrary intersecting polygons:
\begin{equation}\label{eq:intersectPolypathClosed}
   % on_plane P1 'a /\ on_plane P2 'a /\ on_plane Q1 'a /\ on_plane Q2 'a 
   % /\ (!X. MEM X Ps ==> on_plane X 'a) /\ (!X. MEM X Qs ==> on_plane X 'a) 
   % /\ ~(?a. on_line P1 a /\ on_line P2 a /\ on_line Q1 a)
   % /\ ~(?a. on_line Q1 a /\ on_line Q2 a /\ on_line P1 a)
   % /\ between P1 X P2 /\ between Q1 X Q2
   % /\ P1 = LAST (CONS P2 Ps) /\ Q1 = LAST (CONS Q2 Qs)
   % ==> ?Y. on_polypath (CONS P2 Ps) Y /\ on_polypath (CONS Q1 (CONS Q2 Qs)) Y
   %         \/ on_polypath (CONS P1 (CONS P2 Ps)) Y /\ on_polypath (CONS Q2 Qs) Y"
  \begin{aligned}
    \vdash&\Triangle{a}{P_1}{P_2}{Q_1}\\
    &\wedge\Triangle{a}{Q_1}{Q_2}{P_1}\\
    &\wedge\between{P_1}{X}{P_2} \wedge \between{Q_1}{X}{Q_2}\\
    &\wedge P_1 = \code{last}\ Ps \wedge Q_1 = \code{last}\ Qs\\
    &\implies \exists Y.\; \code{on\_polypath}\ (\cons{P_2}{Ps})\ Y \wedge \code{on\_polypath}\ (\cons{Q_1}{\cons{Q_2}{Qs}})\ Y\\
    &\qquad\qquad\vee \code{on\_polypath}\ (\cons{P_1}{\cons{P_2}{Ps}})\ Y \wedge \code{on\_polypath}\ (\cons{Q_2}{Qs})\ Y.
  \end{aligned}
\end{equation}

In words, if we have polygons $P_1P_2\ldots P_1$ and $Q_1Q_2\ldots Q_1$ such that the segment $P_1P_2$ and $Q_1Q_2$ intersect, then one of the polygons intersects a non-trivial suffix of the other.

\begin{figure}
\centering\includegraphics[scale=0.5]{jordanVerification1/Theorem}
\caption{Arbitrary intersecting polygons}
\label{fig:IntersectingPolygons}
\end{figure}

The theorem places no constraints on the polygons other than that they cross at their first edges. They can have repeated vertices; they can self-intersect; they could even be the trivial polygons $P_1P_2P_1$ and $Q_1Q_2Q_1$. The point to visualise is that if two segments $P_1P_2$ and $Q_1Q_2$ cross one another, and we attempt to connect $P_2$ back to $P_1$ whilst attempting to connect $Q_2$ back to $Q_1$, we will find another point of intersection. See Figure~\ref{fig:IntersectingPolygons}.

\section{The Plane Divides into at Least Two Regions}\label{sec:FinalProofJordan1}
We can now verify the main theorem for this chapter. We assume a simple polygon $P_1P_2\ldots P_n$, and we must find two points off this polygon which cannot be connected by a polygonal path without crossing the simple polygon. Equivalently, any polygonal path connecting the two chosen points must intersect the simple polygon.

The strategy we use to prove this has already been covered in \S\ref{sec:ParityProofInformal}, and the verification respects the structure. In the sketch proof, we consider two rays emerging on either side of the edge $P_1P_2$. We find the points where these rays intersect $Ps$, and pick the point of intersection closest to $AB$. In our verification, this step is handled by a ``ray-casting'' theorem which we discuss in \S\ref{sec:RayCasting}. 

Ray-casting is the one and only place where we need to assume the simplicity of the simple polygon. In fact, we do not need to assume that much. All we really need to know is that there is \emph{some} edge of a polygon, and \emph{some} point $P$ inside that edge, such that $P$ does not lie on the rest of the polygon. Under these circumstances, we know that the polygon divides the plane into at least two regions.

This is to be contrasted with the verification in the next section. There, the assumption of a polygon's simplicity will feature heavily. The reason is that there are many ways for a polygon to divide the plane into multiple regions, but fewer ways for a polygon to restrict the number of regions to \emph{two}.

We have come this far by building a large tower of abstractions, complicated and unwieldy definitions, and theorems containing an irritatingly large number of hypotheses. The pay-off from this sort of verification is a result which throws out the scaffolding and brings us to a neat and easily grasped theorem.

\begin{equation}
  \begin{split}
    \vdash&\code{simple\_polygon}\ \alpha\ Ps\\
    &\implies \exists P.\;\exists Q.\; \code{on\_plane}\ P\ \alpha \wedge \code{on\_plane}\ Q\ \alpha\\
    &\qquad\wedge \neg\code{on\_polypath}\ Ps\ P \wedge \neg\code{on\_polypath}\ Ps\ Q\\
    &\qquad\wedge \neg\code{polypath\_connected}\ \alpha\ (\code{on\_polypath}\ Ps)\ P\ Q.
  \end{split}\tag{\ref{eq:jordanFormal1}}
\end{equation}

%%% Local Variables: 
%%% mode: latex
%%% TeX-master: "../thesis"
%%% End: 
