% MAKE SURE YOU MENTION ORDERED GEOMETRY HERE See paper notes
\chapter{Group II}\label{chapter:Group2Eval}
\section{Pasch's Axiom}
The first of Hilbert's \emph{proofs} appear in Group~II, and they make extensive use of an axiom due to Pasch. This axiom asserts that any line $a$ which enters a triangle $ABC$ on one side must leave by one of the other two sides, such as in the case depicted in Figure \ref{PaschDiagram}.

\begin{figure}\label{PaschDiagram}
% \begin{pspicture}(-2,0)(5,3.5)
% \psset{xunit=0.75cm}
% \psset{yunit=0.5cm}
% \put(1.3,0.2){\parbox{5cm}{$A$}}
% \put(6,0.2){\parbox{5cm}{$B$}}
% \put(5.2,2.6){\parbox{5cm}{$C$}}
% \put(4.3,0){\parbox{5cm}{$a$}}
% \psline[linewidth=0.15mm](2,1)(8,1)
% \psline[linewidth=0.15mm](2,1)(7,5)
% \psline[linewidth=0.15mm](7,5)(8,1)
% \psline[linewidth=0.25mm](5.5,0)(8,6)
% \end{pspicture}
% \caption{Axiom II,4}
\end{figure}

This is a complex axiom to apply. The claim that a line cannot intersect any of the triangle's vertices cashes out as three claims of \emph{non-incidence}, and as we explained in \S\ref{NonIncidenceTriangle}, a claim of non-incidence just asserts that three points form a triangle. Together with the triangle that is being intersected, this means we need to find four triangles every time we apply Pasch's axiom. The matter was formalised when we derived the following version of the axiom:

\begin{align*}
&\neg\code{collinear}\{A, B, C\}\\
\wedge\,&\neg\code{collinear}\{A, D, E\}\\
\wedge\,&\neg\code{collinear}\{B, D, E\}\\
\wedge\,&\neg\code{collinear}\{C, D, E\}\\
&\quad\rightarrow\between\,A\,B\,C \\
&\qquad\rightarrow\exists F.\; \code{collinear} \{D, E, F\} \wedge (\between\,A\,F\,C \vee \between\,B\,F\,C).
\end{align*}

The axiom is further complicated by its existential and disjunctive conclusion. Every time this axiom is applied in Hilbert's proofs, at least one of the disjuncts must be eliminated, by deriving a falsehood on its assumption. We are then left with an existential, which is often eliminated by showing that the point in question has already been constructed. In all but one case, Hilbert elides the case-splitting and the existential elimination. In our Isabelle formalisation, this elided reasoning turned into often complicated incidence arguments based on point sets.

\subsection{Pasch as a Tactic}
When Hilbert draws the reader's attention to an application of Pasch's axiom, he generally does so in the following way:
\begin{quote}By an application of Axiom~II,4 to the triangle $BCG$ and to the line $AD$....\end{quote}

The wording here suggests that Pasch is not being treated as a mere axiom, but as a two-argument \emph{procedure}, taking a triangle and a line (in this case, $BCG$ and $AD$). We followed up this idea by implementing our own two argument Pasch procedure, allowing us to write proof steps in the form

\vspace{0.5cm}
\texttt{\qquad\qquad by pasch\_on $\neg$COLLINEAR \{B, C, G\} and A,D}
\vspace{0.5cm}

Our procedure first identifies the necessary hypotheses needed to apply Pasch's axiom from the two supplied arguments. It then looks up these hypotheses in the step's justifying theorems. Once it has the existential disjunction, it runs a case-split on the two disjuncts, looking for an obvious contradiction using the automated inferencing tool we discuss in \S\ref{sec:DiscoveryImplementation}. It then eliminates the existential to introduce a new point.

\section{Analysis}
In this subsection, we discuss what we have learned about incidence reasoning with point sets based on our discovery tool. We then discuss two proofs which required significant effort to formalise in Isabelle, and show what their formalisation looks like when written in tandem with our discovery tool. We compare these new proofs with their prose counterparts, suggest weaknesses of the prose, and discuss an alternative proof which the automation allowed us to explore.

\subsection{Idle-Time Discoveries}\label{sec:IdleTimeDiscoveries}
We first describe some of the combinatorial details of the incidence reasoning. We implemented our tool with book-keeping, retaining information about which rules had been applied when and in what order, as it derived all the discovered facts. The results revealed fairly complicated chains of inference. 

\begin{figure}
  % \mbox{\subfigure[Replace $E$ with $G$]{
  %     \begin{pspicture}(-1,0)(5,3)
  %       \psset{yunit=0.5cm}
  %       \psset{xunit=0.5cm}
  %       \put(0.2,0.2){\parbox{5cm}{\scriptsize$A$}}
  %       \put(1.8,0.2){\parbox{5cm}{\scriptsize$B$}}
  %       \put(4.3,0.2){\parbox{5cm}{\scriptsize$D$}}
  %       \put(2.7,2.5){\parbox{5cm}{\scriptsize$E$}}
  %       \put(2.1,2){\parbox{5cm}{\scriptsize$G$}}
  %       \put(3.4,1.3){\parbox{5cm}{\scriptsize$H$}}
  %       \psline[linewidth=0.25mm](\ptfvA)(\ptfvD)
  %       \psline[linewidth=0.25mm](\ptfvD)(\ptfvG)
  %       \psline[linestyle=dashed,linewidth=0.4mm](\ptfvA)(\ptfvB)
  %       \psline[linestyle=dashed,linewidth=0.4mm](\ptfvB)(\ptfvG)
  %       \psline[linestyle=dashed,linewidth=0.4mm](\ptfvG)(\ptfvA)
  %       \psline[linestyle=dashed,linewidth=0.25mm](\ptfvB)(\ptfvE)
  %       \psline[linestyle=dashed,linewidth=0.25mm](\ptfvG)(\ptfvE)
  %     \end{pspicture}\label{fig:deriveAEH1}}}
  % \quad
  % \mbox{\subfigure[Replace $B$ with $D$]{
  %     \begin{pspicture}(0,0)(5,3)
  %       \psset{yunit=0.5cm}
  %       \psset{xunit=0.5cm}
  %       \put(0.2,0.2){\parbox{5cm}{\scriptsize$A$}}
  %       \put(1.8,0.2){\parbox{5cm}{\scriptsize$B$}}
  %       \put(4.3,0.2){\parbox{5cm}{\scriptsize$D$}}
  %       \put(2.1,2){\parbox{5cm}{\scriptsize$G$}}
  %       \put(3.4,1.3){\parbox{5cm}{\scriptsize$H$}}
  %       \psline[linewidth=0.25mm](\ptfvA)(\ptfvD)
  %       \psline[linestyle=dashed,linewidth=0.25mm](\ptfvA)(\ptfvB)
  %       \psline[linestyle=dashed,linewidth=0.25mm](\ptfvB)(\ptfvG)
  %       \psline[linestyle=dashed,linewidth=0.4mm](\ptfvA)(\ptfvD)
  %       \psline[linestyle=dashed,linewidth=0.4mm](\ptfvD)(\ptfvG)
  %       \psline[linestyle=dashed,linewidth=0.4mm](\ptfvG)(\ptfvA)
  %     \end{pspicture}}}

  % \mbox{\subfigure[Replace $D$ with $H$]{
  %     \begin{pspicture}(-1,0)(5,3)
  %       \psset{yunit=0.5cm}
  %       \psset{xunit=0.5cm}
  %       \put(0.2,0.2){\parbox{5cm}{\scriptsize$A$}}
  %       \put(4.3,0.2){\parbox{5cm}{\scriptsize$D$}}
  %       \put(2.1,2){\parbox{5cm}{\scriptsize$G$}}
  %       \put(3.4,1.3){\parbox{5cm}{\scriptsize$H$}}
  %       \psline[linewidth=0.25mm](\ptfvA)(\ptfvD)
  %       \psline[linestyle=dashed,linewidth=0.25mm](\ptfvD)(\ptfvH)
  %       \psline[linestyle=dashed,linewidth=0.4mm](\ptfvA)(\ptfvG)
  %       \psline[linestyle=dashed,linewidth=0.4mm](\ptfvG)(\ptfvH)
  %       \psline[linestyle=dashed,linewidth=0.4mm](\ptfvH)(\ptfvA)
  %     \end{pspicture}}}
  % \quad
  % \mbox{\subfigure[Replace $G$ with $E$]{
  %     \begin{pspicture}(0,0)(5,3)
  %       \psset{yunit=0.5cm}
  %       \psset{xunit=0.5cm}
  %       \put(0.2,0.2){\parbox{5cm}{\scriptsize$A$}}
  %       \put(2.7,2.5){\parbox{5cm}{\scriptsize$E$}}
  %       \put(2.1,2){\parbox{5cm}{\scriptsize$G$}}
  %       \put(3.4,1.3){\parbox{5cm}{\scriptsize$H$}}
  %       \psline[linestyle=dashed,linewidth=0.25mm](\ptfvA)(\ptfvG)
  %       \psline[linestyle=dashed,linewidth=0.25mm](\ptfvG)(\ptfvH)
  %       \psline[linestyle=dashed,linewidth=0.25mm](\ptfvH)(\ptfvA)
  %       \psline[linestyle=dashed,linewidth=0.4mm](\ptfvA)(\ptfvE)
  %       \psline[linestyle=dashed,linewidth=0.4mm](\ptfvE)(\ptfvH)
  %       \psline[linestyle=dashed,linewidth=0.4mm](\ptfvH)(\ptfvA)
  %     \end{pspicture}}}
\caption{Finding $AEH$ from $AEG$}
\label{fig:deriveAEH}
\end{figure}

There was one fact needed during a proof which particularly interested us, because in our proposal, we erroneously stated that \emph{it could not be derived}. The example comes from a step in the proof of Theorem~5, where we must infer that three points $A$, $E$ and $H$ form a triangle. The derivations are depicted in Figure \ref{fig:deriveAEH}. Each step corresponds to an application of our triangle introduction rule and can be understood as substituting points of a non-collinear triple one-at-a-time, until we have rewritten the initial triangle $ABE$ to $AEH$.

It might seem that we can just replace the point $B$ with the point $H$ to rewrite $ABE$ to $AEH$, but the triangle introduction rule requires a hypothesis about an appropriate collinear set and an appropriate point inequality. So the inference is more indirect. At first, we can substitute $E$ for $G$ using the line $AGE$. We then substitute $B$ for $D$ using the line $ABD$, and then $D$ for $H$ using the line $DGH$. Finally, we must use our point-inequality introduction rule to show that $E$ and $H$ are distinct on the basis of the triangle $AGH$ and the line $AGE$, after which we can substitute $G$ for $H$ using the line $AGE$.

We had not been able to realise this chain of argument manually in the Isabelle formalisation, though we had written many like it. In some of these other arguments, we would have to experiment with different starting triangles to get the right conclusions, and manually determine the instantiations for the point variables in our rules. But once we had our automated tool, the inferences could be discovered for us, and we were relieved of the complex combinatorial steps on which they depend.

\subsection{Assisted Formalisations in Group II}\label{sec:FormalisationAnalysis}
The opening theorems in Group~II, Theorems~3, 4 and 5 (see Appendix~\ref{app:GroupII}), are the first theorems Hilbert explicitly proves in the tenth edition. This group is concerned with \emph{order}. Theorem~3 tells us that given two points, we can put a third point in between them. It was absent in the first edition, instead paired with Axiom~II,2. Similarly, Theorem~4 was originally part of Axiom~II,3. Originally, this axiom stated that given three points on a line, exactly one lies between the other two. This was weakened by the tenth edition to merely say \emph{no more than one} lies between the other two. Finally, Theorem~5 was an additional axiom in Group~II verbatim. The derivations of these axioms was not just due to Hilbert. As he mentions in the footnotes to the tenth edition, it was respectively Wald and Moore who derived Theorems~4 and~5.

The fact that Hilbert originally assumed these theorems were unprovable can be taken as evidence that they are non-trivial. After all, Hilbert did not want redundant axioms --- the second chapter of \emph{Foundations of Geometry} is concerned in part with their mutual independence. Moreover, according to our Isabelle formalisation, these arguments were further complicated by a great deal of fussy incidence reasoning. In any case, we argue that they are a suitable test-bed for our tool's effectiveness, both in deferring implicit reasoning to automation, and in exploring and analysing the logic of the proofs.

\subsubsection{Analysis of Theorem~4}
We give the prose version of the proof as it appears in the tenth edition:

\begin{figure}
  % \begin{pspicture}(-1,0)(5,7)
  %   \put(0.8,0.6){\parbox{5cm}{\scriptsize$A$}}
  %   \put(3.8,0.6){\parbox{5cm}{\scriptsize$B$}}
  %   \put(7.0,0.6){\parbox{5cm}{\scriptsize$C$}}
  %   \put(4.4,2.6){\parbox{5cm}{\scriptsize$D$}}
  %   \put(5.6,3.8){\parbox{5cm}{\scriptsize$E$}}
  %   \put(3.2,3.7){\parbox{5cm}{\scriptsize$F$}}
  %   \put(4.7,5.2){\parbox{5cm}{\scriptsize$G$}}
    
  %   \psline[linewidth=0.25mm](\ptfrA)(\ptfrC)
  %   \psline[linewidth=0.25mm](\ptfrB)(\ptfrG)
  %   \psline[linewidth=0.25mm](\ptfrA)(\ptfrE)
  %   \psline[linewidth=0.25mm](\ptfrC)(\ptfrF)
  %   \psline[linewidth=0.25mm](\ptfrA)(\ptfrG)
  %   \psline[linewidth=0.25mm](\ptfrC)(\ptfrG)
  % \end{pspicture}
\caption{Proof of Theorem~4}
\end{figure}

\begin{quote}
THEOREM 4. Of any three points $A$, $B$, $C$ on a line there always is one that lies between the other two.

Let $A$ not lie between $B$ and $C$ and let also $C$ not lie between $A$ and $B$. Join a point $D$ that does not lie on the line $AC$ with $B$ and choose by Axiom~II,2 a point $G$ on the connecting line such that $D$ lies between $B$ and $G$. By an application of Axiom~II,4 to the triangle $BCG$ and to the line $AD$ it follows that the lines $AD$ and $CG$ intersect at a point $E$ that lies between $C$ and $G$ . In the same way, it follows that the lines $CD$ and $AG$ meet at a point $F$ that lies between $A$ and $G$.

If Axiom~II,4 [(Pasch's axiom)] is applied now to the triangle $AEG$ and to the line $CF$ it becomes evident that $D$ lies between $A$ and $E$, and by an application of the same axiom to the triangle $AEC$ and to the line $BG$ one realizes that $B$ lies between $A$ and $C$.
\end{quote}

The original Isabelle proof ran to sixty-nine steps. Most of the steps were unilluminating, consisting of a complex combination of tactics and picky variable instantiations. We had to use comments judiciously to show how the prose translated into the formal proof steps. But our new proof has just thirteen lines, each readily understandable, and matching the prose in an almost one-to-one fashion (the formalisation of theorems \texttt{construct\_triangle} and \texttt{g22} are given in Appendix~\ref{app:GroupIIFormalisations}).

\vspace{0.5cm}
\begin{minipage}{\linewidth}
  \footnotesize 
  \noindent\texttt{prove collinear \{A, B, C\} $\wedge$ A $\neq$ B $\wedge$ A $\neq$ C $\wedge$ B $\neq$ C}

  \texttt{\qquad\qquad$\neg$ between A C B $\wedge$ $\neg$between B A C $\Longrightarrow$ between A B C}

  \texttt{assume collinear \{A, B, C\} $\wedge$ A $\neq$ B $\wedge$ A $\neq$ C $\wedge$ B $\neq$ C}

  \texttt{\qquad\qquad $\neg$ between A C B $\wedge$ $\neg$between B A C}

  \texttt{so consider D such that $\neg$collinear \{A, B, D\} by construct\_triangle}

  \texttt{obviously consider G such that between B D G by g22}

  \texttt{obviously consider E such that collinear \{A, D, E\} $\wedge$ between C E G}

  \texttt{\qquad\qquad by pasch\_on $\neg$collinear \{B, C, G\} and A,D}

  \texttt{obviously consider F such that collinear \{C, D, F\} $\wedge$ between A F G}

  \texttt{\qquad\qquad by pasch\_on $\neg$collinear \{A, B, G\} and C,D}

  \texttt{obviously consider H such that collinear \{C, F, H\} $\wedge$ between A H E}

  \texttt{\qquad\qquad by pasch\_on $\neg$collinear \{A, E, G\} and C,F}

  \texttt{obviously have H $=$ D}

  \texttt{obviously consider I such that COLLINEAR \{B, D, I\} $\wedge$ between A I C}

  \texttt{\qquad\qquad by pasch\_on $\neg$COLLINEAR \{A, C, E\} and B,G}

  \texttt{obviously have I $=$ B}

  \texttt{qed}
\end{minipage}
\vspace{0.5cm}

Notice that we mention points $H$ and $I$ in our proof. This is because our Pasch tactic always introduces a fresh variable when eliminating the existential conclusion of Pasch's axiom. We intend to correct this flaw. If the Pasch tactic can prove that the new point is equal to an existing point, then the existential hypothesis can be eliminated entirely. Rather than write

\vspace{0.5cm}
\begin{minipage}{\linewidth}
  \footnotesize
  \texttt{obviously consider H such that COLLINEAR \{C, F, H\} $\wedge$ between A H E}

  \texttt{\qquad\qquad by pasch\_on $\neg$COLLINEAR \{A, E, G\} and C,F}

  \texttt{obviously have H $=$ D}
\end{minipage}
\vspace{0.5cm}

\noindent we can instead give a more direct formalisation which is more faithful to the prose, and does not involve unnecessary labels for points in the construction:

\vspace{0.5cm}
\begin{minipage}{\linewidth}
  \footnotesize
  \texttt{obviously have COLLINEAR \{C, D, F\} $\wedge$ between A D E}

  \texttt{\qquad\qquad by pasch\_on $\neg$COLLINEAR \{A, E, G\} and C,F}
\end{minipage}
\vspace{0.5cm}

\subsubsection{An Alternative Proof of Theorem 4}\label{sec:Theorem4Alt}
Having reproduced this theorem, we extended our automated discovery tool to apply the Pasch tactic non-deterministically. The tool searched through all ways to make four or fewer successive applications of Pasch's axiom, as in the original proof, looking for alternative ways to prove the theorem.

The tool found several ``alternatives'', but most of these yield symmetries of the original proof. In some cases, two independent applications of Pasch's axioms were applied in reverse order. In other cases, the proof was identical to the original up to a symmetric relabelling of the points involved in the construction. Only one new proof was revealed up to symmetry. We give it now in a prose formulation with an accompanying diagram (Figure \ref{fig:Theorem4Alt}).

\begin{figure}
  % \begin{pspicture}(-1.5,0)(5,7)
  %   \psset{xunit=0.75cm}

  %   \put(0.5,0.8){\parbox{5cm}{\scriptsize$A$}}
  %   \put(1.5,3.0){\parbox{5cm}{\scriptsize$B$}}
  %   \put(2.8,5.1){\parbox{5cm}{\scriptsize$C$}}
  %   \put(3.3,2.2){\parbox{5cm}{\scriptsize$D$}}
  %   \put(5.0,3.4){\parbox{5cm}{\scriptsize$E$}}
  %   \put(3.3,3.3){\parbox{5cm}{\scriptsize$F$}}
  %   \put(8.0,0.8){\parbox{5cm}{\scriptsize$G$}}
    
  %   \psline[linewidth=0.25mm](\ptaA)(\ptaC)
  %   \psline[linewidth=0.25mm](\ptaC)(\ptaG)
  %   \psline[linewidth=0.25mm](\ptaB)(\ptaG)
  %   \psline[linewidth=0.25mm](\ptaA)(\ptaE)
  %   \psline[linewidth=0.25mm](\ptaC)(\ptaD)
  %   \psline[linewidth=0.25mm](\ptaB)(\ptaE)
  % \end{pspicture}
\caption{Alternative Proof of Theorem~4}
\label{fig:Theorem4Alt}
\end{figure}

\begin{quote}
Assume $A$, $B$ and $C$ are collinear, with $A$ not between $B$ or $C$ and $C$ not between $A$ or $B$. We find a point $D$ off the line $AC$ and extend it to $G$ using Axiom~II,2. We then use Axiom~II,4 on the triangle $BCG$ and the line $AD$ to find the point $E$ between $C$ and $G$. We use Axiom~II,4 on the triangle $BEG$ and the line $CD$ to find the point $F$ between $B$ and $E$. We use the axiom again on the triangle $ABE$ and the line $CF$ to show that $D$ lies between $A$ and $E$. Finally, we can use the axiom on the triangle $ACE$ and the line $BG$ to find $B$ between $A$ and $C$.
\end{quote}

In some sense, the basic strategy of our alternative proof is the same as the original. We first construct a point $D$ off the line $AC$ and extend the line $BD$ to $G$. This tells us that $D$ lies between $B$ and $G$, which gives us our first opportunities to use Pasch's axiom. In both cases, our goal is to use this axiom in order to place the point $D$ between $A$ and $E$, so that a final application of Pasch's axiom to the triangle $ACE$ and the line $BG$ will place $B$ between $A$ and $C$. The two proofs only differ in how they construct the point $F$, and how they use $F$ to place $D$ between $A$ and $E$.

In Hilbert's proof, $F$ is found on the edge of the outer triangle $ACG$, and is placed symmetrically with $E$. Indeed, the proof is valid when exchanging all references of $E$ and $F$, whereas in our proof, $F$ is placed in the interior of $ACG$ while $E$ lies asymmetrically on the triangle's edge. 

So Hilbert's proof has a lot of symmetry: $E$ could be replaced with $F$; and it is clear that the third application of Pasch could be made on the triangle $CFG$ and the line $AE$, instead of $AEG$ and the line $CF$. Our proof makes it clear that, while $E$ and $F$ can be constructed symmetrically and independently, only one of these points is distinguished in the final few steps. 

It is worth drawing some attention to the subtlety of the incidence reasoning here. We could have applied Pasch differently to find the point $F$, using the triangle $CDG$ and the line $BE$. This would tell us that $F$ lies on the line $BE$ between $C$ and $D$ (before, it told us that $F$ lies on the line $CD$ between $B$ and $E$). Now it might seem that we can use a symmetrical application of Pasch on the same line $BE$ and the triangle $ACD$, which would solve the goal putting $B$ between $A$ and $C$. But at this stage in the proof, we must consider the possibility that $BF$ exits the triangle $ACD$ between $A$ and $D$. This possibility is not yet eliminable by incidence reasoning alone.

This observation is not apparent in the proof. In his eleven uses of Pasch across Theorems~3, 4 and 5, Hilbert only considers the case-split implied by the axiom twice. And yet it takes up a significant amount of combinatorial reasoning about incidence. It is difficult to justify leaving this complexity implicit, when it has consequences on the shape of the proof which we find difficult to argue as \emph{obvious}. The reasoning may be laborious, but we demand rigour when Hilbert's geometrical intuition is supposed to have been replaced by uninterpreted symbolic reasoning. So it is not clear whether these incidence proof steps can be justified as being left implicit. Of course, with machine-checked automation, we can now be safely confident in the correctness of the implicit steps.

\subsubsection{Analysis of Theorem~5}\label{sec:Theorem5}
Theorem~5 is the most complex of the three theorems, taking up almost an entire page of the English translation. Effectively, the theorem gives a transitivity property for point ordering. It is split into three parts. 

% \begin{pspicture}(0,0)(8,8)
%   \put(0.8,0.6){\parbox{5cm}{\scriptsize$A$}}
%   \put(3.8,0.6){\parbox{5cm}{\scriptsize$B$}}
%   \put(7.0,0.6){\parbox{5cm}{\scriptsize$C$}}
%   \put(9.0,0.6){\parbox{5cm}{\scriptsize$D$}}
%   \put(6.2,5.0){\parbox{5cm}{\scriptsize$E$}}
%   \put(5.5,7.2){\parbox{5cm}{\scriptsize$F$}}
%   \put(4.4,4.1){\parbox{5cm}{\scriptsize$G$}}
%   \put(6.6,2.8){\parbox{5cm}{\scriptsize$H$}}
%   \psline[linewidth=0.25mm](\ptfvA)(\ptfvD)
%   \psline[linewidth=0.25mm](\ptfvD)(\ptfvG)
%   \psline[linewidth=0.25mm](\ptfvA)(\ptfvE)
%   \psline[linewidth=0.25mm](\ptfvB)(\ptfvF)
%   \psline[linewidth=0.25mm](\ptfvC)(\ptfvF)
% \end{pspicture}

\begin{quote}
THEOREM 5. Given any four points on a line, it is always possible to label them $A$, $B$, $C$, $D$ in such a way that the point labeled $B$ lies between $A$ and $C$ and also between $A$ and $D$, and furthermore, that the point labeled $C$ lies between $A$ ad $D$ and also between $B$ and $D$.

PROOF. Let $A$, $B$, $C$, $D$ be four points on a line $g$. The following will now be shown:

1. If $B$ lies on the segment $AC$ and $C$ lies on the segment $BD$ then the points $B$ and $C$ also lie on the segment $AD$. By Axioms~I,3 and II,2 choose a point $E$ that does not lie on $g$, [and] a point $F$ such that $E$ lies between $C$ and $F$. By repeated applications of Axioms~II,3 and II,4 it follows that the segments $AE$ and $BF$ meet at a point $G$, and moreover, that the line $CF$ meets the segment $GD$ at a point $H$. Since $H$ thus lies on the segment $GD$ and since, however, by Axiom~II,3, $E$ does not lie on the segment $AG$, the line $EH$ by Axiom~II,4 meets the segment $AD$, i.e. $C$ lies on the segment $AD$. In exactly the same way one shows analogously that $B$ also lies on this segment.
\end{quote}

\subsubsection{Formalisation of Theorem 5}
We have given the first part of Hilbert's proof. We now give its formalisation. Again, the formalisations of named theorems are given in Appendix~\ref{app:GroupIIFormalisations}.

\vspace{0.5cm}
\begin{minipage}{\linewidth}
  \footnotesize 
  \texttt{prove between A B C $\wedge$ between B C D $\Longrightarrow$ between A C D}

  \texttt{assume between A B C $\wedge$ between B C D}

  \texttt{so have A$\neq$D by swap\_order and no\_between\_permute}

  \texttt{obviously consider E such that $\neg$collinear \{A, B, E\} by}

  \texttt{$\qquad\qquad$ construct\_triangle}

  \texttt{obviously consider F such that collinear \{C, E, F\} by g22}

  \texttt{obviously consider G such that collinear \{B, F, G\} $\wedge$ between A G E}

  \texttt{$\qquad\qquad$ by pasch\_on $\neg$collinear \{A, C, E\} and B,F}

  \texttt{obviously consider G2 such that collinear \{A, E, G2\} $\wedge$}

  \texttt{$\qquad\qquad$ BETWEEN B G2 F by pasch\_on $\neg$collinear \{B, C, F\} and A,E}

  \texttt{obviously have G $=$ G2}

  \texttt{obviously consider H such that collinear \{C, E, H\} $\wedge$ between D H G}

  \texttt{$\qquad\qquad$ by pasch\_on $\neg$collinear \{B, D, G\} and C,F}

  \texttt{obviously consider I such that collinear \{E, H, I\} $\wedge$ between A I D}

  \texttt{$\qquad\qquad$ by pasch\_on $\neg$collinear \{E, H, I\} and E,H}

  \texttt{obviously have C $=$ I}

  \texttt{qed}
\end{minipage}
\vspace{0.5cm}

\subsubsection{A Comparison with the Isabelle Formalisation}
In our Isabelle formalisation, the proof of Theorem~5 runs to approximately 135 lines of often complicated formal proof steps. Most of these steps are required to derive the non-degeneracy conditions needed for Pasch's axiom, and all of them have been eliminated in the HOL~Light formalisation. 

The complexity of the original Isabelle formalisation had got the better of us. As we mentioned in \S\ref{sec:IdleTimeDiscoveries}, we had not been able to show that $E \neq H$, a fact which is needed for the final application of Pasch in the above proof. In fact, in our proposal, we had suggested its derivation was \emph{impossible}, and we had expected our algorithm to confirm this. It refuted it! After inspecting its history of inferences, we found that the algorithm derived $E\neq H$ after step (c) in Figure \ref{fig:deriveAEH}. At this step, we know that $AGH$ forms a triangle while $AEG$ is a line. Hence, $E$ and $H$ must be distinct.

This takes us back to a recurring question. While we have shown that Hilbert's prose proof is logically valid, we must ask \emph{is it \emph{clearly} valid?} We had not thought so originally, and the confirmation only came after complex incidence reasoning which we had to leave to automation. It could be argued that these chains of reasoning should have been explicit in the prose, perhaps with warnings about when Pasch cannot be applied, such as the one we gave in \S\ref{sec:Theorem4Alt}. This would help guide the reader through the subtleties of reasoning about incidence at this stage of the text, where we have so little geometric intuition that we can validly draw upon. This would have an advantageous side-effect on the proof. Assisting the reader will also assist the automated tool if it is rerun in batch fashion: it will be able to prune its search space down to a single path.

\subsubsection{A Comparison with Hilbert's Prose}
Our new formalisation gives an almost one-to-one correspondence between our formal proof and Hilbert's prose, but there are still differences which draw our attention.

Firstly, notice that in the prose, Hilbert compresses his applications of Pasch's axiom (Axiom II,4):

\begin{quote}
By repeated applications of Axioms II,3 and II,4 it follows that the segments $AE$ and $BF$ meet at a point $G$, and moreover, that the line $CF$ meets the segment at the point $GD$ at a point $H$.
\end{quote}

We can see from our formal version that this cashes out in exactly three applications of Pasch's axiom. While Hilbert does not give this number explicitly, it is implied by a subtle use of language in the English translation: ``the segments $AE$ and $BF$ meet at a point $G$'' while ``the \emph{line} $CF$ meets the segment $GD$ at a point $H$'' (our emphasis). Now if we are to show that two \emph{segments} intersect, we must derive \emph{two} facts of betweeness, and therefore we need \emph{two} applications of Pasch. But if we are to show that a \emph{line} and a segment intersect, we only need only one fact of betweeness. It seems that Hilbert was aware of this detail, and choose his words carefully, and as with Theorem~4, it indicates a subtlety of the logic which is not immediately apparent from the diagram.

Secondly, notice that the first step in our proof is needed to show that $A$ and $D$ are distinct. If they were, then $B$ would lie between $A$ and $C$, whilst $C$ would lie between $A$ and $B$, which is impossible. This argument does not appear in the prose. Hilbert does not even assert that the points are distinct, and so it might seem that our automated tool should take care of the matter. However, we do not think this would be an appropriate use of automation, since at this stage, we cannot make such arguments \emph{systematic}. We discuss the matter further in Section \ref{sec:Chains}.

Next, note the following step in Hilbert's proof: ``...and since, however, by Axiom II,3, $E$ does not lie on the segment $AG$,....'' Here, Hilbert is eliminating a case-split in an application of Pasch. In our formal proof, there was no need to. As far as our tool is concerned, there is nothing special about this case-split. It is eliminated by the same systematic reasoning that allows all the other cases to be eliminated. We would therefore argue that, for the sake of consistency of presentation, it should have been omitted in Hilbert's prose.

Finally, we were able to elide the final step, where Hilbert proves that $B$ lies between $A$ and $D$. Hilbert implies this can be achieved by an analogous proof, but it is actually a corollary of the above: we just use Hilbert's first axiom in the group which tells us that \mbox{\texttt{BETWEEN A B C}} implies \mbox{\texttt{BETWEEN C B A}}. Therefore, we take the theorem, swap $A$ and $D$, and swap $B$ and $C$, and obtain the required result.

We briefly mention the second part of the proof, which derives the following:

\begin{quote}2. If $B$ lies on the segment $AC$ and $C$ lies on the segment $AD$ then $C$ also lies on the segment $BD$ and $B$ also lies on the segment $AD$.\end{quote}

We omit the details of its formalisation. Suffice to say, we noticed the same neat correspondence between our formal proof steps and Hilbert's prose. 

We \emph{do} have a good deal to say about the final part of the proof, which in our HOL Light formalisation has become the base case of its later inductive generalisation. We discuss it in detail in the next section.

% \vspace{0.5cm}
% {\footnotesize
% \texttt{prove BETWEEN A B C $\wedge$ between A C D $\Longrightarrow$ between B C D`}

% \texttt{BETWEEN A B C $\wedge$ BETWEEN A C D}

% \texttt{assist consider G such that $\neg$COLLINEAR {A, B, G}}

% \texttt{$\qquad\qquad$ by construct\_triangle}

% \texttt{assist consider F such that BETWEEN B G Fa by g22}

% \texttt{consider H such that COLLINEAR \{C, F, H\} $\wedge$}

% \texttt{$\qquad\qquad$ BETWEEN A H G $\vee$ BETWEEN B H G}

% \texttt{$\qquad\qquad$ by pasch\_on $\neg$COLLINEAR \{E, H, I\} and E and H at 1}

% \texttt{case BETWEEN A H G}

% \texttt{$\quad$ qed by pasch\_on $\neg$COLLINEAR \{A, B, G\} and C and F}

% \texttt{case BETWEEN B H G} 

% \texttt{$\quad$ consider $I$ such that COLLINEAR \{Fa, H, Ia\} $\wedge$ BETWEEN B Ia D}

% \texttt{$\qquad\qquad$ by pasch\_on $\neg$COLLINEAR \{B, D, G\} and F and H}

% \texttt{$\quad$ assist qed}

% \texttt{finished from 1}

% \texttt{qed}}
% \vspace{0.5cm}

\subsection{Results}\label{sec:Results}
A relatively simple description of a discoverer can now systematically recover the implicit incidence-reasoning in Hilbert's \emph{Foundations of Geometry}. We show its results through part of an example proof. Here, we are trying to prove a transitivity property of Hilbert's three-place \emph{between} relation on points: if $B$ lies between $A$ and $C$, and $C$ between $B$ and $D$, then $C$ lies between $A$ and $D$.

\vspace{0.2cm}
\noindent\fbox{\begin{minipage}{11.9cm}
\footnotesize\code{prove between A B C $\wedge$ between B C D $\Longrightarrow$ between A C D}

\code{assume between A B C $\wedge$ between B C D at 0}
\code{consider E such that such that \ttriangle{A}{B}{E} from 0 by II,1 and triangle}
\end{minipage}}
\vspace{0.1cm}

The \code{assume} step adds the goal's antecedent to the current hypotheses, while the \code{consider} step introduces a non-collinear point $E$ using one of Hilbert's axioms and a lemma \code{triangle} (see Appendix~\ref{app:axioms}). These hypotheses form the context for our discoverer. They are automatically picked up by \code{monitor} and then fed through our incidence network to produce the following theorems within 0.31 seconds:\footnote{We have tested this on an Intel Core 2 2.53GHz machine.}

\vspace{0.2cm}
\noindent\doublebox{\begin{minipage}{11.9cm}
\footnotesize\code{A$\neq$E, B$\neq$E, between A B C, between B C D, A$\neq$B, A$\neq$C, B$\neq$C, B$\neq$D,}\\
\code{C$\neq$D, \ttriangle{A}{B}{E},\ttriangle{A}{C}{E},}\\
\code{\ttriangle{B}{C}{E}, C$\neq$E,}\\
\code{\collinearfour{A}{B}{C}{D},}\\
\code{\ttriangle{B}{D}{E}, \ttriangle{C}{D}{E},}\\
\code{\planarfive{A}{B}{C}{D}{E}, D$\neq$E}
\end{minipage}}
\vspace{0.1cm}

The \code{obviously} keyword picks up these theorems, and from $C\neq E$ we are able to find a point $F$:

\vspace{0.2cm}
\noindent\fbox{\begin{minipage}{11.9cm}
\footnotesize\code{obviously by\_incidence consider F such that between C E F by II,2 at 1}
\end{minipage}}
\vspace{0.1cm}

The next set of discovered theorems are found within 1.21 seconds:

\vspace{0.2cm}
\noindent\doublebox{\begin{minipage}{11.9cm}
\footnotesize\code{between C E F, \collinearthree{C}{E}{F}, C$\neq$F, E$\neq$F,}\\
\footnotesize\code{\planarsix{A}{B}{C}{D}{E}{F},}\\
\footnotesize\code{A$\neq$F, B$\neq$F, D$\neq$F}\\
\footnotesize\code{\ttriangle{A}{C}{F}, \ttriangle{A}{E}{F},}\\
\footnotesize\code{\ttriangle{B}{C}{F}, \ttriangle{B}{E}{F},}\\
\footnotesize\code{\ttriangle{C}{D}{F}, \ttriangle{D}{E}{F},}\\
\footnotesize\code{\ttriangle{A}{B}{F}, \ttriangle{B}{D}{F}},
\end{minipage}}
\vspace{0.1cm}

The rest of the proof consists of repeatedly applying a complex axiom due to Pasch. The axiom says that if a line enters one side of a triangle then it leaves by one of the other two sides. By cleverly applying this axiom, it is possible to prove our original theorem (this is not a trivial matter, and the proof had eluded Hilbert in the first edition of \emph{Foundations~of~Geometry} where the theorem was an axiom; the proof was later supplied by Moore~\cite{MooreProof}). 

The challenge, however, lies in verifying when all the preconditions on Pasch's Axiom have been met, something we handle by adding a discover \code{by\_pasch} to our existing incidence discovery (we omit the definition for space). It reveals the following additional theorems, found within 2.82 seconds.

\vspace{0.2cm}
\noindent\doublebox{\begin{minipage}{11.9cm}
\footnotesize
\code{$\exists$G. \collinearthree{B}{E}{G} $\wedge$ (between A G C $\vee$ between A G F)}\\ 
\code{$\exists$G. \collinearthree{A}{E}{G} $\wedge$ (between B G C $\vee$ between B G F)}\\
\code{$\exists$G. \collinearthree{B}{E}{G} $\wedge$ (between A G F $\vee$ between C G F)}\\
\code{$\exists$G. \collinearthree{B}{E}{G} $\wedge$ (between C G D $\vee$ between D G F)}\\
\code{$\exists$G. \collinearthree{B}{F}{G} $\wedge$ (between A G E $\vee$ between C G E)}\\
\code{$\exists$G. \collinearthree{D}{E}{G} $\wedge$ (between B G C $\vee$ between B G F)}
\end{minipage}}
\vspace{0.1cm}

Further exploration of the proof involves applying one of these theorems. We can, for instance, try the penultimate instance with the step

\vspace{0.2cm}
\noindent\fbox{\begin{minipage}{11.9cm}
\footnotesize\code{obviously by\_pasch consider G such that \collinearthree{B}{F}{G}\\
$\wedge$ (between A G E $\vee$ between C G E)}
\end{minipage}}
\vspace{0.1cm}

% The discoverer is now faced with a case-split, but in the right-branch, it infers $\code{between C G E} \rightarrow \bot$, after which further discovery stops. In the left branch, we obtain the following:

The \code{monitor} now picks up the disjunction and creates a tree to represent a case-split. As this feeds into the discoverer, our combinators will automatically partition the search on the two assumptions. Our discoverer then produces three sets of theorems in 9.58 seconds \footnote{Note that the outputs from the three sets are interleaved, and are not generated simultaneously. While the full set of theorems requires 9.58 seconds, most of the theorems shown here are actually generated in under 1 second. The theorem required to advance the proof, \code{between C G E $\rightarrow$ F = G} is generated in 6.19 seconds.}

The first set of theorems are proven independently of the case-split:

\vspace{0.2cm}
\noindent\doublebox{\begin{minipage}{11.9cm}
\footnotesize
\code{\planarseven{A}{B}{C}{D}{E}{F}{G}}\\
\code{C$\neq$G $\wedge$ E$\neq$G $\wedge$ A$\neq$G $\wedge$ D$\neq$G}
\end{minipage}}
\vspace{0.1cm}

The next set of theorems are discovered in a branch on the assumption of \code{between A G E}:

\vspace{0.2cm}
\noindent\doublebox{\begin{minipage}{11.9cm}
\footnotesize
\code{between A G E, \collinearthree{A}{G}{E}, B$\neq$G, F$\neq$G,}\\
\code{\ttriangle{A}{B}{G}, \ttriangle{B}{E}{G}}\\
\code{\ttriangle{A}{C}{G}, \ttriangle{C}{E}{G}}\\
\code{\ttriangle{E}{F}{G}, \ttriangle{B}{D}{G}}\\
\code{\ttriangle{C}{D}{G}, \ttriangle{C}{F}{G}}\\
\code{\ttriangle{D}{F}{G}}
\end{minipage}}
\vspace{0.1cm}

The final set of theorems are discovered in a branch on the assumption of \code{between C G E}:

\vspace{0.2cm}
\noindent\doublebox{\begin{minipage}{11.9cm}
\footnotesize
\code{between C G E, \collinearfour{C}{E}{F}{G}, B$\neq$G,}\\
\code{\ttriangle{A}{C}{G}, \ttriangle{A}{E}{G}}\\
\code{\ttriangle{B}{C}{G}, \ttriangle{B}{E}{G}}\\
\code{\ttriangle{C}{D}{G}, \ttriangle{D}{E}{G}}\\
\code{\ttriangle{A}{B}{G}, \ttriangle{B}{D}{G}}\\
\code{F = G, between C F E}
\end{minipage}}
\vspace{0.1cm}

The \code{obviously} step collapses the stream of trees, pushing the branch labels into the theorems as antecedents, and then uses the resulting lemmas to justify the step. Thus, the fact \code{F = G} becomes \code{between C G E $\rightarrow$ F = G}. This fact is sufficient to derive a contradiction with \code{between C E F} and thus eliminate the case-split:

\vspace{0.2cm}
\noindent\fbox{\begin{minipage}{11.9cm}
\footnotesize\code{obviously by\_incidence have between A G E from 1 by II,3}
\end{minipage}}
\vspace{0.1cm}

The rest of the proof proceeds similarly. While the prose proof has 9 steps and our earlier formalisation without discovery runs to over 80 steps, the new formalisation has just 17 steps. We found this roughly 80\% reduction in proof length across all 18 theorems from our earlier formalisation, with the new formalisations comparing much more favourably with the prose.
