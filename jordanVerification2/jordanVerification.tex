\chapter{Verifying the Polygonal JCT: Part II}\label{chapter:JordanVerification2}
We are nearing the end of our verifications. All that remains is to verify the second half of the Jordan Curve Theorem for polygons based on the axioms of Hilbert's ordered geometry. In this half of the verification, we must prove that a simple polygon separates its plane into at most two regions. As discussed when we gave the formalisation of this theorem in Chapter~\ref{chapter:JordanFormalisation}, it amounts to proving that given three points in the plane and not on the polygon, at least two of them are connected by a polygonal path.

This is effectively a maze navigation problem, lively and visual, with lots of geometrically interesting lemmas. Unlike the ``crossings'' of the last chapter, the basic concepts we appeal to are reflected cleanly in the low-level details of the verification, rather than being obscured by case-analyses and edge cases.

In discussing our verification, we will cover the same basic ground as we did in the last chapter. In \S\ref{sec:SketchProofJordan2}, we shall lay out the general approach of the proof. In \S\ref{sec:Jordan2Formulation}, we will look more closely at some of the basic machinery we will need, and formalise the key concepts in higher-order logic. Then, in \S\ref{sec:Jordan2Lemmas}, we shall cover the key lemmas that support our basic formalised concepts. As in the last chapter, these lemmas can be divided into those which introduce points in a geometrical configuration, and those which allow us to infer properties of the resulting configurations. 

In the rest of the chapter, we shall look in more detail at how the lemmas are applied to recover all the details of the sketch proof. We provide a few readable extracts of interesting verifications, demonstrating how faithfully we can formalise the intuitive synthetic arguments.

\section{Strategy}\label{sec:SketchProofJordan2}
The basic intuition behind the proof is similar to the ones presented by Veblen~\cite{Veblenphd} and Feigl~\cite{FeiglJordan}. We follow Veblen's proof the most closely. Contrary to Guggenheimer's~\cite{GuggenheimerJordanCurve} claim that Veblen's proof only holds for convex polygons, we believe the evidence of this chapter shows that Veblen was basically correct. That said, our verification is based on a more thorough analysis than presented by Veblen.

We are required to show that, given three points in the plane and not on a polygon, at least two of them are connected by a polygonal path. Let us reinterpret this and understand the three points as three players trying to navigate a polygonal maze.

Our basic goal is to get the three players ``next to'' the same edge. Then we just need to find a path between whichever of the two players are on the same side of that edge. We will find that the difficulty here lies in getting the players through potentially very tight corridors, and around difficult corners. We must show how to obtain paths for the players without recourse to notions such as comparable directions, parallel lines or distances. This will rule out common approaches to the theorem, such as the one given by Tverberg~\cite{TverbergJordan}. In Tverberg's proof, we just need to consider a sufficiently small region around the edges of the maze (an ``offset curve'' ), which we know to be polygonal path-connected. Without notions of distance, this description is out of scope of Hilbert's ordered geometry.

One way to formulate the idea that the players are ``next to'' the same edge of the maze is to assert that all three have line-of-sight to that edge. This metaphor does not appear explicitly in Veblen or Feigl's proof, but it can be read into both, and we found it extremely helpful in providing an intuitive grasp of the formalisation.

To make clear the idea about lines-of-sight, we will depict our players as \emph{eyes}, with a dashed line-of-sight to a point of the maze. In Figure~\ref{fig:SketchProofJordan2}, we show players $Red$, $Black$ and $Blue$ situated and staring at various points of a maze. Players $Red$ and $Black$ are inside the maze, while $Blue$ is on the outside. In the figure, we depict the paths they follow as they traverse the maze so that they have line-of-sight to the edge $P_iP_{i+1}$. Since $Red$ and $Black$ end up on the same side of $P_iP_{i+1}$, we can connect them by a polygonal path.

\begin{figure}
  \centering\includegraphics[scale=0.75]{jordanVerification2/SketchProof}
  \caption{Navigating a maze}
  \label{fig:SketchProofJordan2}
\end{figure}

The paths we have drawn through the maze are potential witnesses to the paths we consider in our verification. In fact, we can take the line-extension axiom 
\begin{equation}
  \tag{\ref{eq:g22}}
  \vdash A \neq B \implies \exists C.\; \between{A}{B}{C}
\end{equation}
and suppose that the witness $C$ in the conclusion is always chosen so that $B$ is half-way between $A$ and $C$. In this case, the paths sketched in Figure~\ref{fig:SketchProofJordan2} are precisely those that we witness in our formal verification.

\section{Formulation and Formalisation}\label{sec:Jordan2Formulation}
Compared to the last chapter, where we introduce the complex idea of a crossing, the basic ideas needed in the verification for this chapter are relatively straightforward. Firstly, given a simple polygon $Ps$, we will say that a point $X$ has line-of-sight to a point $X'$ if there is no point of $Ps$ which lies strictly between $X$ and $X'$. We shall say that when the point $X'$ lies between vertices $P_i$ and $P_{i+1}$ of $Ps$, then the point $X$ has line-of-sight to the \emph{edge} $P_iP_{i+1}$. The situation is formalised as
\begin{displaymath}
  \begin{aligned}
    &\neg\code{on\_polypath}\ Ps\ X \wedge \between{P_i}{X'}{P_{i+1}}\\
    &\wedge \neg\exists Z.\; \between{X}{Z}{X'} \wedge \code{on\_polypath}\ Ps\ Z.
  \end{aligned}
\end{displaymath}

Our verification breaks down into three parts. Firstly, we must show how every point not on a simple polygon has a line-of-sight to some edge of the maze. Next, we must show how, if a point $X$ has line-of-sight to an edge $P_iP_{i+1}$, then there is a polygonal path to a point $Y$ which has line-of-sight to the next edge $P_{i+1}P_{i+2}$. As such, for any edge and any point $X$, there is a polygonal path from $X$ to another point which has line-of-sight to that edge. Finally, we must show that if two players have line-of-sight to the same edge, and lie on the same side of that edge, then there is a polygonal path between them.

We shall describe the informal proofs and verifications of each part in \S\ref{sec:NavigationVerification}. First, we consider the crucial supporting lemmas.

\section{Obtaining Lines-of-Sight}\label{sec:Jordan2Lemmas}
We have two key theorems: a ray-casting theorem which obtains a new line-of-sight, and a theorem dubbed ``squeeze'' which handles narrow cracks in corridors. In proving the squeeze theorem, we shall need recourse to many of our earlier theorems about triangles and their interiors, and a new theorem about a triangle containing another triangle.

\subsection{Ray-casting}\label{sec:RayCasting}
Our ray-casting theorem gives us a line-of-sight to a polygonal path, aimed in an arbitrary direction towards that path. To achieve this, we must find the first point of intersection that the ray makes with the polygonal path. Ray-casting is actually needed in the first half of the Polygonal Jordan Curve Theorem (see \S\ref{sec:FinalProofJordan1}), but it makes more sense to explain it here where we are appealing to metaphors from computer graphics.

Ray-casting appears in a weakened form in Veblen's proof, but there he only considers casting a ray which does not intersect any \emph{vertex} of the polygonal path. This can be generalised by considering a few additional cases, after which we have a much more useful theorem.

This ray-casting theorem relies almost exclusively on linear reasoning and we found it particularly tricky to verify. As with our verification of Theorem~\ref{eq:IH3} in the last chapter, our linear reasoning tactic came through as a powerful tool for dealing with these problems, but first it has to be fed the right starting hypotheses. The trouble we had in the verification was deciding which hypotheses were needed. 

It often ended up being a matter of trial and error, but fortunately, the linear reasoning tactic is based on a decision procedure. When there were not enough facts available, it would promptly terminate and announce that the goal was not solvable. We could then go back through the problem and try to identify additional facts to feed the tactic and then retry. 

The need for this sort of manual labour is not particularly worthy of an automated proof assistant, and in future, some feedback on why the tactic failed would be useful and reasonably straightforward to implement. We are only working here with small sets of points (and the tactic struggles anyway when larger sets are considered), so it is feasible to enumerate all their permutations, including permutations when some combination of the points are equal. For instance, in the case of 5 points, there are only 431 possible arrangements. These serve as models, which can be filtered down by finding those which satisfy both the current hypotheses and the negation of the conclusion. After filtering, we are left with just the counterexamples, which would help if we could identify features in them which we can recognise as impossible given our diagrams and general intuition about the proof. We leave such a counterexample-checker for future work.

Sometimes, such counterexamples mean that a case-split must be considered. Sometimes, this reflected two different linear reasoning problems, which meant providing two subproofs. But sometimes we got lucky. If the case-split led to a single linear reasoning problem, we could let our incidence-discoverer handle the case-analyses automatically using its internal representation of proof trees.

The verification applies structural induction on the vertex list, and reduces the problem to that of ray-casting to a single edge. The basic case-analyses are shown in Figure~\ref{fig:RayCast}. We cast rays from the points $A$, $B$ and $C$ to the polygonal path $P_1P_2\ldots P_5$. The salient differences between the three lines-of-sight are as follows: point $A$ has line-of-sight to an endpoint of an edge, but the edge itself is not on the line-of-sight. Point $B$ has line-of-sight to an endpoint of an edge, and the edge itself \emph{is} on the line-of-sight. Finally, point $C$ has line-of-sight to the interior of an edge $P_2P_3$. 

\begin{figure}
\centering\includegraphics[scale=1]{jordanVerification2/RayCast}
\caption{Ray-casting}
\label{fig:RayCast}
\end{figure}

We now give the formalisation of the theorem. From a point $X$, we fire a ray to an arbitrary point $P$ on the polygonal path, and then obtain a point $Y$ to which $X$ has line-of-sight. Though it does not prove necessary in our verifications, we provide some extra information about the point $Y$, namely that it is either strictly between $X$ and $P$, or else we already had line-of-sight to $P$. 
\begin{multline}\label{eq:RayCast}
% !Ps X Y. ~(on_polyseg Ps X) /\ on_polyseg Ps Y
%                     ==> ?Z. on_polyseg Ps Z /\ (between X Z Y \/ Y = Z)
%                     /\ ~(?R. between X R Z /\ on_polyseg Ps R)
  \vdash\neg\code{on\_polypath}\ Ps\ X \wedge \code{on\_polypath}\ Ps\ P\\
  \implies \left(
    \begin{aligned}&\exists Y.\; \code{on\_polypath}\ Ps\ Y \wedge (\between{P}{Y}{X} \vee P = Y)\\
      &\qquad\qquad\wedge \neg(\exists Q.\; \between{X}{Q}{Y} \wedge \code{on\_polypath}\ Ps\ Q)
    \end{aligned}\right).
\end{multline}

Note the first conjunct in the hypothesis of this theorem. We can only cast rays to a polygonal path if we are not on that path. This should clarify a point made in \S\ref{sec:FinalProofJordan1} of the last chapter. There, we said that the final part of the verification showing that there are at least two regions of a simple polygon is based on ray-casting from some point $X$ on that polygon. In particular, we are ray-casting to the rest of the polygon, and thus, if we are to ray-cast, we need to assume that the rest of the polygon does not have a self-intersection at $X$. This we guarantee based on the fact that the polygon is assumed to be simple.

\subsection{Squeeze}\label{sec:Squeeze}
The most powerful theorem in our arsenal is one we dubbed ``squeeze'', since the intuition is that it allows us to find segments which squeeze through arbitrarily narrow gaps of a maze. What counts as a narrow gap in the abstract world of ordered geometry is  determined by the betweenness relation, and so the basic axioms governing this relation limit our powers in navigating such gaps. We can get some idea of the challenge by realising that, on some interpretations, these gaps are \emph{infinitesimally} narrow. Hilbert's axioms are independent of Archimedes' axiom.

Abstractly, our \emph{squeeze} theorem, Theorem~\ref{eq:Squeeze}, tells us that if we have a polygonal path $[A,B,C]$ which is not intersected by the polygonal path $Ps$, then we can introduce a point $A'$ between $A$ and $B$ such that $Ps$ intersects $A'C$ in at most one place. We can apply and interpret this theorem in a number of ways. In \S\ref{sec:SqueezeEye}, we will show how to interpret it in terms of lines-of-sight. Here, we interpret it in terms of finding diagonals that divide a polygon into two simple polygons.
\begin{equation}\label{eq:Squeeze}
  \begin{aligned}
      % "!P start goal 'a polyseg.
      % ~(?a. on_line P a /\ on_line start a /\ on_line goal a)
      % /\ on_plane P 'a /\ on_plane start 'a /\ on_plane goal 'a
      % /\ (!X. MEM X polyseg ==> on_plane X 'a)
      % /\ ~on_polyseg polyseg P
      % /\ (!X. between P X start ==> ~on_polyseg polyseg X)
      % /\ (!X. between P X goal ==> ~on_polyseg polyseg X)
      % ==> ?s. between P s start
      %         /\ !X. in_triangle (P,s,goal) X ==> ~on_polyseg polyseg X"
    \vdash&\neg\code{on\_polypath}\ Ps\ B\\
    &\wedge \neg(\exists X.\; \between{A}{X}{B} \wedge \code{on\_polypath}\ Ps\ X)\\
    &\wedge \neg(\exists X.\; \between{B}{X}{C} \wedge \code{on\_polypath}\ Ps\ X)\\
    &\implies \exists A'.\; \between{A}{A'}{B} \wedge \neg\exists X.\; \code{in\_triangle}\ (A',B,C)\ X\wedge \code{on\_polypath}\ Ps\ X.
  \end{aligned}
\end{equation}

\begin{figure}
\centering\includegraphics[scale=0.75]{jordanVerification2/SqueezeDiagonal}
\caption{Squeezing a diagonal}
\label{fig:SqueezeDiagonal}
\end{figure}

In Figure~\ref{fig:SqueezeDiagonal}, we use squeeze to find a diagonal of the polygon $P_1P_2\ldots P_{11}$. Here, we have set $A = P_1$, $B = P_{11}$, and $C = P_{10}$, while we set $path$ to be the rest of the polygon $P_1P_2P_3\ldots P_{10}$. 

The form of the conclusion in Theorem~\ref{eq:Squeeze} reflects its verification. We make a slightly different claim than declaring the existence of a diagonal. We say instead that the polygonal path does not lie in the interior of $\triangle A'BC$, which means that any point between $A'$ and $B$ yields a segment with the point $C$ which does not intersect $Ps$. We prove this starting with the triangle $ABC$, and find the first vertex in $Ps$ which lies inside this triangle. In the case shown, this would be the vertex $P_3$. We draw a line through $P_{10}$ and $P_3$ to the point $A_3$, and continue the argument with this new triangle. Eventually, we will be left with a triangle whose interior contains no point of $Ps$. In a sense, the triangle $ABC$ has been ``squeezed'' by $Ps$ into the triangle $A'BC$.

Unhappily, proving that $\triangle A'BC$ contains no point of $Ps$ has us boiled down in case-splits similar to those needed to analyse triangle crossings in the last chapter (\S\ref{sec:CrossingVerification}). Rather than go into the details of these, we shall focus on the more illuminating verification that $\triangle A'BC$ contains no \emph{vertex} of $Ps$. This only boils down to two more lemmas for triangle interiors.

\subsubsection{Another ``Inner Pasch'' Rule}
Our first supporting theorem allows us to introduce the intersection points $A_3$, $A_4$, $A_6$ and $A'$ in Figure~\ref{fig:SqueezeDiagonal}. This theorem is similar in spirit to the Pasch axioms (\ref{eq:OuterPasch}, \ref{eq:InnerPasch}) and its variant for triangle interiors \eqref{eq:tricut1}. It has a very succinct formalisation, but a non-trivial verification:
\begin{equation}\label{eq:TriCut3}
\vdash\code{in\_triangle}\ (A,B,C)\ P \implies \exists X.\; \between{B}{X}{C} \wedge \between{A}{P}{X}.
\end{equation}

The verification is the most interesting for these point introduction theorems. Unlike the verification of Theorem~\ref{eq:tricut1} from the last chapter, we find ourselves back employing Pasch's axiom \eqref{eq:g24} rather than exploiting our theorems of half-planes. First, we use \ref{eq:three} to find a point $X$ on $AB$. Next, we apply \eqref{eq:g24} to the triangle $ABC$ and the line $PX$. This gives us a point $Y$ which is either between $B$ and $C$ or between $A$ and $C$. Here is one of the rare times where both cases are possible. Normally, we would be able to refute one of the cases based on incidence reasoning. Here, we need two subproofs.

\begin{figure}
\centering\includegraphics[scale=1.2]{jordanVerification2/TriCut3}
\caption{Drawing a line from a vertex to the opposite side}
\label{fig:TriCut3}
\end{figure}

If $Y$ lies between $B$ and $C$ as in case (a) of Figure~\ref{fig:TriCut3}, then we apply \eqref{eq:g24} to the triangle $BXY$ and the line $AP$ to find a point $Z$ between $B$ and $Y$. This point is then between $B$ and $C$ (via linear reasoning on $B$, $C$, $Y$ and $Z$). Furthermore, $P$ is between $A$ and $Z$ by~\eqref{eq:inTriangle2}.

In case (b) of Figure~\ref{fig:TriCut3}, we find that $P$ is now an interior point according to Veblen's definition and the position of the points $X$ and $Y$. Here, we apply \eqref{eq:g24} to the triangle $CXY$ and the line $AP$ to find a point $P'$ on $CX$. By the same axiom applied to $\triangle BCX$ and the line $AP'$, we find the desired point $Z$ on $BC$.

\subsubsection{Subtriangles}\label{sec:Subtriangles}
In our verification of Theorem~\ref{eq:Squeeze}, we consider a sequence of triangles such as those of Figure~\ref{fig:SqueezeDiagonal}, namely $\triangle ABC$, $\triangle A_3BC$, $\triangle A_4BC$, $\triangle A_6BC$ and $\triangle A'BC$. Each of these is intended to exclude one vertex of $Ps$ from its interior. By the time we reach the last triangle, we can conclude that all vertices of $Ps$ will lie outside the interior. This conclusion assumes a transitivity property, which is given by our second lemma. It shows that the ordering of $A$, $A_3$, $A_4$, $A_6$ and $A'$ along the segment $AB$ yields a sequence of triangles with \emph{nested} interiors. 

The verification, given in Figure~\ref{fig:SubTriangleProof}, is extremely short and readable, making use of a number of lemmas we have previously considered, including Theorem~\ref{eq:TriCut3} from the last subsection. We show that any point $P$ interior to a triangle $AA'C$ is interior to any triangle $ABC$ where $A'$ is between $A$ and $B$. That is, the interior of $AA'C$ is nested in $ABC$. To prove it, we put a point $Q$ on the line $A'C$, and show that it must be interior to $\triangle ABC$ on the basis of Theorem~\ref{eq:inTriangle1}. It then follows by Theorem~\ref{eq:inTriangle2} that $P$ must also be interior. See Figure~\ref{fig:SubTriangle}.

\begin{figure}
\centering\includegraphics[scale=1.2]{jordanVerification2/SubTriangle}
\caption{Any interior point $P$ of $\triangle AA'C$ is an interior point of $\triangle ABC$.}
\label{fig:SubTriangle}
\end{figure}

\begin{boxedfigure}
\small
\begin{align*}
&\code{theorem }\code{in\_triangle}\ (A,A',C)\ P \wedge \between{A}{A'}{B} 
                  \implies \code{in\_triangle}\ (A,B,C)\ P\\
&\code{assume}\ \code{in\_triangle}\ (A,A',C)\ P \wedge \between{A}{A'}{B} & 0 \\
&\code{so consider}\ Q\ \code{such that}\ \between{A'}{Q}{C} \wedge \between{A}{P}{Q} \ \code{by}\ \eqref{eq:TriCut3} & 1\\
&\code{obviously}\ \code{(by\_ncols}\ \circ\ \code{add\_in\_triangle}\ \circ\ \code{conjuncts})\\
&\qquad \code{hence}\ \code{in\_triangle}\ (A,B,C)\ Q\ \code{by}\ \eqref{eq:triSyms},\eqref{eq:inTriangle1},\eqref{eq:g21}\ \code{from}\ 0\\
&\code{qed from}\ 1\ \code{by}\ \eqref{eq:onTriangleDef},\eqref{eq:inTriangle2}
\end{align*}
\caption{Subtriangles}
\label{fig:SubTriangleProof}
\end{boxedfigure}

\subsection{Obtaining lines-of-sight via Squeeze}\label{sec:SqueezeEye}
As we suggested earlier, Theorem~\ref{eq:Squeeze} is quite powerful. In this section, we will show two applications of the theorem in terms of lines-of-sight, both of which we shall need for the core verifications of this chapter. 

\subsubsection{Moving to a new line-of-sight}\label{sec:MoveToNew}
\begin{figure}
\centering\includegraphics[scale=0.75]{jordanVerification2/Squeeze1}
\caption{Obtaining line-of-sight to $C$}
\label{fig:Squeeze1}
\end{figure}

Consider Figure~\ref{fig:Squeeze1}. Here, we have slightly modified and relabelled the diagram from Figure~\ref{fig:SqueezeDiagonal} so that we can interpret the point $A$ as our player's starting location, with line-of-sight to a point $B$ on the edge $P_{11}P_{12}$ of a simple polygon. The goal is to obtain line-of-sight to a given point $C$ on that same edge.

The idea is that if the player proceeds far enough down their initial line-of-sight, we can easily rotate them to the new line-of-sight. We do this by applying Theorem~\ref{eq:Squeeze} and taking $path$ to be the fragment of the polygon $P_{12}P_1P_2\ldots P_{11}$. Let us think about how to fulfil the hypotheses. We are basically being asked to rule out the possibility that the polygonal path $[A,B,C]$ is not intersected by this fragment, save for possible intersections at $A$ and at $C$. 

Well, we know that the segment $AB$ does not lie on the fragment $P_{12}P_1P_2\ldots P_{11}$, because $AB$ is supposed to be a line-of-sight. We also know that $BC$ does not intersect the fragment, because we are assuming \emph{simple} polygons: the polygon should not self-intersect the edge $P_{11}P_{12}$, and this edge contains $BC$. We leave open the possibility that $P_{12}P_1P_2\ldots P_{11}$ intersects the lone point $C$, since, as we shall see in \S\ref{sec:ConcaveMove}, we will sometimes need to obtain line-of-sight to a vertex, where we will set $C=P_{11}$.

Applying Theorem~\ref{eq:Squeeze} is not quite enough. We now have a point $A'$ which \emph{almost} has line-of-sight to $C$. We just need to use \ref{eq:three}, obtaining a point $A''$ and a segment $A''C$ which lies properly in the triangle $A'BC$, and, as per the conclusion of Theorem~\ref{eq:Squeeze}, a segment which is a line-of-sight to the point $C$. We verify the argument here as a corollary of Theorem~\ref{eq:Squeeze}:
\begin{equation}\label{eq:Squeeze2}
  \begin{aligned}
  % "!P start goal 'a polyseg.
  %     ~(?a. on_line P a /\ on_line start a /\ on_line goal a)
  %     /\ on_plane P 'a /\ on_plane start 'a /\ on_plane goal 'a
  %     /\ (!X. MEM X polyseg ==> on_plane X 'a)
  %     /\ ~on_polyseg polyseg P
  %     /\ (!X. between P X start ==> ~on_polyseg polyseg X)
  %     /\ (!X. between P X goal ==> ~on_polyseg polyseg X)
  %     ==> (?s. between P s start
  %              /\ !X. between s X goal  ==> ~on_polyseg polyseg X)"
\vdash    &\neg\code{on\_polypath}\ path\ B\\
    &\wedge \neg(\exists X.\; \between{A}{X}{B} \wedge \code{on\_polypath}\ path\ X)\\
    &\wedge \neg(\exists X.\; \between{B}{X}{C} \wedge \code{on\_polypath}\ path\ X)\\
    &\implies \exists A'.\; \between{A}{A'}{B} \wedge \neg\exists X.\; \between{A'}{X}{C} \wedge \code{on\_polypath}\ path\ X.
  \end{aligned}
\end{equation}

Now because $AB$ is a line-of-sight, we know that $AA''$ is part of a polygonal path which does not intersect $P_1P_2\ldots P_{11}P_{12}P_{1}$. What we have exploited here is the fact that we can always move along our lines-of-sight without intersecting the polygon. The use of squeeze \eqref{eq:Squeeze} in this section is therefore telling us how to build up paths through a maze, by getting the player to move far enough along their line-of-sight that they will be able to see a point further down the current edge. 

The next application of squeeze will be just as critical for our verifications.

\subsubsection{Rotating to a new line-of-sight}\label{sec:RotateToNew}
\begin{figure}
\centering\includegraphics[scale=0.75]{jordanVerification2/Squeeze2}
\caption{Obtaining line-of-sight to the interior of $AB$}
\label{fig:Squeeze2}
\end{figure}
In Figure~\ref{fig:Squeeze2}, we have again modified and relabelled the diagram. The idea here is that our player is situated at a point $C$ and initially has line-of-sight to the vertex $B$. The problem with this scenario is that, in our verifications, we are more concerned about having lines-of-sight to points on a polygon which are \emph{not} vertices (for a start, this forces the player off the line of the edge).  So we want our player to rotate their line-of-sight slightly. 

Therefore, we apply Theorem~\ref{eq:Squeeze} by reversing the order of $A$, $B$ and $C$. The polygonal fragment we are concerned with here is again $P_{12}P_1P_2\ldots P_{11}$, and as before, we have to be sure that the hypotheses of Theorem~\ref{eq:Squeeze} have been met. That is, we must show that the polygonal path $[A,B,C]$ is not intersected by the fragment $P_{12}P_1P_2\ldots P_{11}$. 

Again, we know that $AB$ is not intersected by the fragment, because we assume that the polygon is simple and so does not have such self-intersections, and we know that $BC$ is not intersected by the fragment because $BC$ is assumed to be a line-of-sight. This means we have met the hypotheses. We now obtain the point $A'$ as shown, and as before, we use \ref{eq:three} to obtain a point $A''$ between $B$ and $A'$. Our player will now have line-of-sight to a non-vertex point of the segment $AB$.

The reader may have noticed that we have not mentioned the segment $BP_{12}$ shown in the diagram. In fact, this segment could prove a problem. If it were chosen to lie on the other side of $BC$, then it might intersect the segment $A''C$, and if this happens, then $A''C$ will not count as a line-of-sight. Fortunately, we shall be able to rule out this circumstance when we come to apply Theorem~\ref{eq:Squeeze2}. We shall always be assuming that, when we rotate our line-of-sight, the points $P_{11}$ and $P_{12}$ are on opposite sides of $BC$, as shown.

To summarise, we can say that the first use of squeeze allows us to reduce the \emph{distance} to the point $B$ sufficiently and thus obtain a new line-of-sight, while this second use of squeeze is telling us that we can reduce the \emph{angle} $ABC$ sufficiently. What the theorem gives us is the ability to reduce distances and angles even without a general theory for comparing them, and without any sort of arithmetic for them. We ``squeeze'' our distances and angles by working exclusively with a weak order relation and properties of incidence.

\section{Edge-to-Edge}\label{sec:NavigationVerification}
Since a simple polygon is just a list of adjacent edges, we can navigate every single edge just by moving between one edge and the next in the adjacency list. In Section~\ref{sec:Jordan2Formulation}, we explained that the movement from an edge $P_{i}P_{i+1}$ to an edge $P_{i+1}P_{i+2}$ amounts to showing three things: (1) that there is a point $X$ with line-of-sight to $P_{i}P_{i+1}$; (2) that there is a point $Y$ with line-of-sight to $P_{i+1}P_{i+2}$; and (3), that there is a polygonal path between $X$ and $Y$.

Now when inside a convex polygon, this matter is completely trivial. Indeed, \emph{every} interior point of a convex polygon has line-of-sight to \emph{every} edge, as in Figure~\ref{fig:ConvexEasy}. The salient fact to notice is that, in a convex polygon, interior points are on the same side of every edge as every other vertex. 

\begin{figure}
\centering\includegraphics[scale=0.75]{jordanVerification2/ConvexEasy}
\caption{Edge-to-edge in a convex polygon}
\label{fig:ConvexEasy}
\end{figure}

Generalising this, we can say that two edges $P_{i}P_{i+1}$ and $P_{i+1}P_{i+2}$ appear ``locally convex'' from the perspective of a point $P$ if the point $P$ is on the same side of $P_{i}P_{i+1}$ as $P_{i+2}$. Otherwise, we can say that the edges appear ``locally concave''. These provide our two cases for how we navigate the edges of a simple polygon.

\subsection{Locally Convex Edges}\label{sec:ConcaveMove}
To explain the case for locally convex edges, we shall assume we have a polygon $P_1P_2P_3\ldots P_n$, and we shall assume that we are moving from edge $P_1P_2$ to edge $P_2P_3$ (we can use the polygon rotations described in \S\ref{sec:PolygonRotation} to consider the other pairs of edges).

\begin{figure}
\centering\includegraphics[scale=0.75]{jordanVerification2/Convex1}
\caption{Edge-to-edge in a locally convex polygon}
\label{fig:Convex1}
\end{figure}

In Figure~\ref{fig:Convex1}, we start at a point $X$ which has line-of-sight to $P_1P_2$. Our goal is to reach a point $Y$ with line-of-sight to $P_2P_3$. We start by moving towards $P_1P_2$ by applying Theorem~\ref{eq:Squeeze2} as described in \S\ref{sec:MoveToNew}, seeking a line-of-sight with the vertex~$P_2$. 

\label{sec:InjectingProcedural}Applying squeeze in this way is usually a tricky business, because there are a number of hypotheses which must be fulfilled and it is not always immediately clear how. Even when we had found all the necessary hypotheses, the default \code{MESON} prover would struggle to apply them all, and so we would have to inject some procedural code as justification. In doing this, we tried to respect the aims of declarative verification. For instance, at no point do we destructively modify the proof context, and we always insist that whenever our tactics apply an assumption, we include the assumption label so that a reader can at least track the dependencies. Here, we use \code{EXISTS\_TAC} followed by \code{MATCH\_MP\_TAC}, which leaves \code{MESON} just having to discharge the assumptions of \eqref{eq:Squeeze2}.

%We then use $\code{MATCH\_MP}$ to retrieve our hypotheses as a big conjunction, and then split each off into its own subgoal. We then tackle these one at a time. Afterwards, we can gather up all the dependent assumptions and prove the original goal in one step. We just need a bit of tactic script to help \code{MESON}, but as usual, we make sure not to destructively modify the goal state, and we reference our justifying theorem \eqref{eq:Squeeze2}. The step is still largely declarative.

\fbox{\begin{minipage}{\boxwidth}\setlength\abovedisplayskip{0cm}
\begin{align*}
\small
&\code{obviously by\_incidence so consider}\ Y\ \code{such that}\ \between{X}{Y}{X'}\\
&\qquad\wedge \forall Z.\; \between{Y}{Z}{P_2} \implies \neg\code{on\_polypath}\ (\cons{P_2}{\cons{P_3}{Ps}})\ Y\\ &\qquad\code{from}\ \ldots\ \code{by}\ \code{on\_polypath},\eqref{eq:g22}\\
&\qquad\code{using K (MATCH\_MP\_TAC}\ \eqref{eq:Squeeze2}\ \code{THEN EXISTS\_TAC}\ \alpha) & 16,17
\end{align*}\end{minipage}}\linebreak

This done, we have our desired path $XY$. We know that this path does not intersect any part of the simple polygon, since it is part of our original line-of-sight. All that remains then is to rotate ourselves slightly to obtain a line-of-sight $YY'$ with some point on $P_2P_3$. This we do in Figure~\ref{fig:Convex2}, using Theorem~\ref{eq:Squeeze2} as described in \S\ref{sec:RotateToNew}. 

\begin{figure}
\centering\includegraphics[scale=0.75]{jordanVerification2/Convex2}
\caption{Edge-to-edge in a locally convex polygon (continued)}
\label{fig:Convex2}
\end{figure}

\begin{boxedfigure}
\small
% so consider ["Y:point"]
%          st "between s Y hand'' /\ on_polyseg [P1;P2;P3] Y"
%          from [22] at [24] by [on_polyseg_CONS2;on_polyseg_pair;BET_SYM]
%        ;obviously (by_ncols o Di.conjuncts)
%          (hence "~on_polyseg [P2;P3] Y"
%             from [4;5;20;23]
%             by [on_polyseg_pair;g11_weak;g21])
%        ;hence "on_polyseg [P1;P2] Y" from [24] at [25]
%          by [on_polyseg_pair;on_polyseg_CONS2]
%        ;have "on_half_plane hp hand''"
%          from [12;14;23] by [bet_on_half_plane]
%        ;hence "on_half_plane hp Y"
%          from [19;24] by [bet_on_half_plane2] at [26]
%        ;hence "between P1 Y P2"
%          from [3;12;25] by [on_polyseg_pair;half_plane_not_on_line]
%        ;qed from [3;12;26] by [BET_NEQS;g12;g21
%                               ;half_plane_not_on_line]]
\begin{align*}
&\code{so consider}\ Z\ \code{such that}\ \between{Y}{Z}{Y'} \wedge \code{on\_polypath}\ [P_1,P_2,P_3]\ Z\\\
&\qquad\qquad\code{from}\ \ldots\ \code{by}\ \eqref{eq:OnPolyPath} & 24\\
&\code{obviously}\ (\code{by\_ncols}\circ\code{conjuncts})\ \code{hence}\ \neg\code{on\_polypath}\ [P_2,P_3]\ Z\ \code{from}\ \ldots\\
&\qquad\code{by}\ \eqref{eq:OnPolyPath},\eqref{eq:g11},\eqref{eq:g21}\\
&\code{hence}\ \code{on\_polypath}\ [P_1,P_2]\ Z\ \code{from}\ 24\ \code{by}\ \eqref{eq:OnPolyPath}&25\\
&\code{have on\_half\_plane}\ hp\ Y'\ \code{from}\ 12,14,23\ \code{by}\ \eqref{eq:betOnHalfPlane1}\\
&\code{hence on\_half\_plane}\ hp\ Z\ \code{from}\ \ldots,24\ \code{by}\ \eqref{eq:betOnHalfPlane2} & 26\\
&\code{hence}\ \between{P_1}{Z}{P_2}\ \code{from}\ \ldots,25\ \code{by}\ \eqref{eq:OnPolyPath},\eqref{eq:halfPlaneNotOnLine}\\
&\code{qed from}\ \ldots,26\ \code{by}\ \eqref{eq:g12},\eqref{eq:g21},\eqref{eq:halfPlaneNotOnLine}
\end{align*}
\caption{Verification extract for the convex case of theorem~\ref{eq:PolygonMove}}
\label{fig:ConvexVerification}
\end{boxedfigure}

This is not the whole story. What is missing is any mention of the local convexity. This is needed to show that our segment $YY'$ does not intersect the polygon, and thus is a genuine line-of-sight. Our second application of \eqref{eq:Squeeze2} only tells us that it does not intersect the fragment of the polygon $P_3P_4\ldots P_1$. Our incidence discoverer tells us further that any point $Z$ between $Y$ and $Y'$ cannot intersect $P_2P_3$. What is left to rule out is a possible intersection with $P_1P_2$. For this, we need to think about half-planes. 

We reason as follows. We know that $Y$ and $P_3$ are on the same side of $P_1P_2$ (since $P_1P_2$ and $P_2P_3$ are locally convex from the perspective of $Y$), and $Y'$, being on $P_2P_3$, must also be on the same side of $P_1P_2$. That means that $Z$ is also on the same side, because it lies on $YY'$. But that means it cannot intersect the line of $P_1P_2$, and, in particular, it cannot be on the segment $P_1P_2$.

The verification shown in Figure~\ref{fig:ConvexVerification} matches this argument's structure.  We start by assuming there is some point $Z$ on $YY'$ which intersects the polygonal path $[P_1,P_2,P_3]$, and proceed by contradiction (we have removed some of the references to earlier steps and inserted ellipses for readability). 

\subsection{Locally Concave Edges}
When the edges are (strictly) locally concave, we have it that our starting point $X$ is on the opposite side of the line $P_1P_2$ as the point $P_3$. In this case, we will generally need to ``round the corner'' $P_1P_2P_3$ to get line-of-sight with the next edge. 

Before we apply Theorem~\ref{eq:Squeeze2}, we will pick our destination. This will be the point $Y'$ shown in Figure~\ref{fig:Concave1}. This point is ``just off'' the vertex $P_2$, and can be found by ray-casting (Theorem~\ref{eq:RayCast}) from $P_2$ in the direction $\overrightarrow{P_1P_2}$. We then use Theorem~\ref{eq:Squeeze2} as described in \S\ref{sec:MoveToNew}, moving along our initial line-of-sight $XX'$ to obtain a new line-of-sight with $Y'$. The segment $XYY'$ will then be the required path.

\begin{figure}
\centering\includegraphics[scale=0.75]{jordanVerification2/Concave1}
\caption{Edge-to-edge in a locally concave polygon}
\label{fig:Concave1}
\end{figure}

Now since $Y'$ was found by ray-casting, we know it has line-of-sight to $P_2$. So all we need to do is rotate this line-of-sight to point at $P_2P_3$, which we know we can do using Theorem~\ref{eq:Squeeze2} as described in \S\ref{sec:RotateToNew}. We thus have the required path and line-of-sight to $Y''$ as shown in Figure~\ref{fig:Concave2}.

We just have to confirm that the path $XYY'$ does not intersect the polygon, and that $Y'Y''$ is a genuine line-of-sight. We know immediately that $XY$ is off the polygon, since it is part of our original line-of-sight. This leaves us having to show that $YY'$ and $Y'Y''$ do not intersect the polygon. 

The way these segments were obtained via Theorem~\ref{eq:Squeeze2} guarantees that they do not intersect the fragment $P_3P_4\ldots P_n$, so that leaves us to consider whether either of the segments $YY'$ and $Y'Y''$ intersect the fragment $P_1P_2P_3$. 

It might seem that these cases would boil down to reasoning with half-planes again, since $YY'$ and $Y'Y''$ are part of two rays emanating in opposite directions from the line of $P_1P_2$. But we were surprised to find that, when we set our discoverer at the problem, it showed that $Y'Y''$ is off the fragment $P_1P_2P_3$ and thus a line-of-sight according to incidence reasoning alone.

The discoverer also showed that $YY'$ does not intersect the edge $P_1P_2$. We only need to use half-planes to show that it does not intersect the edge $P_2P_3$, using the same sort of tidy declarative verification that we saw in the last subsection: since $Y$ lies between $X$ and $P_1P_2$, and $X$ and $P_3$ lie on opposite sides of $P_1P_2$, it must be the case that $Y$ and $P_3$ lie on opposite sides of this line as well. And then, since $Y'$ is on the line of $P_1P_2$, it follows from Theorem~\ref{eq:betOnHalfPlane1} that every point on $YY'$ lies on the opposite side to $P_3$, and thus, does not lie on the line of $P_1P_2$. 

\begin{figure}
\centering\includegraphics[scale=0.75]{jordanVerification2/Concave2}
\caption{Edge-to-edge in a locally concave polygon (continued)}
\label{fig:Concave2}
\end{figure}

This almost completes the verification. We have not talked about the case that $P_1$, $P_2$ and $P_3$ are collinear, which requires one application of Theorem~\ref{eq:Squeeze2}, but we hope by this stage the reader is convinced we can achieve this, and is further convinced of how powerful our squeeze theorem is in allowing us to move between edges of a maze in a geometrically intuitive way.

\subsection{Putting it all Together}
We have packaged all the details discussed so far in this section into a single declarative verification consisting of 119 steps. That we can manage such long verifications shows how easily Mizar~Light, with our additional automated tools, can scale.

We would like to draw special attention to the hypotheses in the verified theorem \eqref{eq:PolygonMove}. Note that we do not need to assume that we are dealing with simple polygons. We just have to assume that $P_1P_2$ does not intersect the rest of the path $P_2P_3\ldots P_n$, and that $P_2P_3$ does not intersect the rest of the path $P_3P_4\ldots P_n$. These are the minimal hypotheses we need to apply Theorem~\ref{eq:Squeeze2} as described in this section. We can think of them as saying that the polygon appears \emph{locally} simple.

\begin{equation}\label{eq:PolygonMove}
  \begin{split}
  % "!P1 P2 P3 Ps X hand 'a.
  %    on_plane P1 'a /\ on_plane P2 'a /\ on_plane P3 'a
  %    /\ (!P. MEM P Ps ==> on_plane P 'a)
  %    /\ on_plane X 'a
  %    /\ between P1 hand P2
  %    /\ ~(P2 = P3)
  %    /\ ~on_polyseg (CONS P1 (CONS P2 (CONS P3 Ps))) X
  %    /\ ~on_polyseg (CONS P3 Ps) P2
  %    /\ ~(?Y. between X Y hand /\ on_polyseg (CONS P1 (CONS P2 (CONS P3 Ps))) Y)
  %    /\ ~(?Y. between P1 Y P2 /\ on_polyseg (CONS P2 (CONS P3 Ps)) Y)
  %    /\ ~(?Y. between P2 Y P3 /\ on_polyseg (CONS P3 Ps) Y)
  %    ==> ?hand' X'.
  %           seg_connected 'a
  %             (on_polyseg (CONS P1 (CONS P2 (CONS P3 Ps)))) X X'
  %           /\ between P2 hand' P3
  %           /\ ~on_polyseg (CONS P1 (CONS P2 (CONS P3 Ps))) X'
  %           /\ ~(?Y. between X' Y hand'
  %           /\ on_polyseg (CONS P1 (CONS P2 (CONS P3 Ps))) Y)"
\vdash    &\between{P_1}{X'}{P_2} \wedge P_2 \neq P_3\\
    &\wedge \neg\code{on\_polypath}\ (\cons{P_1}{\cons{P_2}{\cons{P_3}{Ps}}})\ X \wedge \neg\code{on\_polypath}\ (\cons{P_3}{Ps})\ P_2\\
    &\wedge \neg(\exists Z.\; \between{X}{Z}{X'} \wedge \code{on\_polypath}\ (\cons{P_1}{\cons{P_2}{\cons{P_3}{Ps}}})\ Z)\\
    &\wedge \neg(\exists Z.\; \between{P_1}{Z}{P_2} \wedge \code{on\_polypath}\ (\cons{P_2}{\cons{P_3}{Ps}})\ Z)\\
    &\wedge \neg(\exists Z.\; \between{P_2}{Z}{P_3} \wedge \code{on\_polypath}\ (\cons{P_3}{Ps})\ Z)\\
    &\implies \exists Y.\;\exists Y'.\; \code{polypath\_connected}\ (\code{on\_polypath}\ (\cons{P_1}{\cons{P_2}{\cons{P_3}{Ps}}}))\ X\ Y\\
    &\qquad\wedge \between{P_2}{Y'}{P_3}\wedge \neg\code{on\_polypath}\ (\cons{P_1}{\cons{P_2}{\cons{P_3}{Ps}}})\ Y)\\
    &\qquad\wedge \neg\exists Z.\; \between{Y}{Z}{Y'} \wedge (\cons{P_1}{\cons{P_2}{\cons{P_3}{Ps}}})\ Z).
  \end{split}
\end{equation}

\section{Without-Loss-of-Generality}
In the last section, we described how to move edge-to-edge in a polygon. However, we effectively made a without-loss-of-generality assumption when stating the theorem, since we assumed that the edges in question were defined by the first three elements of the vertex list. More generally, they could be any two edges defined by  three adjacent elements of the list, or, if we are considering moving forward from the last edge of a polygon back to the first, then the vertices defining the edges come from the last two elements and the first two elements.

We can follow Harrison's approach to dealing with without-loss-of-generality assumptions~\cite{HarrisonWLOG} if we can find an equivalence relation which preserves the properties of interest to us. In effect, we say that before we move a player to the next edge, we must be able to transform the polygon's vertex list into the player's perspective, much as we would perform a coordinate transform.

\subsection{Polygon Rotations: Formulation}\label{sec:PolygonRotation}
To make the idea work formally, we will need a way to represent a polygon rotation. Since our polygons are being represented by vertex lists, we must briefly leave the pleasant world of synthetic geometry and instead look at computing with lists. On the upside, some of these ideas are a general contribution to the theory of lists, and can be applied in other contexts. 

It is not uncommon, for instance, to need to rotate a list. This can be understood in terms of splitting the list at some point and then switching the two halves. Formally:
\begin{displaymath}
  \begin{split}
&\code{rotation\_of}\; :\; [\alpha] \rightarrow [\alpha] \rightarrow \code{bool}\\
&\code{rotation\_of}\ ps\ qs \iff \exists xs.\;\exists ys.\; ps = \append{xs}{ys} \wedge qs = \append{ys}{xs}.\\
  \end{split}
\end{displaymath}
This defines an equivalence relation, but our use-case for list rotations gives us a complication. We represent a polygon by a vertex list where we assume that the first and last elements are duplicated. We cannot simply rotate this vertex list, since this will generally give us duplicate vertices and endpoints which do not match. Instead, we should only rotate the largest non-trivial prefix of the list, and then duplicate the new head at the end. 

The definition of this sort of rotation is slightly more complex, but assuming the rotation is not the identity, we can understand it as taking some vertex inside the list and swapping it with the first and last elements. The sublists in between are then exchanged. In other words:
\begin{align*}
    \vdash_{def}\;&\code{rotation\_of}\ Ps\ Qs \iff\\
    &\qquad\begin{aligned}
    \exists P.\;\exists Q.\;\exists Ps.\;\exists Qs.\; Ps &= (\append{[P]}{\append{Ps'}{\append{[Q]}{\append{Ps''}{[P]}}}})\\
    \wedge\; Qs &= (\append{[Q]}{\append{Ps''}{\append{[P]}{\append{Ps'}{[Q]}}}})\\
    \vee\; Ps &= Qs.
  \end{aligned}
\end{align*}

We verified that this is an equivalence relation.

\subsection{Invariance}
In order to use these list rotations to justify without-loss-of-generality, we need to know that a rotation preserves the properties of interest. In our case, we need to know that a rotated polygon is still the same figure as a set of points, and we need to know that it is still \emph{simple} as a set of segments. 

The proofs are not exactly straightforward, since the set of points of a polygon and the simplicity of a polygon are defined in terms of list functions such as $\code{adjacent}$, $\code{mem}$ and $\code{pairwise}$. The reasoning involves the interplay with these functions, for which we have had to contribute many new lemmas. We found it despairing to be doing this at such a late stage, when so much of the earlier verification was focused on synthetic declarative geometry. But with perseverance, we obtained the necessary theorem:
\begin{equation}\label{eq:PolygonRotateInvariant}
\begin{aligned}
\vdash      &\code{rotation\_of}\ Ps\ Qs\\
    &\qquad\implies (\code{simple\_polygon}\ Ps \implies \code{simple\_polygon}\ Qs)\\
    &\qquad\qquad\quad\wedge \code{on\_polypath}\ Ps = \code{on\_polypath}\ Qs.
  \end{aligned}
\end{equation}

\subsection{Example}
We end this section by explaining some of the reasoning that goes into using polygon rotations in the context of the main proof for this chapter. Suppose we have a polygon $Ps$ as a vertex list, and we know we have line-of-sight to a point $X'$ on an edge of the polygon. We can formally describe the position of $X'$ with:
\begin{displaymath}
  \exists P_i.\;\exists P_{i+1}.\; \code{mem}\ (P_i,P_{i+1})\ (\code{adjacent}\ Ps) \wedge \between{P_i}{X'}{P_{i+1}}.
\end{displaymath}

The presence of $\code{adjacent}$ is troublesome, but we have verified a theorem to help us out here:
\begin{equation*}
  % (`!x:a y xs. MEM (x,y) (ADJACENT xs)
  %      <=> ?ys zs. xs = APPEND ys (CONS x (CONS y zs))`,
\vdash  \code{mem}\ (x,y)\ (\code{adjacent}\ xs) \iff 
\exists ws.\;\exists zs.\; xs = \append{ws}{\append{[x,y]}{zs}}.
\end{equation*}

This gives us just about everything we need to perform a rotation. We know that if the witnessed $ys$ is empty, then our edge $[x,y]$ must already be at the front of the list. Otherwise, we know that the list $xs$ is of the form
\begin{displaymath}
  xs = \append{[w]}{\append{ws}{\append{[x,y]}{\append{zs}{[w]}}}}
\end{displaymath}

which is a rotation of
\begin{displaymath}
  xs = \append{[x,y]}{\append{zs}{\append{[w]}{\append{ws}{[x]}}}}.
\end{displaymath}

This puts the edge at the front of the list, as we require to apply Theorem~\ref{eq:PolygonMove}. Our invariance theorem \eqref{eq:PolygonRotateInvariant} assures us that the rotation will not change the set of points defined by the vertex list, nor affect the simplicity of the polygon.

\section{Moving to Any Edge}
We are now able to move between any two edges of a polygon. This takes us a significant way towards a verification of this last half of the Polygonal Jordan Curve Theorem. 

In particular, we have shown that, given a point $X$ with line-of-sight to a point $X'$ on some edge, there must be a polygonal path-connected point $Y$ which has line-of-sight to a point $Y'$ on the \emph{first} edge. Formally, we verify:
\begin{equation*}
  % "!'a Ps X P Q hand.
  %  simple_polygon 'a Ps
  %  /\ on_plane X 'a /\ ~on_polyseg Ps X
  %  /\ MEM (P,Q) (ADJACENT Ps) /\ between P hand Q
  %  /\ ~(?Y. between hand Y X /\ on_polyseg Ps Y)
  %  ==> ?hand' X'. seg_connected 'a (on_polyseg Ps) X X'
  %                 /\ between (HD Ps) hand' (HD (TL Ps))
  %                 /\ ~on_polyseg Ps X'
  %                 /\ ~(?Y. between hand' Y X' /\ on_polyseg Ps Y)"
  \begin{split}
\vdash    &\code{simple\_polygon}\ Ps \wedge \neg\code{on\_polypath}\ Ps\ X\\
    &\wedge \code{mem}\ (P,Q)\ (\code{adjacent}\ Ps) \wedge \between{P}{X'}{Q}\\
    &\wedge\neg(\exists Z.\; \between{X}{Z}{X'} \wedge \code{on\_polypath}\ Ps\ Z)\\
    &\qquad\implies \exists Y.\;\exists Y'.\; \code{polypath\_connected}\ (\code{on\_polypath}\ Ps)\ X\ Y\\
    &\qquad\qquad\qquad\wedge\neg\code{on\_polypath}\ Ps\ Y\\
    &\qquad\qquad\qquad\wedge \between{(\code{head}\ Ps)}{Y'}{(\code{head}\ (\code{tail}\ Ps))}\\
    &\qquad\qquad\qquad\wedge \neg\exists Z.\; \between{Y}{Z}{Y'} \wedge \code{on\_polypath}\ Ps\ Z.
  \end{split}
\end{equation*}

Note that for the very first time we are working on the hypothesis that our vertex list $Ps$ defines a simple polygon. We can see why this hypothesis is needed by considering two of the hypotheses from Theorem~\ref{eq:PolygonMove}, which we have said assumes that a polygon is only ``locally'' simple:
\begin{enumerate}
\item $\neg\code{on\_polypath}\ (\cons{P_3}{Ps})\ P_2$;
\item $\neg(\exists Z.\; \between{P_1}{Z}{P_2} \wedge \code{on\_polypath}\ (\cons{P_2}{\cons{P_3}{Ps}})\ Z)$.
\end{enumerate}

By allowing $Ps$ to be an arbitrarily rotated polygon, the first of these assumptions will require that all vertices are distinct and that no vertex appears on another edge. The second will require that the edges themselves do not intersect. This is just to say that the polygonal segment $Ps$ must be simple.

To conclude this section, we will give some perspective on what is involved in the proof so far. In Figure~\ref{fig:SketchProofJordan2Full}, we have reproduced the example from \S\ref{sec:SketchProofJordan2}, showing three players navigating a maze in order to have line-of-sight to the edge $P_1P_2$. However, we can now explain the peculiar shape of the red and blue paths, understanding that they have been obtained by precise applications of Theorem~\ref{eq:Squeeze2} using the auxiliary dashed lines. Each dashed line marks a point on a player's line-of-sight that they move towards, or a point on an edge towards which they rotate. Thus, we see each player navigating without compass or ruler, using only incidence and ordering.

\begin{figure}
  \centering\includegraphics[scale=0.9]{jordanVerification2/SketchProofFull}
  \caption{Navigating a maze with Theorem~\ref{eq:Squeeze2}}
  \label{fig:SketchProofJordan2Full}
\end{figure}

\section{Final Steps}
We are almost there. To finish the proof, we will need to show that when two points are on the same side and looking at the same edge, then they are polygonal path-connected. We shall deal with this matter in \S\ref{sec:SameSideEdgeConnected}. First, we shall deal with the neglected matter of how we are able to obtain line-of-sight to an edge in the first place. Both problems will have us reusing our near ubiquitous squeeze theorem~\eqref{eq:Squeeze2}.

\subsection{Getting onto the Maze}
If we take an arbitrary point $X$ in the plane and an arbitrary point $X'$ on a maze, then a simple ray-cast gives us line-of-sight to some other point $X''$ of the maze. If this point $X''$ lies between two other vertices, we have a line-of-sight to an edge. 

However, if $X''$ coincides with a vertex as shown in Figure~\ref{fig:SightToEdge}, we will need to rotate our line-of-sight slightly. We will first assume, without loss-of-generality (or more accurately, we will assume we have rotated the polygon's vertex list) so that the vertex $X''$ coincides with the vertex $P_2$. We will want to obtain line-of-sight to either the edge $P_1P_2$ or $P_2P_3$.

\begin{figure}
  \centering\includegraphics[scale=0.75]{jordanVerification2/SightToWall}
  \caption{Obtaining line-of-sight with an edge}
  \label{fig:SightToEdge}
\end{figure}

We can do this by applying Theorem~\ref{eq:Squeeze2} to the edge $P_1P_2$ and the polygonal fragment $P_3P_4\ldots P_n$ as described in \S\ref{sec:RotateToNew}. This will not generally give us our required point of intersection. In fact, it might be that there is \emph{no} line-of-sight from $X$ to the edge $P_1P_2$, because the edge $P_2P_3$ is in the way. 

This does not pose much of a problem. If the segment $XY$ intersects $P_3X''$ at the point $Y'$, then we shall just take $Y'$ as our line-of-sight. We just need two steps for this, one exploiting our linear reasoning tactic and the other our incidence discoverer, to show that the segment $XY'$ is our required line-of-sight. On the other hand, if $XY$ does not intersect $P_3X''$, then one step with our incidence discoverer verifies that it is the required line-of-sight. 

We have formalised the result in Figure~\ref{fig:RotateToEdge}, retaining the without-loss-of-generality assumption. This allows us to review which hypotheses are actually needed for this particular theorem, and is one of the great benefits of formal verification. The labour involved in teasing out the necessary hypotheses means we usually end up with very tight lemmas with which to easily trace dependencies. Consider that this theorem appears in our theory file before simple polygons are even \emph{defined}.

\begin{figure}
\begin{equation}\label{eq:RotateToEdge}
  \begin{split}
  % "!P1 P2 P3 Ps X 'a.
  %  on_plane P1 'a /\ on_plane P2 'a /\ on_plane P3 'a /\ on_plane X 'a
  %  /\ (!P. MEM P Ps ==> on_plane P 'a)
  %  /\ ~between P1 P3 P2 /\ ~between P2 P1 P3
  %  /\ ~(P1 = P2) /\ ~(P1 = P3)
  %  /\ ~on_polyseg (CONS P3 Ps) P2
  %  /\ ~(?Z. between P1 Z P2 /\ on_polyseg (CONS P2 (CONS P3 Ps)) Z)
  %  /\ ~(?Z. between P2 Z P3 /\ on_polyseg (CONS P3 Ps) Z)
  %  /\ ~on_polyseg (CONS P1 (CONS P2 (CONS P3 Ps))) X
  %  /\ ~(?Z. between P2 Z X /\ on_polyseg (CONS P1 (CONS P2 (CONS P3 Ps))) Z)
  %  ==> ?Y. (between P1 Y P2 \/ between P2 Y P3)
  %          /\ ~(?Z. between X Z Y
  %                   /\ on_polyseg (CONS P1 (CONS P2 (CONS P3 Ps))) Z)"
\vdash&\neg\between{P_1}{P_3}{P_2} \wedge \neg\between{P_2}{P_1}{P_3} \wedge P_1 \neq P_2 \wedge P_1 \neq P_3\\
    &\wedge\neg\code{on\_polypath}\ (\cons{P_3}{Ps})\ P_2\\
    &\wedge\neg(\exists Z.\; \between{P_1}{Z}{P_2} \wedge \code{on\_polypath}\ (\cons{P_2}{\cons{P_3}{Ps}})\ Z)\\
    &\wedge\neg(\exists Z.\; \between{P_2}{Z}{P_3} \wedge \code{on\_polypath}\ (\cons{P_3}{Ps})\ Z)\\
    &\wedge\neg\code{on\_polypath}\ (\cons{P_1}{\cons{P_2}{\cons{P_3}{Ps}}})\ X\\
    &\wedge\neg(\exists Z.\; \between{P_2}{Z}{X} \wedge \code{on\_polypath}\ (\cons{P_1}{\cons{P_2}{\cons{P_3}{Ps}}})\ Z)\\
    &\implies\exists Y.\; (\between{P_1}{Y}{P_2} \vee \between{P_2}{Y}{P_3})\\
    &\qquad \wedge \neg(\exists Z.\; \between{X}{Z}{Y} \wedge \code{on\_polypath}\ (\cons{P_1}{\cons{P_2}{\cons{P_3}{Ps}}})\ Z).
  \end{split}
\end{equation}
\caption{Rotating a line-of-sight to an edge}
\label{fig:RotateToEdge}
\end{figure}

We can now get any point to have line-of-sight to the maze, and thus we can connect every point in the plane to another point with line-of-sight to any edge. This means that, if we have three points in the plane, we can find three paths which bring them together at the same edge. It is enough now to verify that two of the three points are polygonal path-connected.

\subsection{There are at Most Two Regions}\label{sec:SameSideEdgeConnected}
Here is where we are: we have three points with line-of-sight to an edge $P_1P_2$ (without loss of generality). We know that two of these points $X$ and $Y$ are on the same side of $P_1P_2$. We will show that these two points can be polygonal path-connected.

First off, we have to consider that $X$ and $Y$ have line-of-sight to \emph{different} points $X'$ and $Y'$. Our diagrams so far in this chapter have given a different impression, but we must remember that the points obtained in our proofs ultimately rely on Axiom~\ref{eq:g22}. This axiom allows us to extend a line-segment, but there are many possible witnesses for its existential conclusion, and we have generally allowed the abstraction to proliferate in our theory, rather than eliminating it with the epsilon-operator. 

\begin{figure}
\centering\includegraphics[scale=0.6]{jordanVerification2/SameSideWallConnected1}
\caption{Connecting the final points}
\label{fig:SameSideEdgeConnected}
\end{figure}

Consider the scenario depicted in Figure~\ref{fig:SameSideEdgeConnected}(a). Here, our points $X$ and $Y$ have line-of-sight to different points on the same edge, so our first order of business is to get $X$ and $Y$ to have line-of-sight to the same point. Again, we use Theorem~\ref{eq:Squeeze2} against the fragment $P_2P_3\ldots P_n$, moving $X$ along its line-of-sight to a point $X''$ so that it now has line-of-sight with $Y'$. A second application of the same theorem, moving along our new line-of-sight sees us facing the point $Y$. With the two points in each other's sights, we have the last piece of our path.

We are almost home free. We still need to factor in the fact that $X$ and $Y$ are on the same side of $P_1P_2$. This is needed because we only applied Theorem~\ref{eq:Squeeze2} to the fragment $P_2P_3\ldots P_n$. The resulting lines-of-sight will not intersect this fragment, but we will need to account for the segment $P_1P_2$ separately.

Yet again, this is settled with our theorems for half-planes. Since $X$ lies on the same side of $P_1P_2$ as $Y$, and $X'$ lies on $P_1P_2$, all points on the segment $XX''$ must also lie on the same side. The same argument applies to the segment $X''Y'$, and finally to $Y'Y$. Since all of these points are on the same side of $P_1P_2$, they must lie off the line $P_1P_2$. 

The final extract of the verified proof, witnessing the final path and showing this line of argument, is reproduced in Figure~\ref{fig:SameSideEdgeConnectedExtract}. We have omitted references to earlier steps.

\begin{equation}
  \label{eq:SameSideEdgeConnected}
  \begin{split}
  % "!P1 P2 Ps X Y hand hand' hp 'a. 
  %    on_plane P1 'a /\ on_plane P2 'a
  %    /\ (!P. MEM P Ps ==> on_plane P 'a)
  %    /\ on_plane X 'a /\ on_plane Y 'a
  %    /\ between P1 hand P2 /\ between P1 hand' P2
  %    /\ ~on_polyseg (CONS P1 (CONS P2 Ps)) X
  %    /\ ~on_polyseg (CONS P1 (CONS P2 Ps)) Y
  %    /\ ~(?Z. between X Z hand /\ on_polyseg (CONS P1 (CONS P2 Ps)) Z)
  %    /\ ~(?Z. between Y Z hand' /\ on_polyseg (CONS P1 (CONS P2 Ps)) Z)
  %    /\ ~(?Z. between P1 Z P2 /\ on_polyseg (CONS P2 Ps) Z)
  %    /\ on_line P1 (line_of_half_plane hp) /\ on_line P2 (line_of_half_plane hp)
  %    /\ on_half_plane hp X /\ on_half_plane hp Y
  %    ==> seg_connected 'a (on_polyseg (CONS P1 (CONS P2 Ps))) X Y"
\vdash &\neg\code{on\_polypath}\ (\cons{P_1}{\cons{P_2}{Ps}})\ X \wedge\neg\code{on\_polypath}\ (\cons{P_1}{\cons{P_2}{Ps}})\ Y\\
    &\wedge\between{P_1}{X'}{P_2}\wedge\between{P_1}{Y'}{P_2}\\
    &\wedge\neg(\exists Z.\; \between{X}{Z}{X'} \wedge \code{on\_polypath}\ (\cons{P_1}{\cons{P_2}{Ps}})\ Z)\\
    &\wedge\neg(\exists Z.\; \between{Y}{Z}{Y'} \wedge \code{on\_polypath}\ (\cons{P_1}{\cons{P_2}{Ps}})\ Z)\\
    &\wedge\neg(\exists Z.\; \between{P_1}{Z}{P_2} \wedge \code{on\_polypath}\ (\cons{P_2}{Ps})\ Z)\\
    &\wedge \code{on\_line}\ P1\ (\code{line\_of\_half\_plane}\ hp) \wedge \code{on\_line}\ P2\ (\code{line\_of\_half\_plane}\ hp)\\
    &\wedge \code{on\_half\_plane}\ hp\ X \wedge \code{on\_half\_plane}\ hp\ Y\\
    &\implies \code{polypath\_connected}\ (\code{on\_polypath}\ (\cons{P_1}{\cons{P_2}{Ps}}))\ X\ Y.
  \end{split}
\end{equation}

\begin{boxedfigure}
\small
\begin{align*}
  % ;take ["[X:point;s:point;s':point;Y:point]"]
  % ;tactics [REWRITE_TAC [NOT_CONS_NIL;GSYM IN_SET_OF_LIST;HD;LAST
  % ;DISJOINT_IMP;set_of_list]
  % THEN REWRITE_TAC [FORALL_IN_INSERT;NOT_IN_EMPTY]]
  % ;obviously (by_planes o Di.conjuncts)
  % (thus "on_plane s 'a /\ on_plane s' 'a 
  % /\ on_plane X 'a /\ on_plane Y 'a"
  % from [0;2;3;11;12;18])
  % ;have "on_line hand (line_of_half_plane hp)
  % /\ on_line hand' (line_of_half_plane hp)"
  % from [3;8] by [g12;g21]
  % ;hence "on_half_plane hp s /\ on_half_plane hp s'
  % /\ !Z. between s Z s' \/ between s' Z Y
  % ==> on_half_plane hp Z"
  % from [8;12;16;18] by [bet_on_half_plane;bet_on_half_plane2]
  % at [20]
  % ;have "!Z. on_half_plane hp Z ==> ~on_polyseg [P1;P2] Z"
  % from [3;8] by [on_polyseg_pair;g12;g21;half_plane_not_on_line]
  % ;hence "!Z. on_polyseg [s;s';Y] Z ==> ~on_polyseg [P1;P2] Z"
  % from [8;20] by [on_polyseg_CONS2;on_polyseg_sing] at [21]
  % ;have "!Z. between s Z s' ==> ~on_polyseg (CONS P2 Ps) Z" proof
  % [fix ["Z:point"]
  % ;assume "between s Z s'"
  % ;hence "between s Z hand'" from [18]
  % using ORDER_TAC `{hand':point,s,s',Z}`
  % ;qed from [13]]
  % ;qed from [4;13;14;18;19;21] by
  % [IN;on_polyseg_CONS2;on_polyseg_sing;BET_SYM]]]];;
  &\code{take}\ [X,X'',X''',Y]\\
  &\code{have}\ \code{on\_line}\ X'\ (\code{line\_of\_half\_plane}\ hp)\\
  &\qquad\wedge \code{on\_line}\ Y'\ (\code{line\_of\_half\_plane}\ hp)\\
  &\qquad\code{from}\ \ldots\ \code{by}\ \eqref{eq:g12},\eqref{eq:g21}\\
  &\code{hence}\ \code{on\_half\_plane}\ hp\ X'' \wedge \code{on\_half\_plane}\ hp\ X'''\\
  &\qquad\wedge\forall Z.\; \between{X''}{Z}{X'''} \vee \between{X'''}{Z}{Y}\ \code{from} \ldots\ \code{by}\ \eqref{eq:betOnHalfPlane1},\eqref{eq:betOnHalfPlane2}\\
  &\code{have}\ \forall Z.\; \code{on\_half\_plane}\ hp\ Z \implies \neg\code{on\_polypath}\ [P_1,P_2]\ Z\
  \code{from}\ 3,8\\
  &\qquad\code{by}\ \eqref{eq:OnPolyPath},\eqref{eq:g12},\eqref{eq:g21},\eqref{eq:halfPlaneNotOnLine}\\
  &\code{hence}\ \forall Z.\; \code{on\_polypath}\ [X'',X'',Y]\ Z \implies \neg\code{on\_polypath}\ [P_1,P_2]\ Z\\
  &\qquad\code{from} \ldots\ \code{by}\ \eqref{eq:OnPolyPath}\ & 21\\
  &\code{have}\ \forall Z.\; \between{X''}{Z}{X'''} \implies \neg\code{on\_polypath}\ (\cons{P_2}{Ps})\ Z\\\
  &\code{proof:}\ \code{fix}\ Z\\
  &\qquad \code{assume}\ \between{X''}{Z}{X'''}\\
  &\qquad \code{hence}\ \between{X''}{Z}{Y'}\ \code{from}\ \ldots\ \code{using}\ \code{ORDER\_TAC}\ \{X'',X''',Y',Z\}\\
  &\qquad \code{qed}\ \code{from}\ \ldots\\
  &\code{qed}\ \code{from}\ \ldots,21\ \code{by}\ \eqref{eq:OnPolyPath},\eqref{eq:g21}
\end{align*}
\caption{Verification Extract for Theorem~\ref{eq:SameSideEdgeConnected}}
\label{fig:SameSideEdgeConnectedExtract}
\end{boxedfigure}

\section{Conclusion}
The proof of the final result, Theorem~\ref{eq:jordanFormal2}, puts all of the pieces together. We start from three arbitrary points in the plane not on the polygonal segment, have each obtain a line-of-sight to an edge of the maze, and then use polygon rotations and Theorem~\ref{eq:PolygonMove} to find three paths to three points with line-of-sight to the first edge of the maze. We then apply  Theorem~\ref{eq:HalfPlaneCover} which says that two of these points must be on the same side of the first edge, and thus, from Theorem~\ref{eq:SameSideEdgeConnected}, we know that these two points are polygonal path-connected.

%As with the final theorem in the last chapter, the final theorem here is much more readable than the lemmas considered up to now. The theory so far builds theorems with large numbers of complicated hypotheses, such as those for Theorem~\ref{eq:SameSideEdgeConnected} (and remember that for brevity, we have omitted all the planar hypotheses needed for every one of these theorems). When developing a theory such as this, where theorems are almost made opaque by their hypotheses, we have to pay particularly close attention to be confident that we are making progress. Fortunately, success in formal verification is unambiguous.
\begin{equation}\tag{\ref{eq:jordanFormal2}}
\begin{aligned}
\vdash &\code{simple\_polygon}\ \alpha\ Ps\\
       &\wedge \code{on\_plane}\ P\ \alpha \wedge \code{on\_plane}\ Q\ \alpha \wedge \code{on\_plane}\ R\ \alpha\\
       &\wedge \neg\code{on\_polypath}\ Ps\ P\wedge \neg\code{on\_polypath}\ Ps\ Q\wedge \neg\code{on\_polypath}\ Ps\ R\\
       &\implies \code{polypath\_connected}\ \alpha\ (\code{on\_polypath}\ Ps)\ P\ Q\\
       &\qquad\quad\vee \code{polypath\_connected}\ \alpha\ (\code{on\_polypath}\ Ps)\ P\ R\\
       &\qquad\quad\vee \code{polypath\_connected}\ \alpha\ (\code{on\_polypath}\ Ps)\ Q\ R
\end{aligned}
\end{equation}

%%% Local Variables: 
%%% TeX-master: "../thesis"
%%% End: 
