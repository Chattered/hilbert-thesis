\chapter{Introduction}\label{chapter:Introduction}
This project began as an attempt to mechanise the \emph{Foundations of Geometry} in a computerised proof assistant. The source text was published by David Hilbert following his lecture series in the foundations of geometry in a culture of careful reflection on the foundations of mathematics and a turn to ever more mathematical rigour. The book has ten editions in both German and English, and for geometry, it was called the most influential book in a hundred years~\cite{BirkhoffHilbertInfluence}. 

At around the time of its publication, there were programmes for codifying the laws of thought in a way which could recover mathematical arguments, being developed by the Italian school under Peano. There were also British engineers such as De Morgan and Stanley Jevons designing machines to automate the application of logical laws such as those of George Boole. It is small wonder then that we find our modern goals of mechanising the \emph{Foundations of Geometry} anticipated in reviews by Veblen and Poincar\'{e}~\cite{VeblenHilbertReview,PoincareReview}, where we read casual claims that Hilbert's proofs could be derived by such simple machines.

Since Russell and Whitehead's Herculean efforts to derive mathematics from axiomatic foundations, which we are told left Russell so exhausted that he never fully recovered, we have known that there is a gap between the practices of rigorous mathematics and the requirements of mechanisation, and that computer assistance is needed in order for mechanisation to be feasible~\cite{FormalizedMathematics}. Since DeBruijn's first mechanisation, those working in formalised mathematics have assigned a numerical value to this gap in the form of the ``DeBruijn factor''. For a perspicious example of the gap, consider Lamport's recent comments about the proof that the sum of the first $n$ triangle is $\tfrac{1}{2}n(n+1)$~\cite{ProofMessageCertificate}.

When they began verifying the \emph{Foundations of Geometry}, Meikle and Fleuriot~\cite{MeikleFleuriotFormalizingHilbert} emphasised the gaps between verification and prose proof, but we wanted to close it down as much as possible, to reach a DeBruijn factor near to 1. The aim then is to produce a \emph{faithful} mechanisation of Hilbert's text. In the present thesis, we try to make the case that the main contribution of a faithful mechanisation lies in the critical appraisal it allows us to make of a prose text.

We should stress that the main contributions of this thesis are intended to be the verifications themselves, available at \url{https://github.com/Chattered}, as a case-study in using the simple declarative combinator language Mizar~Light to reproduce declarative, synthetic geometric proofs, and the automation which allows these proofs to achieve a very low DeBruijn factor. In describing these verifications, we 

