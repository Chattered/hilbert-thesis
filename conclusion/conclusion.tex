\chapter{Conclusion}\label{chapter:Conclusion}
In an article published in the \emph{American Mathematical Monthly}~\cite{GuggenheimerJordanProof}, on the validity of the Polygonal Jordan Curve Theorem in ordered geometry, almost forgotten, Guggenheimer laments:
\begin{quotation}It is astonishing that none of the textbooks of elementary axiomatic geometry gives a proof.
\end{quotation}

We find it more astonishing considering that the axioms of ordered geometry mark the first major contribution to the modern rigorisation of geometry by Pasch, ``the father of rigour in geometry''. It stretches all our credibility when those same axioms were delineated in the most influential textbook on modern axiomatic geometry, nearly eight decades earlier,  and the theorem stated as derivable, by David Hilbert, one of the greatest mathematicians who ever lived.

Such is hindsight. We confess that we had little interest in the theorem initially. Hilbert implied it was trivial, that, in the \emph{Grundlagen}, the interesting proofs only appear after the later groups of axioms. If not for the demands of formal verification, we would have missed the literal labyrinths that were left unexplored in the wake of his first two groups. We would have almost certainly glossed over the details, letting ourselves be convinced by invalid proofs --- Veblen's, our own, and whatever Hilbert had in mind when he declared the result obtainable without much difficulty. Had it not been for formal verification, we would not have done justice to ordered geometry.

Formal verification forces us through the labyrinths, and the final outcome is a confidence and authority that would make paragons of logical rigour such as Pasch and Hilbert envious. This is a new \emph{standard} of rigour. It has been said~\cite{ConvinceEnemy} that a mathematical proof must make it through three filters: it must convince oneself; it must convince a friend; and it must convince an enemy. There is a final filter, originating with the LCF tradition~\cite{LCF}: it must convince a blind and merciless symbol crunching machine. Here, we cannot exploit the faith of friends, nor protest our way through the intelligent scrutiny of our enemies. There is no appeal. We either mechanically construct our theorem or we do not.

The standard is unequivocal. A theorem prover does not implement a \emph{test} of verification; it is the \emph{definition} of verification. Finishing is as decisive as finishing a jigsaw puzzle. The practitioners know precisely what success means before they ever embark. In our case, once the formalisations from Chapter~\ref{chapter:JordanFormalisation} were input to the computer, the criterion of success could not be more tangible: our theorem prover must simply announce ``No subgoals.'' It was easy to imagine the precise moment with giddy anticipation, knowing it would mean that a matter had been forever closed, that our task had been completed without any possibility of there being a single loose end. Consulting a third-party would be a spectacular redundancy. We do not need mere humans to check our work. Their judgements are irrelevant. 

Rarely is success so definitive, expectation so intense, and the final pay-off so rewarding. But let us be clear: there were bleak moments. Mechanical theorem proving may have a definitive criterion of success, but it is useless at measuring ongoing progress. Our adventures in verifying the Polygonal Jordan Curve Theorem were undertaken with \emph{faith}, for we were always haunted by doubt, of the thought that our efforts were for nought, our verifications to be consigned to the bin. Even at the very end, with the key lemmas verified, we were not completely convinced that the theorem was even \emph{valid} in ordered geometry. We had to accept that we could have been misled by sketch proofs and intuition, like our predecessors who had far better credentials. 

We could sweat out a major lemma such as \eqref{eq:changeTriangle} from Appendix~\ref{app:JordanVerificationExtra}, but what real assurance could this give? We were tottering on a rickety scaffold of symbolic definitions, pinning our hopes on theorems with half a dozen opaque hypotheses that seemed to be miles from our intended destination. Had we missed something crucial? Would we find that our strategy was irrevocably thwarted at the final hurdle? This was uncharted territory.

And certainly, it was risky. Sage advice says you should not verify proofs of which you are not already absolutely convinced. The time investment is just too much. But we had ignored this advice and based our verification on a proof we were certain was \emph{wrong}. It was a proof which Guggenheimer claims only works for convex polygons, and had we failed to verify it, we would have only had ourselves to blame.

We only dared the venture because we were confident that we had the right automation. Otherwise, we could scarcely imagine how we would have completed the verification. Of our roughly 1400 declarative proof steps, 111 were handled by the incidence automation of Chapter~\ref{chapter:Automation} and 82 by the linear reasoning automation of Chapter~\ref{chapter:LinearOrder}. This automation relieved us of tedium that we know \emph{dominates} the underlying verifications~\cite{ScottMScThesis}. Once available, the evidence of Chapter~\ref{chapter:Group2Eval} convinced us that we could produce readable verifications of a granularity comparable to those of a model mathematician's proofs. This was to be no grind. It was an \emph{exploration}.

And the outcome of our exploration is that, with huge relief, we finally found the theorem. It is really there, a concrete object that can be reconstructed any number of times by rerunning our code, or rediscovered by following the detailed signposts of our declarative verification. The proof of its existence is a fitting end to a story which begins  with Hilbert setting a new standard of rigour which first made it possible to state the theorem. With machines, we set the \emph{ultimate} standard of rigour, and conclude the story, confirming once and for all that we have scoured every corridor of the mazes of ordered geometry.

%%% Local Variables: 
%%% mode: latex
%%% TeX-master: "../thesis"
%%% End: 